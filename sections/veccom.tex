% !TEX root = ../main.tex
\section{Homomorphic Vector Commitments}\label{sec:veccom}

\begin{definition}\label{def:hvc}
  Let $\ring$ be a ring and let $q=q(\secpar)\in\NN$.
  A homomorphic vector commitment scheme (HVC) for domain $\ring^\domlen_q$ is defined by four PPT algorithms $(\setup,\commit,\allowbreak\open,\verify)$.
\begin{description}
    \item[$\params\gets\setup(\secparam,\tau)$] The setup algorithm takes as input the security parameter and the binary logarithm of the length of the committed vectors and outputs public parameters.
    \item[$c \gets \commit(\params,\vec{m})$] The commitment algorithm gets as input the public parameters and a vector $\vec{m}\in(\ring_q^\domlen)^{2^\tau}$ and outputs a commitment $c\in\ring_q^\comlen$.
    \item[$d \gets \open(\params,c,\vec{m},t)$] The opening algorithm gets as input the public parameters, a commitment, the committed vector, and an index and outputs a decommitment $d\in\ring_q^{\oplen}$.
    \item[$\vec{m}/\bot\gets \wverify(\params,c,t,d)$] The weak verification algorithm takes as input public parameters, a commitment, an index, and a decommitment and outputs either $\vec{m}\in\ring^\domlen_q$ or an error symbol.
    \item[$\vec{m}/\bot\gets \sverify(\params,c,t,d)$] The strong verification algorithm takes as input public parameters, a commitment, an index, and a decommitment and outputs either $\vec{m}\in\ring^\domlen_q$ or an error symbol.
  \end{description}
  Let $\rho\in\NN$ and $W\subseteq\ring$.
  A vector commitment is $(\rho,W)$-homomorphically correct, if for all security parameters $\secpar\in\NN$, vector lengths $2^\tau=\poly$, $\ell\in[\rho]$, vectors $\vec{m}^0,\dots,\vec{m}^{\ell-1}\in(\ring_q^\domlen)^{2^\tau}$, ring elements $w^0,\dots,w^{\ell-1}\in W$, and indices $t\in[2^\tau-1]$ it holds that
  \[
    \Pr\left[
      \begin{aligned}
      \params\gets&\mspace{\medmuskip}\setup(\secparam,\tau);\\
      \vec{c}^i \gets&\mspace{\medmuskip} \commit(\params,\vec{m}^i);\\
      \vec{d}^i \gets&\mspace{\medmuskip} \open(\params,\vec{c}^i,\vec{m}^i,t)\\
      \end{aligned}\!\!\!\!\!\!:
      \sverify\Bigl(\params,\sum_{i=0}^{\ell-1}w^i\cdot \vec{c}^i,t,\sum_{i=0}^{\ell-1}w^i\cdot \vec{d}^i\Bigr) = \sum_{i=0}^{\ell-1}w^i\cdot\vec{m}^i_{t}
    \right]=1
  \]
\end{definition}
\begin{remark}
  Note that the homomorphic correctness definition above implies regular correctness of unaggregated commitments with $\ell=1$ and $1\in W$.
\end{remark}
\begin{definition}[Position-Binding]
  An HVC is position binding if for all security parameters $\secpar$ and all PPT algorithms $\adv$ it holds that
  \[
    \Pr\left[
      \begin{aligned}
      \params\gets&\mspace{\medmuskip}\setup(\secparam,\tau);\\
      (c,t,d_0,d_1) \gets&\mspace{\medmuskip} \adv(\params);\\
      \vec{m}_0 \gets&\mspace{\medmuskip} \wverify(\params,c,t,d_0);\\
      \vec{m}_1 \gets&\mspace{\medmuskip} \wverify(\params,c,t,d_1)
      \end{aligned}:
      \begin{aligned}&\mspace{\medmuskip}\vec{m}_0\neq \vec{m}_1\\ \land&\mspace{\medmuskip}\bot\not\in\{\vec{m}_0,\vec{m}_1\}\end{aligned}
    \right]\leq \negl.
  \]
\end{definition}
We require that a limited homomorphism holds, even for malicious commitments.
For any two, even malicious, commitments and their two respective openings that \emph{strongly} verify, their difference will still weakly verify.

\begin{definition}
  \label{def:malhomhvc}
  Let $\hvc$ be a vector commitment scheme (HVC) for domain $\ring^\domlen_q$ with commitment length $\comlen$ and decommitment length $\oplen$.
  $\hvc$ is robustly homomorphic if for all security parameters $\secpar\in\NN$, vector lengths $2^\tau=\poly$, public parameters $\params\gets\setup(\secparam,\tau)$, indices $t\in[2^\tau-1]$, (possibly malformed) commitments $\vec{c}^0,\vec{c}^1 \in \ring_q^\comlen$, and (possibly malformed) decommitments $\vec{d}^0,\vec{d}^1\in\ring_q^\oplen$ such that
  \[
    \sverify(\params,\vec{c}^0, t,\vec{d}^0)=\vec{m}^0 \quad \text{and} \quad \sverify(\params,\vec{c}^1, t,\vec{d}^1)=\vec{m}^1
  \]
  with $\vec{m}^0,\vec{m}^1\neq \bot$ it holds that
  \[
    \wverify(\params,\vec{c}^0 - \vec{c}^1, t,\vec{d}^0 - \vec{d}^1)=\vec{m}^0 - \vec{m}^1.
  \]
\end{definition}


\subsection{Homomorphic Vector Commitment for $\ring_q$}

We now define a decomposition function that allows us to map a ring element with possibly large norm to a vector of low norm ring elements.
To be able to use the greates arity while minimizing the infinity norm, we use a balanced $(2\eta+1)$-ary decomposition.
We note that any even arity, such as the binary decomposition used by Squirrel~\cite{CCS:FleSimZha22} is strictly worse than the next greater odd arity.
We then show that the decomposition function has nice homomorphic properties.
\begin{definition}[Balanced $(2\eta+1)$-ary decomposition of $\ring_q$ elements]
    For any $a = \sum_{i=0}^{n-1} a_i\cdot x^i  \in \ring_q$,
    denote by $(a_{i,0},\dots,a_{i,\lceil \log_{2\eta+1} q \rceil-1})^\intercal\in \{-\eta,\dots,\eta\}^{\lceil\log_{2\eta+1} q\rceil}$ the balanced $(2\eta+1)$-ary decomposition of $a_i$, i.e.,
    \[a_{i} := \smashoperator{\sum_{j=0}^{\lceil\log_{2\eta+1} q\rceil - 1}} a_{i,j}\cdot (2\eta+1)^j.\]
    We define the following decomposition of $a$ into polynomials with coefficients in $\{-\eta,\dots,\eta\}$:
    \begin{equation*}
        \decomp_q: \ring_q \to \ring_q^{\lceil\log_{2\eta+1} q\rceil},\quad
        \decomp_q(a) = \left(\sum_{i=0}^{n-1} a_{i,0}\cdot x^i, \dots, \sum_{i=0}^{n-1} a_{i,\lceil\log q\rceil -1}\cdot x^i \right).
    \end{equation*}
\end{definition}

\begin{definition}[Projection onto $\ring_q$ elements]\label{def:proj}
  For any $\vec{b} \in \ring_q^{\lceil\log_{2\eta+1} q\rceil}$ we define the function 
  \[
    \proj_q : \ring_q^{\lceil\log q\rceil} \to \ring_q,\quad \proj(\vec{b}) = \smashoperator{\sum_{j=0}^{\lceil\log_{2\eta+1} q\rceil -1}} (2\eta+1)^j\cdot b_j.
  \]
\end{definition}

For the sake of readability we will omit $q$ and simply write $\decomp$ and $\proj$ whenever the modulus is clear from context.

The following two simple lemmas effectively states that the projection function is the inverse of the decomposition function and that the projection function is $\ring_q$-linear.
\begin{lemma}\label{lem:projinvofbin}
  For all $a = \sum_{i=0}^{n-1} a_i\cdot x^i \in\ring_q$, it holds that $\proj(\decomp(a))=a$.
\end{lemma}

\begin{proof}
    \begin{align*}
      \proj(\decomp(a)) =& \proj\left(\sum_{i=0}^{n-1} a_{i,0}\cdot x^i, \dots, \sum_{i=0}^{n-1} a_{i,\lceil\log_{2\eta+1} q\rceil -1}\cdot x^i\right)\\
      =& \smashoperator{\sum_{j=0}^{\lceil\log_{2\eta+1} q\rceil -1}} (2\eta+1)^j \cdot \sum_{i=0}^{n-1} a_{i,j}\cdot x^i\\
      =& \sum_{i=0}^{n-1} x^i\cdot \smashoperator{\sum_{j=0}^{\lceil\log_{2\eta+1} q\rceil -1}} (2\eta+1)^j \cdot a_{i,j}\\
      =& \sum_{i=0}^{n-1} x^i\cdot a_i=a\tag*{\qed}
    \end{align*}
\end{proof}
  

\begin{lemma}\label{lem:projislin}
  The projection function $\proj$ is $\ring_q$-linear, i.e., for any $\vec{b}^0,\vec{b}^1 \in \ring_q^{\lceil\log_{2\eta+1} q\rceil}$ and any $w^0,w^1 \in \ring_q$, $\proj(w^0\cdot\vec{b}^0+w^1\cdot\vec{b^1}) = w^0\cdot\proj(\vec{b^0}) + w^1\cdot\proj(\vec{b}^1)$.
\end{lemma}
\begin{proof}
  \begin{align*}
    \vspace{-1cm}
    \proj(w^0\cdot\vec{b}^0+w^1\cdot\vec{b^1})
    =&\smashoperator{\sum_{j=0}^{\lceil\log_{2\eta+1} q\rceil -1}} (2\eta+1)^j\cdot (w^0\cdot b^0_j+w^1\cdot b^1_j)\tag{\autoref{def:proj}}\\
    =&w^0\cdot \Bigl(\smashoperator{\sum_{j=0}^{\lceil\log_{2\eta+1} q\rceil -1}} (2\eta+1)^j\cdot b^0_j\Bigr) + w^1\cdot\Bigl( \smashoperator{\sum_{j=0}^{\lceil\log_{2\eta+1} q\rceil -1}} (2\eta+1)^j\cdot b^1_j\Bigr)\\
    =&w^0\cdot \proj(\vec{b}^0) + w^1\cdot\proj(\vec{b}^1)\tag{\autoref{def:proj}}
  \end{align*}
  \qed
\end{proof}
