% !TEX root = ../main.tex
\section{Homomorphic Vector Commitments}\label{sec:veccom}

In this section, we define and instantiate the notion of a homomorphic vector commitment.
A vector commitment in general allows to to commit to a very long vector with a very short commitment value.
Positions in the vector can then be individually opened using a short opening value.

Following very closely the definitions of~\cite{CCS:FleSimZha22} vector commitments are extended with homomorphic properties such that linear combination of commitments to different vectors will open to the linear combination of the entries of the individual vectors.

Our definitions depart from \cite{CCS:FleSimZha22} in that we have incomparable requirements on the homomorphic properties.
On the one hand, \cite{CCS:FleSimZha22} only requires that honestly generated have homomorphic properties whereas we have the stronger requirement that the homomorphism works for any individually verifying commitments and openings.
On the other hand, \cite{CCS:FleSimZha22} requires that the homomorphism works with probability $1$, whereas we allow some noticeable error.
This weakening, among other changes, allows us to instantiate the primitive more compactly.

\begin{definition}\label{def:hvc}
  Let $\ring$ be a ring and let $q,q',\tau\in\NN$.
  A homomorphic vector commitment scheme (HVC) for domain $\ring^\domlen_{q'}$ and vectors of length $2^\tau$ is defined by four PPT algorithms $\hvc=(\setup,\commit,\allowbreak\open,\verify)$.
\begin{description}
    \item[$\params\gets\setup(\secparam)$] The setup algorithm takes as input the security parameter and outputs public parameters.
    \item[$c \gets \commit(\params,\vec{m})$] The commitment algorithm gets as input the public parameters and a vector $\vec{m}\in(\ring_{q'}^\domlen)^{2^\tau}$ and outputs a commitment $c\in\ring_q^\comlen$.
    \item[$d \gets \open(\params,c,\vec{m},t)$] The opening algorithm gets as input the public parameters, a commitment, the committed vector, and an index and outputs a decommitment $d\in\ring_q^{\oplen}$.
    \item[$\vec{m}/\bot\gets \iverify(\params,c,t,d)$] The individual verification algorithm takes as input public parameters, a commitment, an index, and a decommitment and outputs either $\vec{m}\in\ring^\domlen_{q'}$ or an error symbol.
    \item[$\vec{m}/\bot\gets \sverify(\params,c,t,d)$] The strong verification algorithm has the same input and output domains as the individual verification algorithm.
    \item[$\vec{m}/\bot\gets \wverify(\params,c,t,d)$] The weak verification algorithm has the same input and output domains as the individual verification algorithm.
  \end{description}
\end{definition}
\begin{definition}[Individual Correctness]
Let $\hvc$ be a vector commitment scheme for domain $\ring^\domlen_{q'}$ and vector length $2^\tau$.
  $\hvc$ is individually correct, if for all security parameters $\secpar\in\NN$, vectors $\vec{m}\in(\ring_{q'}^\domlen)^{2^\tau}\!\!$, indices $t\in[2^\tau]$, parameters $\params \gets \setup(\secparam)$, commitments $\vec{c} \gets \commit(\params,\vec{m})$, and decommitments $\vec{d} \gets \open(\params,\vec{c},\vec{m},t)$ it holds that
  \[
    \iverify\Bigl(\params,\vec{c},t,\vec{d}\Bigr) = \vec{m}_{t}.
  \]
\end{definition}
%
We require that individually verifying commitments and their respective decommitments can be homomorphically aggregated by computing a random linear combination of them.
Such aggregated commitments and decommitments should still \emph{strongly} verify with high probability over the choice of the random linear combination.

\begin{definition}[Probabilistic Homomorphism]
Let $\hvc$ be a vector commitment scheme for domain $\ring^\domlen_{q'}$ and vector length $2^\tau$ with commitment length $\comlen$ and decommitment length $\oplen$.
Let $\rho,\varepsilon\in\NN$ and $W\subseteq\ring$.
$\hvc$ is $(\rho,W,\varepsilon)$-probabilistically homomorphic, if for all security parameters $\secpar\in\NN$, number of aggregated commitments $\ell\in[\rho]$, indices $t\in[2^\tau]$, parameters $\params \gets \setup(\secparam)$, commitments $\vec{c}^i\in\ring_q^{\comlen}$, and decommitments $\vec{d}^i\in\ring_q^{\oplen}$ with $\iverify(\params,\vec{c}^i,t,\vec{d}^i) = \vec{m}^i$ such that $\vec{m}^i \neq \bot$ it holds that
  \[
    \Pr\mleft[
      w^1,\dots,w^{\ell} \gets W :
      \sverify\Bigl(\params,\sum_{i=1}^{\ell}w^i\cdot \vec{c}^i,t,\sum_{i=1}^{\ell}w^i\cdot \vec{d}^i\Bigr) = \sum_{i=1}^{\ell}w^i\cdot\vec{m}^i_{t}
    \mright] \geq 1-2^{-\varepsilon}
  \]
\end{definition}
%
We additionally require that a further limited homomorphism still holds, even for maliciously \emph{aggregated} commitments.
For any two, even maliciously generated, commitments and their two respective openings that \emph{strongly} verify, their difference will still \emph{weakly} verify.

\begin{definition}[Robust Homomorphism]
  \label{def:malhomhvc}
  Let $\hvc$ be a vector commitment scheme for domain $\ring^\domlen_{q'}$ and vector length $2^\tau$ with commitment length $\comlen$ and decommitment length $\oplen$.
  $\hvc$ is robustly homomorphic if for all security parameters $\secpar\in\NN$, public parameters $\params\gets\setup(\secparam)$, indices $t\in[2^\tau]$, (possibly malformed) commitments $\vec{c}^0,\vec{c}^1 \in \ring_q^\comlen$, and (possibly malformed) decommitments $\vec{d}^0,\vec{d}^1\in\ring_q^\oplen$ with
  \[
    \sverify(\params,\vec{c}^0, t,\vec{d}^0)=\vec{m}^0 \quad \text{and} \quad \sverify(\params,\vec{c}^1, t,\vec{d}^1)=\vec{m}^1
  \]
  such that $\vec{m}^0,\vec{m}^1\neq \bot$ it holds that
  \[
    \wverify(\params,\vec{c}^0 - \vec{c}^1, t,\vec{d}^0 - \vec{d}^1)=\vec{m}^0 - \vec{m}^1.
  \]
\end{definition}
%
Finally we require the commitments to be position binding.
\begin{definition}[Position-Binding]
  Let $\hvc$ be a vector commitment scheme.
  $\hvc$ is position binding if for all security parameters $\secpar$ and all PPT algorithms $\adv$ it holds that
  \[
    \Pr\mleft[
      \begin{aligned}
      \params\gets{}&\setup(\secparam);\\
      (\vec{c},t,\vec{d}_0,\vec{d}_1) \gets{}& \adv(\params);\\
      \vec{m}_0 \gets{}& \wverify(\params,c,t,d_0);\\
      \vec{m}_1 \gets{}& \wverify(\params,c,t,d_1)
      \end{aligned}:
      \vec{m}_0\neq \vec{m}_1 \land \bot\not\in\{\vec{m}_0,\vec{m}_1\}
    \mright]\leq \negl.
  \]
\end{definition}


\subsection{A Homomorphic Vector Commitment based on Ring-SIS}
To construct a homomorphic vector commitment with the properties defined above we first define a decomposition function that allows us to map a ring element with possibly large norm to a vector of low norm ring elements.
To be able to use the greatest arity while minimizing the infinity norm, we use a balanced $(2\eta+1)$-ary decomposition.
We note that any even arity, such as the binary decomposition used by Squirrel~\cite{CCS:FleSimZha22} is strictly worse than the next greater odd arity.
We then show that the decomposition function has nice homomorphic properties.
\begin{definition}[Balanced $(2\eta+1)$-ary decomposition of $\ring_q$ elements]
    For any $a = \sum_{i=1}^{n} a_i\cdot x^{i-1}  \in \ring_q$,
    denote by $(a_{i,1},\dots,a_{i,\lceil \log_{2\eta+1} q \rceil})^\intercal\in \{-\eta,\dots,\eta\}^{\lceil\log_{2\eta+1} q\rceil}$ the balanced $(2\eta+1)$-ary decomposition of $a_i$, i.e.,
    \[a_{i} := \smashoperator{\sum_{j=1}^{\lceil\log_{2\eta+1} q\rceil}} a_{i,j}\cdot (2\eta+1)^{j-1}.\]
    We define the following decomposition of $a$ into polynomials with coefficients in $\{-\eta,\dots,\eta\}$:
    \begin{equation*}
        \decomp_q: \ring_q \to \ring^{\lceil\log_{2\eta+1} q\rceil},\quad
        \decomp_q(a) = \left(\sum_{i=1}^{n} a_{i,1}\cdot x^{i-1}, \dots, \sum_{i=1}^{n} a_{i,\lceil\log_{2\eta+1} q\rceil}\cdot x^{i-1} \right).
    \end{equation*}
\end{definition}

\begin{definition}[Projection onto $\ring_q$ elements]\label{def:proj}
  For any $\vec{b} \in \ring_q^{\lceil\log_{2\eta+1} q\rceil}$ we define the function 
  \[
    \proj_q : \ring^{\lceil\log_{2\eta+1} q\rceil} \to \ring_q,\quad \proj_q(\vec{b}) = \smashoperator{\sum_{j=1}^{\lceil\log_{2\eta+1} q\rceil}} b_j\cdot(2\eta+1)^{j-1}.
  \]
\end{definition}
%
The following two simple lemmas effectively state that the projection function is the inverse of the decomposition function and that the projection function is $\ring_q$-linear.
\begin{lemma}\label{lem:projinvofbin}
  For all primes $q$ and $a = \sum_{i=1}^{n} a_i\cdot x^{i-1} \in\ring_q$, it holds that $\proj_q(\decomp_q(a))=a$.
\end{lemma}
\begin{proof}
    \begin{align*}
      \proj_q(\decomp_q(a)) ={}& \proj_q\left(\sum_{i=1}^{n} a_{i,1}\cdot x^{i-1}, \dots, \sum_{i=1}^{n} a_{i,\lceil\log_{2\eta+1} q\rceil}\cdot x^{i-1}\right)\\
      ={}& \smashoperator{\sum_{j=1}^{\lceil\log_{2\eta+1} q\rceil}} \Bigl((2\eta+1)^{j-1} \cdot \sum_{i=1}^{n} a_{i,j}\cdot x^{i-1}\Bigr)\\
      ={}& \sum_{i=1}^{n} \Bigl(x^{i-1}\cdot \smashoperator{\sum_{j=1}^{\lceil\log_{2\eta+1} q\rceil}} (2\eta+1)^{j-1} \cdot a_{i,j}\Bigr)\\
      ={}& \sum_{i=1}^{n} x^{i-1}\cdot a_i=a\tag*{\qed}
    \end{align*}
\end{proof}
  

\begin{lemma}\label{lem:projislin}
  For all primes $q$ the projection function $\proj_q$ is $\ring_q$-linear, i.e., for any $\vec{b}^0,\vec{b}^1 \in \ring_q^{\lceil\log_{2\eta+1} q\rceil}$ and any $w^0,w^1 \in \ring_q$, $\proj_q(w^0\cdot\vec{b}^0+w^1\cdot\vec{b^1}) = w^0\cdot\proj_q(\vec{b^0}) + w^1\cdot\proj_q(\vec{b}^1)$.
\end{lemma}
\begin{proof}
  \begin{align*}
    \vspace{-1cm}
    \proj_q(w^0\cdot\vec{b}^0+w^1\cdot\vec{b^1})
    =&\smashoperator{\sum_{j=1}^{\lceil\log_{2\eta+1} q\rceil}} (2\eta+1)^{j-1}\cdot (w^0\cdot b^0_j+w^1\cdot b^1_j)\tag{\autoref{def:proj}}\\
    =&w^0\cdot \Bigl(\smashoperator{\sum_{j=1}^{\lceil\log_{2\eta+1} q\rceil}} b^0_j\cdot (2\eta+1)^{j-1}\Bigr) + w^1\cdot\Bigl( \smashoperator{\sum_{j=1}^{\lceil\log_{2\eta+1} q\rceil}} b^1_j\cdot (2\eta+1)^{j-1}\Bigr)\\
    =&w^0\cdot \proj_q(\vec{b}^0) + w^1\cdot\proj_q(\vec{b}^1)\tag{\autoref{def:proj}}
  \end{align*}
  \qed
\end{proof}
%
For the sake of readability we will at times abuse notation slightly and apply $\decomp_q$ to \emph{vectors} of $\ring_q$ elements, which is to be understood as the componentwise application of $\decomp_q$ with subsequent concatenation of the resulting vectors.
Similarly, $\proj_j$ may be applied to vectors of a length that is a \emph{multiple} of $\lceil\log_{2\eta+1} q\rceil$ to result in a vector of $\ring_q$ elements.
The above lemmas generalize to this extension.

\begin{definition}[Labeled Full Binary Tree]\label{def:label}
  Let $n,q,q',\xi\in\NN$ with $n$ a power of two and $q,q'$ primes.
  Let $\vec{m}=(\vec{m}_{1},\dots,\vec{m}_{{2^{\tau}}})^\intercal\in(\ring_{q'}^\xi)^{2^{\tau}}$, $\vec{g} \in \ring_{q}^{\xi\lceil\log_{2\eta+1} q'\rceil}$ and $\vec{h}_0,\vec{h}_1 \in \ring_q^{\lceil\log_{2\eta+1} q\rceil}$ be fixed.
  We define the labeling function $\lbl : \ring_{q}^{\xi\lceil\log_{2\eta+1} q'\rceil} \times (\ring_q^{\lceil\log_{2\eta+1} q\rceil})^2 \times (\ring^\xi_{q'})^{2^{\tau}}\times \bin^{\leq \tau} \to \ring^{\lceil \log_{2\eta+1} q \rceil}$ for a labeled full binary tree of depth $\tau$ as
  \[
    \lbl(\vec{g}, \vec{h}_0,\vec{h}_1,\vec{m},v) := \begin{cases} \decomp_{q}(\vec{g}^\intercal\cdot \decomp_{q'}(\vec{m}_{v})) & \text{if } \abs{v}=\tau\\ \decomp_q\mleft(\begin{aligned}\vec{h}_0^\intercal \cdot\lbl(\vec{g},\vec{h}_0,\vec{h}_1,\vec{m},v\Vert 0)\\ + \vec{h}_1^\intercal \cdot\lbl(\vec{g},\vec{h}_0,\vec{h}_1,\vec{m},v\Vert 1)\end{aligned}\mright)& \text{if } \abs{v}< \tau\end{cases}
  \]
\end{definition}

\begin{figure}[t]
\centering
\begin{pcvstack}[center,boxed]
\begin{pchstack}[center]
  \procedure{$\setup(\secparam)$}{
    \vec{g} \gets \ring_q^{\xi\lceil\log_{2\eta+1} q'\rceil}\\
    \vec{h}_0 \gets \ring_q^{\lceil\log_{2\eta+1} q\rceil}\\
    \vec{h}_1 \gets \ring_q^{\lceil\log_{2\eta+1} q\rceil}\\
    \pcreturn (\vec{g},\vec{h}_0,\vec{h}_1)
  }
  \pchspace
  \procedure{$\commit(\params,\vec{m})$}{
    \vec{c} := \lbl(\vec{g},\vec{h}_0,\vec{h}_1,\vec{m},\epsilon)\\
    \pcreturn \vec{c}
  }
\end{pchstack}
\begin{pchstack}[center]
    \procedure{$\open(\params,\vec{c},\vec{m},t)$}{
      \tilde t := \binsca(t)\\
      \pcfor j \in [\tau]\\
      \quad \vec{p}_{j} := \lbl(\vec{h}_0,\vec{h}_1,\vec{m},\tilde t_{< j}\concat \tilde t_{j})\\
      \quad \vec{s}_{j} := \lbl(\vec{h}_0,\vec{h}_1,\vec{m},\tilde t_{< j}\concat (\tilde t_{j}\xor 1))\\
      \vec{u} := \decomp_{q'}(\vec{m}_t)\\
      \pcreturn (\vec{p}_1,\dots,\vec{p}_\tau,\vec{s}_1,\dots,\vec{s}_\tau,\vec{u})
    }
  \pchspace
  \procedure{$\verify(\params,\vec{c},t, \vec{d},\beta)$}{
    \pcparse \vec{d} \pcas (\vec{p}_1,\dots,\vec{p}_\tau,\vec{s}_1,\dots,\vec{s}_\tau,\vec{u})\\
    \tilde t := \binsca(t)\\
    \vec{p}_0 := \vec{c}\\
    \pcfor j \in [\tau]\\
    \quad \pcif \norm{\vec{p}_{j}} > \beta \pcor \norm{\vec{s}_{j}} > \beta\\
    \quad \quad \pcreturn \bot \\
    \quad\pcif \proj_q(\vec{p}_{j-1}) \neq \vec{h}_{\tilde t_j}^\intercal\cdot \vec{p}_{j} + \vec{h}_{\tilde t_j \xor 1}^\intercal \cdot \vec{s}_{j}\\
    \quad\quad\pcreturn \bot\\
    \pcif \norm{\vec{u}} > \beta \pcor \proj_q(\vec{p}_\tau) \neq \vec{g}^\intercal \cdot \vec{u}\\
    \quad \pcreturn \bot\\
    \pcreturn \proj_{q'}(\vec{u})
  }
\end{pchstack}
  \pcvspace
\begin{pchstack}
    \procedure{$\iverify(\params,\vec{c},t, \vec{d})$}{
      \pcreturn \verify(\params,\vec{c},t, \vec{d},\eta)
    }
    \pchspace
    \procedure{$\sverify(\params,\vec{c},t, \vec{d})$}{
      \pcreturn \verify(\params,\vec{c},t, \vec{d},\bagg)
    }
    \pchspace
    \procedure{$\wverify(\params,\vec{c},t, \vec{d})$}{
      \pcreturn \verify(\params,\vec{c},t, \vec{d},2\bagg)
    }
\end{pchstack}
\end{pcvstack}
\caption{The construction of a homomorphic vector commitment for message space $\ring^\xi_{q'}$ based on a labeled binary tree.}
\label{fig:hvcinst}
\end{figure}

\begin{theorem}\label{theo:veccom}
  Let $n,q,q',\alpha_w,\rho,\eta,\tau,\xi,\bagg,\varepsilon$ be positive integers such that $n$ is a power of two, $q,q'$ are prime, and $\bagg \geq \eta\sqrt{2\alpha_w\rho(\varepsilon+1+\log_2 n + \log_2(2\tau \lceil\log_{2\eta+1}q\rceil + \xi\lceil\log_{2\eta+1}q'\rceil))\cdot\ln2}$.
  Let $\ring_q,\ring_{q'}$ be the polynomial rings $\ZZ_q[x]/\langle x^n+1\rangle$ and $\ZZ_{q'}[x]/\langle x^n+1\rangle$ respectively.
%  Let $\alpha$ be the smallest integer, such that $\binom{n}{\alpha}\cdot 2^\alpha \geq 2^\secpar$.
  If the $\sis_{\ring,q,2\lceil \log_{2\eta+1} q \rceil,4\bagg}$ problem and the $\sis_{\ring,q,\xi\lceil \log_{2\eta+1} q' \rceil,4\bagg}$ problem are hard, then the construction from \autoref{fig:hvcinst} is an individually correct, $(\rho,\tern_\alpha,\varepsilon)$-probabilistically homomorphic, robustly homomorphic, and position binding HVC for domain $\ring^{\xi}_{q'}$ and vector length $2^\tau$.
\end{theorem}
\begin{proof}
  The theorem follows from \autoref{lem:veccomcorrectness}, \autoref{lem:hvcprobhom}, \autoref{lem:hvcrobhom}, and \autoref{lem:hvcposbind} proven below. \qed
\end{proof}

\begin{lemma}\label{lem:veccomcorrectness}
  Let $n,q,q',\alpha_w,\rho,\eta,\tau,\xi,\bagg,\varepsilon$ be positive integers such that $n$ is a power of two, $q,q'$ are prime.
  Let $\ring_q,\ring_{q'}$ be the polynomial rings $\ZZ_q[x]/\langle x^n+1\rangle$ and $\ZZ_{q'}[x]/\langle x^n+1\rangle$ respectively.
  The construction from \autoref{fig:hvcinst} is an individually correct HVC for domain $\ring^\xi_{q'}$ and vector length $2^\tau$.
\end{lemma}
\begin{proof}
Let $\vec{m} \in (\ring_{q'}^{\xi})^{2^\tau}$, $\vec{c} = \vec{p}_0 = \commit(\params,\vec{m})$, $t\in[2^\tau]$, $(\vec{p}_1,\dots,\vec{p}_{\tau},\vec{s}_1, \dots, \vec{s}_{\tau},\vec{u})^\intercal = \open(\params,\vec{c},\vec{m},t)$.
We first observe that for all $j\in[\tau]$ it holds that
\begin{align*}
  \proj_q(\vec{p}_{j-1})
  ={}&\proj_q\left(\lbl(\vec{g},\vec{h}_0,\vec{h}_1,\vec{m},\tilde t_{< j})\tag{Def. of $\commit$ and $\open$}\right)\\
  ={}&\proj_q\left(\decomp_q\left(\begin{aligned}
  &\vec{h}_0^\intercal \cdot\lbl(\vec{g},\vec{h}_0,\vec{h}_1,\vec{m},\tilde t_{< j}\Vert 0)\\ +& \vec{h}_1^\intercal \cdot\lbl(\vec{g},\vec{h}_0,\vec{h}_1,\vec{m},\tilde t_{< j}\Vert 1)
  \end{aligned}\right)\right)\tag{\autoref{def:label}}\\
  ={}&\vec{h}_0^\intercal \cdot\lbl(\vec{g},\vec{h}_0,\vec{h}_1,\vec{m},\tilde t_{< j}\Vert 0) + \vec{h}_1^\intercal \cdot\lbl(\vec{g},\vec{h}_0,\vec{h}_1,\vec{m},\tilde t_{< j}\Vert 1)\tag{\autoref{lem:projinvofbin}}\\
  ={}&\vec{h}_{\tilde t_{j}}^\intercal \cdot\lbl(\vec{g},\vec{h}_0,\vec{h}_1,\vec{m},\tilde t_{< j}\Vert \tilde t_{j}) + \vec{h}_{\tilde t_{j}\xor 1}^\intercal \cdot\lbl(\vec{g},\vec{h}_0,\vec{h}_1,\vec{m},\tilde t_{< j}\Vert (\tilde t_{j}\xor 1))\\
  ={}&\vec{h}_{\tilde t_{j}}^\intercal \cdot\vec{p}_{j} + \vec{h}_{\tilde t_{j}\xor 1}^\intercal \cdot\vec{s}_{j}\tag{Def. of $\open$}.
\end{align*}

Further it holds that
\begin{align*}
  \proj_q(\vec{p}_{\tau})
  ={}&\proj_q(\lbl(\vec{g},\vec{h}_0,\vec{h}_1,\vec{m},\tilde t)\tag{Def. of $\commit$ and $\open$})\\
  ={}&\proj_q(\decomp_q(\vec{g}^\intercal\cdot \decomp_{q'}(\vec{m}_t)))\tag{\autoref{def:label}}\\
  ={}&\vec{g}^\intercal\cdot \decomp_{q'}(\vec{m}_t)\tag{\autoref{lem:projinvofbin}}\\
  ={}&\vec{g}^\intercal\cdot\vec{u}\tag{Def. of $\open$}.
\end{align*}


Therefore it only remains to check that the norm bounds are not violated. For every $j\in[\tau]$, $\vec{p}_j$ and $\vec{s}_j$ are outputs of the $\lbl$ function and thus in the range of $\decomp_q$.
Similarly $\vec{u}$ is the output of $\decomp_{q'}$.
By design the range of $\decomp_q$ and $\decomp_{q'}$ are sets of vectors of polynomials with coefficients in $\{-\eta,\dots,\eta\}$ and the norm of each $\vec{p}_j$ and $\vec{s}_j$ as well as $\vec{u}$ is at most $\eta$.\qed
\end{proof}

\begin{lemma}\label{lem:hvcprobhom}
  Let $n,q,q',\alpha_w,\rho,\eta,\tau,\xi,\bagg,\varepsilon$ be positive integers such that $n$ is a power of two, $q,q'$ are prime, and \[\bagg \geq \eta\sqrt{2\alpha_w\rho(\varepsilon+1+\log_2 n + \log_2(2\tau \lceil\log_{2\eta+1}q\rceil + \xi\lceil\log_{2\eta+1}q'\rceil))\cdot\ln2}.\]
  Let $\ring_q,\ring_{q'}$ be the polynomial rings $\ZZ_q[x]/\langle x^n+1\rangle$ and $\ZZ_{q'}[x]/\langle x^n+1\rangle$ respectively.
  The construction from \autoref{fig:hvcinst} is a $(\rho,\tern_\alpha,\varepsilon)$-probabilistically homomorphic HVC for domain $\ring^\xi_{q'}$ and vector length $2^\tau$.
\end{lemma}
\begin{proof}
Let $\params \gets \setup(\secparam)$, $\vec{c}^i = \vec{p}_0^i \in \ring_q^{\lceil\log_{2\eta+1} q\rceil}$, $t\in[2^\tau]$, $\vec{d}^i = (\vec{p}^i_1,\dots,\vec{p}^i_{\tau},\vec{s}^i_1, \dots, \vec{s}^i_{\tau},\vec{u})^\intercal\in (\ring^{\lceil\log_{2\eta+1} q\rceil})^{2\tau} \times \ring^{\xi\lceil\log_{2\eta+1} q'\rceil}$ with $\iverify(\params,\vec{c}^i,t,\vec{d}^i) = \vec{m}_t^i \neq \bot$
as specified in \autoref{def:hvc}.
%
We first note that even for arbitrary $w^1,\dots,w^{\ell}\in\tern_{\alpha_w}$ it holds for all $j\in[\tau]$ that
\begin{align*}
  \proj_q\Bigl(\sum_{i=1}^{\ell}w^i\cdot\vec{p}_{j-1}^i\Bigr) ={}&\sum_{i=1}^{\ell}w^i\cdot\proj_q(\vec{p}_{j-1}^i)\tag{\autoref{lem:projislin}}\\
  ={}&\sum_{i=1}^{\ell}w^i\cdot(\vec{h}_{\tilde t_j}^\intercal\cdot \vec{p}^i_{j} + \vec{h}_{\tilde t_j\xor 1}^\intercal \cdot \vec{s}^i_{j})\tag{Def. of $\iverify$}\\
  ={}&\sum_{i=1}^{\ell}\vec{h}_{\tilde t_j}^\intercal\cdot w^i \vec{p}^i_{j} + \vec{h}_{\tilde t_j\xor 1}^\intercal\cdot w^i  \vec{s}^i_{j}\\  
  ={}& \vec{h}_{\tilde t_j}^\intercal\cdot \Bigl(\sum_{i=1}^{\ell}w^i\cdot\vec{p}^i_{j}\Bigr) + \vec{h}_{\tilde t_j \xor 1}^\intercal \cdot \Bigl(\sum_{i=1}^{\ell}w^i\cdot\vec{s}^i_{j}\Bigr).
\end{align*}
and similarly
\begin{align*}
  \proj_q\Bigl(\sum_{i=1}^{\ell}w^i\cdot\vec{p}_{\tau}^i\Bigr) ={}&\sum_{i=1}^{\ell}w^i\cdot\proj_q(\vec{p}_{\tau}^i)\tag{\autoref{lem:projislin}}\\
  ={}&\sum_{i=1}^{\ell}w^i\cdot(\vec{g}^\intercal\cdot \vec{u}^i)\tag{Def. of $\iverify$}\\
  ={}&\vec{g}^\intercal\cdot\sum_{i=1}^{\ell}w^i\vec{u}^i 
\end{align*}
Therefore it only remains to verify that the norm-checks go through with sufficient probability. I.e., that

\[
    \Pr\mleft[
      w^1,\dots,w^{\ell} \gets \tern_{\alpha_w} : \exists j\in[\tau]\ldotp
      \bigl\Vert\sum_{i=1}^{\ell}w^i\cdot \vec{p}_j^i\bigr\Vert > \bagg \lor \bigl\Vert\sum_{i=1}^{\ell}w^i\cdot \vec{s}_j^i\bigr\Vert \lor \bigl\Vert\sum_{i=1}^{\ell}w^i\cdot\vec{u}^i\bigr\Vert > \bagg
    \mright] \leq 2^{-\varepsilon}
  \]
  
  To bound this probability, consider that the norm-bound is violated iff the absolute value of at least one of the $n(2\tau\lceil\log_{2\eta+1} q\rceil  + \xi\lceil\log_{2\eta+1} q'\rceil)$ coefficients in one of the sums $\sum_{i=1}^{\ell} w^i\cdot\vec{p}^i$, $\sum_{i=1}^{\ell} w^i\cdot\vec{s}^i$, and $\sum_{i=1}^{\ell} w^i\cdot\vec{u}^i$ is greater than $\bagg$.
  By a union bound it is thus sufficient to show that each individual coefficient violates the bound with probability at most $2^{-\varepsilon}/(n(2\tau\lceil\log_{2\eta+1} q\rceil + \xi\lceil\log_{2\eta+1} q'\rceil))$.
  
  For each individual $\vec{p}^i$, $\vec{s}^i$, $\vec{u}^i$ it holds by the definition of $\iverify$ that
  \[
    \norm{\vec{p}^i} \leq \eta, \quad \norm{\vec{s}^i} \leq \eta,\quad \text{and} \quad\norm{\vec{u}^i}\leq \eta.
  \]
  Recall that each $w^i$ is a ternary polynomial with weight $\alpha_w$.
  Therefore, each coefficient is a sum of the form
  \(
    \sum_{j=1}^{\alpha_w\ell}b_j c_j
  \)
  where $\abs{c_j}\leq \eta$ and $b_j$ is chosen uniformly from $\{-1,1\}$.
  By linearity of expectation, the expected value of this sum is always zero and changing any summand can vary the sum by at most $2\eta$. We can thus apply McDiarmid's inequality~\cite{McDiarmid89} and the lower bound on $\bagg$ from the lemma statement to obtain the following bound on the probability that each individual coefficient exceeds the norm bound $\bagg$
  \begin{align*}
    &\Pr\Bigl[\vec{b}\gets\{-1,1\}^{\alpha_w\ell} : \Bigl|\smashoperator{\sum_{j=1}^{\alpha_w\ell}}b_j c_j\Bigr| > \bagg\Bigr]\\
    \leq{}& 2\cdot\exp\Bigl(-\frac{2\bagg^2}{\alpha_w\ell\cdot (2\eta)^2}\Bigr)\\
    ={}& 2\cdot\exp\Bigl(-\frac{\bagg^2}{2\alpha_w\rho \eta^2}\Bigr)\\
    \leq{}& 2\cdot\exp\Bigl(-\frac{\eta^2 2\alpha_w\rho(\varepsilon + 1 +\log_2 n + \log_2(2\tau \lceil\log_{2\eta+1}q\rceil + \xi\lceil\log_{2\eta+1}q'\rceil))\cdot\ln2}{2\alpha_w\rho \eta^2}\Bigr)\\
    ={}& 2\cdot 2^{-(\varepsilon + 1 +\log_2 n + \log_2(2\tau \lceil\log_{2\eta+1}q\rceil + \xi\lceil\log_{2\eta+1}q'\rceil))\cdot}\\
    ={}& 2^{-\varepsilon}\cdot\frac{1}{n\cdot(2\tau \lceil\log_{2\eta+1}q\rceil + \xi\lceil\log_{2\eta+1}q'\rceil)}.
  \end{align*}
  It thus follows that with probability at least $1-2^{-\varepsilon}$ the strong verification algorithm outputs
  \begin{align*}
    \proj_{q'}\bigl(\sum_{i=1}^{\ell}w^i \cdot \vec{u}^i\bigr)
    ={}&\sum_{i=1}^{\ell}w^i \cdot \proj_{q'}(\vec{u}^i)\tag{\autoref{lem:projislin}}\\
    ={}&\sum_{i=1}^{\ell}w^i \cdot \iverify(\params,\vec{c}^i,t,\vec{d}^i)\tag{Def. of $\iverify$}\\
    ={}&\sum_{i=1}^{\ell}w^i \cdot \vec{m}_t^i
  \end{align*}
  as required.
  \qed
\end{proof}

\begin{lemma}\label{lem:hvcrobhom}
  Let $n,q,q',\alpha_w,\rho,\eta,\tau,\xi,\bagg,\varepsilon$ be positive integers such that $n$ is a power of two, $q,q'$ are prime.
  Let $\ring_q,\ring_{q'}$ be the polynomial rings $\ZZ_q[x]/\langle x^n+1\rangle$ and $\ZZ_{q'}[x]/\langle x^n+1\rangle$ respectively.
  Then the construction from \autoref{fig:hvcinst} is a robustly homomorphic HVC.
\end{lemma}
\begin{proof}
The proof of this lemma is taken almost verbatim from \cite{CCS:FleSimZha22}.
It deviates only insofar as the full construction and proof was split in two in \cite{CCS:FleSimZha22}, whereas is is combined here.
Since the proof is short, we include it here for the sake of completeness.
Let $\vec{c}^0,\vec{c}^1 \in \ring_q^\comlen$, and $\vec{d}^0, \vec{d}^1 \in \ring_q^\oplen$, and $t\in[2^\tau]$ be arbitrary, such that
\begin{equation}
    \sverify(\params,\vec{c}^0, t,\vec{d}^0)=\vec{m}^0 \quad \text{and} \quad \sverify(\params,\vec{c}^1, t,\vec{d}^1)=\vec{m}^1\label{eq:outputofsverify}
\end{equation}
with $\vec{m}^0,\vec{m}^1\neq \bot$.
Let $\vec{d}^i$ parse as $(\vec{p}^i_1,\dots,\vec{p}^i_{\tau},\vec{s}^i_1, \dots, \vec{s}^i_{\tau},\vec{u}^i)^\intercal$ for $i\in\bin$.
We first note that \emph{if} $\wverify(\params,\vec{c}^0-\vec{c}^1, t,\vec{d}^0-\vec{d}^1)\neq\bot$, then
\begin{align*}
  &\wverify(\params,\vec{c}^0-\vec{c}^1, t,\vec{d}^0-\vec{d}^1)\\
  =&\proj_{q'}(\vec{u}^0-\vec{u}^1)\tag{Def of $\sverify$}\\
  =&\proj_{q'}(\vec{u}^0)-\proj_{q'}(\vec{u}^1)\tag{\autoref{lem:projislin}}\\
  =&\sverify(\params,\vec{c}^0, t,\vec{d}^0)-\sverify(\params,\vec{c}^1, t,\vec{d}^1)\tag{Def. of $\sverify$}\\
  =&\vec{m}^0- \vec{m}^1 \tag{\autoref{eq:outputofsverify}}.
\end{align*}
%
It thus remains to show that $\wverify(\params,\vec{c}^0-\vec{c}^1, t,\vec{d}^0-\vec{d}^1)\neq\bot$.
For this, let further $\vec{p}^i_0 = \vec{c}^i$.
By definition of the strong verification algorithm, and since $\vec{m}^0,\vec{m}^1\neq\bot$ it holds that for $i\in\bin$ 
and $j \in [\tau]$ that the following two conditions hold
\begin{align}
  \norm{\vec{p}^i_{j}} \leq \bagg\quad\text{and}\quad
  \norm{\vec{s}^i_{j}} \leq \bagg\label{eq:robhomnormcheck}\\
  \proj_q(\vec{p}^i_{j-1}) = \vec{h}_{\tilde t_j}^\intercal\cdot \vec{p}^i_{j} + \vec{h}_{\tilde t_j\xor 1}^\intercal \cdot \vec{s}^i_{j}.\label{eq:robhompathcheck}
  \end{align}
  Similarly it holds that
  \begin{align}
  \norm{\vec{u}^i} \leq \bagg\quad \text{and}\quad \proj_q(\vec{p}^i_{\tau}) = \vec{g}^\intercal\cdot \vec{u}^i.\label{eq:robhompaycheck}
  \end{align}
  From \autoref{eq:robhomnormcheck} and \autoref{eq:robhompaycheck} it follows that for all $j \in [\tau]$
  \begin{align*}
    \norm{\vec{p}_j^0-\vec{p}_j^1} \leq& \norm{\vec{p}_j^0} + \norm{\vec{p}_j^1} \leq 2\bagg\\
    \norm{\vec{s}_j^0-\vec{s}_j^1} \leq& \norm{\vec{s}_j^0} + \norm{\vec{s}_j^1} \leq 2\bagg
  \end{align*}
  and
  \[
      \norm{\vec{u}^0-\vec{u}^1} \leq \norm{\vec{u}^0} + \norm{\vec{u}^1} \leq 2\bagg.
  \]
  
  From \autoref{eq:robhompathcheck} and \autoref{eq:robhompaycheck} and the linearity of $\proj_q$ it follows that for all $j \in [\tau]$
  \begin{align*}
    \proj_q(\vec{p}^0_{j-1}-\vec{p}^1_{j-1})
    ={}&\proj_q(\vec{p}^0_{j-1})-\proj_q(\vec{p}^1_{j-1})\tag{\autoref{lem:projislin}}\\
    ={}&(\vec{h}_{\tilde t_j}^\intercal\cdot \vec{p}^0_{j} + \vec{h}_{\tilde t_j\xor 1}^\intercal \cdot \vec{s}^0_{j})- (\vec{h}_{\tilde t_j}^\intercal\cdot \vec{p}^1_{j} + \vec{h}_{\tilde t_j\xor 1}^\intercal \cdot \vec{s}^1_{j})\tag{\autoref{eq:robhompathcheck}}\\
  ={}&\vec{h}_{\tilde t_j}^\intercal\cdot (\vec{p}^0_{j} - \vec{p}^1_{j}) + \vec{h}_{\tilde t_j\xor 1}^\intercal \cdot (\vec{s}^0_{j} - \vec{s}^1_{j}).
  \end{align*}
  and
  \begin{align*}
    \proj_q(\vec{p}^0_{\tau}-\vec{p}^1_{\tau})
    ={}&\proj_q(\vec{p}^0_{\tau})-\proj_q(\vec{p}^1_{\tau})\tag{\autoref{lem:projislin}}\\
    ={}&(\vec{g}^\intercal\cdot \vec{u}^0 - \vec{g}^\intercal \cdot \vec{u}^1)\tag{\autoref{eq:robhompaycheck}}\\
  ={}&\vec{g}^\intercal\cdot (\vec{u}^0 - \vec{u}^1).
  \end{align*}
  Thus, all checks in the weak verification algorithm go through and $\wverify(\params,\vec{c}^0-\vec{c}^1, t,\vec{d}^0-\vec{d}^1)\neq\bot$.\qed
\end{proof}

\begin{lemma}\label{lem:hvcposbind}
  Let $n,q,q',\alpha_w,\rho,\eta,\tau,\xi,\bagg,\varepsilon$ be positive integers such that $n$ is a power of two, $q,q'$ are prime.
  Let $\ring_q,\ring_{q'}$ be the polynomial rings $\ZZ_q[x]/\langle x^n+1\rangle$ and $\ZZ_{q'}[x]/\langle x^n+1\rangle$ respectively.
  If the $\sis_{\ring,q,2\lceil \log_{2\eta+1} q \rceil,4\bagg}$ problem and the $\sis_{\ring,q,\xi\lceil \log_{2\eta+1} q' \rceil,4\bagg}$ problem are hard, then the construction from \autoref{fig:hvcinst} is position binding.
\end{lemma}
\begin{proof}
This proof once again follows very closely the proof shown in \cite{CCS:FleSimZha22}.
We will prove this lemma by leveraging that any pair of valid decommitments for different messages will lead to a collision somewhere in the generalized hash tree, which can be turned into a solution for one of the SIS instances. 

  Let $\adv$ be an arbitrary PPT adversary against the position binding property of the construction.
  
  By the law of total probability it holds that
  \begin{align*}
    &\Pr[\vec{m}_0\neq \vec{m}_1 \land \bot\not\in\{\vec{m}_0,\vec{m}_1\}]\\
    ={}& \Pr[\vec{m}_0\neq \vec{m}_1 \land \bot\not\in\{\vec{m}_0,\vec{m}_1\} \land \proj_q(\vec{p}_\tau^0) = \proj_q(\vec{p}_\tau^1)]\\& + \Pr[\vec{m}_0\neq \vec{m}_1 \land \bot\not\in\{\vec{m}_0,\vec{m}_1\} \land \proj_q(\vec{p}_\tau^0) \neq \proj_q(\vec{p}_\tau^1)].
  \end{align*}
  
  We now bound the two probabilities seperately.
  
  \begin{align*}
    &\Pr[\vec{m}_0\neq \vec{m}_1 \land \bot\not\in\{\vec{m}_0,\vec{m}_1\} \land \proj_q(\vec{p}_\tau^0) = \proj_q(\vec{p}_\tau^1)]\\
    \leq{}&\Pr[\proj_{q'}(\vec{u}^0) \neq \proj_{q'}(\vec{u}^1) \land \vec{g}^\intercal\cdot \vec{u}^0 = \vec{g}^\intercal\cdot \vec{u}^1 \land \norm{\vec{u}^0}\leq 2\bagg \land \norm{\vec{u}^1}\leq 2\bagg]\tag{Def. of $\wverify$}\\
    \leq{}& \Pr[\vec{u}^0 \neq \vec{u}^1 \land \vec{g}^\intercal\cdot (\vec{u}^0-\vec{u}^1) = 0 \land \norm{\vec{u}^0-\vec{u}^1}\leq 4\bagg]\\
    = {}& \Pr[(\vec{u}^0-\vec{u}^1) \in \ball_{4\bagg}^{\xi\lceil\log_{2\eta+1}q'\rceil}\setminus \{\vec{0}\} \land \vec{g}^\intercal\cdot (\vec{u}^0-\vec{u}^1) = 0]\\
    \leq{}& \negl,
  \end{align*}
  where the last inequality follows from the assumed hardness of the $\sis_{\ring,q,\xi\lceil \log_{2\eta+1} q' \rceil,4\bagg}$ problem and the fact that all involved algorithms are PPT.
  
  We now analyze 
  \[
    \Pr[\vec{m}_0\neq \vec{m}_1 \land \bot\not\in\{\vec{m}_0,\vec{m}_1\} \land \vec{p}_\tau^0 \neq \vec{p}_\tau^1]
  \]
  We construct a PPT algorithm that solves the $\sis_{\ring,q,2\lceil\log_{2\eta+1} q\rceil,4\bagg}$ problem as follows.
  Upon input $\vec{a}=(a_0,\dots,a_{2\lceil\log_{2\eta+1} q\rceil-1})^\intercal$, $\bdv$ sets $\vec{h}_0:=(a_0,\dots,a_{\lceil\log_{2\eta+1} q\rceil-1})^\intercal$ and $\vec{h}_1 := (a_{\lceil\log_{2\eta+1} q\rceil},\dots,a_{2\lceil\log_{2\eta+1} q\rceil-1})^\intercal$, samples $\vec{g} \gets \ring_q^{\xi\lceil\log_{2\eta+1} q'\rceil}$ and runs $(\vec{c},t,\vec{d}^0,\vec{d}^1) \gets \adv((\vec{g},\vec{h}_0,\vec{h}_1))$.
  For $i\in\bin$ let $m^i := \wverify((\vec{g},\vec{h}_0,\vec{h}_1),\vec{c},t,\vec{d}^i)$.
  If $\vec{m}^0 = \vec{m}^1$, $\bot\in\{\vec{m}^0,\vec{m}^1\}$, or $\proj_q(\vec{p}_\tau^0) = \proj_q(\vec{p}_\tau^1)$, $\bdv$ aborts.
  Otherwise, parse $\vec{d}^i$ as $(\vec{p}^i_1,\dots,\vec{p}^i_{\tau},\vec{s}^i_1,\dots,\vec{s}^i_\tau,\vec{u}^i)$, set $\vec{p}^i_0 := \vec{c}$.
  
  Let $j^*\in [\tau+1]$ be the \emph{largest} index, such that $\proj_q(\vec{p}^0_{j^*-1})=\proj_q(\vec{p}^1_{j^*-1})$.
  Note that such an index always exists, since $\vec{p}^0_0 = \vec{c} = \vec{p}^1_0$ and that $j^* < \tau$, since $\proj_q(\vec{p}_\tau^0) \neq \proj_q(\vec{p}_\tau^1)$.
  If $\tilde t_{j^*-1} = 0$, $\bdv$ outputs $
  \vec{z}:=(\vec{p}_{j^*}^0, \vec{s}_{j^*}^0)^\intercal - (\vec{p}_{j^*}^1, \vec{s}_{j^*}^1)^\intercal$, if $\tilde t_{j^*-1} = 1$, $\bdv$ outputs $\vec{z}:=(\vec{s}_{j^*}^0, \vec{p}_{j^*}^0)^\intercal - (\vec{s}_{j^*}^1, \vec{p}_{j^*}^1)^\intercal$.
  
  We now analyze the success probability of $\bdv$.
  It holds that $\proj_q(\vec{p}_{j^*-1}^0) = \proj_q(\vec{p}_{j^*-1}^1)$ and by the definition of the weak verification algorithm that
  \begin{align*}
    &\vec{h}_{\tilde t_{j^*}}^\intercal\cdot\vec{p}_{j^*}^0 + \vec{h}_{\tilde t_{j^*}\xor 1}^\intercal\cdot\vec{s}_{j^*}^0 = \vec{h}_{\tilde t_{j^*}}^\intercal\cdot\vec{p}_{j^*}^1 + \vec{h}_{\tilde t_{j^*}\xor 1}^\intercal\cdot\vec{s}_{j^*}^1\\
    \iff&\vec{h}_{\tilde t_{j^*}}^\intercal\cdot(\vec{p}_{j^*}^0-\vec{p}_{j^*}^1) + \vec{h}_{\tilde t_{j^*}\xor 1}^\intercal\cdot(\vec{s}_{j^*}^0-\vec{s}_{j^*}^1) = 0\\
    \iff&\vec{a}^\intercal\cdot\vec{z}=\vec{0} 
  \end{align*}
  It further holds by the definition of the weak verification algorithm that 
  \[\norm{\vec{p}_{j^*}^0} \leq 2\bagg,\quad \norm{\vec{s}_{j^*}^0} \leq 2\bagg, \quad\norm{\vec{p}_{j^*}^1} \leq 2\bagg, \quad \norm{\vec{s}_{j^*}^1} \leq 2\bagg.\]
  Therefore, the norm of $\vec{z}$ can be bounded as \[\norm{\vec{z}} \leq \max\{\norm{\vec{p}_{j^*}^0},\norm{\vec{s}_{j^*}^0}\}+\max\{\norm{\vec{p}_{j^*}^1},\norm{\vec{s}_{j^*}^1}\} \leq 4\bagg.\]
  It remains to show that $\vec{z}\neq 0$.
  Since $j^*$ is the \emph{largest} index such that 
  \[
    \proj_q(\vec{p}_{j^*-1}^0) = \proj_q(\vec{p}_{j^*-1}^1)
  \]
  it holds that
  \[
    \proj_q(\vec{p}_{j^*}^0) \neq \proj_q(\vec{p}_{j^*}^1)
  \]
  and thereby that
  \[
    \vec{p}_{j^*}^0 \neq \vec{p}_{j^*}^1.
  \]
  Therefore $\vec{z}\neq\vec{0}$.
  Thus, whenever $\adv$ is successful, $\bdv$ is successful with probability $1$ and we can conclude that
  \begin{align*}
  \negl \geq{}& \Pr[
      \vec{a} \gets \ring_q^{2\lceil\log_{2\eta+1} q\rceil}; \vec{z}\gets\bdv(\vec{a}) : \vec{z}\in\ball_{4\bagg}^{2\lceil\log q\rceil}\setminus\{\vec{0}\} \land \vec{a}^\intercal\vec{z}=0
    ]\\
    ={}&
    \Pr[\vec{m}_0\neq \vec{m}_1 \land \bot\not\in\{\vec{m}_0,\vec{m}_1\} \land \vec{p}_\tau^0 \neq \vec{p}_\tau^1]
  \end{align*}.
  
  Combining the above, it follows that
  \[
    \Pr[\vec{m}_0\neq \vec{m}_1 \land \bot\not\in\{\vec{m}_0,\vec{m}_1\}] \leq \negl
  \]
  as required.\qed
\end{proof}


