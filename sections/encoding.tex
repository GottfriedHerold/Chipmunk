% !TEX root = ../main.tex
\section{Encoding HVC openings}\label{sect:efficientencoding}

% \gnote{This section just is not colored / versioned due to LaTeX issues I'm too lazy to solve. It is not present in the camera-ready version due to the \textbackslash input being conditional}

% \gnote{TODOs: Explain algorithm, make picture, remark on encoded HVC}
% \gnote{Explain relationship to Babai's algorithm}

To reduce the size needed to transmit openings in our final HVC construction, we employ a non-trivial encoding scheme.
Let us first sketch the idea and how it relates to lattice enumeration, before defining it more formally in \autoref{fig:EncodingCoeff}.

Our HVC construction is, except for projections and decompositions, a Merkle tree with a homomorphic hash function.
Time slots $t$ correspond to paths in the Merkle tree and our openings contain the labels $\vec{p}_i$ along the path, together with the sibling nodes' labels $\vec{s}_i$.
Usually, when opening a path of a Merkle tree, it is not necessary to actually include most of the nodes $\vec{p}_i$ along the Merkle path in the opening, but only the sibling nodes $\vec{s}_i$ (ignoring possible special handling at the root or leaf).
The reason is that for any valid opening of a usual Merkle tree, we have 
\[
 H(\vec{p}_i,\vec{s}_i) = \vec{p}_{i-1} \quad\text{or}\quad H(\vec{s}_i,\vec{p}_i) = \vec{p}_{i-1}\enspace, 
\]
where $H$ is the hash function used in the construction (concretely for us, Ajtai's hash function $\hashajtai$).
This allows the verifier to compute $\vec{p}_{i-1}$ from $\vec{p}_{i}$ by itself, if given $\vec{s}_i$.

For us, the corresponding relation (ignoring smallness constraints) instead reads
\[
 H(\vec{p}_i,\vec{s}_i) =  \projmod{q}(\vec{p}_{i-1})  \quad\text{or}\quad H(\vec{s}_i,\vec{p}_i) = \projmod{q}(\vec{p}_{i-1})
\]
throwing $\projmod{q}$, i.e.\ $\projring$ and reduction modulo $q$, in the mix, which complicates things.

Now, for individually verifying openings, the above idea still works out due the size constraints:
the bounds $\norm{\vec{p}_i} \leq \eta$ and $\norm{\projring(\vec{p}_i)}\leq \tfrac{q-1}{2}$ for individually verifying openings imply that $\vec{p}_i$ is actually uniquely determined by $\projmod{q}(\vec{p}_i)$.
Indeed, $\vec{p}_i$ is given by $\vec{p}_i = \decompmod{q}(\projmod{q}(\vec{p}_i))$, leading to
\[
 \vec{p}_{i-1} = \decompmod{q}(H(\vec{p}_i,\vec{s}_i))\quad\text{or}\quad \vec{p}_{i-1} = \decompmod{q}(H(\vec{s}_i,\vec{p}_i))\enspace.
\]
For aggregate openings, this unfortunately no longer holds:
for any given output of Ajtai's hash function, which we will denote as $\hint\in\ring_q$ in our algorithm, the equation $\projmod{q}(\vec{p}_i) = \hint$ can have many solutions that satisfy the more relaxed size constraints that we impose on aggregate openings.

Ignoring any size constraints, for a given $\hint\in\ring_q$, the set of solutions to $\projmod{q}(\vec{p}_i) = \hint$ is a lattice coset $\coset$ of the form $\coset = \latmodular + \vec{t}$, where $\latmodular \coloneqq \{\vec{x}\in\ring^\limbs \mid \projring(\vec{x})\bmod q = 0\}$, with $\limbs = \ceil{\log_{2\eta+1}q}$. The vector $\vec{t}$ depends on $\hint$.
Note that while the coset $\coset$ is uniquely determined by $\hint$, there are multiple possible choices for $\vec{t}$. The different possible choices differ exactly by elements from $\latmodular$.
Our task now boils down to effciently encode small elements $\vec{p}_{i}$ (in $\norm{.}_{\infty}$-norm) from this lattice coset $\coset$. Both encoder and decoder know $\hint$ (and hence $\coset$) from hashing the child nodes; note that for notational convenience, our encoder defined in \autoref{fig:EncodingCoeff} instead determines $\hint$ by computing $\projmod{q}(\vec{p}_i)$.

\medskip
Before formally defining our encoding and decoding algorithms, let us give some more informal remark that explains the basic idea and how it relates to Babai's algorithm and lattice enumeration. Note that our actual algorithm in \autoref{fig:EncodingCoeff} and its formal analysis given below will be fully self-contained and do not rely on this remark in any way.

\begin{remark}[Lattice enumeration]\label{rmk:RelationshipToBabai}
% In our notation the $B$ in $\latencode$ and $\latdecode$ is for Babai.
Let us now explain how our encoding and decoding is connected to Babai's algorithm~\cite{DBLP:journals/combinatorica/Babai86} (more precisely, the generalization in~\cite{RSA:LinPei11}) and lattice enumeration.
% Note that, to keep our exposition self-contained, both our actual algorithm in \autoref{fig:EncodingCoeff} and its analysis make no mention of this connection and can be understood without it.
The problem we need to solve is to encode some $\vec{v}\in\coset = \vec{t} + \latgeneral$ via an encoding $\overline{\vec{v}}$, where short $\vec{v}$ should correspond to short(er) $\overline{\vec{v}}$. Here, $\latgeneral$ can be an arbitrary (full-rank) lattice at first that we will later take to be $\latmodular$. Let us assume we have some pre-agreed basis $\kernelbasis = \{\vec{b}_1,\vec{b}_2,\ldots\}$ for $\latgeneral$, either over $\ZZ$ or (if $\latgeneral$ is a free $\ring$-module) over our ring $\ring$.

The most simple idea, corresponding to what's called Babai rounding, is to deterministically (so both encoder and decoder agree on it) determine some $\tref$ from $\hint$, or ,equivalently, from $\coset$, such that $\coset = \tref + \latgeneral$. Then $\vec{v} - \tref \in \latgeneral$ and we can simply encode $\vec{v}$ by the coordinate vector $(\alpha_1,\alpha_2\ldots)$ of $\vec{v}-\tref$ with respect to the basis $\kernelbasis$.

How good this is (i.e.\ how small the $\alpha_i$ are) depends on both how ``good'' the basis $\kernelbasis$ is and how we choose $\tref$. For the latter, we want $\vec{v}-\tref$ to be small, so $\tref$ should itself be small. One way (if working with a $\ZZ$-basis) called Babai rounding to choose $\tref$ is to set $\tref = \sum_i  t_i\vec{b}_i$ with all real-valued) coeffients $-\tfrac12 < t_i \leq \tfrac12$. Observe that we have $\vec{v} = \sum_i (t_i + \alpha_i)\vec{b}_i$.
% \znote{should this be $\vec{v} = \vec{t} + \sum_i (t_i + \alpha_i)\vec{b}_i$ as $\vec{v} = \vec{t} + \latgeneral$?} %%% Gotti: v - t has coos \alpha_i, t has coos t_i, so v has coos \alpha_i + t_i.
This means that this approach essentially computes the $\alpha_i$ by writing $\vec{v}$ with respect to $\kernelbasis$ and rounding the coefficients to the nearest integers.

Note that this approach first computes a short reference $\tref\in\coset$ in some way and then separately encodes $\vec{v}$ by encoding the difference.
An equivalent, useful view, is to consider this as a single algorithm akin to lattice enumeration: we want to output not just a single short vector $\tref$ of a lattice coset, but rather enumerate (candidate) short vectors $\vec{v}$. For this, we don't set a single $t_i$'s with $-\tfrac12 < t_i \leq \tfrac12$, but rather enumerate possible candidate short vectors by also trying larger values of $t_i$, parameterized by $\alpha_i$'s. In lattice enumeration, where the goal is to find a vector as short as possible, we usually try a large number of such candidates and settle for the shortest one we found (usually, this takes super-polynomial time). Here, we are given $\vec{v}$ and we encode $\vec{v}$ by the branch (parameterized by $\alpha_i$) a lattice enumeration algorithm would need to take to output $\vec{v}$.

This point of view lets us use Babai's algorithm proper rather than the more naive Babai rounding: here, we (greedily) choose coefficients wrt.\ a given basis $\kernelbasis$, but differently to Babai rounding, we choose coefficients one-by-one and
our choices are affected by the previous choice.

For a recursive description of Babai's algorithm / enumeration, pick one%
\footnote{The choice of $\vec{b}_\qbasis$ matters. This algorithm is typically applied to bases $(\vec{b}_1,\ldots)$ obtained from lattice reduction. Lattice reduction outputs an ordered basis and with the ordering convention from lattice reduction, the appropriate choice for Babai's algorithm is to choose the \emph{last} basis element as $\vec{b}_\qbasis$}
of the basis vectors $\vec{b}_\qbasis\in\kernelbasis$ and decompose $\kernelbasis$ into a disjoint union
$\kernelbasis = \{\vec{b}_\qbasis\}\cup \kernelbasis'$. This decomposes $\latgeneral$ as $\latgeneral = \latgeneral'\oplus \Span\vec{b}_{\qbasis}$, where $\latgeneral'$ is the lattice generated by $\kernelbasis'$. We now choose $\tref^{(1)}\in\coset$ and then pick the unique $\alpha_\qbasis$ such that $\vec{v} \in \alpha_\qbasis\vec{b}_\qbasis + \tref^{(1)} + \latgeneral'$. This is similar to the approach before, except that now we only determine a single coefficient $\alpha_\qbasis$. Note that the choice of $\tref^{(1)}$ only matters modulo $\latgeneral'$, so only the single (real-valued) $\vec{b}_\qbasis$-component of $\tref^{(1)}$ matters. This is hence a 1-dimensional problem and Babai's algorithm (which only works for $\ZZ$-coeffients and is designed for the $\norm{.}_2$-norm) chooses $\tref^{(1)}$ such that the $\norm{.}_2$-distance between $\tref^{(1)}$ and $\Span_{\RR}\kernelbasis'$ is minimized (with some arbitrary determistic tie-breakers).
We then recurse into the new problem instance given by $\vec{v} \in \coset'$ with $\coset' = \latgeneral' + \alpha_\qbasis\vec{b}_\qbasis + \tref^{(1)}$ of dimension 1 less than the original. Note that to get a full-rank lattice, we may orthogonally project out the orthgonal complement to $\Span_\RR\latgeneral'$. The 0-dimensional base case is trivial. Babai's algorithm itself only considers $\alpha_\qbasis = 0$, lattice enumeration branches into several candidate $\alpha_\qbasis$ and our approach sets $\alpha_\qbasis$ from $\vec{v}$. Importantly, the next $\tref^{(2)}$ chosen in the next step during the recursion depends on $\coset'$ and hence on the previous choice of $\alpha_\qbasis$.

For our problem with $\latgeneral = \latmodular$ and $\coset$ determined by $\hint\in\ring_q$, we do not work over the $\norm{.}_2$-norm, but rather the $\norm{.}_\infty$-norm. Fortunately, our lattice has a very good basis for this norm (close to the coordinate axes). We take a very similar general approach, but instead of choosing $\tref^{(1)}$ via a $\norm{.}_2$-minimization problem, we directly choose $\tref^{(1)} \coloneqq \decompmod{q}(\hint)$. Note that this equals $\tref^{(1)}=\decompmod{q}(\projmod{q}(\vec{v}))$. In the next recusion step, we set $\tref^{(2)}$ as $\decompring(\projring(\vec{v}))$. Note here that our basis element $\vec{b}_\qbasis$ in the first step is given by $(q,0,0,\ldots)$. Essentially, determining $\alpha_\qbasis$ in the first recursion step allows us to ``update'' the information $\hint$ we are given from something modulo $q$ to something unreduced, which helps the algorithm by choosing a better $\tref^{(2)}$.
Our algorithm does not need to perform any orthogonal projections and works over $\ring$. We also only perform this recursive step once and use the Babai rounding approach after one step; the reason is that the shape of our basis is so good that this is good enough.

% We mention that our formal descripion in \autoref{fig:EncodingCoeff} of the algorithm is self-contained and actually makes no explicit mention of $\tref^{(1)}$ or $\tref^{(2)}$. However, it is easy to see that (in the notation given in the algorithm) $\deltav = \vec{v}-\tref^{(2)}$ and we just encode that by coefficients as in Babai rounding. With $\tref^{(1)}=\decompmod{q}(\hint)$, the approach described above finds $\tilde{\alpha}_\qbasis$ such that $\vec{v} = \alpha_\qbasis\vec{b}_\qbasis + \decompmod{q}(\hint) + \vec{v}'$ with $\vec{v}'\in\\latgeneral' = \latring$. With


% The problem we need to solve is, essentially, to encode $\proj_q(\vec{v}) = h$ by encodings $\overline{\vec{v}}$, where short solutions $\vec{v}$ should be encoded by small $\overline{\vec{v}}$.
% The set of all such solutions is a lattice coset for the kernel of $\proj_q$. So this is essentially a close vector problem for a lattice.

% Let $\kernelbasis = \{\vec{b}_0,\vec{b}_1\ldots\}$ be a $\ZZ$-basis for the kernel of $\proj_q$ (which could be any linear map here).
% Writing everything wrt.\ this basis, we are looking at solutions $\vec{c} = (c_0,c_1\ldots)$ of
% \[
% \proj_q\Bigl(\sum_{i\geq 0} c_i\vec{b}_i\Bigr) = h\enspace,
% \]
% where $\sum_{i\geq 0} c_i\vec{b}_i$ is integral.
% Note that the $c_i$ themselves do not need to be integers, but are rational numbers.
% By construction, if $\vec{c}$ is a solution then so is $\vec{c} + \vec{\delta}$ for any integral $\vec{\delta}$.

% Consequently, if any solution exists, we can find a reference solution $\vec{c}'$ where each $c_i\in [-\tfrac12,+\tfrac12[$, which corresponds to what is often called Babai rounding.
% A naive way to encode a given $\vec{c}$ is then by encoding it by $\vec{\delta} \coloneqq \vec{c} - \vec{c}'$, which is an integral vector.
%
% A better way, especially if the basis is very non-orthogonal, is to use Babai's nearest plane(s) algorithm~\cite{DBLP:journals/combinatorica/Babai86,RSA:LinPei11}:
% we only use the naive construction to obtain the first coefficient $\delta_0$. For fixed $\delta_0$, we can plug this into the equation and consider solutions to
% \[
% \proj_q\Bigl(\sum_{i\geq 1} c_i\vec{b}_i\Bigr) = h - \proj_q(c'_0 + \delta_0\vec{b}_0)\enspace,
% \]
% where $\sum_{i\geq 1} c_i\vec{b}_i$ is integral. We then use recursion to solve a problem in dimension 1 less.
% Note that the reference solution obtained in a given step during this recursion depends on the choices of $\delta_0,\ldots$ made in previous steps.
%
% Our actual algorithms $\latencode$ and $\latdecode$ correspond precisely to this, with the only difference being that we work over\footnote{Normally, we would have to expand everything over $\ZZ$ and increase dimension $n$-fold, but since both $\proj_q$ and our given $\kernelbasis$ can be written with $\ZZ$ coefficients, all monomials are independent and we can just work in $\ring$.} $\ring$ and that we only perform the above recursive step \emph{once} to get a single coefficient $\delta_q$ before using the naive algorithm to get the rest of the coefficients $(\alpha_1,\alpha_2,\ldots)$ in one go.
% The reason we perform the recursive step only once is that $(\vec{b}_1,\vec{b}_2,\ldots)$ is almost orthogonal and parallel to the coordinate axes (the latter condition is relevant, because we work in $\norm{.}_\infty$, not $\norm{.}_2$), so the naive way is ``good enough''.
% Only $\vec{b}_0$ needs special handling.
\end{remark}


% So our task boils down to efficiently encode small elements (in $\norm{.}_{\infty}$-norm) from a lattice coset for the lattice $\latmodular \coloneqq \{\vec{x}\in\ring^\limbs \mid \projring(\vec{x})\bmod q = 0\}$, where $\limbs = \ceil{\log_{2\eta+1}q}$.
% We do this, essentially, by performing lattice enumeration and encoding the given solution by the path in the search tree that lattice enumeration would use to find it.\gnote{Review this sentence}
% \gnote{TODO: Ref to remark}

% While our presentation is self-contained and does not require knowledge of lattice enumeration, the following 


\medskip
Let us now proceed to define our encoding and decoding formally. For this, we need bases of the relevant lattices.
% Still, we can define a reference solution $\vec{p}_i' \coloneqq \decompmod{q}(h)$ and consider the difference $\vec{\delta}_i = \vec{p}_i - \vec{p}'_i$.
% The differences are short (because both $\norm{\vec{p}_i}$ and $\norm{\vec{p}'_i}$ are small) and, by construction, satisfy $\projmod{q}(\vec{\delta}_i) = 0$.
% We can then replace transmission of most $\vec{p}_i$'s by transmission of $\vec{\delta}_i$'s.
% So our task boils down to efficiently encode small elements (in $\norm{.}_{\infty}$-norm) from the lattice $\latmodular \coloneqq \{\vec{x}\in\ring^\limbs \mid \projring(\vec{x})\bmod q = 0\}$, where $\limbs = \ceil{\log_{2\eta+1}q}$.

\begin{proposition}\label{prop:kernellattices}
Let $q,\eta$ be positive integers with $q$ prime. Set $\limbs \coloneqq \ceil{ \log_{2\eta+1}q }$ and define lattices
\begin{align*}
\latring    \coloneqq &\{\vec{v}\in\ring^\limbs \mid \projring(\vec{v}) = 0\}\\
\latmodular \coloneqq &\{\vec{v}\in\ring^\limbs \mid \projmod{q}(\vec{v}) = 0\}\\
\end{align*}
for the kernels of $\projring$ and $\projmod{q}$, respectively.
Define vectors $\vec{b}_\qbasis$ and $\vec{b}_1,\ldots,\vec{b}_{\limbs-1} \in \ring^\limbs$ as
\begin{align*}
\vec{b}_\qbasis &= (q,0,0,\ldots, 0)\\
\vec{b}_1       &= (-(2\eta+1), 1, 0,0,\ldots, 0)\\
\vec{b}_2       &= (0, -(2\eta+1), 1, 0,\ldots, 0)\\
\ldots\\
\vec{b}_{\limbs-1}&=(0,0,\ldots,0, -(2\eta+1), 1)\enspace.
\end{align*}
Then $\latring$ and $\latmodular$ are $\ring$-module lattices (i.e.\ lattices that are also free $\ring$-modules). A basis (over $\ring$) for $\latring$ is given by $\kernelbasis \coloneqq \{\vec{b}_1,\ldots,\vec{b}_{\limbs-1}\}$ and
a basis over $\ring$ for $\latmodular$ is given by $\{\vec{b}_\qbasis,\vec{b}_1,\ldots,\vec{b}_{\limbs-1}\}$.
\end{proposition}
\begin{proof}
Note that $\ring$ is a free $\ZZ$-module, i.e.\ $\ring\cong\ZZ^n$ as a $\ZZ$-module (but not as a ring), so the whole notion of $\ring$-module lattice even makes sense.
Being kernels of appropriate $\ring$-linear maps, $\latring$ and $\latmodular$ are clearly $\ring$-module lattices. Recall that $\projring$ is defined as
\[
\projring\colon\,\ring^\limbs\to\ring, \quad \projring(v_1,\ldots, v_{\limbs}) = \sum_{i=1}^{\limbs} (2\eta+1)^{i-1} \cdot v_i\enspace.
\]
We easily compute that $\projring(\vec{b}_\qbasis) = q$ and $\projring(\vec{b}_i) = 0$ for $1\leq i \leq \limbs-1$.
Hence, all $\vec{b}_i$ and $\vec{b}_\qbasis$ are in the appropriate lattices.
They are also clearly linearly independent due to the triangular shape of $\{\vec{b}_\qbasis\}\cup\kernelbasis$.
% , even over the rationals/reals\footnote{We need linear independence over the reals, which means, in mathematical terms, linear independence over $\ring\tensor_{\ZZ}\reals$.
% Concretely this means we write out}.

\smallskip
We now need to show that they also span $\latring$ resp.\ $\latmodular$, i.e.\ that $\latring \subset \Span_\ring\kernelbasis$ and $\latmodular\subset\Span_\ring(\kernelbasis\cup\{\vec{b}_\qbasis\})$.
For this, consider any $\vec{x}\in\ring^\limbs$.
Without the first column, $\kernelbasis$, viewed as a matrix, is a $(\limbs-1)\times (\limbs-1)$ lower triangular matrix with 1's on the diagonal.
This implies that the projection of $\Span_\ring\kernelbasis$ onto the last $\limbs-1$ coefficients (over $\ring$) is all of $\ring^{\limbs-1}$, and this projection is bijective.
This means we can find a unique $\tilde{\vec{x}}\in\Span_\ring\kernelbasis$ matching $\vec{x}$ on the last $\limbs-1$ coefficients, giving a unique decomposition
\[
 \vec{x} = (x',0,\ldots,\ 0) + \tilde{\vec{x}}
\]
with $x'\in\ring, \tilde{\vec{x}}\in\Span_\ring\kernelbasis$. Since $\Span_\ring\kernelbasis\subset \latring$, we have
\[
 \projring(\vec{x}) = \projring((x',0,\ldots,0)) + \projring(\tilde{\vec{x}}) = x' + 0 = x'\enspace.
\]
Consequently, if $\vec{x}\in\latring$, we have by definition $\projring(\vec{x}) = 0$, so $x'=0$ and $\vec{x} = \tilde{\vec{x}}$. This means $\vec{x} = \tilde{\vec{x}}\in\Span_\ring\kernelbasis$, giving $\latring \subset \Span_\ring\kernelbasis$.
Similarly, if $\vec{x}\in\latmodular$, we have $\projring{\vec{x}}\equiv 0 \mod q$, so $x'\bmod q = 0$. This gives $(x',0,\ldots, 0)\in\Span_\ring\vec{b}_\qbasis$. Together with $\tilde{\vec{x}}\in\Span_\ring\kernelbasis$, this implies $\vec{x}\in\Span_\ring(\kernelbasis\cup\{\vec{b}_\qbasis\})$.

% This means that we can reduce $\vec{x}$ modulo $\Span_\ring \kernelbasis$ to obtain
% \[
%   \vec{x}\equiv (x',0,0,\ldots, 0) \mod \Span\nolimits_\ring\kernelbasis
% \]
% for some uniquely determined $x'\in\ring$.
% Since $\Span_\ring\kernelbasis\subset \latring$, we have 
% \[
%  \projring(\vec{x}) = \projring(x',0,\ldots, 0) = x'\enspace. 
% \]
% Consequently, iff $\vec{x}\in\latring$, we have $\vec{x}\equiv 0 \mod \Span_\ring\kernelbasis$, so $\latring = \Span_\ring\kernelbasis$.
% Similarly, iff $\vec{x}\in\latmodular$, we have $\vec{x}\equiv (x',0,\ldots, 0) \mod \Span_\ring\kernelbasis$ with $x'\bmod q = 0$. This shows that $\kernelbasis\cup\{\vec{b}_0\}$ is a basis of $\latmodular$.
\qed
\end{proof}

\if 0
Following the ideas presented above, the naive approach would correspond to encoding vectors from $\latmodular$ by their coefficients (in $\ring$) wrt.\ $\{\vec{b}_0,\ldots,\vec{b}_{\limbs-1}\}$.
While this would indeed work out, the size of the resulting coefficients depends on how good the chosen basis is.
For the $\norm{.}_\infty$-norm, we would ideally want the basis vectors to be (nearly) orthogonal and parallel to the coordinate axes.
Observe that for large $\eta$, this holds true at least for $\kernelbasis$, but not for $\vec{b}_0$, which is why we treat this coordinate specially\footnote{We could replace $\vec{b}_0$ by a different, more ``reduced'', basis vector here. A canonical choice would be $\decompring(q)$. However, this would not solve the issue.}.
\fi

Formally, we define encoding and decoding algorithms $\latencode$ and $\latdecode$ as in \autoref{fig:EncodingCoeff}.

% \gnote{Define lattice(s), $b_i$'s and extra basis vector.}

% \gnote{Note about extra basis vector}

% \gnote{Intro: Define encoding / decoding of points from lattice}

\begin{figure}[ht]
 \begin{pchstack}[center,boxed]
   \procedure{$\latencode(\vec{v})$}{
   \limbs \coloneqq \ceil{\log_{2\eta+1}q}\\
   \hint\coloneqq \projmod{q}(\vec{v})\in\ring_q\\
%    \text{Obtain $\eta, q, \limbs$ from $\params$ with $\limbs = \ceil{\log_{2\eta+1}q}$}\\
   \text{Represent $\hint$ by $\hint'\in\ring$, $\norm{\hint'}\leq \tfrac{q-1}{2}$}\\
%    \pcif \projmod{q}(\vec{v}) \neq \hint\\
%    \quad \pcreturn \bot\\
   \alpha_\qbasis\coloneqq\frac{\projring(\vec{v})-\hint'}{q}\in\ring
   \pccomment{numerator divisible by $q$}\\
   \deltav\coloneqq \vec{v}- \decompring(\projring(\vec{v}))\in\ring^\limbs\\
   \text{Find $\alpha_1,\ldots, \alpha_{\limbs-1}\in\ring$ s.t.}\pccomment{Exist by \autoref{lem:latencode}}\\
   \quad \deltav = \alpha_1 \vec{b}_1 + \ldots + \alpha_{\limbs-1}\vec{b}_{\limbs-1}\\
   \pcreturn \overline{\vec{v}} = (\alpha_\qbasis, \alpha_1, \ldots, \alpha_{\limbs-1})
   }
  \pchspace
   \procedure{$\latdecode(\overline{\vec{v}}, \hint)$}{
   \limbs \coloneqq \ceil{\log_{2\eta+1}q}\\
%    \text{Obtain $\eta, q, \limbs$ from $\params$ with $\limbs = \ceil{\log_{2\eta+1}q}$}\\
   \text{Represent $\hint\in\ring_q$ by $\hint'\in\ring$, $\norm{\hint'}\leq \tfrac{q-1}{2}$}\\
   \pcparse \overline{\vec{v}} \pcas (\alpha_\qbasis, \alpha_1,\ldots, \alpha_{\limbs-1})\\
   h''\coloneqq \hint' + q\cdot \alpha_\qbasis \pccomment{We show $h''=\projring(\vec{v})$}\\
   \deltav \coloneqq \alpha_1 \vec{b}_1 + \ldots + \alpha_{\limbs-1}\vec{b}_{\limbs-1}\\
   \pcreturn \decompring(h'') + \deltav \in\ring^\limbs
   }
 \end{pchstack}
 \caption{%
    Algorithms for encoding and decoding an element $\vec{v}\in\ring^\limbs$. Decoding requires $\hint = \projmod{q}(\vec{v})$.
%     We assume that the public parameters $\params$ define $q, \eta, \limbs = \ceil{\log_{2\eta+1}q}$.
    The vectors $\vec{b}_1,\ldots, \vec{b}_{\limbs-1}$ are the basis of $\latring$ as defined in \autoref{prop:kernellattices}.
    }
 \label{fig:EncodingCoeff}
\end{figure}

\begin{remark}
We mention that our formal descripion in \autoref{fig:EncodingCoeff} of the algorithm is self-contained and actually makes no explicit mention of $\tref^{(1)}$ or $\tref^{(2)}$ as defined in \autoref{rmk:RelationshipToBabai}. However, it is easy to see that (in the notation given in the algorithm) $\deltav = \vec{v}-\tref^{(2)}$ and we just encode that by coefficients as in Babai rounding. With $\tref^{(1)}=\decompmod{q}(\hint)$, the definition of $\alpha_\qbasis$ given in \autoref{rmk:RelationshipToBabai} means that this $\alpha_\qbasis$ is such that $\vec{v} = \alpha_\qbasis\vec{b}_\qbasis + \decompmod{q}(\hint) + \vec{v}'$ with $\vec{v}'\in\latgeneral' = \latring$.

Observe that $\projring(\vec{b}_\qbasis) = q$, $\projring(\decompmod{q}(\hint)) = \hint'$ and $\projring(\vec{v}') = 0$. This gives
\[
%  \frac{\projring(\vec{v}) - \hint'}{q} = \frac{\projring(\alpha_\qbasis\vec{b}_\qbasis) + \projring(\decompmod{q}(\hint)) + \projring(\vec{v}') - \hint'}{q} = \frac{\alpha_\qbasis q + \hint' - \hint'}{q} = \alpha_\qbasis
\projring(\vec{v}) = \projring(\alpha_\qbasis\vec{b}_\qbasis) + \projring(\decompmod{q}(\hint)) + \projring(\vec{v}') = \alpha_\qbasis q + \hint'\enspace.
\]
From this, it follows that the definition of $\alpha_\qbasis$ from \autoref{fig:EncodingCoeff} as $\alpha_\qbasis \coloneqq \frac{\projring(\vec{v})-\hint'}{q}$ and the one from \autoref{rmk:RelationshipToBabai} actually coincide.
\end{remark}


\begin{lemma}[Properties of $\latencode$ and $\latdecode$]\label{lem:latencode}
Let $q,\eta,\limbs$ be positive integers with $q$ prime and $\limbs = \ceil{\log_{2\eta+1}q}$.
% Let $\params$ be some public parameters determining those integers.
Then the deterministic encoding and decoding algorithms defined in \autoref{fig:EncodingCoeff} satisfy the following properties:
\begin{enumerate}
 \item Coefficients $\alpha_\qbasis,\alpha_1\ldots,\alpha_{\limbs-1}$ as required in $\latencode$ exist, are unique and can be found in polynomial time.\label{item:basis}
 \item For any $\vec{v}\in\ring^\limbs$, $\hint = \projmod{q}(\vec{v})$ and $\overline{\vec{v}} \gets \latencode(\vec{v})$ we have
 $\latdecode(\overline{\vec{v}}, \hint) = \vec{v}$.\label{item:inverse1}
 \item For any $\overline{\vec{v}}\in\ring^\limbs$, $\hint\in\ring_q$ and $\vec{v}\gets\latdecode(\overline{\vec{v}}, \hint)$, we have\\
 $\projmod{q}(\vec{v})  = \hint$ and $\latencode(\vec{v}) = \overline{\vec{v}}$.\label{item:inverse2}
 \item For any $\vec{v}\in\ring^\limbs$ and $(\alpha_\qbasis,\alpha_1,\ldots,\alpha_{\limbs-1}) \gets \latencode(\vec{v})$, we have
 \begin{align*}
   \norm{\alpha_\qbasis} <{}& \frac{\norm{\projring(\vec{v})}}{q} + \frac{1}{2}\qquad\text{and}\\
   \norm{\alpha_i} <{}& \frac{\norm{\vec{v}}}{2\eta} + \frac{1}{2} \qquad\text{for $1\leq i \leq \limbs-1$}\enspace.
 \end{align*}\label{item:latencodebounds}
\end{enumerate}
\end{lemma}
\begin{proof}
For each of the individual claims, let notation be as in the definitions of the algorithms in \autoref{fig:EncodingCoeff}.
Note that variables $\alpha_\qbasis,\deltav, \hint, \hint', \alpha_1\ldots,\alpha_{\limbs-1}$ appearing in both $\latencode$ and $\latdecode$ with the same name actually have the same value as far as this proof is concerned.
This only matters for \autoref{item:inverse1}, where it is obvious, and for \autoref{item:inverse2}, where we actually need to prove it for $\deltav, \alpha_\qbasis,\alpha_1,\ldots,\alpha_{\limbs-1}$.

\medskip\noindent
Let us now prove each individual claim in order.

\bigskip\noindent
For \autoref{item:basis}, note that $\hint' \equiv \projring(\vec{v}) \mod q$ by definition, so the division by $q$ makes sense in $\ring$.
For the $\alpha_i$, note that $\projring(\decompring(\vec{x})) = \vec{x}$ for all $\vec{x}\in\ring$. This implies that
\[
 \projring(\deltav) = \projring(\vec{v}) - \projring(\decompring(\projring(\vec{v}))) = \projring(\vec{v}) - \projring(\vec{v}) = 0\enspace.
\]
So $\deltav\in\latring$ and, by \autoref{prop:kernellattices}, coefficients $\alpha_i$ exist and are unique.
They can clearly be found by solving an (overdetermined) system of linear equations (over $\ring$).
In fact, $\kernelbasis$ is already in appropriate echelon form, so this can even be done without divisions in $\ring$ and is clearly polynomial time. We will write down another solution explicitly alongside the proof of \autoref{item:latencodebounds}.
% \gnote{Review!}

\bigskip\noindent
% For \autoref{item:inverse1}, we clearly have $\overline{\vec{v}} \neq \bot$.
Let us now tackle \autoref{item:inverse1}.
During decoding, note that
\[
 h'' = \hint' + q\cdot\alpha_\qbasis = \hint'+q\cdot\frac{\projring(\vec{v})-\hint'}{q} = \projring(\vec{v})\enspace.
\]
It follows that
\[
 \latdecode(\overline{\vec{v}}, h) = \decompring(h'') + \deltav = \decompring(\projring(\vec{v})) + (\vec{v} - \decompring(\projring(\vec{v}))) = \vec{v}\enspace,
\]
as desired.


\bigskip\noindent
For \autoref{item:inverse2}, let $\alpha_\qbasis, \deltav, \alpha_i, \hint, \hint'$ denote the values used during the computation by $\latdecode$.
While the equally named values in $\latencode$ are actually the same, we need to prove this.
First, note that $\projring(\vec{b}_i) = 0$ for all $1\leq i \leq \limbs-1$.
By $\ring$-linearity, it follows that $\projring(\deltav) = 0$. From this, we get
\[
 \projring(\vec{v}) = \projring(\decompring(h'') + \deltav) = h'' = \hint' + q\cdot \alpha_\qbasis \equiv \hint' \equiv \hint \mod q\enspace,
\]
so $\projmod{q}(\vec{v}) = \hint$. In particular, the values of $\hint$ and $\hint'$ used in $\latencode$ match those in $\latdecode$.
During the computation of $\latencode(\vec{v})$ with $\vec{v} = \decompring(h'')+\deltav$ output by $\latdecode$, we compute
\[
 \frac{\projring(\vec{v}) - \hint'}{q} = \frac{\projring(\decompring(h'')+\deltav)-\hint'}{q} = \frac{h''-\hint'}{q} = \alpha_\qbasis\enspace,
\]
using again linearity and that $\projring(\deltav) = 0$.
So the value of $\alpha_\qbasis$ recovered inside $\latencode$ is the same as the value of $\alpha_\qbasis$ in $\latdecode$.
Similarly, during the computation in $\latencode$, we have
\begin{align*}
 \vec{v}-\decompring(\projring(\vec{v})) 
 ={}& \decompring(h'') + \deltav - \decompring(\projring(\decompring(h'')+\deltav))\\
 ={}& \decompring(h'') + \deltav - \decompring(\projring(\decompring(h''))+\projring(\deltav))\\
 ={}& \decompring(h'') + \deltav - \decompring(\projring(\decompring(h''))+ 0)\\
 ={}&\decompring(h'') + \deltav - \decompring(h'') = \deltav\enspace,
\end{align*}
so the value for $\deltav$ obtained in $\latencode$ is the same as that in $\latdecode$.
Since $\kernelbasis$ is an $\ring$-basis, it follows that the values of the $\alpha_i$ are also the same, which proves this item.

\bigskip\noindent
We now tackle the last \autoref{item:latencodebounds}, giving bounds on the encodings.
The first bound is just an easy application of the triangle inequality:
\[
 \norm{\alpha_\qbasis} = \norm{ \frac{\projring(\vec{v}) - \hint'}{q}} \leq \frac{1}{q} \Bigl( \norm{\projring(\vec{v})} + \norm{\hint'}\Bigr)
 \leq \frac{1}{q}\Bigl(\norm{\projring(\vec{v})} + \tfrac{q-1}{2}\Bigr)
 < \frac{\norm{\projring(\vec{v})}}{q} + \frac{1}{2}
\]
For the other bound, let us look at the individual components of $\vec{v}$ and $\decompring(\projring(\vec{v})) \in\ring^\limbs$.
For this, set $(w_1,\ldots, w_{\limbs})  \coloneqq \decompring(\projring(\vec{v}))$ and 
$(v_1,\ldots,v_{\limbs}) \coloneqq \vec{v}$ with $v_i,w_i\in\ring$.
By definition of $\decompring$, we have $\norm{w_i}\leq \eta$ for $1\leq i < \limbs$.
Note that this bound excludes the most significant limb.
Writing the equation $\deltav = \alpha_1\vec{b}_1 + \ldots + \alpha_{\limbs-1}\vec{b}_{\limbs-1}$ in its components, using the definition of $\kernelbasis$ gives the following linear system of equations (over $\ring$) in unknowns $\alpha_1,\ldots,\alpha_{\limbs-1}$.
\begin{align*}
v_1 - w_1 &= -(2\eta+1)\alpha_1\\
v_2 - w_2 &= \alpha_1 - (2\eta+1)\alpha_2\\
v_3 - w_3 &= \alpha_2 - (2\eta+1)\alpha_3\\
\ldots\\
v_{\limbs-1} - w_{\limbs-1} &= \alpha_{\limbs-2} - (2\eta+1)\alpha_{\limbs-1}\\
v_{\limbs} - w_{\limbs} &= \alpha_{\limbs-1}
\end{align*}
We now prove the bound for $\norm{\alpha_i}$ by induction over $i$, using those equations\footnote{We have more equations than variables $\alpha_i$ and we will not use the last equation.}\footnote{It may be helpful to think of these equations as equations between polynomials where we allow \emph{rational} coefficients. By \autoref{item:basis}, we know a priori that the (unique) rational solution for the $\alpha_i$ will turn out integral.}.

\smallskip\noindent
For $i=1$, the first equation above gives $\alpha_1 = -\frac{v_1-w_1}{2\eta+1}$. This implies
\[
  \norm{\alpha_1} \leq \frac{\norm{v_1} + \norm{w_1}}{2\eta+1} \leq \frac{\norm{\vec{v}}}{2\eta+1} + \frac{\eta}{2\eta+1} < \frac{\norm{\vec{v}}}{2\eta} + \frac12\enspace.
\]
For $1\leq i\leq \limbs-1$, we have $\alpha_i = -\frac{v_{i}-w_{i} - \alpha_{i-1}}{2\eta+1}$. By induction, we can bound this as
\[
 \norm{\alpha_i} 
 \leq \frac{ \norm{v_{i}} + \norm{w_{i}} + \norm{\alpha_{i-1}} } {2\eta+1}  
 < \frac{ \norm{\vec{v}} + \eta + \frac{\norm{\vec{v}}}{2\eta} + \frac{1}{2} }{2\eta+1} = \frac{\norm{\vec{v}}}{2\eta} + \frac{1}{2}\enspace,
\]
which finishes the proof.\qed
\end{proof}
% We remark that for $\vec{v}\in\ring^\limbs, h=\proj_q(\vec{v})$, with $\norm{\vec{v}}\leq \eta$, $\norm{\proj_q(\vec{v})} < \tfrac{q}{2}$, the bounds from \autoref{lem:latencode} implies
% $\norm{\overline{\vec{v}}} < 1$ for $\overline{\vec{v}}\gets\latencode(\params, \vec{v}, h)$. This corresponds to the situation we have for individually verifying signatures

\begin{definition}\label{def:linearlygoodopenings}
Let $n,\eta, q,q',\xi\in\NN$ with $n$ a power of two, $q,q'$ primes.
Consider the HVC construction $\hvcplain$ from \autoref{fig:hvcinst} with openings from
$\modopening = (\ring^\limbs)^{2\tau} \times (\ring^{\limbs'})^\xi$, where $\limbs = \ceil{\log_{2\eta+1}q}$, $\limbs' = \ceil{\log_{2\eta+1}q'}$.
Let $\params\gets\setup(\secparam)$ be fixed, defining coefficients $\vec{g},\vec{h}_0,\vec{h}_1$ for Ajtai's hash functions.

We call an opening $\vec{d} = (\vec{p}_1,\ldots,\vec{p}_\tau, \vec{s}_1,\ldots,\vec{s}_\tau,\vec{u})$ linearly verifying for time slot $t, 1\leq t\leq 2^\tau$ iff the following conditions hold
\begin{align*}
\vec{g}^\transpose \cdot \vec{u} & =  \projmod{q}(\vec{p}_\tau)\\
\projmod{q}(\vec{p}_{j-1}) & = \vec{h}_{\tilde t_j}^\transpose\cdot \vec{p}_{j} + \vec{h}_{\tilde t_j \xor 1}^\transpose \cdot \vec{s}_{j}\quad \text{for all $2\leq j \leq \tau$}\enspace,
\end{align*}
where $\tilde{t} = \binsca(t-1)$ is the binary decomposition of $t-1$. We define $\modlinear{t}\subset\modopening$ to be the subset of all linearly verifying openings for $t$.
Since $\modlinear{t}$ is defined via $\ring$-linear constraints, $\modlinear{t}$ is an $\ring$-submodule of $\modopening$.
\end{definition}
By construction, any opening that passes either individual, weak or strong verification must be linearly verifying.
Whether a given opening $\vec{d}$ is linearly verifying or not can be checked in polynomial time, given only $\vec{d}$ and public data $\params$ and $t$.

We now define an efficient encoding scheme for linearly verifying openings in \autoref{fig:EncodingOpenings}.
% \begin{definition}\label{def:latencode}
% Let $n, q, q', \eta, \xi,\tau \in\NN$ with $n$ a power of two, $q,q'$ primes.
% \end{definition}
\begin{figure}[ht]\centering
\begin{pchstack}[center,boxed]
  \procedure{$\openencode(\params, t, \vec{d})$}{
     % \tilde t \coloneqq \binsca(t-1)\\
     \pcparse \vec{d} \pcas (\vec{p}_1,\dots,\vec{p}_\tau,\vec{s}_1,\dots,\vec{s}_\tau,\vec{u})\\
     \pcif \vec{d}\notin \modlinear{t}\\
     \quad\pcreturn \bot\\
%      \pcfor j\in\{2,\ldots, \tau\}\\
%      \quad \hint_{j-1} \coloneqq \vec{h}_{\tilde t_j}^\transpose\cdot \vec{p}_{j} + \vec{h}_{\tilde t_j \xor 1}^\transpose \cdot \vec{s}_{j} \in\ring_q\\
%      \hint_\tau \coloneqq \vec{g}^\transpose\cdot \vec{u} \in\ring_q\\
     \pcfor j\in\{1,\ldots, \tau\}\\
     \quad \overline{\vec{p}}_j \coloneqq \latencode(\vec{p}_j)\\
%      \quad \pcif \overline{\vec{p}}_j = \bot\\
%      \quad\quad \pcreturn \bot\\
     \overline{\vec{d}}\coloneqq (\overline{\vec{p}}_1,\ldots,\overline{\vec{p}}_\tau, \vec{s}_1,\ldots,\vec{s}_\tau,\vec{u})\\
     \pcreturn \overline{\vec{d}}
   }
  \pchspace
    \procedure{$\opendecode(\params, t, \overline{\vec{d}})$}{
      \tilde t \coloneqq \binsca(t-1)\\
      \pcparse \overline{\vec{d}} \pcas (\overline{\vec{p}}_1,\dots,\overline{\vec{p}}_\tau,\vec{s}_1,\dots,\vec{s}_\tau,\vec{u})\\
      \hint_\tau \coloneqq \vec{g}^\transpose\cdot \vec{u}  \in\ring_q\\
      \pcfor j\in\{\tau,\ldots, 1\} \pccomment{loop downward}\\
      \quad \vec{p}_j \coloneqq \latdecode(\overline{\vec{p}}_j, \hint_j)\\
      \quad \hint_{j-1} \coloneqq \vec{h}_{\tilde t_j}^\transpose\cdot \vec{p}_{j} + \vec{h}_{\tilde t_j \xor 1}^\transpose \cdot \vec{s}_{j} \in\ring_q\\
      \pccomment{Note: $\hint_0$ is unused}\\
      \vec{d} \coloneqq (\vec{p}_1,\ldots,\vec{p}_\tau,\vec{s}_1,\ldots,\vec{s}_\tau,\vec{u})\\
      \pcreturn \vec{d}
    }
  \end{pchstack}
 \caption{%
    Algorithms for encoding and decoding openings for a given time slot.
%     We assume that the public parameters $\params$ determine $q, \eta, \limbs = \ceil{\log_{2\eta+1}q}$.
%     The vectors $\vec{b}_1,\ldots, \vec{b}_{\limbs-1}$ are the basis of $\latring$ as defined in \autoref{prop:kernellattices}.
    }
 \label{fig:EncodingOpenings}
\end{figure}

% \gnote{Double check that the order of arguments is consistent.}

\begin{theorem}[Efficient encoding of decommitments]\label{thm:EncodingOfOpenings}
Let $n,q,q',\eta,\xi\in\NN$ with $n$ a power of two, $q,q'$ primes.
Let $\params\gets\setup(\secparam)$ be fixed, defining coefficients $\vec{g},\vec{h}_0,\vec{h}_1$ for Ajtai's hash functions.
Fix a time slot $t$ with $1\leq t \leq 2^\tau$.
Define $\modlinear{t}\subset \modopening$ as in \autoref{def:linearlygoodopenings}, where\footnote{We write the domain of $\openencode$ as $\modopening$ resp.\ $\modlinear{t}$ and the range as $\ring^\oplen$. Even though those are the same as sets, we only view the domain as an $\ring$-module in the usual way.} $\modopening = \ring^\oplen$.
Then $\openencode$ and $\opendecode$, as defined in \autoref{fig:EncodingOpenings} satisfy the following properties.
\begin{enumerate}
 \item $\openencode$ and $\opendecode$ are deterministic polynomial time algorithms. \label{item:openencodeispt}
 \item For any $\vec{d}\in\modopening,\overline{\vec{d}} \coloneqq \openencode(\params,t, \vec{d})$, we have $\overline{\vec{d}} = \bot$ iff $\vec{d}\notin \modlinear{t}$.\label{item:openencodeworks}
 \item For any $\overline{\vec{d}}\in\ring^\oplen, \vec{d}\coloneqq \opendecode(\params,t, \overline{\vec{d}})$, we have $\vec{d}\in\modlinear{t}$. \label{item:opendecodeworks}
 \item For fixed $\params$ and time slot $t$, the functions 
 \begin{align*}
 \openencode(\params, t, .)\colon\,\modlinear{t} \to \ring^\oplen&,\ \vec{d}\mapsto \openencode(\params,t, \vec{d})\\
 \opendecode(\params, t, .)\colon\,\ring^\oplen \to \modlinear{t}&,\ \overline{\vec{d}}\mapsto \opendecode(\params,t, \overline{\vec{d}})
 \end{align*}
are inverses to each other.\label{item:openencodeinverse}
\item Let $\vec{d}\in\modopening$ and $(\overline{\vec{p}}_1,\ldots,\overline{\vec{p}}_\tau,\vec{s}_1,\ldots,\vec{s}_\tau,\vec{u}) \coloneqq \overline{\vec{d}}\coloneqq \openencode(\params,t, \vec{d})$. Let $\vec{c}\in\modcommit$ be any (possibly maliciously generated) commitment and let $\verify$ be as in \autoref{fig:hvcinst}. If $\verify(\params, \vec{c}, t,\vec{d}, \beta) \neq \bot$ for some $\beta$, then \label{item:openencodebounds}
\[
 \norm{\overline{\vec{p}}_i} < \frac{\beta}{2\eta} + \frac12\quad\text{for all $i$}\enspace.
\]
\end{enumerate}
We remark that for individually verifying openings, we have $\beta=\eta$ above, so the bound reads $\norm{\overline{\vec{p}}_i} < 1$ for this case, meaning that $\overline{\vec{p}}_i = 0$.
This just captures the fact that in this case, we can use the usual Merkle tree trick of not transmitting the $\vec{p}_i$, but letting the verifier compute them.
\end{theorem}
\begin{proof}
For \autoref{item:openencodeispt} and for \autoref{item:openencodeworks}, there is nothing to show, really.

\bigskip\noindent
Let us show \autoref{item:opendecodeworks} now:
by \autoref{lem:latencode}, \autoref{item:inverse2}, the $\vec{p}_i$ constructed by $\opendecode$ satisfy $\projmod{q}(\vec{p}_j) = \hint_j$ for all $1\leq j \leq \tau$.
With the way $\opendecode$ chooses $\hint_j$, the definition of what it means for an opening to be linearly verifying precisely reads $\projmod{q}(\vec{p}_j) = \hint_j$. So $\vec{d}$ output by $\opendecode$ is linearly verifying.

% For \autoref{item:openencodeworks}, this follows directly from the definitions.
% By \autoref{lem:latencode} resp.\ the definition of $\openencode$, we have that $\overline{\vec{p}}_j = \bot$ iff $\projmod{q}(\vec{p}_j) \neq \hint_j$.
% Our definition of what it means for an opening to be linearly verifying precisely matches this.

\bigskip\noindent
For \autoref{item:openencodeinverse}, let us first show that $\opendecode(\params, t, \openencode(\params,t, \vec{d})) = \vec{d}$ for all $\vec{d}\in\modlinear{t}$.
To disambiguate, we temporarily denote the equally named values $\vec{p}_j$ and $\vec{d}$ appearing in both $\openencode$ and $\opendecode$ by $\vec{p}_j^{\ENC}$ and $\vec{d}^{\ENC}$ resp.\ $\vec{p}_j^{\DEC}$ and $\vec{d}^{\DEC}$.
Of course, these are equal, but that's precisely what we need to show here. For this, we show that
\begin{itemize}
 \item $\hint_j$ constructed by $\opendecode$ satisfies $\hint_j = \projmod{q}(\vec{p}_j^{\ENC})$\qquad and %, where $\vec{p}_j$ is the value parsed from $\vec{d}$.
 \item $\vec{p}_j^{\ENC}$ = $\vec{p}_j^{\DEC}$
\end{itemize}
for each $\tau \geq j \geq 1$ by induction over $j$ (starting at $j=\tau$ and going down):

For $j=\tau$, since $\vec{d}^{\ENC}$ is linearly verifying, $\projmod{q}(\vec{p}_{\tau}^{\ENC}) = \vec{g}^\transpose\cdot\vec{u}$. From this we get $\projmod{q}(\vec{p}_{\tau}^{\ENC}) = \hint_\tau$.
Using \autoref{lem:latencode}, \autoref{item:inverse1}, this gives $\vec{p}_{\tau}^{\DEC} = \vec{p}_{\tau}^{\ENC}$.
For $\tau > j \geq 1$, we have
\begin{align*}
 \projmod{q}(\vec{p}_{j}^{\ENC}) &= \vec{h}_{\tilde t_{j+1}}^\transpose\cdot \vec{p}_{j+1}^{\ENC} + \vec{h}_{\tilde t_{j+1} \xor 1}^\transpose \cdot \vec{s}_{j+1} \tag{$\vec{d}^{\ENC}$ linearly verifying}\\
                               {}&= \vec{h}_{\tilde t_{j+1}}^\transpose\cdot \vec{p}_{j+1}^{\DEC} + \vec{h}_{\tilde t_{j+1} \xor 1}^\transpose \cdot \vec{s}_{j+1} \tag{induction hypothesis}\\
                               {}&= \vec{\hint}_j\enspace. \tag{Definition of $\hint_j$ in $\opendecode$}
\end{align*}
Again, using \autoref{lem:latencode}, \autoref{item:inverse1}, we can conclude $\vec{p}_j^{\DEC} = \vec{p}_j^{\ENC}$, finishing the induction. This gives $\vec{d}^{\DEC} = \vec{d}^{\ENC}$, showing $\opendecode(\params, t, \openencode(\params,t, \vec{d})) = \vec{d}$.

% since $\vec{d}^{\ENC}$ is linearly verifying, $\projmod{q}(\vec{p}_{j+1}^{\ENC}) = \vec{h}_{\tilde t_{j+1}}^\transpose\cdot \vec{p}_{j+1}^{\ENC} + \vec{h}_{\tilde t_{j+1} \xor 1}^\transpose \cdot \vec{s}_{j+1}$. By induction hypothesis, this is equal to

% observe that (by induction over $j$, starting at $j=\tau$ and going down) $\hint_j = \projmode{q}(\vec{p}_j)$ for the and, by \autoref{item:inverse1} of \autoref{lem:latencode}, the $\vec{p}_j$ constructed during $\opendecode$ are the same as in $\openencode$.

Now consider the other direction, i.e.\ that $\openencode(\params, t,\opendecode(\params,t,\overline{\vec{d}})) = \overline{\vec{d}}$ for all $\overline{\vec{d}}\in\ring^{\oplen}$.
By \autoref{lem:latencode}, \autoref{item:inverse2}, we get $\projmod{q}(\vec{p}_j) = \hint_j$ (so $\vec{d}$ is linearly verifying and $\openencode$ does not abort) and that the values of $\overline{\vec{p}}_j$ constructed by $\openencode$ are the same as those in $\opendecode$, which shows the claim.
% Here, we can see directly from the definition that the values of $\hint_j$ constructed by $\openencode$ are the same as in $\opendecode$.
% From \autoref{item:inverse2} of \autoref{lem:latencode}, we then get that all $\overline{\vec{p}}_j$ constructed by $\openencode$ are the same as the input to $\opendecode$, showing that claim.
% Observe that we also showed here that $\openencode$ does not output $\bot$, which proves \autoref{item:opendecodeworks} along the way.

\bigskip\noindent
For \autoref{item:openencodebounds}, write $\vec{d} = (\vec{p}_1,\ldots,\vec{p}_\tau,\vec{s}_1,\ldots, \vec{s}_\tau,\vec{u})$.
Recall that $\verify(\params, \vec{c},t,\vec{d},\beta) \neq \bot$ checks among other things that
\[
 \norm{\vec{p}_j} \leq \beta \quad\text{and}\quad \norm{\projring(\vec{p}_j)}\leq \frac{q\beta}{2\eta} \quad\text{for all $1\leq i \leq \tau$}\enspace.
\]
Plugging this into \autoref{item:latencodebounds} of \autoref{lem:latencode} directly gives $\norm{\overline{\vec{p}}_j} < \frac{\beta}{2\eta}+\frac12$ for all $j$.
\qed
\end{proof}

We can use $\openencode(\params, t, .)$ and $\opendecode(\params, t, .)$ to store and transmit openings:
By \autoref{item:openencodebounds} of \autoref{thm:EncodingOfOpenings}, the encoded openings have smaller norm than the unencoded versions, which can be used to save space.
The restriction that $\openencode(\params, t, .)$ only works for linearly verifying openings is immaterial, as openings violating this condition will never be valid anyway.

% \gnote{I wonder to what extent \autoref{thm:encodedhvcworks} and possibly even \autoref{def:encodedhvc} are kind-of obvious and treating this so formally may not be terribly helpful.}
Formally, we can define an encoded version of Chipmunk's HVC as follows:
\begin{definition}\label{def:encodedhvc}
Let $n,q,q',\alpha_w,\rho,\eta,\tau,\xi,\bagg$ be positive integers such that $n$ is a power of two, $q,q'$ are prime.
Let $\hvcplain = (\setup,\commit,\open,\iverify,\sverify, \wverify)$ be the HVC from \autoref{fig:hvcinst} for its domain $\moddomain$ and vectors of length $2^\tau$.
Denote by $\modopening$ and $\modcommit$ the $\ring$-modules where the openings and commitments are from.
Recall that $\modopening$ has the form $\modopening = \ring^\oplen$.

We can then define an encoded version $\hvcencoded = (\setup',\commit',\open',\iverify',\sverify', \wverify')$ by simply encoding/decoding the openings as follows:
\begin{description}
    \item[$\setup'(\secparam)$:] Identical to $\setup$.
    \item[$\commit'(\params, \vec{m})$:] Identical to $\commit$.
    \item[$\open'(\params,\vec{c},\vec{m},t)$:] Run $\vec{d}\gets\open(\params,\vec{c},\vec{m},t)$ and output $\overline{\vec{d}} = \openencode(\params, t, \vec{d})$.
    \item[$\iverify'(\params, \vec{c},t,\overline{\vec{d}})$:] Run $\vec{d}\gets\opendecode(\params,t,\overline{\vec{d}})$. Output whatever $\iverify(\params,\vec{c},t,\vec{d})$ outputs.
    \item[$\sverify'(\params, \vec{c},t,\overline{\vec{d}})$:] Run $\vec{d}\gets\opendecode(\params,t,\overline{\vec{d}})$. Output whatever $\sverify(\params,\vec{c},t,\vec{d})$ outputs.
    \item[$\wverify'(\params, \vec{c},t,\overline{\vec{d}})$:] Run $\vec{d}\gets\opendecode(\params,t,\overline{\vec{d}})$. Output whatever $\wverify(\params,\vec{c},t,\vec{d})$ outputs.
\end{description}
\end{definition}
Note that the opening space of $\hvcencoded$ is $\ring^\oplen$.
However, to perform homomorphic operations on openings, we need to operate on the \emph{unencoded} values, since encoding/decoding is not a $\ring$-linear operation with the usual $\ring$-module structure on $\ring^\oplen$.
To formally satisfy the homomorphism requirements, we therefore need to endow the set $\ring^\oplen$ with a (non-standard) $\ring$-module structure $\modencoded{t} = (\ring^\oplen, \odot, \oplus)$, where scalar multiplication $\odot$ by ring elements and the addition $\oplus$ are given by
\begin{align*}
 \odot\colon\,\ring\times\modencoded{t}\to\modencoded{t},&\quad w\odot \overline{\vec{d}} \coloneqq \openencode(\params,t, w\cdot\opendecode(\params, t, \overline{\vec{d}}))\\
 \oplus\colon\,\modencoded{t}\times\modencoded{t},&\quad \overline{\vec{d}}_1 \oplus \overline{\vec{d}}_2 \coloneqq \openencode(\params, t, \opendecode(\params,t,\overline{\vec{d}}_1) + \opendecode(\params, t,\overline{\vec{d}}_2))
\end{align*}
This gives an $\ring$-module $\modencoded{t}$, which is the opening space of $\hvcencoded$.
Note that it depends on $t$.

\begin{theorem}\label{thm:encodedhvcworks}
Let $n, q, q',\alpha_w,\rho, \eta, \tau, \xi, \bagg$ be positive integers and $0<\errorbound \leq 1$ such that $n$ is a power of two and $q,q'$ prime.
Let $\hvcplain = (\setup,\commit,\open,\iverify,\sverify, \wverify)$ be the homomorphic vector commitment from \autoref{fig:hvcinst} and $\hvcencoded = (\setup,\commit,\open',\iverify',\sverify', \wverify')$ be the encoded version from \autoref{def:encodedhvc} based on it for those parameters.
Then $\hvcencoded$ is individually correct, $(\rho,\tern_{\alpha_w},\errorbound)$-probabilistically homomorphic, robustly homomorphic and position binding for domain $\ring_{q'}^\xi$ and vector length $2^\tau$, provided $\hvcplain$ has those properties.
\end{theorem}
\begin{proof}
There is really not much to show here.

Individual correctness follows directly from \autoref{item:openencodeinverse} of \autoref{thm:EncodingOfOpenings}.

The robust homomorphism properties also follows from this, together with the way we defined $\modencoded{t}$.
Notably, assume we have $\params\gets\setup(\secparam)$, $1\leq t \leq 2^\tau$, commitments $\vec{c}^0,\vec{c}^1\in\moddomain$ and $\overline{\vec{d}}^0,\overline{\vec{d}}^1\in\modencoded{t}$ with
\[
 \sverify'(\params,\vec{c}^0,t,\overline{\vec{d}}^0) = \vec{m}^0
 \quad\text{and}\quad
 \sverify'(\params,\vec{c}^1,t,\overline{\vec{d}}^1) = \vec{m}^1
\]
such that $\vec{m}^0,\vec{m}^1\neq \bot$.
We need to show that
\[
 \wverify'(\params, \vec{c}^0-\vec{c}^1, t, \overline{\vec{d}}^0 \ominus \overline{\vec{d}}^1) = \vec{m}^0 - \vec{m}^1\enspace,
\]
where $\ominus$ is the subtraction in $\modencoded{t}$ corresponding to $\oplus$.

Let $\vec{d}^0 \coloneqq \opendecode(\params, t, \overline{\vec{d}}^0), \vec{d}^1\coloneqq \opendecode(\params, t, \overline{\vec{d}}^1)$. By definition of $\sverify'$, we have
\[
 \vec{m}^0 = \sverify(\params,\vec{c}^0,t,\vec{d}^0)
 \quad\text{and}\quad
 \vec{m}^1 = \sverify(\params,\vec{c}^1,t,\vec{d}^1)\enspace.
\]
Since $\hvc$ is robustly homomorphic, this yields
\[
 \wverify(\params, \vec{c}^0-\vec{c}^1, t, \vec{d}^0 - \vec{d}^1) = \vec{m}^0 - \vec{m}^1\enspace.
\]
By definition, $\opendecode(\params, t, \overline{\vec{d}}^0 \ominus \overline{\vec{d}}^1) = \vec{d}^0 - \vec{d}^1$. Putting these together, we obtain
\begin{align*}
   &\wverify'(\params, \vec{c}^0-\vec{c}^1, t, \overline{\vec{d}}^0 \ominus \overline{\vec{d}}^1) \\
 = &\wverify (\params, \vec{c}^0 - \vec{c}^1, t, \opendecode(\params, t, \overline{\vec{d}}^0 \ominus \overline{\vec{d}}^1) )\\
 = &\wverify (\params, \vec{c}^0 - \vec{c}^1, t, \vec{d}^0 - \vec{d}^1)\\x
 = &\vec{m}^0 - \vec{m}^1\enspace.
\end{align*}

For the probabilistic homomorphism property, let $\params\gets\setup(\secparam), \ell < \rho$ and $1\leq t \leq 2^\tau$.

For $1\leq i \leq \ell$, consider commitments $\vec{c}^i\in\modcommit$, decommitments $\overline{\vec{d}}^i\in\modencoded{t}$ with $\iverify'(\params, \vec{c}^i,t,\overline{\vec{d}}^i) = \vec{m}^i$ such that $\vec{m}^i\neq \bot$.
We need to show that 
\[
    \Pr\mleft[
      w^1,\dots,w^{\ell} \gets W\colon\;
      \sverify'\Bigl(\params,\sum_{i=1}^{\ell}w^i\cdot \vec{c}^i,t,\bigoplus_{i=1}^{\ell}w^i\odot\overline{\vec{d}}^i\Bigr)
      = \sum_{i=1}^{\ell}w^i\cdot\vec{m}^i_{t}
    \mright] \geq 1-\errorbound \enspace.
\]
For this, set $\vec{d}^i\coloneqq \opendecode(\params, t, \overline{\vec{d}}^i)$. By definition of $\iverify'$, we have $\iverify(\params, \vec{c}^i, t, \vec{d}^i) = \vec{m}^i$. Then we get
\begin{align*}
  &\sverify'\Bigl(\params,\sum_{i=1}^{\ell}w^i\cdot \vec{c}^i,t,                             \bigoplus_{i=1}^{\ell}w^i\odot\overline{\vec{d}}^i        \Bigr)\\
 =&\sverify \Bigl(\params,\sum_{i=1}^{\ell}w^i\cdot \vec{c}^i,t,\opendecode\Bigl(\params, t, \bigoplus_{i=1}^{\ell}w^i\odot\overline{\vec{d}}^i \Bigr) \Bigr)\\
 =&\sverify \Bigl(\params,\sum_{i=1}^{\ell}w^i\cdot \vec{c}^i,t,                                  \sum_{i=1}^{\ell}w^i\cdot          \vec{d}^i         \Bigr)\\
\end{align*}
The claim then follows from the probabilistic homomorphism property of $\hvc$.

Let us look at the position-binding property. For any adversary $\adv$ against the position-binding property of $\hvcencoded$, we construct an adversary $\bdv$ against $\hvc$ as follows.

$\bdv(\params)$ runs $(\vec{c}, t, \overline{\vec{d}}_0,\overline{\vec{d}}_1)\gets\adv(\params)$ and outputs $(\vec{c}, t, \opendecode(\params, t,\overline{\vec{d}}_0), \opendecode(\params, t, \overline{\vec{d}}_1))$. It is easy to see that $\adv$ is successful iff $\bdv$ is.\qed
%Note here that $\openencode(\params, t, \overline{\vec{d}}_0)$ or $\openencode(\params, t, \overline{\vec{d}}_1)$ may be $\bot$. This is fine: in that case, 
\end{proof}

\begin{remark}
Let $n, q, q',\alpha_w,\rho, \eta, \tau, \xi, \bagg$ be positive integers such that $n$ is a power of two and $q,q'$ prime.
Let us collect in \autoref{table:hvcsizes} the individual components of our HVC constructions and look at the bit-sizes of commitments and (individually or strongly verifying) openings as functions of the parameters.
Note that the strongly verifying case will correspond to the contribution for the size of aggregated signatures later in \autoref{sec:nidv}, and this size is the most important metric we want to minimize.

A commitment is a (non-short) single element from $\ring_q$. This means we can use $n\ceil{\log q}$ bits to store it.\footnote{In principle, we could do $\ceil{n\log q}$ by using some clever arithmetic encoding; however, for simplicity, we assume here that every coefficient is stored individually.}
In the variant without our elaborate encodings, an opening consists of $\oplen = 2\tau\ceil{\log_{2\eta+1}q} + \xi\ceil{\log_{2\eta+1}q'}$ many elements from $\ring$.
Each element is $\norm{.}_\infty$-bounded: for individually verifying openings, the bound is $\eta$, giving $n\oplen\ceil{\log (2\eta+1)}$ bits. For strongly verifying opening, the bound is $\bagg$, giving $n\oplen\ceil{\log (2\bagg+1)}$ bits.

For $\hvcencoded$, an opening consists of the same number of elements from $\ring$, but we have tighter size constraints for $\tau\ceil{\log_{2\eta+1}q}$ of them. For the individually verifying case, those elements are actually 0. In the strongly verifying case, the (non-attained) bound is $\tfrac{\bagg}{2\eta}+\frac12$, giving $n\tau\ceil{\log_{2\eta+1}q}\ceil{\log(2\lfloor\frac{\bagg}{2\eta}+\frac12\rfloor +1 )} \leq n\tau\ceil{\log_{2\eta+1}q}\ceil{\log (\lfloor\tfrac{\bagg}{\eta}\rfloor+2)}$ many bits for the $\overline{\vec{p}}_i$'s.

\begin{table}\centering
 \begin{tabular}{ccc@{\hskip 3.5ex}l}
  & & & size in bits \\\toprule
  commitments & & & $n\ceil{\log q}$\\
  \hline
  \multirow{4}{*}{opening}& \multirow{2}{*}{$\hvcplain$} &individually verifying & $\bigl(2\tau\limbs + \xi\limbs'\bigr)n\cdot \ceil{\log(2\eta+1)}$\\\cline{3-4}
                          &                                   &strongly verifying & $\bigl(2\tau\limbs + \xi\limbs'\bigr)n\cdot \ceil{\log(2\bagg+1)}$  \\\cline{2-4}
                          & \multirow{2}{*}{$\hvcencoded$}    &individually verifying & $\bigl(\tau\limbs + \xi\limbs'\bigr)n\cdot \ceil{\log(2\eta+1)}$\\\cline{3-4}
                          &                                   &strongly verifying & $\bigl(\tau\limbs + \xi\limbs'\bigr)n\cdot \ceil{\log(2\bagg+1)}+ \tau\limbs n\normalceil{\log (\lfloor\frac{\bagg}{\eta}\rfloor+2)}$\\
                          \hline
 \end{tabular}
 \medskip % Alternatively, put caption on top (some style guides suggest this)
 \caption{bitlength of our HVC constructions. We denote by $\limbs = \ceil{\log_{2\eta+1}q}$ and $\limbs' = \ceil{\log_{2\eta+1}q'}$ the number of limbs for the decompositions of $\ring_q$ resp.\ $\ring_{q'}$ elements.}
 \label{table:hvcsizes}
\end{table}
\end{remark}


\if 0
\begin{remark}[Lattice enumeration]\label{rmk:RelationshipToBabai}
% In our notation the $B$ in $\latencode$ and $\latdecode$ is for Babai.
Let us now explain how our encoding and decoding is connected to Babai's algorithm~\cite{DBLP:journals/combinatorica/Babai86} (or more precisely, the generalization in~\cite{RSA:LinPei11}) and lattice enumeration.
Note that, to keep our exposition self-contained, both our actual algorithm in \autoref{fig:EncodingCoeff} and its analysis make no mention of this connection and can be understood without it.
The problem we need to solve is, essentially, to enumerate the solutions to $\proj_q(\vec{v}) = h$ by encodings $\overline{\vec{v}}$, where short solutions $\vec{v}$ should be encoded by small $\overline{\vec{v}}$.
The set of all such solutions is a lattice coset for the kernel of $\proj_q$. So this is essentially a close vector problem for a lattice.

Let $\kernelbasis = \{\vec{b}_0,\vec{b}_1\ldots\}$ be a $\ZZ$-basis for the kernel of $\proj_q$ (which could be any linear map here).
Writing everything wrt.\ this basis, we are looking at solutions $\vec{c} = (c_0,c_1\ldots)$ of
\[
\proj_q\Bigl(\sum_{i\geq 0} c_i\vec{b}_i\Bigr) = h\enspace,
\]
where $\sum_{i\geq 0} c_i\vec{b}_i$ is integral.
Note that the $c_i$ themselves do not need to be integers, but are rational numbers.
By construction, if $\vec{c}$ is a solution then so is $\vec{c} + \vec{\delta}$ for any integral $\vec{\delta}$.

Consequently, if any solution exists, we can find a reference solution $\vec{c}'$ where each $c_i\in [-\tfrac12,+\tfrac12[$, which corresponds to what is often called Babai rounding.
A naive way to encode a given $\vec{c}$ is then by encoding it by $\vec{\delta} \coloneqq \vec{c} - \vec{c}'$, which is an integral vector.

A better way, especially if the basis is very non-orthogonal, is to use Babai's nearest plane(s) algorithm~\cite{DBLP:journals/combinatorica/Babai86,RSA:LinPei11}:
we only use the naive construction to obtain the first coefficient $\delta_0$. For fixed $\delta_0$, we can plug this into the equation and consider solutions to
\[
\proj_q\Bigl(\sum_{i\geq 1} c_i\vec{b}_i\Bigr) = h - \proj_q(c'_0 + \delta_0\vec{b}_0)\enspace,
\]
where $\sum_{i\geq 1} c_i\vec{b}_i$ is integral. We then use recursion to solve a problem in dimension 1 less.
Note that the reference solution obtained in a given step during this recursion depends on the choices of $\delta_0,\ldots$ made in previous steps.

Our actual algorithms $\latencode$ and $\latdecode$ correspond precisely to this, with the only difference being that we work over\footnote{Normally, we would have to expand everything over $\ZZ$ and increase dimension $n$-fold, but since both $\proj_q$ and our given $\kernelbasis$ can be written with $\ZZ$ coefficients, all monomials are independent and we can just work in $\ring$.} $\ring$ and that we only perform the above recursive step \emph{once} to get a single coefficient $\delta_q$ before using the naive algorithm to get the rest of the coefficients $(\alpha_1,\alpha_2,\ldots)$ in one go.
The reason we perform the recursive step only once is that $(\vec{b}_1,\vec{b}_2,\ldots)$ is almost orthogonal and parallel to the coordinate axes (the latter condition is relevant, because we work in $\norm{.}_\infty$, not $\norm{.}_2$), so the naive way is ``good enough''.
Only $\vec{b}_0$ needs special handling.
% Enumeration-based lattice algorithms for the closest vector problem enumerate those solutions in a tree-like fashion in what is a generalization of Babai's algorithm.
% So we can encode solutions by the path in the enumeration tree that was taken.
%
%
%
%
% One way to write Babai's algorithm and its enumeration generalization for full-rank lattices is as follows:
% We pick the solution to $\proj_q(\vec{v}) = h$ by fixing one coordinate (wrt. $\kernelbasis$) of the solutions after another.
% For this, consider the set $S_0$ of all $c_0$ for which $\proj_q(c_0\vec{b}_0 + \sum_{i\geq 1} c_i\vec{b}_i) = h$ has a solution.
% The set $S_0$ has the form $c_0\in\{\alpha_0\cdot v + c_0' \mid \alpha_0\in\ZZ\}$ for some $v\in\ZZ$ and any fixed solution $c_0'\in S_0$.
% Babai's algorithm only considers the particular value of $c_0'$ that greedily minimizes the $\norm{.}_2$ that this choice will ensure (by looking at the contribution in orthogonal projection to the span of $\vec{b}_0$).
% Enumeration algorithms also consider other solutions, parameterized by $\alpha$.
%
% Once $\alpha_0$ is fixed, we can shift the target by considering $\proj_q(\sum_{i\geq 1} c_i\vec{b}_i) = h - \proj_q(c_0\vec{b}_0)$ and recursing this approach.
% The sequence of $\alpha_i$ then encodes the particular solution.
%
% Our encoding is an adaption of this approach to $\norm{.}_{\infty}$.
% The difference is that we greedily minimize the $\norm{.}_{\infty}$ norm that a given choice of $c_0'$ will contribute, meaning we minimize $\abs{c_0'}$.
% Then $\alpha_0$ corresponds to $\delta_q$.
\end{remark}
\fi
