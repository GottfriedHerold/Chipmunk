% !TEX root = ../main.tex
\section{Preliminaries}\label{sec:prelim}
This section introduces notation, some basic definitions and a few basic lemmas that we will use throughout this work.
We denote by $\secpar\in\NN$ the security parameter and by $\poly$ any function that is bounded by a polynomial in $\secpar$.
A function $f$ in $\secpar$ is negligible, if for every $c \in \NN$, there exists some $N\in\NN$, such that for all $\secpar>N$ it holds that $f(\secpar) < 1/\secpar^c$.
We denote by $\negl$ any negligible function.
An algorithm is PPT if it is modeled by a probabilistic Turing machine with a running time bounded by $\poly$.

Let $S$ be a set.
We write $x\gets S$ for the process of sampling an element of $S$ uniformly at random.
Let $T$ be a full binary tree of depth $d$.
We denote the root node of $T$ by the empty string $\emptylist$, and for any node $v$, $v\concat 0$ and $v\concat 1$ denotes the left and right child of $v$ respectively.
In particular, $\bin^d$ is the set of leaves of $T$.
A labeled full binary tree with labels in $S$ is represented by a labeling function $\lbl\colon\, \bin^{\leq d} \to S$.

Let $\vec{v},\vec{u}$ be vectors of length $m$.
We throughoutly use 1-based indices in this work.
We write $\vec{v}^\transpose$ to denote the transpose of $\vec{v}$ and $v_i$ to denote the $i$-th entry in the vector for $1\leq i \leq m$.
%\in\interval{\abs{\vec{v}}}$. %%%
We generalize this notation and write $\vec{v}_{\subrange{i}}$ to denote the $(i-1)$-length prefix of $\vec{v}$.
We use the same notation for a bit-string $s$, denoting by $s_i$ the $i$-th bit and by $s_{\subrange{i}}$ the prefix consisting of the first $i-1$ bits of $s$.
From time to time we will slightly abuse this notation and use a bit-string $s$ as an index.
In this case the index is to be understood as the canonical interpretation of $s$ as an integer in big-endian encoding.\gnote{Check actual usage of this abuse of notation.}
Conversely, for $0\leq t \leq 2^\tau-1$ we denote by $\binsca(t)\in\{0,1\}^\tau$ the big-endian binary decomposition of $t$ (possibly with leading zeros to ensure a fixed length of $\tau$).
% \gnote{We actually mix 0-based indices and 1-based indices, so this is not true as stated here. This whole section needs discussion and review.}

For $n\in\NN$, we denote by $\interval{n}$ the set $\{1,\dots,n\}$.

We define the function $\zip$ that \enquote{zips} up two vectors into a single vector of pairs, i.e.
\[
\zip(\vec{a},\vec{b}) := \begin{pmatrix}(a_1,b_1)\\
\vdots\\
(a_\ell,b_\ell) \end{pmatrix}.
\]

\gnote{zip is only used once and much later in the document on page \pageref{fig:nidvconst}. The reader will have forgotten this. Maybe just spell it out at place of usage?}

% Without loss of generality, we work on a power-of-two cyclotomic polynomial ring.
Our concrete construction works over a power-of-two cyclotomic polynomial ring.
Let $\Phi_{2n} = X^n+1$ be the cyclotomic polynomial with $n$ a power of 2. 
We work in the polynomial ring $\ring =  \defring$. For the purpose of taking norms and transmitting data, we represent elements of $\ring$ as $n$-dimensional vectors $\ZZ^n$ with $(c_1,\dots,c_{n})^\transpose\in \ZZ^n$ representing the ring element $\sum_{i=1}^{n} c_{i} X^{i-1}$.
For any odd prime number $q$, we always represent $\ZZ_q$ by the set $\{-\tfrac{q-1}{2},\ldots,\tfrac{q-1}{2}\}$ centered around 0.
% Let $q$ be some odd prime. % such that $q \equiv 1 \bmod 2n$. \gnote{We do not actually need this restriction in our constructions. It only matters for efficiency. Move to Benchmark section?}
For an odd prime $q$, we denote by $\ring_q$ the quotient ring of $\ring$ modulo $q$, represented by vectors in $\ZZ_q^n$. Whenever we need to take representatives to view these as elements from $\ring$, we do so by taking representatives centered around 0 as above. For efficiency reasons, our parameter choice will always satisfy $q \equiv 1 \bmod 2n$, so we can use more efficient NTT-based multiplication in $\ring_q$.
Let $\vec{c} \in \ring$ be a ring element with coefficients $(c_1,\ldots, c_{n})$.
We work, unless specified otherwise, with the $\norm{.}_\infty$-norm.
We define $\norm{\vec{c}}\coloneqq\norm{\vec{c}}_\infty = \max_{i}\abs{c_i}$ and $\norm{\vec{c}}_1 = \sum_{i}\abs{c_i}$ on $\ring$ by taking the norm of the coefficient vector in the monomial basis.
We extend these definitions to norms on $\ring_q$ by taking representatives in $\ring$ using coefficients in $\{-\tfrac{q-1}{2},\ldots,\tfrac{q-1}{2}\}$.
We also extend the definition of $\norm{.}_\infty$ to $\ring^m$ for any $m$ by $\norm{\vec{c}}_{\infty}= \max_i\norm{\vec{c}_i}_\infty$.

Our convention is that mixed multiplication of an element from $\ring$ with an element from $\ring_q$ gives an element from $\ring_q$, thereby viewing $\ring_q$ as an $\ring$-module.

We denote by $\ball{\beta}{q}$ the ball $\ball{\beta}{q} = \{a\in\ring_q\mid \norm{a}\leq\beta\}$. We are only interested in the case $\beta < \tfrac{q}{2}$.
By 
\(
\tern_\alpha = \{a=(a_1+a_2\cdot X + \dots + a_{n}X^{n-1}) \in\ring \mid \norm{a}_\infty\leq 1 \land \sum_{i=1}^{n} \abs{a_i} = \alpha\}
\) %Note (Gotti): Making the \infty-norm explicit in \tern, but not in \ball is intentional. -- changing the norm in our whole construction would not change it for \tern.
we denote the set of polynomials with ternary coefficients, i.e. coefficients from $\{-1,0,1\}$, and with exactly $\alpha$ non-zero coefficients.


Observe that for our choices of ring $\ring$ and norm, for any $a\in\ring$, we have $\norm{a} = \norm{X\cdot a}$, because multiplication by $X$ acts on the coefficient vector as a cyclic shift (up to sign).
For such rings and norms, we can make use of the following simple lemma that allows us to bound the norm of the product of two polynomials.
\begin{lemma}[\cite{Mic07}]\label{lem:ternbound}
  Let $a,b\in\ring$ be two polynomials. Then $\norm{b\cdot a} \leq \norm{a}_1\cdot\norm{b}$.
\end{lemma}

The security of our constructions relies on the hardness of the short integer solution problem defined over rings as follows.
\begin{definition}[Ring Short Integer Solution Problem]\label{def:ring-sis}
  For a ring $\ring$ and parameters $\mu,q,\beta \in \NN$, the $\sis_{\ring,q,\mu,\beta}$ problem is hard if for all PPT algorithms $\adv$ it holds that
  \[
    \Pr[
      \vec{a} \gets \ring_q^\mu; \vec{s}\gets\adv(\vec{a}) \colon \vec{s}\in\ball{\beta}{q}^\mu\setminus\{\vec{0}\} \land \vec{a}^\transpose\vec{s}=0
    ] \leq \negl
  \]
\end{definition}


We will construct a particular homomorphic vector commitment (HVC) denoted by $\hvcplain$ in \autoref{sec:veccom}, then improve it to a more compact $\hvcencoded$ in \autoref{sect:efficientencoding}.
In \autoref{sec:otms}, we construct a key-homomorphic one-time signature scheme (KOTS) denoted by $\otschip$. In \autoref{sec:nidv}, we combine those components to construct a synchronized aggregatable signature scheme.
Our concrete constructions depends on a significant number of tunable parameters, whose choices affects both security, efficiency and functionality. \autoref{table:meaning_of_parameters} gives an overview and is intended as a reference for later, to aid the reader.

\begin{table}
\centering
\begin{tabular}{c@{\hskip 2ex}l}
 parameter & meaning\\\toprule
 $n$ & Dimension of the ring $\ring = \defring$ we are working over; $n$ is a power of 2.\\
 $q$ & Prime number for HVCs. Our HVCs works modulo $q$, i.e. with $\ring_q = \ring/\langle q\rangle$.\\
     & Note that our HVCs can be used to commit to elements from $\ring_{q'}$.\\
 $q'$ & Prime number for KOTS. Our KOTS works modulo $q'$, i.e. with $\ring_{q'} = \ring/\langle q'\rangle$.\\
 $\tau$ & Depth of our Merkle tree. $2^\tau$ is the number of indices for the HVCs.\\
 & This is also the number of time slots for the synchronized multi-signature.\\
%  $\params$ & public parameters. \\
%  $c$ & HVC commitment\\
%  $d$ & HVC opening\\
%  $\vec{m}$ &\\
%  $\moddomain$ & \\
%  $\modcommit$ & \\
%  $\modopening$ & \\
%  $\oplen$ & \\
%  $\comlen$ & \\
%  $\domlen$ & \\
 $\rho$ & Maximum number of homomorphic vector commitments or signatures that we support aggregating.\\
%  $\limbs$ \\
%  $\limbs$' &  \\
 $\eta$ & Arity parameter. Our constructions make use of $2\eta+1$-ary decomposition.\\
 $\xi$ & Dimension (over $\ring_{q'}$) of the elements our HVCs commits to.\\
       & When constructing the synchronized multi-signatures, $\xi = 2\otspkkeylen$.\\
 $\otspkkeylen$ & Dimension (over $\ring_{q'}$ of elements in the KOTS.\\
                & When constructing the synchronized multi-signatures, $\xi = 2\otspkkeylen$.\\
 $\bagg$ & Norm bound for HVCs after aggregation/homomorphic addition.\\
 $\alpha_w$ & Norm bound for ring elements used as coefficients for homomorphic addition of our HVCs or KOTS.\\
 $\alpha_H$ & Norm bound for ring elements used as randomized in the construction of each individual KOTS signature.\\
  $\errorbound$ & Error bound. Homomorphic addition may fail with at most this probability for our HVCs or KOTS.\\
 %
%  $\norm{.}$ & infinity-norm. For polynomials, we use monomial coordinates.\\
%  $\ring$ & polynomial ring $\ring = \ZZ[X]/\langle X^n + 1\rangle$\\
%  $\ring_q$ & polynomial ring $\ring = \ZZ_q[X]/\langle X^+n + 1\rangle$, where we work modulo $q$.\\
\end{tabular}
\medskip % Alternatively, put caption on top.
 \caption{Parameters used in our concrete homomorphic vector commitment (HVC) and key-homomorphic one-time signature (KOTS) schemes. Since we combine those to a synchronized aggregatable signature scheme later, the parameters are related.\label{table:meaning_of_parameters}}
\end{table}
