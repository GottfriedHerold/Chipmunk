% !TEX root = ../main.tex

\section{Key-Homomorphic One-Time Signatures}\label{sec:otms}

In this section, we define and instantiate key-homomorphic one-time signatures, which are a weak form of a digital signature scheme that is only guaranteed to be unforgeable, if at most one signature is published under any given public key.
A one-time signature is called key-homomorphic, if the linear combination of separate signatures for the same message verifies under the linear combination of the corresponding public keys.

Our definitions again follow the definitions of \cite{CCS:FleSimZha22} closely, but are incomparable just as in \autoref{sec:veccom}.
As with the vector commitments, we have the stronger requirement that the homomorphism works for any individually verifying signature, not just honestly created ones.
But, this homomorphism is allowed to have a noticeable correctness error.

The construction presented in this section is a modification of the construction of \cite{CCS:FleSimZha22}, which itself was a modification of the one-time signature schemes by Boneh and Kim~\cite{BonKim2020} and Lyubashevsky and Micciancio~\cite{TCC:LyuMic08}.

% % \gnote{Should define key and signature spaces as $\ring$-modules rather than talking about $\siglen$ and $\opklen$ (and say that what those modules are for our cases), similarly to what HVC def does. The current definition talks about arbitrary rings, for which we did does not define $\ring_q$.}
\begin{definition}[Key-Homomorphic One-Time Signature]\label{def:hots}
  Let $\ring$ be a ring. Let $\modopk$ and $\modsig$ be $\ring$-modules denoting the spaces where the public keys and signatures are from.
  A key-homomorphic one-time signature scheme (KOTS) over $\ring$ with public key space $\modopk$ and signatures from $\modsig$
%   $\opklen$ and signature length $\siglen$ 
  is defined by six PPT algorithms $\hots=(\setup,\kgen,\sign,\iverify, \sverify, \wverify)$.
  \begin{description}
    \item[$\params\gets\setup(\secparam)$] The setup algorithm takes as input the security parameter and outputs public parameters.
    \item[$(\osk,\opk) \gets \kgen(\params)$] The key generation algorithm takes as input the public parameters and outputs a key pair with $\opk\in\modopk$.
    \item[$\vec{\sigma} \gets \sign(\params,\osk,m)$] The signing algorithm takes as input the public parameters, a one-time signing key, and a message and outputs a signature $\sigma\in\modsig$.
    \item[$b\gets \iverify(\params,\opk,m,\sigma)$] The individual verification algorithm takes as input the public parameters, a verification key, a message, and a candidate signature and outputs a bit indicating acceptance/rejection.
    \item[$b\gets \sverify(\params,\opk,m,\sigma)$] The strong verification algorithm has the same input and output domains as the individual verification algorithm.
    \item[$b\gets \wverify(\params,\opk,m,\sigma)$] The weak verification algorithm has the same input and output domains as the individual verification algorithm.
  \end{description}
\end{definition}
Note that for us, we will always have $\ring = \ZZ[X]/\langle X^n+1\rangle$ for $n$ a power of two and $\modsig = \ring_{q'}^\siglen$, $\modopk = \ring_{q'}^\opklen$ for some prime $q'$.

\begin{definition}[Individual Correctness]
Let $\hots$ be a key-homorphic one-time signature scheme.
$\hots$ is individually correct, if for all security parameters $\secpar\in\NN$, parameters $\params \gets \setup(\secparam)$, key pairs $(\osk,\opk) \gets \kgen(\params)$, messages $m\in\bin^*$, and signatures $\vec{\sigma} \gets \sign(\params,\osk,m)$ it holds that
\[
  \iverify(\params,\opk,m,\sigma)=1\enspace.
\]
\end{definition}

We require that individually verifying signatures can be homomorphically aggregated by computing a random linear combination of them.
Such aggregated signatures should still \emph{strongly} verify with high probability over the choice of the random linear combination.

\begin{definition}[Probabilistic Homomorphism]
  Let $\hots$ be a one-time signature scheme over a ring $\ring$ with public key space $\modopk$ and signatures from $\modsig$.
  Let $\rho,\varepsilon \in \NN$ and $W\subseteq \ring$.
  $\hots$ is $(\rho,W,\varepsilon)$-probabilistically homomorphic, if 
  for all security parameters $\secpar\in\NN$, number of aggregated signatures $\ell\in[\rho]$, parameters $\params \gets \setup(\secparam)$, public keys $\opk^i \in \modopk$, messages $m\in\bin^*$ and signatures $\vec{\sigma}^i\in\modsig$ with $\iverify(\params,\opk^i,m,\vec{\sigma}^i)$ it holds that
  \[
    \Pr\mleft[
      w^1,\dots,w^\ell\gets W
      :
      \sverify(\params,\sum_{i=1}^{\ell}w^i\cdot\opk^i,m,\sum_{i=1}^{\ell}w^i\cdot\vec{\sigma}^i) = 1
    \mright]\geq 1-2^{-\varepsilon}\enspace.
  \]
\end{definition}

As with the vector commitments from the previous section, we additionally require that a further limited homomorphism still holds, even for maliciously \emph{aggregated} signatures.
For any two, even maliciously generated, signatures that \emph{strongly} verify under potentially maliciously generated public keys, their difference will still \emph{weakly} verify.

\begin{definition}[Robust Homomorphism]
  \label{def:malhomhots}
  Let $\hots$ be a key-homomorphic one-time signature scheme over a ring $\ring$ with public key space $\modopk$ and signatures from $\modsig$.
  $\hots$ is robustly homomorphic if for all security parameters $\secpar\in\NN$, public parameters $\params\gets\setup(\secparam)$, messages $m\in\bin^*$, (possibly malformed) public keys $\opk^0,\opk^1 \in \modopk$, and (possibly malformed) signatures $\vec{\sigma}^0,\vec{\sigma}^1\in\modsig$ with
  \[
    \sverify(\params,\opk^0, m,\vec{\sigma}^0)=1 \quad \text{and} \quad \sverify(\params,\opk^1, m,\vec{\sigma}^1)=1
  \]
  it holds that
  \[
    \wverify(\params,\opk^0-\opk^1, m,(\vec{\sigma}^0-\vec{\sigma}^1))=1.
  \]
\end{definition}

The following definition of a multi-user version of (one-time) existential unforgeability under rerandomized keys is taken directly from \cite{CCS:FleSimZha22}.

\begin{definition}[Multi-User Existential Unforgeability under Rerandomized Keys]
  A $(\rho,W,\varepsilon)$-homomorphically correct KOTS is $W'$-existentially unforgeable under rerandomized keys (EUF-RK), if for all security parameters $\secpar$, any $T=\poly\in\NN$ and all stateful PPT algorithms $\adv$ it holds that
  \[
    \Pr\mleft[
      \begin{aligned}
      \params \gets \setup(\secparam);\\
      \forall i \in [T-1]\ldotp (\osk_i,\opk_i) \gets \kgen(\params);\\
      (i^*,m^*,\sigma^*,w^*) \gets \adv^{\widetilde\sign(\cdot,\cdot)}(\params,\opk_0,\dots,\opk_{T-1});\\
%      \sigma \gets \sign(\params,\osk,m);\\
%      (m^*,\sigma^*,w_0^*,w_1^*) \gets \adv(\sigma)\\
      \end{aligned}:
      \begin{aligned}
      \wverify(\params,w^*\cdot\opk_{i^*},m^*,\sigma^*) = 1\\
      \land m^*\not\in Q_{i^*} \land \abs{Q_{i^*}}\leq 1 \land w^* \in W'
      \end{aligned}
    \mright]\leq \negl,
  \]
  where the oracle $\widetilde\sign(\cdot,\cdot)$ is defined as $\widetilde\sign(i,m) := \sign(\osk_i,m)$ and $Q_i$ denotes the set of messages for which a signing query with index $i$ has been made.
\end{definition}

\begin{figure}
\centering
\begin{pcvstack}[center,boxed]
\begin{pchstack}[center]
  \procedure{$\setup(\secparam)$}{
    \vec{a} \gets \ring_{q'}^\gamma\\
    \pcreturn \vec{a}
  }
  \pchspace
  \procedure{$\kgen(\params)$}{
    \vec{s}_0 \gets \ball_{\varphi}^\gamma\\
    \vec{s}_1 \gets \ball_{\varphi\cdot\alpha_H}^{\gamma}\\
    v_0 := \vec{a}^\intercal\cdot \vec{s}_0\\
    v_1 := \vec{a}^\intercal\cdot \vec{s}_1\\
    \pcreturn ((\vec{s}_0,\vec{s}_1)(v_0,v_1))
  }
  \pchspace
  \procedure{$\sign(\params,\osk,m)$}{
    \pcparse \osk \pcas (\vec{s}_0,\vec{s}_1)\\
    \vec{\sigma} := \vec{s}_0\cdot H(m)+\vec{s}_1\\
    \pcreturn \vec{\sigma}
  }
  \end{pchstack}
  \pchspace
  \begin{pchstack}
  \begin{pcvstack}
  \procedure{$\iverify(\params,\opk,m,\vec{\sigma})$}{
    \pcreturn \verify(\params,\opk,m,\vec{\sigma},2\varphi\alpha_H)
  }
  \pcvspace
  \procedure{$\sverify(\params,\opk,m,\vec{\sigma})$}{
    \pcreturn \verify(\params,\opk,m,\vec{\sigma},\beta_\sigma)
  }
  \pcvspace
  \procedure{$\wverify(\params,\opk,m,\vec{\sigma})$}{
    \pcreturn \verify(\params,\opk,m,\vec{\sigma},2\beta_\sigma)
  }
  \end{pcvstack}
  \pchspace
  \procedure{$\verify(\params,\opk,m,\vec{\sigma},\beta')$}{
    \pcparse \opk \pcas (v_0,v_1)\\
    \pcif \norm{\vec{\sigma}} > \beta'\\
    \quad \pcreturn 0\\
    \pcif \vec{a}^\intercal\cdot \vec{\sigma} \neq v_0\cdot H(m) + v_1\\
    \quad \pcreturn 0\\
    \pcreturn 1
  }
\end{pchstack}
\end{pcvstack}
\caption{%
Description of our key-homomorphic one-time signature scheme.
$H$ is a collision-resistant hash function mapping bit-strings to $\tern_{\alpha_H}$.
Our key space is $\modopk = \ring_{q'}^{2\gamma}$.
The signature space is $\modsig = \ring_{q'}^\gamma$.
}
\label{fig:otsconstruction}
\end{figure}

\autoref{fig:otsconstruction} shows the construction of the KOTS we will use.
The construction is almost identical to the construction from \cite{CCS:FleSimZha22} but differs from it in its choice of the ball from which the secret keys are chosen.
Specifically, the components of the secret keys are allowed to have a larger infinity norm.
This is beneficial, because the security proof partially relies on fact that the function mapping secret keys to public keys and signatures is highly compressing.
With a larger secret key-space the compression ratio increases, allowing us to reduce the size of other parameters, ultimately decreasing the size of the signatures.

\begin{theorem}\label{theo:kots}
Let $\secpar, \varepsilon, \alpha_w$, $\alpha_H$, $\delta$, $\varphi$, $\gamma$, $\rho$, $\beta_\sigma$, $n$, $q'$ be integers such that $q'$ is prime and $q' > 16\alpha_w\alpha_H\varphi$, $n$ is a power of two, $2^{2\secpar}\leq \tern_{\alpha_H} \leq 2^{2\secpar+\delta}$, $\beta_\sigma \geq 4\varphi\alpha_H\sqrt{\tfrac{1}{2}\alpha_w\rho(\varepsilon+1+\log_2n\gamma)\cdot\ln2}$, and $\gamma\geq((3\secpar+\delta)/n+\log_2q')\log^{-1}_2(\varphi+\tfrac{1}{2})$.
Let $H\colon\; \bin^* \to \tern_{\alpha_H}$ be a hash function.
Let $W' = \{w_0-w_1\mid w_0,w_1\in\tern_{\alpha_w} \land w_0\neq w_1\}$.
If the $\sis_{\ring,q',\gamma,2\beta_\sigma+4\alpha_w\alpha_H\varphi}$ problem is hard and $H$ is collision resistant, then the construction from~\autoref{fig:otsconstruction} is an individually correct, $(\rho,\tern_{\alpha_w},\varepsilon)$-probabilistically homomorphic, robustly homomorphic KOTS that is $W'$-multi-user existentially unforgeable under rerandomized keys.
\end{theorem}
\begin{proof}
The theorem follows from \autoref{lem:kots_ind_correct}, \autoref{lem:kots_correct}, \autoref{lem:kots_homomorphic}, and \autoref{lem:kots_sis}.
\qed
\end{proof}

The following four lemmas state that our construction satisfies the desired homomorphic properties and that it is unforgeable.
\begin{lemma}\label{lem:kots_ind_correct}
Let $\secpar, \varepsilon, \alpha_w$, $\alpha_H$, $\varphi$, $\gamma$, $\rho$, $\beta_\sigma,n,q'$ be positive integers, such that $n$ is a power of two, $q'$ is prime.
  Let $\ring_{q'}$ be the polynomial ring $\ZZ_{q'}[x]/\langle x^n+1\rangle$.
Let $H : \bin^* \to \tern_{\alpha_H}$ be a hash function.
  Then the construction from \autoref{fig:otsconstruction} is a individually correct one time signature scheme.
\end{lemma}
\begin{proof}
  Let $\params \gets \setup(\secparam)$, $(\osk,\opk) \gets \kgen(\params)$, $m\in\bin^*$ and $\vec{\sigma} \gets \sign(\params,\osk,m)$ be arbitrary.
  We first observe that the check on the \emph{value} of the signature goes through, as
  \begin{align*}
    \vec{a}^\intercal\vec{\sigma}
    ={}&\vec{a}^\intercal(\vec{s}_0\cdot H(m) + \vec{s}_1)\tag{Def. of $\sign$}\\
    ={}&\vec{a}^\intercal\vec{s}_0\cdot H(m) + \vec{a}^\intercal\vec{s}_1 \tag{Distributivity}\\
    ={}&v_0\cdot H(m) + v_1. \tag{Def. of $\kgen$}
  \end{align*}
  The signature also does not violate the norm bound, as
  \begin{align*}
    \norm{\vec{\sigma}}
    ={}&\norm{\vec{s}_0\cdot H(m) + \vec{s}_1} \tag{Def. of $\sign$}\\
    \leq{}&\norm{\vec{s}_0\cdot H(m)} + \norm{\vec{s}_1}\\
    \leq{}&\norm{\vec{s}_0}\cdot\norm{H(m)}_1 + \norm{\vec{s}_1} \tag{\autoref{lem:ternbound}}\\
    ={}&2\varphi\alpha_H \tag{Def. of $\kgen$}.
  \end{align*}
  The lemma thus follows.
  \qed
\end{proof}


\begin{lemma}\label{lem:kots_correct}
Let $\secpar, \varepsilon, \alpha_w$, $\alpha_H$, $\varphi$, $\gamma$, $\rho$, $\beta_\sigma$, $n$, $q'$ be positive integers, such that \[\beta_\sigma \geq 4\varphi\alpha_H\sqrt{\tfrac{1}{2}\alpha_w\rho(\varepsilon+1+\log_2n\gamma)\cdot\ln2}.\]
  Let $\ring_{q'}$ be the polynomial ring $\ZZ_{q'}[x]/\langle x^n+1\rangle$.
  Let $H\colon\; \bin^* \to \tern_{\alpha_H}$ be a hash function.
  Then the construction from \autoref{fig:otsconstruction} is a $(\rho,\tern_{\alpha_w},\varepsilon)$-probabilistically homomorphic one time signature scheme.
\end{lemma}

\begin{proof}
  Let $\ell\in[\rho]$, $m\in\bin^*$, and $\params\gets\setup(\secparam)$ and for $i\in[\ell]$ let $\opk^i=(v_0,v_1) \in \ring_{q'}^2$ and $\vec{\sigma}^i \in \ring_{q'}^\gamma$ be arbitrary such that for all $i\in[\ell]$, $\iverify(\params,\opk^i,m,\vec{\sigma}^i)=1$.
  
  We first note that even for arbitrary $w_1,\dots,w_\ell \in \tern_\alpha$ it holds that
  \begin{align*}
    \vec{a}^\intercal\cdot \sum_{i=1}^{\ell-1}w^i\cdot\vec{\sigma}^i
    ={}&\sum_{i=1}^{\ell}w^i\cdot\vec{a}^\intercal\vec{\sigma}^i \tag{Distributivity}\\
    ={}&\sum_{i=1}^{\ell}w^i\cdot(v^i_0\cdot H(m)+v^i_1) \tag{Def. of $\iverify$}\\
%    ={}&\sum_{i=1}^{\ell}w^iv^i_0\cdot H(m)+ \sum_{i=1}^{\ell}w^iv^i_1 \tag{Distributivity}\\
    ={}&\Bigl(\sum_{i=1}^{\ell}w^iv^i_0\Bigr)\cdot H(m)+ \Bigl(\sum_{i=1}^{\ell}w^iv^i_1\Bigr) \tag{Distributivity}.
  \end{align*}
  Therefore, it only remains to verify that the norm-check goes through with sufficient probability,
  i.e., that
  \[
    P\coloneqq \Pr\mleft[
      w^1,\dots,w^\ell\gets \tern_{\alpha_w}
      \colon
      \Bigl\Vert\sum_{i=1}^{\ell} w^i\cdot\vec{\sigma}^i\Bigr\Vert \leq \beta_\sigma\mright] \leq 2^{-\varepsilon}\enspace.
  \]
  For each individual $\vec{\sigma}^i$ it holds by the definition of $\iverify$ that $\norm{\vec{\sigma}^i} \leq 2\varphi\alpha_H$.
  What we need to show here is a bound for vectors from $\ring^\gamma$.
  Using \autoref{lem:normgrowth} with $\zeta = \tfrac{\beta_\sigma}{2\varphi\alpha_H}$ and taking a union bound over all $\gamma$ entries immediately gives
  \begin{align}\label{eq:appliedMcD}
   P \leq \gamma\cdot 2n\exp\Bigl(-\frac{\beta_\sigma^2}{8\varphi^2\alpha_H^2\alpha_w\ell}\Bigr) \leq 2\gamma n\exp\Bigl(-\frac{\beta_\sigma^2}{8\varphi^2\alpha_H^2\alpha_w\rho}\Bigr)\enspace.
  \end{align}
Our condition $\beta_\sigma \geq 4\varphi\alpha_H\sqrt{\tfrac{1}{2}\alpha_w\rho(\varepsilon+1+\log_2n\gamma)\cdot\ln2}$ is chosen as to be equivalent to
\[
 \frac{\beta_\sigma^2}{8\varphi^2\alpha^2_H\alpha_w\rho} \geq \ln\Bigl( 2\gamma n \cdot 2^\varepsilon \Bigr)\enspace.
\]
Plugging this into \autoref{eq:appliedMcD} directly gives $P \leq 2^{-\varepsilon}$.
%   To bound this probability, consider that the norm-bound is violated iff the absolute value of at least one of the $n\gamma$ coefficients in $\sum_{i=1}^{\ell} w^i\cdot\vec{\sigma}^i$ is greater than $\beta_\sigma$.
%   By a union bound it is thus sufficient to show that each individual coefficient violates the bound with probability at most $2^{-\varepsilon}/(n\gamma)$
%   
%   For each individual $\vec{\sigma}^i$ it holds by the definition of $\iverify$ that
%     $\norm{\vec{\sigma}^i} \leq 2\varphi\alpha_H$
%   Therefore, each coefficient is a sum of the form
%   \(
%     \sum_{j=1}^{\alpha_w\ell}b_j c_j
%   \)
%   where $\abs{c_j}\leq 2\varphi\alpha_H$ and $b_j$ is chosen uniformly from $\{-1,1\}$.
%   By linearity of expectation, the expected value of this sum is always zero and changing any summand can vary the sum by at most $4\varphi\alpha_H$. We can thus apply McDiarmid's inequality~\cite{McDiarmid89} and the lower bound on $\beta_\sigma$ from the lemma statement to obtain the following bound on the probability that each individual coefficient exceeds the norm bound $\beta_\sigma$:
%   \begin{align*}
%     \Pr\Bigl[\vec{b}\gets\{-1,1\}^{\alpha_w\ell} : \Bigl|\smashoperator{\sum_{j=1}^{\alpha_w\ell}}b_j c_j\Bigr| > \beta_\sigma\Bigr]
%     \leq{}& 2\cdot\exp\Bigl(-\frac{2\beta_\sigma^2}{\alpha_w\ell\cdot (4\varphi\alpha_H)^2}\Bigr)\\
%     \leq{}& 2\cdot\exp\Bigl(-\frac{2\beta_\sigma^2}{\alpha_w\rho\cdot (4\varphi\alpha_H)^2}\Bigr)\\
%     \leq{}& 2\cdot\exp\Bigl(-\frac{2(4\varphi\alpha_H)^2\cdot\tfrac{1}{2}\alpha_w\rho(\varepsilon+1+\log_2n\gamma)\cdot\ln2}{\alpha_w\rho\cdot (4\varphi\alpha_H)^2}\Bigr)\\
%     ={}& 2\cdot\exp(-(\varepsilon+1+\log_2n\gamma)\cdot\ln2)\\
%     ={}& 2^{-\varepsilon-\log_2n\gamma} = 2^{-\varepsilon}\cdot \frac{1}{n\gamma}
%   \end{align*}
It thus follows that with probability at least $1-2^\varepsilon$ the strong verification algorithm outputs $1$ as required.
\end{proof}


\begin{lemma}\label{lem:kots_homomorphic}
  Let $\secpar, \alpha_H$, $\varphi$, $\gamma$, $\beta_\sigma,q',n$ be positive integers.
  Let $\ring_{q'}$ be the polynomial ring $\ZZ_{q'}[x]/\langle x^n+1\rangle$.
  Let $H\colon\; \bin^* \to \tern_{\alpha_H}$ be a hash function.
  Then the construction from \autoref{fig:otsconstruction} is robustly homomorphic.
\end{lemma}

\begin{proof}
  Let $\params\gets\setup(\secparam)$, $m\in\bin^*$, $\opk^0=(v^0_0,v^0_1),\opk^1=(v^1_0,v^1_1)\in \ring^2_q$, and $\vec{\sigma}^0,\vec{\sigma}^1 \in \ring_{q'}^\gamma$ be arbitrary such that $\sverify(\params,\opk^0,m,\vec{\sigma}^0)=1$ and $\sverify(\params,\opk^1,m,\vec{\sigma}^1)=1$.
  
  By the definition of the strong verification algorithm, it holds that
  \begin{equation*}
     \norm{(\vec{\sigma}^0-\vec{\sigma}^1)}
    \leq \norm{\vec{\sigma}^0}+\norm{\vec{\sigma}^1}
    \leq 2\beta_\sigma\enspace,
  \end{equation*}
  thus the norm check goes through.
  It remains to verify that the second check also goes through.
  \begin{align*}
     \vec{a}^\intercal\cdot (\vec{\sigma}^0-\vec{\sigma}^1)
    ={}& \vec{a}^\intercal\cdot \vec{\sigma}^0- \vec{a}^\intercal\cdot\vec{\sigma}^1\\
    ={}& (v^0_0\cdot H(m) + v^0_1) - (v^1_0\cdot H(m) + v^1_1)\tag{Def of $\sverify$}\\
    ={}& (v^0_0-v^1_0)\cdot H(m) + (v^0_1-v^1_1)\enspace.
  \end{align*}
  Therefore, the lemma statement follows.
  \qed
\end{proof}


\begin{lemma}\label{lem:kots_sis}
  Let $n,\gamma,q',\alpha_H,\alpha_w,\delta,\secpar$ be positive integers with $q'$ prime and $n$ a power of two, such that $q' > 16 \alpha_w \alpha_H\varphi$, $\gamma\geq((3\secpar+\delta)/n+\log_2q)\log^{-1}_2(\varphi+\tfrac{1}{2})$, and $2^{2\secpar} \leq \abs{\tern_{\alpha_H}} \leq 2^{2\secpar + \delta}$.
  Let $H\colon\; \bin^* \to \tern_{\alpha_H}$ be a hash function.
  If the $\sis_{\ring,q',\gamma,2\beta_\sigma + 4\alpha_w\alpha_H\varphi}$ problem is hard and $H$ is collision resistant, then the construction from \autoref{fig:otsconstruction} is existentially unforgeable under rerandomized keys.
\end{lemma}

\begin{proof}
  Let $\adv$ be an arbitrary adversary against the multi-user $W'$-existentially unforgeability under rerandomized keys with success probability $\nu(\secpar)$.
  We construct an algorithm $\sisadvkots$ that solves $\sis_{\ring,q',\gamma,2\beta_\sigma + 4\alpha_w\alpha_H\varphi}$ as follows.
  Given $\vec{a}\in\ring_{q'}^\gamma$, $\sisadvkots$ honestly chooses secret keys $(\vec{s}^i_0,\vec{s}^i_1) \in \ball_{\varphi}^\gamma\times \ball_{\varphi\alpha_H}^\gamma$ uniformly at random for $i\in[T-1]$ and invokes $\adv$ on public keys $(v^i_0,v^i_1)$, with $v^i_b := \vec{a}^\intercal\cdot s^i_b$.
  Whenever $\adv$ sends a signing query $(i,m)$, $\sisadvkots$ will respond with the honestly computed signature $\vec{\sigma}:=\vec{s}^i_0\cdot H(m)+ \vec{s}^i_1$.
  Eventually $\adv$ outputs a candidate forgery $(i^*,m^*,\vec{\sigma}^*,w^*)$ and $\sisadvkots$ will compute a signature on the same message as $\vec{\sigma}' := w^*\cdot\vec{s}^{i^*}_0\cdot H(m^*)+ w^*\cdot\vec{s}^{i^*}_1$.
  It then outputs $\vec{\sigma}^*-\vec{\sigma}'$.
  
  To analyze the success probability of $\sisadvkots$ suppose that $\adv$ outputs a \emph{valid} forgery.
  I.e., at most a single query was asked for index $i^*$, said query was \emph{not} $m^*$, $w^*\in W'$ and $\wverify(\vec{a},(w^*v^{i^*}_0,w^*v^{i^*}_1),m^*,\allowbreak\vec{\sigma}^*)=1$.
  From this and the definition of $\vec{\sigma}'$ above it follows that
  \begin{align*}
       \vec{a}^\intercal\cdot(\vec{\sigma}^*-\vec{\sigma}') ={}& \vec{a}^\intercal\vec{\sigma}^*-\vec{a}^\intercal\vec{\sigma}'\\ 
    ={}& (w^*\cdot v^{i^*}_0 H(m) + w^*\cdot v^{i^*}_1) - \vec{a}^\intercal(w^*\cdot \vec{s}^{i^*}_0\cdot H(m^*)+ w^*\cdot\vec{s}^{i^*}_1)\\
    ={}& (w^*\cdot v^{i^*}_0 H(m) + w^*\cdot v^{i^*}_1) - (w^*\cdot\vec{a}^\intercal\vec{s}^{i^*}_0\cdot H(m^*)+ w^*\cdot\vec{a}^\intercal\vec{s}^{i^*}_1)\\
    ={}& (w^*\cdot v^{i^*}_0\cdot H(m) + w^*\cdot v^{i^*}_1) - (w^*\cdot v^{i^*}_0\cdot H(m) + w^*\cdot v^{i^*}_1) = 0.
  \end{align*}
  as required for a solution to the SIS problem.
  
  Next, to argue that $\norm{\vec{\sigma}^*-\vec{\sigma}'}\leq 2\beta_\sigma + 4\alpha_w\alpha_H\varphi$, note that the weak verification algorithm guarantees that $\norm{\vec{\sigma}^*} \leq 2\beta_\sigma$.
  Further, since $w^*\in W'$ there exist $w_0,w_1 \in \tern_{\alpha_w}$ such that $w^* = w_0-w_1$ and $\norm{w^*}_1 \leq \norm{w_0}_1 + \norm{w_1}_1 = 2\alpha_w$.
  We can thus bound the norm of $\vec{\sigma}'$ as
  \begin{align*}
    \norm{\vec{\sigma}'} ={}& \norm{w^*\cdot\vec{s}^{i^*}_0\cdot H(m^*)+ w^*\cdot\vec{s}^{i^*}_1}\tag{Def. of $\sign$}\\
    ={}&\norm{w^*\cdot\vec{s}^{i^*}_0\cdot H(m^*)}+ \norm{w^*\cdot\vec{s}^{i^*}_1}\tag{Triangle Inequality}\\
    ={}&\norm{w^*}_1\cdot\norm{H(m^*)}_1\cdot\norm{\vec{s}^{i^*}_0} + \norm{w^*}_1\cdot\norm{\vec{s}^{i^*}_1}\tag{\autoref{lem:ternbound}}\\
    ={}&4\alpha_w\alpha_H\varphi.\tag{$w^*\in W'$ and $H(m^*)\in\tern_{\alpha_H}$}
  \end{align*}
  It follows that $\norm{\vec{\sigma}^*-\vec{\sigma}'} \leq \norm{\vec{\sigma}^*}+\norm{\vec{\sigma}'} \leq 2\beta_\sigma + 4\alpha_w\alpha_H\varphi$ as required.

  Finally, we need to argue that $\vec{\sigma}^*-\vec{\sigma}'\neq 0$.
  This is the case iff $\vec{\sigma}^* \neq \vec{\sigma}'$.
  It thus suffices to bound the probability, that $\vec{\sigma}^*=\vec{\sigma}'$.

  To this end, we observe by \autoref{lem:keyhidden} that, since $\adv$ has learned at most a single signature under $(v_0^{i^*},v_1^{i^*})$, the corresponding $(\vec{s}_0^{i^*},\vec{s}_1^{i^*})$ remains information-theoretically hidden from $\adv$ among at least 2 possible secret keys.
  Once $\adv$ outputs a valid forgery $(i^*,m^*,\vec{\sigma}^*,w^*)$ the signing key used for the forgery becomes uniquely determined by \autoref{lem:nilssupportivechildsupport} as long as $H(m^*)\neq H(m)$ which is guaranteed with overwhelming probability by the collision resistance of $H$.
  It follows that $\sigma^* \neq \sigma'$ with probability at least $1/2 - \negl$.
  Therefore, the success probability of our reduction $\sisadvkots$ is $(1/2 - \negl) \nu(\secpar)$ and since the SIS problem is assumed to be hard, $\nu(\secpar)$ must therefore be negligible in $\secpar$.
  \qed
\end{proof}

\begin{lemma}\label{lem:keyhidden}
  Let $n,\gamma,q',\alpha_H, \delta, \varphi,\secpar$ be positive integers such that $\gamma\geq((3\secpar+\delta)/n+\log_2q)\log^{-1}_2(\varphi+\tfrac{1}{2})$ and $\abs{\tern_{\alpha_H}} \leq 2^{2\secpar + \delta}$, let $\ring=\ZZ[x]/(x^n+1)$. Then for any $\vec{a}\in\ring_{q'}^\gamma$ and uniformly chosen $(\vec{s}_0,\vec{s}_1)\in \ball_{\varphi}^\gamma \times \ball_{\varphi\alpha_H}^\gamma$ it holds with probability at least $1-2^{-\lambda}$ that for every $c\in \tern_{\alpha_H}$ there exists $(\vec{s}'_0,\vec{s}'_1)\in \ball_{\varphi}^\gamma \times \ball_{\varphi\alpha_H}^\gamma$such that $(\vec{s}'_0,\vec{s}'_1)\neq(\vec{s}_0,\vec{s}_1)$, $(\vec{a}^\intercal\cdot\vec{s}'_0,\vec{a}^\intercal\cdot\vec{s}'_1) = (\vec{a}^\intercal\cdot\vec{s}_0,\vec{a}^\intercal\cdot\vec{s}_1)$ and $\vec{s}'_0\cdot c + \vec{s}'_1 = \vec{s}_0\cdot c + \vec{s}_1$.
\end{lemma}
\begin{proof}
  We define a function $f_{\vec{a}, c}$ that maps any secret key $(\vec{s}_0, \vec{s}_1)$ to a pair of public key and signature defined as $((\vec{a}^\intercal\cdot\vec{s}_0,\vec{a}^\intercal\cdot\vec{s}_1), \vec{s}_0\cdot c + \vec{s}_1)$.
  We will show that the domain of this function is at least $2^{3\secpar + \delta}$ times larger than the range.
  The number of possible secret keys is $(2\varphi+1)^{n\gamma} \cdot (2\varphi\alpha_H+1)^{n\gamma}$.
  The number of possible signatures is at most $(4\varphi\alpha_H + 1)^{n\gamma}$.
  For fixed values $\vec{a}, c, \vec{s}_0\cdot c + \vec{s}_1$, we observe that once $\vec{a}^\intercal\cdot\vec{s}_0$ is fixed, the second component $\vec{a}^\intercal\cdot\vec{s}_1 = \vec{a}^\intercal \cdot ((\vec{s}_0\cdot c + \vec{s}_1) - \vec{s}_0 \cdot c)$ is uniquely determined.
  Thus for a fixed signature, there are at most $q'^n$ many possible public keys and therefore the size of the range of $f_{\vec{a}, c}$ is at most $(4 \varphi\alpha_H + 1)^{n\gamma} \cdot q'^n$.
  We observe that
  \begin{align*}\frac{(2\varphi+1)^{n\gamma} \cdot (2\varphi\alpha_H+1)^{n\gamma}}{(4 \varphi\alpha_H + 1)^{n\gamma} \cdot q'^n}
   \geq{}& \frac{(2\varphi+1)^{n\gamma} \cdot (2\varphi\alpha_H+1)^{n\gamma}}{(4 \varphi\alpha_H + 2)^{n\gamma} \cdot q'^n}\\
   ={}& \frac{(2\varphi+1)^{n\gamma}}{2^{n\gamma} \cdot q'^n}\\
   ={}& \left(\varphi+\tfrac{1}{2}\right)^{n\gamma} \cdot \frac{1}{q'^n}\\
   ={}& 2^{\log_2(\varphi+\frac{1}{2})\cdot{n\gamma} - n \log_2 q'}
  \end{align*}
  Using the condition on $\gamma$ from the lemma statement, one can see that
  \[
    \log_2(\varphi+\tfrac{1}{2})\cdot{n\gamma} - n \log_2 q'
    \geq n\Bigl(\frac{3\secpar+\delta}{n}+\log_2q\Bigr) - n \log_2 q' = 3\secpar+\delta
  \]
  and thus, as claimed the domain of $f_{\vec{a},c}$ is at least $2^{3\secpar + \delta}$ times larger than its range.
  
  Using Lemma 4.1 from~\cite{TCC:LyuMic08}, the probability, over a uniformly chosen secret key, that there exists $(\vec{s}'_0,\vec{s}'_1)\in \ball_{1}^\gamma \times \ball_{\beta_s}^\gamma$ such that $(\vec{s}'_0,\vec{s}'_1)\neq(\vec{s}_0,\vec{s}_1)$, $(\vec{a}^\intercal\cdot\vec{s}'_0,\vec{a}^\intercal\cdot\vec{s}'_1) = (\vec{a}^\intercal\cdot\vec{s}_0,\vec{a}^\intercal\cdot\vec{s}_1)$ and $\vec{s}'_0\cdot c + \vec{s}'_1 = \vec{s}_0\cdot c + \vec{s}_1$ is at least $1-2^{-3\secpar-\delta}$.
  By union bounding over all possible hash values $c \in \tern_{\alpha_H}$ and observing that $\tern_{\alpha_H} \leq 2^{2\secpar + \delta}$ the lemma statement follows.
  \qed
\end{proof}

\begin{lemma}\label{lem:nilssupportivechildsupport}
Let $n,\gamma,q',\alpha_H, \alpha_w$ be positive integers with $q'$ prime and $n$ a power of two such that $q' > 16 \alpha_w \alpha_H\varphi$ and let $\ring=\ZZ[x]/(x^n+1)$. Let $\vec{a}\in\ring_{q'}^\gamma$, $c_0,c_1 \in \tern_{\alpha_{H}}$, $w_0, w_1 \in \tern_{\alpha_w}$, and $\sigma_0,\sigma_1 \in \ring$ be arbitrary ring elements such that $c_0\neq c_1$ and $w_0 \neq w_1$.
Then there exists at most a single pair of vectors $(\vec{s}_0,\vec{s}_1)\in\ball^\gamma_{\varphi}\times \ball^\gamma_{\varphi\alpha_H}$, such that
    \[
    \vec{s}_0\cdot c_0 + \vec{s}_1 = \sigma_0 \quad\text{and}\quad (w_0 - w_1) \cdot (\vec{s}_0\cdot c_1 + \vec{s}_1) = \sigma_1\enspace.
    \]
\end{lemma}
 \begin{proof}
    Let $(\vec{s}_0, \vec{s}_1)\in\ball^\gamma_{\varphi}\times \ball^\gamma_{\varphi\alpha_H}$ and $(\vec{s}'_0, \vec{s}'_1)\in\ball^\gamma_{\varphi}\times \ball^\gamma_{\varphi\alpha_H}$ be two secret keys, such that 
    \begin{equation}
    \vec{s}_0\cdot c_0 + \vec{s}_1 = \vec{s}'_0\cdot c_0 + \vec{s}'_1 \implies (\vec{s}_0 - \vec{s}'_0)\cdot c_0 + (\vec{s}_1 - \vec{s}'_1) = 0 \label{hello}
    \end{equation}
    and 
    \begin{equation}
    \begin{aligned}
    &(w_0 - w_1) \cdot (\vec{s}_0\cdot c_1 + \vec{s}_1) = (w_0 - w_1) \cdot (\vec{s}'_0\cdot c_1 + \vec{s}'_1)\\ \implies& (w_0 - w_1)((\vec{s}_0 - \vec{s}'_0)\cdot c_1 + (\vec{s}_1 - \vec{s}'_1)) = 0 \label{kitty}
    \end{aligned}
    \end{equation}
    Equation~\ref{hello} implies that 
    \[
    (w_0 - w_1)((\vec{s}_0 - \vec{s}'_0)\cdot c_0 + (\vec{s}_1 - \vec{s}'_1)) = 0.
    \]
    Combined with Equation~\ref{kitty}, we get that in $\ring_{q'}$
    \begin{equation}
    (w_0 - w_1)(\vec{s}_0 - \vec{s}'_0) (c_0 - c_1)  = 0 \label{herekittykitty}
    \end{equation}
    Since $w_0,w_1\in\tern_{\alpha_w}$, $\vec{s}_0,\vec{s}'_0 \in \ball^\gamma_{\varphi}$, and $c_0,c_1\in\tern_{\alpha_{H}}$, it holds by \autoref{lem:ternbound} that
    \[
      \norm{(w_0 - w_1)(\vec{s}_0 - \vec{s}'_0) (c_0 - c_1)} \leq \norm{w_0 - w_1}_1\cdot\norm{c_0 - c_1}_1\cdot \norm{(\vec{s}_0 - \vec{s}'_0)} \leq 8\alpha_w\alpha_H\varphi \leq \tfrac{q'-1}{2}\enspace.
    \]
    Therefore \autoref{herekittykitty} also holds in $\ring$.
    Since $w_0 \neq w_1$, $c_0 \neq c_1$, and $\ring$ is an integral domain, it follows that $\vec{s}_0 = \vec{s}'_0$.
    By Equation~\ref{hello}, it must therefore hold that $(\vec{s}_0, \vec{s}_1) = (\vec{s}'_0, \vec{s}'_1)$.
    \qed
\end{proof}
