% !TEX root = ../main.tex

\section{Key-Homomorphic One-Time Signatures}\label{sec:otms}

In this section, we define and instantiate the notion of a key-homomorphic one-time signature scheme.
A one-time signature is a weak form of a digital signature scheme that is only guaranteed to be unforgeable if at most one signature is published under any given public key.
A one-time signature is called key-homomorphic, if the linear combination of separate signatures for the same message verifies under the linear combination of the corresponding public keys.

Our definitions again follow the definitions of \cite{CCS:FleSimZha22} very closely, but as in \autoref{sec:veccom} are incomparable.
As with the vector commitments, we have the stronger requirement that the homomorphism works for any individually verifying signature, not just honestly created ones.
But, this homomorphism is allowed to have a noticeable correctness error.

The construction presented in the latter part of this section derives from the construction of \cite{CCS:FleSimZha22} which was itself very similar to previous one-time signature schemes by Boneh and Kim~\cite{BonKim2020} and Lyubashevsky and Micciancio~\cite{TCC:LyuMic08}.

\begin{definition}[One-Time Signature]\label{def:hots}
  Let $\ring$ be a ring.
  A key-homomorphic one-time signature scheme (KOTS) over $\ring$ with public key length $\opklen$ and signature length $\siglen$ is defined by four PPT algorithms $\hots=(\setup,\kgen,\sign,\verify)$.
  \begin{description}
    \item[$\params\gets\setup(\secparam)$] The setup algorithm takes as input the security parameter and outputs public parameters.
    \item[$(\osk,\opk) \gets \kgen(\params)$] The key generation algorithm takes as input the public parameters and outputs a key pair with $\opk\in\ring_q^\opklen$.
    \item[$\vec{\sigma} \gets \sign(\params,\osk,m)$] The signing algorithm takes as input the public parameters, a one-time signing key, and a message and outputs a signature $\sigma\in\ring_q^\siglen$.
    \item[$b\gets \iverify(\params,\opk,m,\sigma)$] The individual verification algorithm takes as input the public parameters, a verification key, a message, and a candidate signature and outputs a bit indicating acceptance/rejection.
    \item[$b\gets \sverify(\params,\opk,m,\sigma)$] The strong verification algorithm has the same input and output domains as the individual verification algorithm.
    \item[$b\gets \wverify(\params,\opk,m,\sigma)$] The weak verification algorithm has the same input and output domains as the individual verification algorithm..
  \end{description}
\end{definition}
\begin{definition}[Individual Correctness]
Let $\hots$ be a one-time signature scheme.
$\hots$ is individually correct, if for all security parameters $\secpar\in\NN$, parameters $\params \gets \setup(\secparam)$, key pairs $(\osk,\opk) \gets \kgen(\params)$, messages $m\in\bin^*$, and signatures $\vec{\sigma} \gets \sign(\params,\osk,m)$ it holds that
\[
  \iverify(\params,\opk,m,\sigma)=1
\]
\end{definition}

We require that individually verifying signatures can be homomorphically aggregated by computing a random linear combination of them.
Such aggregated signatures should still \emph{strongly} verify with high probability over the choice of the random linear combination.

\begin{definition}[Probabilistic Homomorphism]
  Let $\hots$ be a one-time signature scheme with public key length $\opklen$ and signature length $\siglen$.
  Let $\rho,\varepsilon \in \NN$ and $W\subseteq \ring$.
  $\hots$ is $(\rho,W,\varepsilon)$-probabilistically homomorphic, if 
  for all security parameters $\secpar\in\NN$, number of aggregated signatures $\ell\in[\rho]$, parameters $\params \gets \setup(\secparam)$, public keys $\opk^i \in \ring_q^\opklen$, messages $m\in\bin^*$ and signatures $\vec{\sigma}^i\in\ring_q^{\siglen}$ with $\iverify(\params,\opk^i,m,\vec{\sigma}^i)$ it holds that
  \[
    \Pr\mleft[
      w^1,\dots,w^\ell\gets W
      :
      \sverify(\params,\sum_{i=1}^{\ell}w^i\cdot\opk^i,m,\sum_{i=1}^{\ell}w^i\cdot\vec{\sigma}^i) = 1
    \mright]\geq 1-2^{-\varepsilon}
  \]
\end{definition}

As with the vector commitments from the previous section, we additionally require that a further limited homomorphism still holds, even for maliciously \emph{aggregated} signatures.
For any two, even maliciously generated, signatures that \emph{strongly} verify under potentially maliciously generated public keys, their difference will still \emph{weakly} verify.

\begin{definition}[Robust Homomorphism]
  \label{def:malhomhots}
  Let $\hots$ be a one-time signature scheme with public key length $\opklen$ and signature length $\siglen$.
  $\hots$ is robustly homomorphic if for all security parameters $\secpar\in\NN$, public parameters $\params\gets\setup(\secparam)$, messages $m\in\bin^*$, (possibly malformed) public keys $\opk^0,\opk^1 \in \ring_q^\opklen$, and (possibly malformed) signatures $\vec{\sigma}^0,\vec{\sigma}^1\in\ring_q^\siglen$ with
  \[
    \sverify(\params,\opk^0, m,\vec{\sigma}^0)=1 \quad \text{and} \quad \sverify(\params,\opk^1, m,\vec{\sigma}^1)=1
  \]
  it holds that
  \[
    \wverify(\params,\opk^0-\opk^1, m,(\vec{\sigma}^0-\vec{\sigma}^1))=1.
  \]
\end{definition}
%
The following definition of a multi-user version of (one-time) existential unforgeability under rerandomized keys is taken directly from \cite{CCS:FleSimZha22}.

\begin{definition}[Multi-User Existential Unforgeability under Rerandomized Keys]
  A $(\rho,W,\varepsilon)$-homomorphically correct KOTS is $W'$-existentially unforgeable under rerandomized keys (EUF-RK), if for all security parameters $\secpar$, any $T=\poly\in\NN$ and all stateful PPT algorithms $\adv$ it holds that
  \[
    \begin{aligned}
    &\Pr\mleft[
      \begin{aligned}
      \params \gets \setup(\secparam);\\
      \forall i \in [T]\ldotp (\osk_i,\opk_i) \gets \kgen(\params);\\
      \begin{pmatrix}i^*,\\m^*,\\\sigma^*,\\w^*\end{pmatrix} \gets \adv^{\widetilde\sign(\cdot,\cdot)}(\params,\opk_1,...,\opk_{T});\\
%      \sigma \gets \sign(\params,\osk,m);\\
%      (m^*,\sigma^*,w_0^*,w_1^*) \gets \adv(\sigma)\\
      \end{aligned}:\!\!\!
      \begin{aligned}
      &\wverify\begin{pmatrix}\params,w^*\opk_{i^*},\\m^*,\sigma^*\end{pmatrix} = 1\\
      &\land m^*\not\in Q_{i^*}\\
      &\land \abs{Q_{i^*}}\leq 1 \land w^* \in W'
      \end{aligned}
    \mright]\\
    &\leq \negl,
    \end{aligned}
  \]
  where the oracle $\widetilde\sign(\cdot,\cdot)$ is defined as $\widetilde\sign(i,m) := \sign(\osk_i,m)$ and $Q_i$ denotes the set of messages for which a signing query with index $i$ has been made.
\end{definition}

\begin{figure}
\centering
\begin{pcvstack}[center,boxed]
\begin{pchstack}[center]
  \procedure{$\setup(\secparam)$}{\\[-3.5ex]
    \vec{a} \gets \ring_q^\gamma\\
    \pcreturn \vec{a}
  }
  \pchspace
  \procedure{$\kgen(\params)$}{\\[-3.5ex]
    \vec{s}_0 \gets \ball_{\varphi}^\gamma\\
    \vec{s}_1 \gets \ball_{\varphi\cdot\alpha_H}^{\gamma}\\
    v_0 := \vec{a}^\intercal\cdot \vec{s}_0\\
    v_1 := \vec{a}^\intercal\cdot \vec{s}_1\\
    \pcreturn ((\vec{s}_0,\vec{s}_1)(v_0,v_1))
  }
  \pchspace
  \procedure{$\sign(\params,\osk,m)$}{\\[-3.5ex]
    \pcparse \osk \pcas (\vec{s}_0,\vec{s}_1)\\
    \vec{\sigma} := \vec{s}_0\cdot H(m)+\vec{s}_1\\
    \pcreturn \vec{\sigma}
  }
  \end{pchstack}
  \pchspace
  \begin{pchstack}
  \begin{pcvstack}
  \procedure{$\iverify(\params,\opk,m,\vec{\sigma})$}{\\[-3.5ex]
    \pcreturn \verify(\params,\opk,m,\vec{\sigma},2\varphi\alpha_H)
  }
  \pcvspace
  \procedure{$\sverify(\params,\opk,m,\vec{\sigma})$}{\\[-3.5ex]
    \pcreturn \verify(\params,\opk,m,\vec{\sigma},\beta_\sigma)
  }
  \pcvspace
  \procedure{$\wverify(\params,\opk,m,\vec{\sigma})$}{\\[-3.5ex]
    \pcreturn \verify(\params,\opk,m,\vec{\sigma},2\beta_\sigma)
  }
  \end{pcvstack}
  \pchspace
  \procedure{$\verify(\params,\opk,m,\vec{\sigma},\beta')$}{\\[-3.5ex]
    \pcparse \opk \pcas (v_0,v_1)\\
    \pcif \norm{\vec{\sigma}} > \beta'\\
    \quad \pcreturn 0\\
    \pcif \vec{a}^\intercal\cdot \vec{\sigma} \neq v_0\cdot H(m) + v_1\\
    \quad \pcreturn 0\\
    \pcreturn 1
  }
\end{pchstack}
\end{pcvstack}
\caption{Description of the key-homomorphic one-time signature scheme. $H$ is a collision-resistant hash function mapping bit-strings to $\tern_{\alpha_H}$.}
\label{fig:otsconstruction}
\end{figure}

\autoref{fig:otsconstruction} shows the construction of a KOTS.
The construction is almost identical to the construction from \cite{CCS:FleSimZha22} but differs from it in its choice of the ball from which the secret keys are chosen.
Specifically, the components of the secret keys are allowed to have a larger infinity norm.
This is beneficial, because the security proof partially relies on fact that the function mapping secret keys to public keys and signatures is highly compressing.
With a larger secret key-space the compression ratio increases, allowing us to reduce the size of other parameters, ultimately decreasing the size of the signatures.

\begin{theorem}\label{theo:kots}
Let $\secpar, \varepsilon, \alpha_w$, $\alpha_H$, $\delta$, $\varphi$, $\gamma$, $\rho$, $\beta_\sigma$, $n$, $q$ be integers such that $q$ is prime and $q > 16\alpha_w\alpha_H\varphi$, $n$ is a power of two, $2^{2\secpar}\leq \tern_{\alpha_H} \leq 2^{2\secpar+\delta}$, $\beta_\sigma \geq 4\varphi\alpha_H\sqrt{\tfrac{1}{2}\alpha_w\rho(\varepsilon+1+\log_2n\gamma)\cdot\ln2}$, and $\gamma\geq((3\secpar+\delta)/n+\log_2q)\log^{-1}_2(\varphi+\tfrac{1}{2})$.
Let $H : \bin^* \to \tern_{\alpha_H}$ be a hash function.
Let $W' = \{w_0-w_1\mid w_0,w_1\in\tern_{\alpha_w} \land w_0\neq w_1\}$.
If the $\sis_{\ring,q,\gamma,2\beta_\sigma+4\alpha_w\alpha_H\varphi}$ problem is hard and $H$ is collision resistant, then the construction from~\autoref{fig:otsconstruction} is an individually correct, $(\rho,\tern_{\alpha_w},\varepsilon)$-probabilistically homomorphic, robustly homomorphic KOTS that is $W'$-multi-user existentially unforgeable under rerandomized keys.
\end{theorem}

Due to space contraints the proof can be found in the \autoref{app:proofs}.

