\documentclass{article}

%generated from my diploma thesis' preamble, adapted (removed most specific stuff)
%moved most command defs to seperate file (preamble-standard-abbrevs.tex)
%assuming these files rest in a directory found by kpathsea  (note that directories in kpathsea's search list starting with !! need to regenerate their cache _manually_ if you put files there)

%Uncomment to save space
% \usepackage{mathptmx}
% \usepackage[scaled=.90]{helvet}

%AMS packages

\usepackage{amscd,amsmath}
\usepackage{amssymb}
% \usepackage{amsthm} %incompatible with LNCS
\usepackage{amsfonts}
\usepackage{stmaryrd} %for more types of brackets; requires texlive-math-extra package in Debian, IIRC.
%stmaryrd changes standard font?
\usepackage{mathrsfs} %fancy script fonts

\usepackage{url}
\usepackage{latexsym}

\usepackage{color}


\usepackage[all]{xy}
% \usepackage{caption} %might cause trouble with llnc
%\usepackage[algo2e, ruled,vlined,noend]{algorithm2e}
\usepackage{algorithm}
\usepackage[noend]{algpseudocode} %noend gets rid of EndIf and the like, saving space
\usepackage{xspace}

%\usepackage{ucs}
%\usepackage[utf8x]{inputenc}
\usepackage[draft]{todonotes} 


\usepackage{paralist} %allows adjustment to enumerations and the like, contained in texlive-extra in Debian
\usepackage{afterpage} %for figure placement - doesn't work too well

%graphic(x) packages
\usepackage{graphicx}
%\usepackage{epsfig} %remove?
%\usepackage{color}

%\graphicspath{{figures/}} %remove?

\usepackage[notref,notcite]{showkeys} %Show labels

\numberwithin{figure}{section}
\numberwithin{table}{section}



% %automatic typos corrections. Those are actually used VERY often, because they cause my editors auto-completion to trigger (which wouldn't trigger at all without those).

\newcommand{\epslion}{\epsilon}
\newcommand{\lamda}{\lambda}
\newcommand{\Lamda}{\Lambda}

%Font selectors


\newcommand{\AlgorithmFont}[1]{\ensuremath{\mathtt{#1}}}
\newcommand*{\MatrixFont}[1]{\ensuremath{\mathbf{#1}}}
\newcommand*{\SpaceFont}[1]{\ensuremath{\mathsf{#1}}} %font for spaces like rings, vector spaces and the like	
\newcommand*{\VarietyFont}[1]{\ensuremath{\mathcal{#1}}} %font for varieties/algebraic sets
\newcommand*{\PolyFont}[1]{\ensuremath{\mathbf{#1}}} %font for polynomials


%complexity classes

\newcommand*{\POLY}{{\mathcal{P}}}
\newcommand*{\NPOLY}{{\mathcal{NP}}}

%probability

\DeclareMathOperator*{\PROB}{\mathbf{Pr}} %has limits!
\DeclareMathOperator*{\EXPECT}{\mathbf{Exp}} %has limits
%need to condition this one on style (language)


\newcommand*{\ScProd}[2]{\ensuremath{\left\langle{#1}{\,,\,}{#2}\right\rangle}}
\newcommand*{\negl}{\textnormal{negl}}
\newcommand*{\poly}{\textnormal{poly}}
\newcommand{\secpar}{\ensuremath{\lambda}} %security parameter
\DeclareMathOperator{\Gen}{\textnormal{\bf Gen}}
\DeclareMathOperator{\Dec}{\textnormal{\bf Dec}}
\DeclareMathOperator{\Enc}{\textnormal{\bf Enc}}
\DeclareMathOperator{\Image}{\mathrm{Im}} %Image (of a usually linear map)
\DeclareMathOperator{\LeadMon}{\mathrm{LM}}
\DeclareMathOperator{\LeadCoeff}{\mathrm{LC}}
\DeclareMathOperator{\LeadTerm}{\mathrm{LT}}
\DeclareMathOperator{\lcm}{\mathrm{lcm}}
\newcommand*{\Messages}{\mathcal{M}}
\newcommand*{\Ciphers}{\mathcal{C}}
\newcommand*{\Grobner}{Gr\"obner\xspace}
\newcommand*{\Keys}{\mathcal{K}}
\newcommand*{\Bit}{\left\{0,1\right\}}
\newcommand*{\OLandau}{O} %Big-O
\newcommand*{\VarOLandau}{\widetilde{O}} %Big-O up to polylog-factors
\newcommand*{\wLandau}{\omega} %grows larger than (i.e. f=\wLandau(g) :<=> lim f/g = +\infty , assuming g>0)
\newcommand*{\concat}{\mid\mid}
\newcommand*{\oLandau}{o} %technically, a small omicron
\DeclareMathOperator{\frob}{frob}

\newcommand*{\getsrandomly}%{\stackrel{\$}{\gets}}   --- too high, makes linespread ugly
{\gets_\$}


\newcommand*{\tensor}{{\otimes}} %tensor product

\newcommand*{\IR}{{\mathbb{R}}} %real numbers
\newcommand*{\IZ}{{\mathbb{Z}}} %set of integers
\newcommand*{\IC}{{\mathbb{C}}} %complex numbers
\newcommand*{\IN}{{\mathbb{N}}} %positive integers
\newcommand*{\IQ}{{\mathbb{Q}}} %rationals
\newcommand*{\Field}{{\mathbb{F}}} %general field, typically finite
\newcommand*{\ddiff} {{\mathrm{d}}} %de-Rham differential etc. I like to have a different font for the d.


\providecommand*{\abs}[1]{{\lvert{#1}\rvert}} %degree
\providecommand*{\norm}[1]{{\lVert{#1}\rVert}}
\newcommand*{\dunion}{{\dot\cup}} %disjoint union of sets
\newcommand*{\dissum}{{\mathbin\sqcup}} %disconnected sum-union
\newcommand*{\VarCHI}{{\mathcal{X}}} %I use \mathcal{X} to replace _capital_ \chi and continue with \Phi, \Psi... I think \mathcal{X} looks better than lowercase \chi (Greek chi) here, because capital \chi would just be X.
\newcommand*{\subspace}{\subset}

\newcommand*{\lact}{\mathbin{\rightharpoonup}}
\newcommand*{\ract}{\mathbin{\leftharpoonup}}


\DeclareMathOperator{\Volume}{Vol} %volume
\DeclareMathOperator{\ev}{ev} %evaluation maps
\DeclareMathOperator{\len}{len}
%\DeclareMathOperator{\Interior}{Int} %interior
\newcommand*{\Interior}[1]{{\mathring{#1}}}
%\DeclareMathOperator{\deg}{deg} %degree - already defined

\newcommand{\dominates}{\ensuremath{\gg}}
\newcommand{\isdominated}{\ensuremath{\ll}}
\newcommand{\divides}{\ensuremath{\mid}}


%\renewcommand{\labelitemii}{{$\cdot$}} %symbol used for 2nd level of nested lists. Standard is -, which is can be confused with a mathematical -.
%should be obsolete with paralist's capabilities

%typeset:
%various subscripts

\DeclareMathOperator*{\LEN}{len}



%To be continued

%theoremstyles:

%\theoremstyle{plain}

%Use \swapnumbers to have theoremnumbers left of the Theorem

% \newtheorem{thm}{Theorem}[section]
% \newtheorem{prop}[thm]{Proposition}
% \newtheorem{proposition}[thm]{Proposition} %hope this works for TOC; might need find&replace otherwise...
% %doesn't enter TOC anyway yet... should it?
% \newtheorem{corollary}[thm]{Corollary}
% 
% \theoremstyle{definition}
% \newtheorem{definition}[thm]{Definition}
% \newtheorem{remark}[thm]{Remark}
% \newtheorem{lemma}[thm]{Lemma}

% \setcaptionmargin{2ex} %There are lots of longer captions surrounded by text. This helps to distinguish the captions from the surrounding text.



%alignment sign subscripts

%\newcommand*{\AlignGlue}{\textnormal{align}}

%paragraph indentation style. I don't like LaTeX's default behaviour

\setlength{\parindent}{0pt} %maybe actually increase?
\setlength{\parskip}{1.1ex plus 0.2ex minus 0.2ex}

%NOTE: might conflict with algorithm typesetting!


\interfootnotelinepenalty=10000 %splitting footnotes across pages is far worse than over/underfull pages

\usepackage{hyperref} %must be LAST preamble command, or so they say.

%automatic typos corrections. Those are actually used VERY often, because they cause my editors auto-completion to trigger (which wouldn't trigger at all without those).

\newcommand{\epslion}{\epsilon}
\newcommand{\lamda}{\lambda}
\newcommand{\Lamda}{\Lambda}

%Font selectors


\newcommand{\AlgorithmFont}[1]{\ensuremath{\mathtt{#1}}}
\newcommand*{\MatrixFont}[1]{\ensuremath{\mathbf{#1}}}
\newcommand*{\SpaceFont}[1]{\ensuremath{\mathsf{#1}}} %font for spaces like rings, vector spaces and the like	
\newcommand*{\VarietyFont}[1]{\ensuremath{\mathcal{#1}}} %font for varieties/algebraic sets
\newcommand*{\PolyFont}[1]{\ensuremath{\mathbf{#1}}} %font for polynomials


%complexity classes

\newcommand*{\POLY}{{\mathcal{P}}}
\newcommand*{\NPOLY}{{\mathcal{NP}}}

%probability

\DeclareMathOperator*{\PROB}{\mathbf{Pr}} %has limits!
\DeclareMathOperator*{\EXPECT}{\mathbf{Exp}} %has limits
%need to condition this one on style (language)


\newcommand*{\ScProd}[2]{\ensuremath{\left\langle{#1}{\,,\,}{#2}\right\rangle}}
\newcommand*{\negl}{\textnormal{negl}}
\newcommand*{\poly}{\textnormal{poly}}
\newcommand{\secpar}{\ensuremath{\lambda}} %security parameter
\DeclareMathOperator{\Gen}{\textnormal{\bf Gen}}
\DeclareMathOperator{\Dec}{\textnormal{\bf Dec}}
\DeclareMathOperator{\Enc}{\textnormal{\bf Enc}}
\DeclareMathOperator{\Image}{\mathrm{Im}} %Image (of a usually linear map)
\DeclareMathOperator{\LeadMon}{\mathrm{LM}}
\DeclareMathOperator{\LeadCoeff}{\mathrm{LC}}
\DeclareMathOperator{\LeadTerm}{\mathrm{LT}}
\DeclareMathOperator{\lcm}{\mathrm{lcm}}
\newcommand*{\Messages}{\mathcal{M}}
\newcommand*{\Ciphers}{\mathcal{C}}
\newcommand*{\Grobner}{Gr\"obner\xspace}
\newcommand*{\Keys}{\mathcal{K}}
\newcommand*{\Bit}{\left\{0,1\right\}}
\newcommand*{\OLandau}{O} %Big-O
\newcommand*{\VarOLandau}{\widetilde{O}} %Big-O up to polylog-factors
\newcommand*{\wLandau}{\omega} %grows larger than (i.e. f=\wLandau(g) :<=> lim f/g = +\infty , assuming g>0)
\newcommand*{\concat}{\mid\mid}
\newcommand*{\oLandau}{o} %technically, a small omicron
\DeclareMathOperator{\frob}{frob}

\newcommand*{\getsrandomly}%{\stackrel{\$}{\gets}}   --- too high, makes linespread ugly
{\gets_\$}


\newcommand*{\tensor}{{\otimes}} %tensor product

\newcommand*{\IR}{{\mathbb{R}}} %real numbers
\newcommand*{\IZ}{{\mathbb{Z}}} %set of integers
\newcommand*{\IC}{{\mathbb{C}}} %complex numbers
\newcommand*{\IN}{{\mathbb{N}}} %positive integers
\newcommand*{\IQ}{{\mathbb{Q}}} %rationals
\newcommand*{\Field}{{\mathbb{F}}} %general field, typically finite
\newcommand*{\ddiff} {{\mathrm{d}}} %de-Rham differential etc. I like to have a different font for the d.


\providecommand*{\abs}[1]{{\lvert{#1}\rvert}} %degree
\providecommand*{\norm}[1]{{\lVert{#1}\rVert}}
\newcommand*{\dunion}{{\dot\cup}} %disjoint union of sets
\newcommand*{\dissum}{{\mathbin\sqcup}} %disconnected sum-union
\newcommand*{\VarCHI}{{\mathcal{X}}} %I use \mathcal{X} to replace _capital_ \chi and continue with \Phi, \Psi... I think \mathcal{X} looks better than lowercase \chi (Greek chi) here, because capital \chi would just be X.
\newcommand*{\subspace}{\subset}

\newcommand*{\lact}{\mathbin{\rightharpoonup}}
\newcommand*{\ract}{\mathbin{\leftharpoonup}}


\DeclareMathOperator{\Volume}{Vol} %volume
\DeclareMathOperator{\ev}{ev} %evaluation maps
\DeclareMathOperator{\len}{len}
%\DeclareMathOperator{\Interior}{Int} %interior
\newcommand*{\Interior}[1]{{\mathring{#1}}}
%\DeclareMathOperator{\deg}{deg} %degree - already defined

\newcommand{\dominates}{\ensuremath{\gg}}
\newcommand{\isdominated}{\ensuremath{\ll}}
\newcommand{\divides}{\ensuremath{\mid}}


%\renewcommand{\labelitemii}{{$\cdot$}} %symbol used for 2nd level of nested lists. Standard is -, which is can be confused with a mathematical -.
%should be obsolete with paralist's capabilities

%typeset:
%various subscripts

\DeclareMathOperator*{\LEN}{len}

\usepackage{amsthm}
\newtheorem{theorem}{Theorem}

\newcommand{\sig}{\mathop{\textrm{sig}}}
\renewcommand{\PROB}{\mathbb{P}}
\renewcommand{\EXPECT}{\mathbb{E}}
\newcommand{\CHI}{\mathcal{X}}

\begin{document}
This document is just an \emph{internal} draft aimed at ourselves; the final proof version that will go eventually into the paper depends on some details earlier sections and will probably look nothing like this. Note that the notation does not fully match Squirrel.

For aggregration of $\ell$ valid signatures $\sig_0,\ldots, \sig_{\ell-1}$ for message $m$ at timeslot $t$ under public keys $\mathcal{P}$, Squirrel chooses a list of randomizers as
\[
\omega_0,\ldots, \omega_{\ell-1} := H(t, m, \mathcal{P})
\]
where each $\omega_i\in R$ is a ternary polynomial with exactly $\alpha$ non-zero coefficients.

The aggregated signature is then essentially $\sum_i \omega_i \cdot \sig_i$. Here, each signature $\sig_i = (\sigma_i, d_i)$ consist of $d_i$, a vector of $2\tau$ elements $p_{i,j}$ for the tree-based vector commitment from $R^{\mu}$ each and 1 element $\sigma_i$ for the OTS from $R^{\mu'}$.
%For this level of the description, we can treat each signature simply as the (concatenated) vector of these.
In Squirrel, the number $\mu$ of small element given per tree label is $\mu = \lceil \log_2 q_{HVC}\rceil$ due to binarization, but in Chipmunk, $\mu$ will be smaller.

We can get better parameters by letting the aggregator randomize the hash function as follows: select $r\in D$ from some sufficiently large domain $D$ and set the randomizer as
\[
\omega_0, \ldots, \omega_{\ell-1} := H(t, m, r, \mathcal{P})
\]
Then check whether each element in the resulting $\sum_i \omega_i \cdot s_i$ is small enough. If not, repeat with a different $r$. If we find a good $r$, we output $\sum_i \omega_i \cdot s_i$ together with $r$ as the signature. Verification is done in the obvious way.

For simplicity of analysis\footnote{If $\abs{D}$ is small, we would rather choose $r$'s in oder. Verification would not (and cannot) check the the $r$ included in a signature is the smallest possible}, assume that $r$ is chosen unformly at random at each attempt.

If I understood Mark's comment about the results of the forking lemma analysis correctly, choosing a large $D$ does not really lose much in terms of security in the proofs, so let's choose $\abs{D} >2^{\secpar+1}$ for simplicity\footnote{This means that for an honest aggregator, the adversary has absolutely no control of the values of $H(t, m, r, \mathcal{P})$ for at least at constant fraction $\tfrac12$ of choices of $r$. Note that much smaller $D$ work as well for the overall argument, but then one has to actual do some computations to bound the adversarial influence.}.

What we need to make sure is that for any valid (but potentially adversarially chosen) list of signatures, message, public key and timeslot, the aggregator will find a good $r$ in reasonable time. Here, $r$ being good means that for each\footnote{We drop the index $j$ running over the $2\tau$ components} of the $2\tau$ components, we have
\[
 \norm{ \sum\nolimits_i  \omega_i \cdot p_i}_{\infty} \leq \beta^{\textrm{agg}}_{\infty}
\]
and similarly for $\sigma_i$ (with a potentially different bound).

Note that $\norm{p_i}_{\infty} \leq 1$ for Squirrel (bigger for Chipmunk).

For the improved analysis, let us switch two a two-norm-bound, i.e.\ we ask
\[
  \norm{\sum\nolimits_i \omega_i \cdot p_i}_{2} \leq \beta^{\textrm{agg}}
\]
and similarly for $\sigma_i$
with \[
\beta^{\textrm{agg}} = \sqrt{n \mu} \beta^{\textrm{agg}}_{\infty}
\] and \[\norm{p_i} \leq \sqrt{n \mu} =: \beta.\] The factor $n=\dim R$ comes from the dimension of the ring. Note that this also means relaxing the verification algorithm (thereby reducing some ``real'' security that the proofs never accounted for anyway) to consider the 2-norm.

Aggregation works if
\[
    \PROB[\norm{\sum\nolimits_i \omega_i \cdot p_i}_{2} > \beta^{\textrm{agg}}] \leq \frac{1}{4\tau +2}
\]
and a similar condition for $\sigma_i$.
Here, the probability is over the choice of the $\omega_i$'s (i.e.\ the output of the hash function $H$, modelled as RO) for given $r$ and the right-hand-side is chosen such that after union-bounding all $2\tau + 1$ conditions, we still have an overall success probability of $\tfrac{1}{2}$ for each choice of $r$.

We can bound this using the (one-sided)
\begin{theorem}[McDiarmid inequality]
Let $X_1,\ldots, X_n$ be independent random variables (not neccessarily uniform), where $X_i$ takes values in $\CHI_i$. Let $f$ be a function
$f\colon \CHI_1\times\dots\times \CHI_n \to \IR$ with the property that
\[
 \abs{f(x_1,\ldots, x_i,\ldots, x_n) - f(x_1,\ldots, x'_i,\ldots, x_n)} \leq c
\]
for some $c>0$ (i.e.\ changing \emph{one} input changes the value at most by $c$).

Then for any $t>0$, we have
\[
  \PROB[f(X_1,\ldots,X_n) - \EXPECT[f(X_1,\ldots,X_n)] \geq t] \leq \mathop{exp}\bigl(-\frac{2t^2}{nc^2}\bigr)
\]
\end{theorem}

We apply this to the function
\[
 f:\{-1,+1\}^{\alpha\cdot \ell} \to \IR, b_1,\ldots,b_{\alpha\ell} \mapsto \norm{\sum_i^{\ell} \omega_i \cdot p_i}
\]
where the $b_i$ denote the signs of the $\pm 1$ coefficients inside the $\omega_i$ (Note that this means that $f$ itself depends on the choice of positions of the $\pm 1$'s and the $p_i$ -- this is fine and corresponds to assuming the worst case here).

For this $f$, the conditions of McDiarmid's inequality are satisfied with $c = 2\max{\norm{\beta_i}} \leq \beta$.

So we get for any $t$
\[
 \PROB\bigl[\norm{\textstyle\sum\omega_i p_i} \geq t + \EXPECT[\norm{\textstyle\sum \omega_i p_i}] \bigr]  \leq \mathop{exp}\bigl(
 -\frac{2t^2}{4\alpha \ell \beta^2}
 \bigr)
\]
Furthermore, we have
\[
 \EXPECT[\norm{\textstyle\sum \omega_i p_i}] \leq \sqrt{\EXPECT[\norm{\textstyle\sum \omega_i p_i}^2]} \leq \sqrt{\alpha\ell}\beta
\]
by expanding the scalar product in terms of $b_j$'s and using linearity of expectation.
Combining these, we get
\[
 \PROB\bigl[\norm{\textstyle\sum\omega_i p_i} \geq t + \sqrt{\ell\alpha}\beta \bigr]  \leq \mathop{exp}\bigl(
 -\frac{t^2}{2\alpha \ell \beta^2}
 \bigr)
\]
Setting $t = \vartheta\sqrt{\ell\alpha}\beta$:
\[
 \PROB\bigl[\norm{\textstyle\sum\omega_i p_i} \geq (\vartheta + 1) \sqrt{\ell\alpha}\beta \bigr]  \leq \mathop{exp}\bigl(
 -\frac{\vartheta^2}{2}
 \bigr)
\]

This means that we need to set $\vartheta$ as
\[
 \vartheta = 2\sqrt{\mathop{ln}(4\tau+2)}
\]
and get a bound
\[
 \beta^{\textrm{agg}} = \beta \cdot (1+\vartheta)\cdot \sqrt{\alpha} \sqrt{\rho}
\]
where $\rho$ is the maximum number of aggregated signatures.

For the Squirrel parameters $\rho = 4096$, $\tau = 26$, $\alpha = 20$, we get $\vartheta \approx 4.3$ and ratio $\frac{\beta^{\textrm{agg}}}{\beta} \approx 1500$, compared to Squirrels 4096.

NOTE: I did not analyze the $\sigma_i$, but neither does Squirrel (?)
\end{document}
