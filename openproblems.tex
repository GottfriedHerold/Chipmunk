\documentclass{article}

%generated from my diploma thesis' preamble, adapted (removed most specific stuff)
%moved most command defs to seperate file (preamble-standard-abbrevs.tex)
%assuming these files rest in a directory found by kpathsea  (note that directories in kpathsea's search list starting with !! need to regenerate their cache _manually_ if you put files there)

%Uncomment to save space
% \usepackage{mathptmx}
% \usepackage[scaled=.90]{helvet}

%AMS packages

\usepackage{amscd,amsmath}
\usepackage{amssymb}
% \usepackage{amsthm} %incompatible with LNCS
\usepackage{amsfonts}
\usepackage{stmaryrd} %for more types of brackets; requires texlive-math-extra package in Debian, IIRC.
%stmaryrd changes standard font?
\usepackage{mathrsfs} %fancy script fonts

\usepackage{url}
\usepackage{latexsym}

\usepackage{color}


\usepackage[all]{xy}
% \usepackage{caption} %might cause trouble with llnc
%\usepackage[algo2e, ruled,vlined,noend]{algorithm2e}
\usepackage{algorithm}
\usepackage[noend]{algpseudocode} %noend gets rid of EndIf and the like, saving space
\usepackage{xspace}

%\usepackage{ucs}
%\usepackage[utf8x]{inputenc}
\usepackage[draft]{todonotes} 


\usepackage{paralist} %allows adjustment to enumerations and the like, contained in texlive-extra in Debian
\usepackage{afterpage} %for figure placement - doesn't work too well

%graphic(x) packages
\usepackage{graphicx}
%\usepackage{epsfig} %remove?
%\usepackage{color}

%\graphicspath{{figures/}} %remove?

\usepackage[notref,notcite]{showkeys} %Show labels

\numberwithin{figure}{section}
\numberwithin{table}{section}



% %automatic typos corrections. Those are actually used VERY often, because they cause my editors auto-completion to trigger (which wouldn't trigger at all without those).

\newcommand{\epslion}{\epsilon}
\newcommand{\lamda}{\lambda}
\newcommand{\Lamda}{\Lambda}

%Font selectors


\newcommand{\AlgorithmFont}[1]{\ensuremath{\mathtt{#1}}}
\newcommand*{\MatrixFont}[1]{\ensuremath{\mathbf{#1}}}
\newcommand*{\SpaceFont}[1]{\ensuremath{\mathsf{#1}}} %font for spaces like rings, vector spaces and the like	
\newcommand*{\VarietyFont}[1]{\ensuremath{\mathcal{#1}}} %font for varieties/algebraic sets
\newcommand*{\PolyFont}[1]{\ensuremath{\mathbf{#1}}} %font for polynomials


%complexity classes

\newcommand*{\POLY}{{\mathcal{P}}}
\newcommand*{\NPOLY}{{\mathcal{NP}}}

%probability

\DeclareMathOperator*{\PROB}{\mathbf{Pr}} %has limits!
\DeclareMathOperator*{\EXPECT}{\mathbf{Exp}} %has limits
%need to condition this one on style (language)


\newcommand*{\ScProd}[2]{\ensuremath{\left\langle{#1}{\,,\,}{#2}\right\rangle}}
\newcommand*{\negl}{\textnormal{negl}}
\newcommand*{\poly}{\textnormal{poly}}
\newcommand{\secpar}{\ensuremath{\lambda}} %security parameter
\DeclareMathOperator{\Gen}{\textnormal{\bf Gen}}
\DeclareMathOperator{\Dec}{\textnormal{\bf Dec}}
\DeclareMathOperator{\Enc}{\textnormal{\bf Enc}}
\DeclareMathOperator{\Image}{\mathrm{Im}} %Image (of a usually linear map)
\DeclareMathOperator{\LeadMon}{\mathrm{LM}}
\DeclareMathOperator{\LeadCoeff}{\mathrm{LC}}
\DeclareMathOperator{\LeadTerm}{\mathrm{LT}}
\DeclareMathOperator{\lcm}{\mathrm{lcm}}
\newcommand*{\Messages}{\mathcal{M}}
\newcommand*{\Ciphers}{\mathcal{C}}
\newcommand*{\Grobner}{Gr\"obner\xspace}
\newcommand*{\Keys}{\mathcal{K}}
\newcommand*{\Bit}{\left\{0,1\right\}}
\newcommand*{\OLandau}{O} %Big-O
\newcommand*{\VarOLandau}{\widetilde{O}} %Big-O up to polylog-factors
\newcommand*{\wLandau}{\omega} %grows larger than (i.e. f=\wLandau(g) :<=> lim f/g = +\infty , assuming g>0)
\newcommand*{\concat}{\mid\mid}
\newcommand*{\oLandau}{o} %technically, a small omicron
\DeclareMathOperator{\frob}{frob}

\newcommand*{\getsrandomly}%{\stackrel{\$}{\gets}}   --- too high, makes linespread ugly
{\gets_\$}


\newcommand*{\tensor}{{\otimes}} %tensor product

\newcommand*{\IR}{{\mathbb{R}}} %real numbers
\newcommand*{\IZ}{{\mathbb{Z}}} %set of integers
\newcommand*{\IC}{{\mathbb{C}}} %complex numbers
\newcommand*{\IN}{{\mathbb{N}}} %positive integers
\newcommand*{\IQ}{{\mathbb{Q}}} %rationals
\newcommand*{\Field}{{\mathbb{F}}} %general field, typically finite
\newcommand*{\ddiff} {{\mathrm{d}}} %de-Rham differential etc. I like to have a different font for the d.


\providecommand*{\abs}[1]{{\lvert{#1}\rvert}} %degree
\providecommand*{\norm}[1]{{\lVert{#1}\rVert}}
\newcommand*{\dunion}{{\dot\cup}} %disjoint union of sets
\newcommand*{\dissum}{{\mathbin\sqcup}} %disconnected sum-union
\newcommand*{\VarCHI}{{\mathcal{X}}} %I use \mathcal{X} to replace _capital_ \chi and continue with \Phi, \Psi... I think \mathcal{X} looks better than lowercase \chi (Greek chi) here, because capital \chi would just be X.
\newcommand*{\subspace}{\subset}

\newcommand*{\lact}{\mathbin{\rightharpoonup}}
\newcommand*{\ract}{\mathbin{\leftharpoonup}}


\DeclareMathOperator{\Volume}{Vol} %volume
\DeclareMathOperator{\ev}{ev} %evaluation maps
\DeclareMathOperator{\len}{len}
%\DeclareMathOperator{\Interior}{Int} %interior
\newcommand*{\Interior}[1]{{\mathring{#1}}}
%\DeclareMathOperator{\deg}{deg} %degree - already defined

\newcommand{\dominates}{\ensuremath{\gg}}
\newcommand{\isdominated}{\ensuremath{\ll}}
\newcommand{\divides}{\ensuremath{\mid}}


%\renewcommand{\labelitemii}{{$\cdot$}} %symbol used for 2nd level of nested lists. Standard is -, which is can be confused with a mathematical -.
%should be obsolete with paralist's capabilities

%typeset:
%various subscripts

\DeclareMathOperator*{\LEN}{len}



%To be continued

%theoremstyles:

%\theoremstyle{plain}

%Use \swapnumbers to have theoremnumbers left of the Theorem

% \newtheorem{thm}{Theorem}[section]
% \newtheorem{prop}[thm]{Proposition}
% \newtheorem{proposition}[thm]{Proposition} %hope this works for TOC; might need find&replace otherwise...
% %doesn't enter TOC anyway yet... should it?
% \newtheorem{corollary}[thm]{Corollary}
% 
% \theoremstyle{definition}
% \newtheorem{definition}[thm]{Definition}
% \newtheorem{remark}[thm]{Remark}
% \newtheorem{lemma}[thm]{Lemma}

% \setcaptionmargin{2ex} %There are lots of longer captions surrounded by text. This helps to distinguish the captions from the surrounding text.



%alignment sign subscripts

%\newcommand*{\AlignGlue}{\textnormal{align}}

%paragraph indentation style. I don't like LaTeX's default behaviour

\setlength{\parindent}{0pt} %maybe actually increase?
\setlength{\parskip}{1.1ex plus 0.2ex minus 0.2ex}

%NOTE: might conflict with algorithm typesetting!


\interfootnotelinepenalty=10000 %splitting footnotes across pages is far worse than over/underfull pages

\usepackage{hyperref} %must be LAST preamble command, or so they say.


\makeatletter
\newcommand\footnoteref[1]{\protected@xdef\@thefnmark{\ref{#1}}\@footnotemark}
\makeatother
%automatic typos corrections. Those are actually used VERY often, because they cause my editors auto-completion to trigger (which wouldn't trigger at all without those).

\newcommand{\epslion}{\epsilon}
\newcommand{\lamda}{\lambda}
\newcommand{\Lamda}{\Lambda}

%Font selectors


\newcommand{\AlgorithmFont}[1]{\ensuremath{\mathtt{#1}}}
\newcommand*{\MatrixFont}[1]{\ensuremath{\mathbf{#1}}}
\newcommand*{\SpaceFont}[1]{\ensuremath{\mathsf{#1}}} %font for spaces like rings, vector spaces and the like	
\newcommand*{\VarietyFont}[1]{\ensuremath{\mathcal{#1}}} %font for varieties/algebraic sets
\newcommand*{\PolyFont}[1]{\ensuremath{\mathbf{#1}}} %font for polynomials


%complexity classes

\newcommand*{\POLY}{{\mathcal{P}}}
\newcommand*{\NPOLY}{{\mathcal{NP}}}

%probability

\DeclareMathOperator*{\PROB}{\mathbf{Pr}} %has limits!
\DeclareMathOperator*{\EXPECT}{\mathbf{Exp}} %has limits
%need to condition this one on style (language)


\newcommand*{\ScProd}[2]{\ensuremath{\left\langle{#1}{\,,\,}{#2}\right\rangle}}
\newcommand*{\negl}{\textnormal{negl}}
\newcommand*{\poly}{\textnormal{poly}}
\newcommand{\secpar}{\ensuremath{\lambda}} %security parameter
\DeclareMathOperator{\Gen}{\textnormal{\bf Gen}}
\DeclareMathOperator{\Dec}{\textnormal{\bf Dec}}
\DeclareMathOperator{\Enc}{\textnormal{\bf Enc}}
\DeclareMathOperator{\Image}{\mathrm{Im}} %Image (of a usually linear map)
\DeclareMathOperator{\LeadMon}{\mathrm{LM}}
\DeclareMathOperator{\LeadCoeff}{\mathrm{LC}}
\DeclareMathOperator{\LeadTerm}{\mathrm{LT}}
\DeclareMathOperator{\lcm}{\mathrm{lcm}}
\newcommand*{\Messages}{\mathcal{M}}
\newcommand*{\Ciphers}{\mathcal{C}}
\newcommand*{\Grobner}{Gr\"obner\xspace}
\newcommand*{\Keys}{\mathcal{K}}
\newcommand*{\Bit}{\left\{0,1\right\}}
\newcommand*{\OLandau}{O} %Big-O
\newcommand*{\VarOLandau}{\widetilde{O}} %Big-O up to polylog-factors
\newcommand*{\wLandau}{\omega} %grows larger than (i.e. f=\wLandau(g) :<=> lim f/g = +\infty , assuming g>0)
\newcommand*{\concat}{\mid\mid}
\newcommand*{\oLandau}{o} %technically, a small omicron
\DeclareMathOperator{\frob}{frob}

\newcommand*{\getsrandomly}%{\stackrel{\$}{\gets}}   --- too high, makes linespread ugly
{\gets_\$}


\newcommand*{\tensor}{{\otimes}} %tensor product

\newcommand*{\IR}{{\mathbb{R}}} %real numbers
\newcommand*{\IZ}{{\mathbb{Z}}} %set of integers
\newcommand*{\IC}{{\mathbb{C}}} %complex numbers
\newcommand*{\IN}{{\mathbb{N}}} %positive integers
\newcommand*{\IQ}{{\mathbb{Q}}} %rationals
\newcommand*{\Field}{{\mathbb{F}}} %general field, typically finite
\newcommand*{\ddiff} {{\mathrm{d}}} %de-Rham differential etc. I like to have a different font for the d.


\providecommand*{\abs}[1]{{\lvert{#1}\rvert}} %degree
\providecommand*{\norm}[1]{{\lVert{#1}\rVert}}
\newcommand*{\dunion}{{\dot\cup}} %disjoint union of sets
\newcommand*{\dissum}{{\mathbin\sqcup}} %disconnected sum-union
\newcommand*{\VarCHI}{{\mathcal{X}}} %I use \mathcal{X} to replace _capital_ \chi and continue with \Phi, \Psi... I think \mathcal{X} looks better than lowercase \chi (Greek chi) here, because capital \chi would just be X.
\newcommand*{\subspace}{\subset}

\newcommand*{\lact}{\mathbin{\rightharpoonup}}
\newcommand*{\ract}{\mathbin{\leftharpoonup}}


\DeclareMathOperator{\Volume}{Vol} %volume
\DeclareMathOperator{\ev}{ev} %evaluation maps
\DeclareMathOperator{\len}{len}
%\DeclareMathOperator{\Interior}{Int} %interior
\newcommand*{\Interior}[1]{{\mathring{#1}}}
%\DeclareMathOperator{\deg}{deg} %degree - already defined

\newcommand{\dominates}{\ensuremath{\gg}}
\newcommand{\isdominated}{\ensuremath{\ll}}
\newcommand{\divides}{\ensuremath{\mid}}


%\renewcommand{\labelitemii}{{$\cdot$}} %symbol used for 2nd level of nested lists. Standard is -, which is can be confused with a mathematical -.
%should be obsolete with paralist's capabilities

%typeset:
%various subscripts

\DeclareMathOperator*{\LEN}{len}


\begin{document}

This document collects some open questions to ponder for Squirrel/Chipmunk. Note that they are not neccessarily difficult and writing them up may be more work that solving them;
still, these are kind-of useful for giving perspective.

\section{Efficient encoding via small elements in 2-norm}

Let $\Field_p$ be some prime field, $n>0$ and consider some $n$-dimensional $\Field_p$-algebra $R$. In practise, $R$ will be a cyclotomic extension of $\Field_p$ with $n$ a power of 2 and we consider the 2-norm on $R$ by looking at individual coefficients wrt.\ some canonical representation as polynomials.

What we want to find is some $m$ and an $R$-linear map
\[
 f\colon R^m \to R
\]
that is surjective and where for every $y\in R$, we can \emph{efficiently} find some \emph{short} preimage $x\in R^m$ with $f(x) = y$ and $\norm{x}\leq \alpha$.

In the original Squirrel, $f$ was the binary reconstruction map $f(x) = x_1 + 2x_2 + 4x_3+ \ldots$ and the algorithm to find a short $x$ was binarization.

For the shortness notion, we will be using $\norm{.} = \norm{.}_2$ throughout. In this sense, binarization/ternarization gives a shortness bound of $\sqrt{nm}$.
The fact that the shortness given by these gives a stronger condition (by bounding every individual coefficient) does not help parameter selection so far.
However, it does ``waste'' a significant portion of the domain $R^m$, because most elements bounded by $\sqrt{nm}$ won't be bounded in infinity-norm by 1.

Note here that the map $f$ needs\footnote{At least that's what I think -- Gotti} to be $R$-linear, not just $\Field_p$-linear. So the map is of the form $f(x) = a_1 x_1 + \ldots + a_m x_m$.
If we restrict ourselves to the case where every $a_i$ is from the base field, the map (and the preimage-finding) acts on every coefficient independently.

\subsection{More efficient maps}
Consider the case where we stick to $a_i$ from the base field. Then essentially we can look at linear
\[
    f\colon (\Field_p)^m \to \Field_p\enspace.
\]
that we want to be surjective with small preimages. There are several questions:
\begin{itemize}
 \item How much can we hope to gain from this approach?
 \item What is the best map $f$
 \item How do we find short preimages.
\end{itemize}
\subsubsection{Sizes of discrete balls}
For the first question, consider the discrete ball
\[
 B_m(\alpha) = \{x \in \IZ^m \mid \norm{x}_2 \leq \alpha\}
\]
vs.\ the corresponding infinity-norm ball
\[
 B'_m(\alpha) = \{x \in \IZ^m \mid \abs{x}_i\leq \frac{\alpha}{\sqrt{m}}\ \text{for all}\ i \}
\]
(Note that $\alpha$ is always for the 2-norm for comparability).

For appropriate odd $k$, the $k$-ary reconstruction map sends $B'_m(\alpha)$ surjectively to $\Field_p$ and this is almost bijective (the only lack of bijectivity comes from $p$ not being a power of $k$).

While for $B_m(\alpha)$, we probably won't find such a good map, clearly, the size of $B_m(\alpha)$ limits how much better we can be: We clearly require \[
\abs{B_m(\alpha)} \leq p
\]
So the question is to compare $\abs{B_m(\alpha)}$ with $\abs{B'_m(\alpha)}$ and $p$.

This needs to be done non-asymptotically, I fear.
\subsubsection{Finding the best f}
If the above question results in that it might be worth it, we would need to find a map $f$ such that $f(B_m(\alpha))$ is all of $\Field_p$. How do you find such a map without making $\alpha$ large.
One approach is to just try random $f$ (see next section on checking whether a given $f$ works) until you find a good one. Is there any more clever way? How much larger than $p$ does $\abs{B_m(\alpha)}$ need to be?

\subsubsection{Efficiently finding preimages}

Note that we essentially hardwire $f$ and we can do expensive precomputations that only depend on $f$. There are multiple approaches to actually finding short preimages:
\begin{itemize}
 \item We could use a structured $f$ that allows some special-purpose algorithm (like binarization)
 \item We could apply $f$ to all of $B_m(\alpha)$ to build a reverse lookup-table. This takes time about $\VarOLandau(p)$, which is actually good enough.
 \item Finding a short preimage is essentially a closest vector problem in dimension $m$. The underlying lattice is \[L=\{z \mid f(z) = 0 \bmod p\},\] which does not depend on the target $y$. Since $m$ is really small, this is completely doable. We could even precompute all Voronoi-relevant vectors of that lattice and have reconstruction use Babai's algorithm and then perform a Voronoi walk. In practise, a subset of the Voronoi-relevant vectors will do and the Voronoi walk will only take 0--2 steps for our parameters.
\end{itemize}
Note that even if we cared about asymptotics, since $p=\poly(\lambda)$ and so $m=\log(\lambda)$ here, using a single-exponential algorithms in $m$ (such as finding all Voronoi-relevant vectors or a Voronoi walk) is actually fine. In practise, $m$ is probably between $5$ and $10$.

The question is which of these approaches works best. I actually suspect the Voronoi one. Generally, this is less scary than it sounds\footnote{it means that for given $y$, we perform some Babai rounding (which really is a few multiplications/additions by constants and $\OLandau(m)$ reductions mod $p$) and then check if the result $x$ is short enough. If not, use the shortest vector among $x\pm v_i$ for some carefully chosen fixed precomputed set of $v_i\in L$'s and repeat this step.} and probably not a bottleneck for the final $f$. This is only really an issue if we try a very large number of $f$ to optimize for a very good $f$.

\section{Use infinity-norm bounds gainfully}

A very different approach from switching to 2-norm throughout would be to embrace the infinity norm and actually use it gainfully. Unfortunately, most hardness results and practical security estimates for lattice problems are really for the 2-norm.
The Squirrel security analysis makes use of the fact that any infinity-norm SIS-solver that finds a vector $x$ with $\norm{x}_{\infty} \leq c$ in dimension $m$ has $\norm{x}_2 \leq \sqrt{m} c$.

Interestingly, we might be able to do better: The reason is that this is a worst-case bound. The question here is the following: Given a random (as defined via the appropriate ring SIS-problem here) lattice $L$, what is the length of the shortest vector (in 2-norm) among those vectors with bounded infinity-norm $c$.
If there are not too many such vectors with bounded infinity norm, one might heuristically assume that they are uniformly distributed in the appropriate cube and what their shortest 2-norm is.

\section{Improved union bound}

The Squirrel security analysis bounds the norm growth incurred by aggregation by making the observation that the average growth is much better than the worst-case growth and since the adversary cannot choose the RO-determined randomizers (except for re-trying). As mentioned, adding a small number $t$ bits of extra inputs to the RO here (and letting the honest aggregator try until they find a good enough one) means that we do not have to hedge against extremely rare events of probability $2^{-\lambda}$, but rather against multiple event with probability $2^{-\frac{\lambda}{t}}$ each occurring simultaneously, which imporoves parameters.
The actual analysis uses a bound for each coordinate and then a union bound.

Some improvement can be probably made here by using the 2-norm rather than the infinity norm. For this to work out, we need to answer the following mathematical problem (which I presume has already been solved by someone)

Let $r, m >0$ and consider $r$ vectors $x_1,\ldots x_r$, each in the $m$-dimensional unit ball.

Choose $r$ uniformly random numbers $a_1,\ldots, a_r$ from $\{-1,+1\}$ and consider
\[
 s= \sum_i a_i x_i \in \IR^m
\]
For any sufficiently large $c$, can you upper-bound the probabilty that $\norm{s} > c$.

Note that the $x_i$ are real-valued. We are concerned with average-case $a_i$, but worst-case $x_i$. It is probably enough to give a bound for, say, $c > 2\sqrt{m}$, as this is really a tail bound. Note that the expected value of $\norm{s}$ is at most $\sqrt{m}$ (with equality attained iff all $x_i$ have length exactly 1).
I conjecture that the worst-case choice for the $x_i$ is attained when all $x_i$ have length 1 and the same (or opposite) direction. Showing that this is indeed the worst-case would be good enough, as in this case, the problem becomes just about bounding a Bernoulli distribution.

\end{document}
