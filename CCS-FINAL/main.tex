 \documentclass[sigconf]{acmart}

\newif\ifcameraready
\newif\ifeprint

% temporarily set both to true
\camerareadytrue
% \camerareadyfalse
\eprintfalse
% \eprintfalse

%%% True definitions here:

\ifcameraready
 \ifeprint % show both
  \newcommand\eprint[1]{{\color{purple}#1}}
  \newcommand\cameraready[1]{{\color{orange}#1}}
 \else % only cameraready
  \newcommand\cameraready[1]{#1}
  \newcommand\eprint[1]\relax
 \fi
\else % cameraready = false
\ifeprint
 \newcommand\eprint[1]{#1}
 \newcommand\cameraready[1]\relax
\else
 \typeout{Neither cameraready nor eprint set to true}
\fi
\fi

%\usepackage{fullpage}
%\usepackage{amssymb}
%\usepackage{amsmath}
%\usepackage[dvipsnames]{xcolor}
%\usepackage{breakcites}
\usepackage{mdframed}
%\usepackage{algorithm}
%\usepackage{algpseudocode}
%\usepackage[colorlinks=true,citecolor=NavyBlue,linkcolor=Black,backref]{hyperref}
%\input{backrefpatch}
\usepackage{nicefrac}
\usepackage{csquotes}
\usepackage{mathtools}
%\usepackage{xspace}
\usepackage{booktabs}
\usepackage{mleftright}
\usepackage{multirow}
\usepackage{aliascnt}
%\usepackage{adjustbox}
%\usepackage[most]{tcolorbox}
\usepackage[adversary,sets,asymptotics,landau,lambda,operators,logic]{cryptocode}
\usepackage{subcaption}
\usepackage{tablefootnote}
\allowdisplaybreaks
\makeatletter
%\let\claim\relax % undefine the environment
%\spnewtheorem{claim}[theorem]{Claim}{\bfseries}{\itshape}
%\let\lemma\relax % undefine the environment
%\let\c@lemma\relax
%\spnewtheorem{lemma}[theorem]{Lemma}{\bfseries}{\itshape}
%\let\corollary\relax % undefine the environment
%\let\c@corollary\relax
%\spnewtheorem{corollary}[theorem]{Corollary}{\bfseries}{\itshape}
%\let\proposition\relax % undefine the environment
%\let\c@proposition\relax
%\spnewtheorem{proposition}[theorem]{Proposition}{\bfseries}{\itshape}
%\let\definition\relax % undefine the environment
%\let\c@definition\relax
%\spnewtheorem{definition}[theorem]{Definition}{\bfseries}{\itshape}
\makeatother
\usepackage{tikz}
\renewcommand{\vec}[1]{\boldsymbol{#1}}
\def\sectionautorefname{Section}
\def\subsectionautorefname{Section}
\def\subsubsectionautorefname{Section}
\AtEndPreamble{%
\theoremstyle{acmdefinition}
\newtheorem{remark}[theorem]{Remark}
\let\definition\relax
\newaliascnt{definition}{theorem}
\newtheorem{definition}[definition]{Definition}
\aliascntresetthe{definition}
\def\definitionautorefname{Definition}
\theoremstyle{acmplain}
\let\lemma\relax
\newaliascnt{lemma}{theorem}
\newtheorem{lemma}[lemma]{Lemma}
\aliascntresetthe{lemma}
\def\lemmaautorefname{Lemma}
\newaliascnt{claim}{theorem}
\newtheorem{claim}[claim]{Claim}
\aliascntresetthe{claim}
\def\claimautorefname{Claim}
}

\preto\tabular{\setcounter{magicrownumbers}{0}}
\newcounter{magicrownumbers}
\newcommand\rownumber{\stepcounter{magicrownumbers}\arabic{magicrownumbers}}

\makeatletter
\newcommand\footnoteref[1]{\protected@xdef\@thefnmark{\ref{#1}}\@footnotemark}
\makeatother

\def\orcidID#1{\kern.08em\href{https://orcid.org/#1}{\protect\includegraphics[keepaspectratio,width=0.7em]{orcid}}}

%% Load macros
\newcommandx*{\pcas}[2][1=\ ,2=\ ]{#1\highlightkeyword[#2]{as}}
\newcommandx*{\pcor}[2][1=\ ,2=\ ]{#1\highlightkeyword[#2]{or}}
\newcommandx*{\pcand}[2][1=\ ,2=\ ]{#1\highlightkeyword[#2]{and}}

% Math hyphen
\mathchardef\mhyphen="2D

\newcommand{\setup}{\mathsf{Setup}}
\newcommand{\kgen}{\mathsf{KGen}}
\newcommand{\sign}{\mathsf{Sign}}
\newcommand{\aggregate}{\mathsf{Aggregate}}
\newcommand{\verify}{\mathsf{Vrfy}}
\newcommand{\wverify}{\mathsf{wVrfy}}
\newcommand{\sverify}{\mathsf{sVrfy}}
\newcommand{\iverify}{\mathsf{iVrfy}}
\newcommand{\averify}{\mathsf{aVrfy}}

\newcommand{\params}{\mathsf{pp}}


%Multi-Sig
\newcommand{\msig}{\mathsf{MSIG}}
\newcommand{\sk}{\mathsf{sk}}
\newcommand{\pk}{\mathsf{pk}}
\newcommand{\aggsig}{\sigma_\mathsf{agg}}
\newcommand{\pubkeys}{\mathcal{P}}
\newcommand{\signatures}{\mathcal{S}}
\newcommand{\queries}{\mathcal{Q}}
\newcommand{\osks}{\mathsf{OSS}}
\newcommand{\opks}{\mathsf{OPK}}

\newcommand{\IG}{\mathsf{IG}}
\newcommand{\fork}{\mathsf{F}}

\newcommand{\Hlist}{\mathcal{L}}
\newcommand{\otsig}{\vec{\sigma}'}

%OTS
\newcommand{\hots}{\mathsf{KOTS}}
\newcommand{\hotsparams}{\params_\hots}
\newcommand{\osk}{\mathsf{osk}}
\newcommand{\opk}{\mathsf{opk}}
\newcommand{\opklen}{{\len_{\mathsf{opk}}}}
\newcommand{\siglen}{{\len_{\mathsf{sig}}}}

%HVC
\newcommand{\hvc}{\mathsf{HVC}}
\newcommand{\hvcparams}{\params_\hvc}
\newcommand{\commit}{\mathsf{Com}}
\newcommand{\open}{\mathsf{Open}}
\newcommand{\len}{\ell}
\newcommand{\domlen}{{\len_{\mathsf{dom}}}}
\newcommand{\comlen}{{\len_{\mathsf{com}}}}
\newcommand{\oplen}{{\len_{\mathsf{op}}}}

\newcommand{\ring}{\mathcal{R}}
\newcommand{\decomp}{\mathsf{dec}}
\newcommand{\binsca}{\mathsf{bin}}
\newcommand{\proj}{\mathsf{proj}}
\newcommand{\lbl}{\mathsf{label}}
\newcommand{\ball}[2]{\mathcal{B}_{#1, #2}}
\newcommand{\sis}{\mathsf{SIS}}
\newcommand{\tern}{\mathcal{T}}

\newcommand{\qkots}{q_\hots}
\newcommand{\qhvc}{q_\hvc}
\newcommand{\betaopen}{\beta_\mathsf{op}}

\newcommand{\bagg}{\beta_{\mathsf{agg}}}

\newcommand{\zip}{\mathsf{zip}}

\newcommand{\nnote}[1]{{\color{red}[\textbf{Nils} #1 ]}}
\newcommand{\gnote}[1]{{\color{red}[\textbf{Gotti:} #1]}}

% spaces

\newcommand{\moddomain}{A_{\mathsf{dom}}}
\newcommand{\modcommit}{A_{\mathsf{com}}}
\newcommand{\modopening}{A_{\mathsf{op}}}
\newcommand{\modlinear}[1]{A_{\mathsf{op,lin}}^{#1}}
\newcommand{\modencoded}[1]{A_{\mathsf{op}}^{#1}}
\newcommand{\modopk}{A_{\mathsf{opk}}}
\newcommand{\modsig}{A_{\mathsf{sig}}}

\newcommand*{\projring}{\proj_{\eta,\limbs}}
\newcommand*{\decompring}{\decomp_{\eta,\limbs}}
\newcommand*{\projmod}[1]{\proj_{#1}}
\newcommand*{\decompmod}[1]{\decomp_{#1}}

\newcommand{\hashajtai}{h_\mathrm{Ajtai}}
\newcommand{\limbs}{\kappa}
\newcommand*{\otspkkeylen}{\gamma}
% Used inside inkscape svg. This uses lots of LaTeX commands because the svg get messy otherwise.
\newcommand{\NODE}[1]{\ensuremath{v_{\scriptscriptstyle #1}}}
\newcommand{\MSG}[1]{\ensuremath{{\scriptstyle m}_{\scriptscriptstyle #1}}}
\newcommand{\MSGU}[1]{\ensuremath{{\scriptstyle u}_{\scriptscriptstyle #1}}} % decomposed message

% semantic markup
\newcommand*{\transpose}{\intercal}
\newcommand{\interval}[1]{[#1]}
\newcommand{\errorbound}{\varepsilon}
\newcommand{\emptylist}{\epsilon}
\newcommand*{\normalceil}[1]{\lceil #1 \rceil} % almost the same as the standard \ceil, but without \left and \right

\newcommand*{\sisadv}{\overline{\adv}}
\newcommand*{\sisadvkots}{\overline{\adv}}
\newcommand*{\hint}{\mathrm{hint}}

\newcommand*{\hvcencoded}{\hvc_{\textup{Encoded}}^{\textup{Chip}}}
\newcommand*{\hvcplain}{\hvc^{\textup{Chip}}_{0}}
\newcommand*{\otschip}{\hots^{\textup{Chip}}}
\newcommand*{\msigchip}{\msig^{\textup{Chip}}}

\newcommand*{\defring}{\ensuremath{\ZZ[X]/\langle X^n + 1\rangle}}
\newcommand*{\defringq}{\ensuremath{\ZZ_q[X]/\langle X^n + 1\rangle}}
\newcommand*{\defringqq}{\ensuremath{\ZZ_{q'}[X]/\langle X^n + 1\rangle}}

\newcommand*{\latring}{\Lambda_\ring}
\newcommand*{\latmodular}{\Lambda_{\ring,q}}

\newcommand*{\kernelbasis}{\mathcal{B}}
\newcommand*{\subrange}[1]{<#1}
% \newcommand*{\concat}{\Vert}

\newcommand*{\latencode}{\ensuremath{\mathsf{Encode}^{\textup{B}}_{\eta,q}}}
\newcommand*{\latdecode}{\ensuremath{\mathsf{Decode}^{\textup{B}}_{\eta,q}}}
\newcommand*{\openencode}{\ensuremath{\mathsf{Encode}_{\mathrm{op}}}}
\newcommand*{\opendecode}{\ensuremath{\mathsf{Decode}_{\mathrm{op}}}}
\DeclareMathOperator*{\Span}{Span}



\title{Chipmunk: Better Synchronized Multi-Signatures from Lattices}

\author{Nils Fleischhacker}
\orcid{0000-0002-2770-5444}
\affiliation{
  \institution{Ruhr University Bochum}
  \city{Bochum}
  \country{Germany}
}
\email{mail@nilsfleischhacker.de}
\author{Gottfried Herold}
\orcid{0009-0005-7089-0883}
\affiliation{
  \institution{Ethereum Foundation}
  \city{Bonn}
  \country{Germany}
}
\email{gottfried.herold@ethereum.org}
\author{Mark Simkin}
\orcid{0000-0002-7325-5261}
\affiliation{
  \institution{Ethereum Foundation}
  \city{Aarhus}
  \country{Denmark}
}
\email{mark.simkin@ethereum.org}
\author{Zhenfei Zhang}
\orcid{0000-0001-5131-5377}
\affiliation{
  \institution{Ethereum Foundation}
  \city{Boston}
  \country{USA}
}
\email{zhenfei.zhang@ethereum.org}

%% Document start
\begin{document}


%% Abstract
\begin{abstract}
Multi-signatures allow for compressing many signatures for the same message that were generated under independent keys into one small aggregated signature. 
This primitive is particularly useful for proof-of-stake blockchains, like Ethereum, where the same block is signed by many signers, who vouch for the block's validity.
Being able to compress all signatures for the same block into a short string significantly reduces the on-chain storage costs, which is an important efficiency metric for blockchains.

In this work, we consider multi-signatures in the synchronized setting, where the signing algorithm takes an additional time parameter as input and it is only required that signatures for the same time step are aggregatable.
The synchronized setting is simpler than the general multi-signature setting, but is sufficient for most blockchain related applications, as signers are naturally synchronized by the length of the chain.

We present Chipmunk, a concretely efficient lattice-based multi-signature scheme in the synchronized setting that allows for signing an a-priori bounded number of messages.
Chipmunk allows for non-interactive aggregation of signatures and is secure against rogue-key attacks.
The construction is plausibly secure against quantum adversaries as our security relies on the assumed hardness of the short integer solution problem.

We significantly improve upon the previously best known construction in this setting by Fleischhacker, Simkin, Zhang (CCS 2022).
Our aggregate signature size is $5 \times$ smaller and for $112$ bits of security our construction allows for compressing 8192 individual signatures into a multi-signature of size less than $200$ KB.
We provide a full implementation of Chipmunk and provide extensive benchmarks studying our construction's efficiency.

\end{abstract}
\maketitle

%% Paper sections
% !TEX root = ../main.tex
\section{Introduction}\label{sec:intro}

Multi-signatures~\cite{NEC:ItaNak83,CCS:MicOhtRey01} allow for compressing distinct signatures for the same message generated by different signers into one small aggregated signature.
Such signature schemes are a powerful tool in distributed systems, like blockchains, where parties vouch for the validity of messages on the network by signing them. 
Rather than storing an amount of signatures that is linear in the number of parties that vouched for a specific messages, multi-signatures allow for storing a much shorter string that vouches for a message on behalf of all signers simultaneously.
Popular proof-of-stake blockchains like Ethereum\footnote{\url{https://github.com/ethereum/annotated-spec/blob/master/phase0/beacon-chain.md\#attestation}} and DFinity\footnote{\url{https://dfinity.org/whitepaper.pdf}} employ multi-signatures at the core of their consensus layer.

The most popular multi-signature scheme used in practice is a construction due to Boneh, Gentry, Lynn, and Shacham~\cite{EC:BGLS03} based on a signature scheme due to Boneh, Lynn, and Shacham (BLS)~\cite{AC:BonLynSha01}.
Their resulting multi-signatures are extremely small, but the security of their construction relies on the assumed hardness of computing discrete logarithms over pairing-friendly groups.
It was shown by Shor~\cite{Shor94} that the discrete logarithm problem can be solved efficiently by quantum computers, meaning that any cryptographic primitive basing its security on such an assumption is insecure in the presence of a quantum adversary.

Luckily, not all computational hardness assumptions are created equal and some seem to remain hard in the presence of quantum adversaries. 
Building multi-signatures from computational hardness assumptions that withstand quantum adversaries is both a theoretically and practically important question.
While it may not be clear, when practically relevant quantum computers will appear, it is important to have secure alternatives for important cryptographic primitives, when the time comes.

One class of cryptographic hardness assumptions that seems to be particularly resilient against quantum adversaries is lattice-based cryptography.
Two of the three post-quantum signature schemes that were standardized by NIST in 2022 base their security on hardness assumptions related to lattices and, not surprisingly, there has also been significant interest in constructing multi-signatures from lattice hardness assumptions~\cite{CANS:ElBStu16,FukHas19,MaJia19,PenDu20,AFRICACRYPT:KanDut20,PROVSEC:FukHas20,PKC:DOTT21,C:BosTakTib22, CCS:FleSimZha22}.
The current multi-signature constructions, however, do still have significant drawbacks that hinder their practical deployment.
The constructions of El Bansarkhani and Sturm~\cite{CANS:ElBStu16} and Ma and Jiang~\cite{MaJia19} assume that the keys of all signers are generated honestly. 
This is not a realistic assumption as an adversarial signer could aim to perform a rogue-key attack by generating a malformed verification key that depends on honest signers' keys and allows for forging aggregated signatures, which falsely claim that both the malicious and the honest parties signed a message that was not actually signed by them.
The scheme of Kansal and Dutta~\cite{AFRICACRYPT:KanDut20} was shown to be insecure by Liu et al.~\cite{EPRINT:LiuTseTso20b}.
The constructions of Fukumitsu and Hasegawa~\cite{FukHas19,PROVSEC:FukHas20}, Ma and Jiang~\cite{MaJia19}, and Peng and Du~\cite{PenDu20}, and Boschini, Takahashi, and Tibouchi~\cite{C:BosTakTib22} all require interaction between the signers for generating a joint multi-signature.
Such an interaction between independent signers is difficult to realize in many distributed systems as the signers may be online at different times and may even not know of each others existence.

Recently, Fleischhacker, Simkin, and Zhang~\cite{CCS:FleSimZha22} presented a multi-signature construction named Squirrel, which allows for non-interactive aggregation and is secure against rogue-key attacks.
They consider a simplified setting, where signer's keys are only able to sign an a-priori bounded number of messages and where signers are synchronized in the sense that aggregation only has to work for signatures that were generated for the same time step and same message.
This simplified setting is still sufficiently strong for most blockchain applications, where signers do not sign more than one message per block and are naturally synchronized by the length of the current chain.
While an a-priori bound on the number of messages that can be signed may seem like a strong limitation, one can simply set this number large enough, e.g. to $2^{24}$ which would allow a signer to sign a message every 10 seconds for 5 years non-stop.
Aiming for $112$ bits of security, their individual signatures are of roughly size $50$ KB and aggregating $4096$ signatures results in a multi-signature of size $771$ KB.

Squirrel represents a significant step forward for multi-signature schemes that are plausibly secure in the presence of a quantum adversary and are concretely efficient.
For real-world practical scenarios their aggregated signatures seem, however, still too large to be really used.
As a point of reference, a full Ethereum block is on average less than $130$ KB large\footnote{\url{https://etherscan.io/chart/blocksize}}, which would mean that one block could not even fit a single multi-signature.

\subsection{Our Contribution}
In this work we present Chipmunk\footnote{Smaller than squirrels, cuter than squirrels.}, a multi-signature scheme in the synchronized setting~\cite{PKC:GenRam06,CCS:AhnGreHoh10,EC:HohWat18,USENIX:DGNW20} with an a-priori bound on the number of signatures that can be issued per key. 
We aim for the exact same setting as Squirrel~\cite{CCS:FleSimZha22}, but provide both theoretical and practical improvements.

On the theoretical side, we strengthen the security notions for multi-signatures by requiring that aggregation involving malformed but verifying adversarial signatures will succeed with high probability. 
In Squirrel aggregation was only required to work for honestly generated individual signatures. 
In principle, their security model would allow an adversary to perform a denial-of-service attack against the signature aggregation procedure by providing a single verifying, but malformed signature.
In a real-world distributed system, such an attack on liveness would be highly problematic.
We strengthen their security definitions to formally ensure that successfully verifying individual signatures will be successfully aggregated, even if they are chosen maliciously. 

On the practical side, our scheme Chipmunk produces smaller individual and aggregated signatures, when compared to Squirrel.
In terms of computational efficiency metrics, Chipmunk either significantly outperforms Squirrel or remains comparable in speed.
In terms of bandwidth, Chipmunk's aggregate signatures are smaller by a factor of $4.8\times$, when compared to Squirrel.
For keys that can generate $2^{24}$ signatures, an individual Chipmunk signatures is $37$ KB and aggregating $8192$ signatures results in an aggregate signature that is $180$ KB large.
%The gap between Chipmunk's and Squirrel's performance increases with an increasing bound on the number of signatures that should issuable with a single key.
%One of Squirrel's main computational bottlenecks was key generation as generating keys that support $2^{21}$ and $2^{26}$ signatures took $4$ minutes and $2$ hours.
%Chipmunk's key generation times are smaller for these parameters by factors of $9 \times$ and $30 \times$.

We have fully implemented Chipmunk and provide extensive benchmarks and comparisons to Squirrel in~\autoref{sec:benchmarks}.
Chipmunk is the most concretely efficient multi-signature based on assumptions that are assumed to remain valid in the presence of a quantum adversary.

\subsection{Technical Overview}

Conceptually, Chipmunk closely follows the blueprint that was introduced by Fleischhacker, Simkin, and Zhang~\cite{CCS:FleSimZha22}.
Recall that in their work and in ours we only aim to issue an a-priori bounded number of signatures, meaning that key generation is parameterized by $\tau$ and produces a public key that can be used to sign $2^\tau$ messages.
Further recall that we are in the synchronized setting, meaning that we only aim to aggregate signatures for the same message that were issued at the same time step.

In Squirrel, each signer's public key $\pk$ is a homomorphic vector commitment of length $2^\tau$, where position $i$ commits to $\pk^i$, which is the public key of a key-homomorphic one-time signature scheme.
To sign message $m$ at time step $i$, the signer opens the commitment $\pk$ to $\pk^i$ at position $i$ and uses the corresponding one-time signing key to sign the message $m$.
The signature is a vector itself that consists of $\pk^i$, the corresponding opening, and the signature of $m$ under this one-time key.
To verify that a message $m$ was signed for time step $i$, the verifier checks that the given public key $\pk^i$ is a valid opening of the $i$-th position of the corresponding signer's public key $\pk$ and that the given signature verifies for message $m$ under the public key $\pk^i$.

Aggregation of such signatures is performed by exploiting the homomorphic properties of the vector commitment and the one-time signature scheme.
To aggregate signatures, roughly speaking, one simply adds up all the individual commitment openings, the one-time keys, and the corresponding one-time signatures.
The homomorphism of the vector commitment scheme ensures that the sum of openings is a valid opening for the sum of committed messages, i.e. the one-time public keys, under the sum of commitments.
The key-homomorphic property of the one-time signature scheme ensures that the sum of signatures for the same message verifies under the sum of one-time public keys.
Chipmunk follows this blueprint, but improves upon all building blocks that are being used and thereby significantly reduces the multi-signature size of Chipmunk.

\paragraph{Key-Homomorphic One-Time Signatures.}
Squirrel uses a key-homomorphic one-time signature scheme that is similar to those of Boneh and Kim~\cite{BonKim2020} and Lyubashevsky and Micciancio~\cite{TCC:LyuMic08}.
The details of the construction are not relevant for now.
For Chipmunk, we use almost the exact same scheme, but observe by carefully inspecting their original security proof that a minor modification of their construction allows for their proof to produce much tighter parameters and thus smaller signatures.

\paragraph{Homomorphic Vector Commitments.}
The vector commitment used by Squirrel is a homomorphic analogue of the classical Merkle tree construction.
To make a Merkle tree homomorphic, the idea is to employ a homomorphic hash function to compute the node's values of the tree.
Now when adding two trees node-wise, one obtains a new valid tree.
Ajtai~\cite{ICALP:Ajtai99} introduced such a homomorphic hash function based on the short integer solution problem.
The main difficulty with using this hash function is the fact that hash output values need to be transformed into valid hash input values in a way that is efficient and maintains the homomorphic properties we would like our tree to have.
Without going into the details, the problem is that inputs for Ajtai's hash function need to have small norm, but outputs have potentially very large norms.
Fleischhacker, Simkin, and Zhang~\cite{CCS:FleSimZha22} solved this problem by effectively performing a binary decomposition of the hash function's output values, which resulted in vectors with infinity norm one, which could then again be used as hash function inputs.
In Chipmunk, we generalize their trick of decomposing values into binary vectors to decomposition into vectors of small norm.
While conceptually simple, we show that this change allows us to significantly reduce the size of our homomorphic vector commitment openings.

\paragraph{Chipmunk Multi-Signatures.}
Our construction of multi-signatures from vector commitments and one-time signatures follows the blueprint that was already outlined above.
One thing we glossed over so far are rogue-key attacks.
If we were to simply add up individual signatures, then our scheme would be susceptible to an adversary that first sees the honest parties public keys and then generates a malicious public key that allows for forging multi-signatures involving honest signers for arbitrary messages.
To avoid this type of attack, the individual signatures are multiplied by randomizer values before being added up.
These randomizing values need to be large enough to be unpredictable for the adversary, but cannot be too large as the multi-signature scheme's efficiency would deteriorate.
In Squirrel, aggregation was a one-shot process.
In Chipmunk, we repeatedly choose fresh randomization values and attempt aggregation until the aggregated signature is ``small enough''.
We show that doing this alllows us to reduce signature size without affecting the security of our construction.



We note that we grossly oversimplified many things in our above overview and that the precise construction and our improvements are more technically involved.
While each individual improvement may seem conceptually simple, they all add up to significantly improved signature sizes, resulting in the by far most efficient multi-signature scheme from lattice assumptions to date.





% !TEX root = ../main.tex
\section{Preliminaries}\label{sec:prelim}

\[\zip(\vec{a},\vec{b}) := \begin{pmatrix}(a_1,b_1)\\
\vdots\\
(a_\ell,b_\ell) \end{pmatrix}\]

\begin{lemma}[\cite{Mic07}]\label{lem:ternbound}
  Let $a,b\in\ring$ be two polynomials. Then $\norm{b\cdot a} \leq \norm{a}_1\cdot\norm{b}$.
\end{lemma}

% !TEX root = ../main.tex
\section{Homomorphic Vector Commitments}\label{sec:veccom}

In this section, we define and instantiate homomorphic vector commitments, which allow for committing to a long vector with a short commitment value.
Positions in the vector can be individually opened using a short opening value.
We follow the definitions for vector commitments of Fleischhacker, Simkin, and Zhang~\cite{CCS:FleSimZha22}, but we require somewhat different and incomparable homomorphic properties.
The definition of \cite{CCS:FleSimZha22} only requires honestly generated commitments to have homomorphic properties, whereas our definition requires the homomorphism to work for any individually verifying commitments and openings.
On the other hand, \cite{CCS:FleSimZha22} requires that the homomorphism works with probability $1$, whereas we allow some noticeable error.
Among other things, this modification of the definition allows us to instantiate homomorphic vector commitments more compactly.

% \gnote{Clarify role of $\params$. Is $\tau$ an input to $\setup$ or hardwired? Do the modules depend on $\secpar$? Do $\params$ determine $\tau$ and the modules?}

%\gnote{Consistent wording of opening vs.\ decommitment?}

\begin{definition}\label{def:hvc}
Let $\tau\in\NN$ be fixed.
  %Let $\ring$ be a ring and let $q,q',\tau\in\NN$.
  %A homomorphic vector commitment scheme (HVC) for domain $\ring^\domlen_{q'}$ and vectors of length $2^\tau$ is defined by four PPT algorithms $\hvc=(\setup,\commit,\allowbreak\open,\verify)$.
  Let $\ring$ be a ring and let $\moddomain$, $\modcommit$, $\modopening$ be $\ring$-modules. % Let $\tau\in\NN$.
  A homomorphic vector commitment scheme (HVC) for domain $\moddomain$ and vectors of length $2^\tau$ is defined by six PPT algorithms $\hvc=(\setup,\commit,\allowbreak\open,\iverify,\sverify, \wverify)$.
\begin{description}
    \item[$\params\gets\setup(\secparam)$] The setup algorithm takes as input the security parameter and outputs public parameters.
    \item[$c \gets \commit(\params,\vec{m})$] The commitment algorithm gets as input the public parameters and a vector $\vec{m}\in\moddomain^{2^\tau}$ and outputs a commitment $c\in\modcommit$.
    \item[$d \gets \open(\params,c,\vec{m},t)$] The opening algorithm gets as input the public parameters, a commitment, the committed vector, and an index and outputs a decommitment $d\in\modopening$.
    \item[$\vec{m}/\bot\gets \iverify(\params,c,t,d)$] The individual verification algorithm takes as input public parameters, a commitment, an index, and a decommitment and outputs either $\vec{m}\in\moddomain$ or an error symbol.
    \item[$\vec{m}/\bot\gets \sverify(\params,c,t,d)$] The strong verification algorithm has the same input and output domains as the individual verification algorithm.
    \item[$\vec{m}/\bot\gets \wverify(\params,c,t,d)$] The weak verification algorithm has the same input and output domains as the individual verification algorithm.
  \end{description}
\end{definition}
For our purposes, $\ring$ will always be $\ring = \defring$ for $n$ a power of 2, as in \autoref{sec:prelim}.
% Indices $0\leq t < 2^\tau$ start at zero.
Our domain, commitment and opening space will always be of the form $\moddomain = \ring_{q'}^\domlen$, $\modcommit = \ring_q^\comlen$, $\modopening = \ring^\oplen$ for some primes $q,q'$.
Note that (correctly verifying) decommitments $d\in\modopening$ will have small coefficients and undergo arithmetic modulo $q$, so the reader may think of them as elements from $\ring_q^\oplen$, as Squirrel \cite{CCS:FleSimZha22} does.
\eprint{However, we will impose some bounds on values that are not reduced modulo $q$ later, so we need to formally treat them as elements from $\ring^\oplen$ and write the modular reduction explicitly.}

We can easily generalize this definition slightly and have $\modcommit$ and $\modopening$ depend on the particular choice of $\params\gets\setup(\secparam)$, but will do not need that.
\eprint{
Furthermore, the opening space $\modopening$ may depend on $t$. The latter is technically needed in \autoref{def:encodedhvc}.
To keep our notation simple, we only track that dependency if relevant.
All our definitions and proofs directly apply to these generalization in a straightforward way.
}
% Generally speaking, some of our proofs require careful distinction between $\ring$ and $\ring_q$, which is why we prefer to write $\equiv$ in place of $=$ whenever there is any chance of confusion.
\begin{definition}[Individual Correctness]\label{def:individual_correctness}
Let $\hvc$ be a vector commitment scheme for domain $\moddomain$ and vector length $2^\tau$.
  $\hvc$ is individually correct, if for all security parameters $\secpar\in\NN$, vectors $\vec{m}\in\moddomain^{2^\tau}$, indices $1\leq t \leq 2^\tau$, parameters $\params \gets \setup(\secparam)$, commitments $\vec{c} \gets \commit(\params,\vec{m})$, and decommitments $\vec{d} \gets \open(\params,\vec{c},\vec{m},t)$ it holds that
  \[
    \iverify\Bigl(\params,\vec{c},t,\vec{d}\Bigr) = \vec{m}_{t}\enspace.
  \]
\end{definition}
%
We require that individually verifying commitments and their respective decommitments can be homomorphically aggregated by computing a random $\ring$-linear combination of them.
Such aggregated commitments and decommitments should still \emph{strongly} verify with high probability over the choice of the random linear combination, provided the coefficients of the linear combination are from some restricted subset $W$ (such as a set of small elements).

\begin{definition}[Probabilistic Homomorphism]\label{def:probabilist_homomorphism}
\eprint{Let $\moddomain,\modcommit,\modopening$ be $\ring$-modules over some ring $\ring$ and $\tau\in\NN$.}
Let $\hvc$ be a vector commitment scheme for domain $\moddomain$ and vector length $2^\tau$.
Let $\rho\in\NN$, $0\leq \errorbound \leq 1$ and $W\subseteq\ring$.
$\hvc$ is $(\rho,W,\errorbound)$-probabilistically homomorphic, if for all security parameters $\secpar\in\NN$, number of aggregated commitments $\ell \leq \rho$, indices
$1\leq t \leq 2^\tau$,
% $t \in \interval{2^\tau}$,
parameters $\params \gets \setup(\secparam)$, commitments $\vec{c}^i\in\modcommit$, and decommitments $\vec{d}^i\in\modopening$ with $\iverify(\params,\vec{c}^i,t,\vec{d}^i) = \vec{m}^i$ such that $\vec{m}^i \neq \bot$ it holds that
  \[
    \Pr\mleft[
      w^1,\dots,w^{\ell} \gets W \colon
      \sverify\Bigl(\params,\sum_{i=1}^{\ell}w^i\cdot \vec{c}^i,t,\sum_{i=1}^{\ell}w^i\cdot \vec{d}^i\Bigr) = \sum_{i=1}^{\ell}w^i\cdot\vec{m}^i_{t}
    \mright] \geq 1-\errorbound\enspace.
  \]
\end{definition}
%
We additionally require that a further limited homomorphism still holds, even for maliciously \emph{aggregated} commitments.
For any two, even maliciously generated, commitments and their two respective openings that \emph{strongly} verify, their difference will still \emph{weakly} verify.

\begin{definition}[Robust Homomorphism] \label{def:malhomhvc}
  Let $\hvc$ be a vector commitment scheme for domain $\moddomain$ and vector length $2^\tau$.
  $\hvc$ is robustly homomorphic if for all security parameters $\secpar\in\NN$, public parameters $\params\gets\setup(\secparam)$, indices $1\leq t \leq 2^\tau$, (possibly malformed) commitments $\vec{c}^0,\vec{c}^1 \in \modcommit$, and (possibly malformed) decommitments $\vec{d}^0,\vec{d}^1\in\modopening$ with
  \[
    \sverify(\params,\vec{c}^0, t,\vec{d}^0)=\vec{m}^0 \quad \text{and} \quad \sverify(\params,\vec{c}^1, t,\vec{d}^1)=\vec{m}^1
  \]
  such that $\vec{m}^0,\vec{m}^1\neq \bot$ it holds that
  \[
    \wverify(\params,\vec{c}^0 - \vec{c}^1, t,\vec{d}^0 - \vec{d}^1)=\vec{m}^0 - \vec{m}^1\enspace.
  \]
\end{definition}
%
Finally, we require the commitments to be position binding.
\begin{definition}[Position-Binding]\label{def:position_binding}
  Let $\hvc$ be a vector commitment scheme.
  $\hvc$ is position binding if for all security parameters $\secpar$ and all PPT algorithms $\adv$ it holds that
  \[
    \Pr\mleft[
      \begin{aligned}
      \params\gets{}&\setup(\secparam);\\
      (\vec{c},t,\vec{d}_0,\vec{d}_1) \gets{}& \adv(\params);\\
      \vec{m}_0 \gets{}& \wverify(\params,c,t,d_0);\\
      \vec{m}_1 \gets{}& \wverify(\params,c,t,d_1)
      \end{aligned}\colon
      \vec{m}_0\neq \vec{m}_1 \land \bot\not\in\{\vec{m}_0,\vec{m}_1\}
    \mright]\leq \negl\enspace.
  \]
\end{definition}

\subsection{Squirrel's Homomorphic Vector Commitment}\label{subsec:squirrelrecap}

\begin{figure}
\centering
%\documentclass[tikz]{standalone}
%\usepackage{fullpage}
%\usepackage{amssymb}
%\usepackage{amsmath}
%\usepackage{breakcites}
%\usepackage{mdframed}
%\usepackage{algorithm}
%\usepackage{algpseudocode}
%\usepackage[notref,notcite]{showkeys} % comment out for final version or add ``final'' as option to the documentclass
%\usepackage[colorlinks=true,citecolor=NavyBlue,linkcolor=Black,backref]{hyperref}
%%\input{backrefpatch}
%\usepackage{nicefrac}
%\usepackage{csquotes}
%\usepackage{mathtools}
%\usepackage{xspace}
%\usepackage{booktabs}
%\usepackage{mleftright}
%\usepackage{multirow}
%\usepackage{adjustbox}
%%\usepackage[most]{tcolorbox}
%\usepackage[adversary,sets,asymptotics,landau,lambda,operators,logic]{cryptocode}
%\usepackage{subcaption}
%\usepackage{tablefootnote}
%\allowdisplaybreaks
%\newcommandx*{\pcas}[2][1=\ ,2=\ ]{#1\highlightkeyword[#2]{as}}
\newcommandx*{\pcor}[2][1=\ ,2=\ ]{#1\highlightkeyword[#2]{or}}
\newcommandx*{\pcand}[2][1=\ ,2=\ ]{#1\highlightkeyword[#2]{and}}

% Math hyphen
\mathchardef\mhyphen="2D

\newcommand{\setup}{\mathsf{Setup}}
\newcommand{\kgen}{\mathsf{KGen}}
\newcommand{\sign}{\mathsf{Sign}}
\newcommand{\aggregate}{\mathsf{Aggregate}}
\newcommand{\verify}{\mathsf{Vrfy}}
\newcommand{\wverify}{\mathsf{wVrfy}}
\newcommand{\sverify}{\mathsf{sVrfy}}
\newcommand{\iverify}{\mathsf{iVrfy}}
\newcommand{\averify}{\mathsf{aVrfy}}

\newcommand{\params}{\mathsf{pp}}


%Multi-Sig
\newcommand{\msig}{\mathsf{MSIG}}
\newcommand{\sk}{\mathsf{sk}}
\newcommand{\pk}{\mathsf{pk}}
\newcommand{\aggsig}{\sigma_\mathsf{agg}}
\newcommand{\pubkeys}{\mathcal{P}}
\newcommand{\signatures}{\mathcal{S}}
\newcommand{\queries}{\mathcal{Q}}
\newcommand{\osks}{\mathsf{OSS}}
\newcommand{\opks}{\mathsf{OPK}}

\newcommand{\IG}{\mathsf{IG}}
\newcommand{\fork}{\mathsf{F}}

\newcommand{\Hlist}{\mathcal{L}}
\newcommand{\otsig}{\vec{\sigma}'}

%OTS
\newcommand{\hots}{\mathsf{KOTS}}
\newcommand{\hotsparams}{\params_\hots}
\newcommand{\osk}{\mathsf{osk}}
\newcommand{\opk}{\mathsf{opk}}
\newcommand{\opklen}{{\len_{\mathsf{opk}}}}
\newcommand{\siglen}{{\len_{\mathsf{sig}}}}

%HVC
\newcommand{\hvc}{\mathsf{HVC}}
\newcommand{\hvcparams}{\params_\hvc}
\newcommand{\commit}{\mathsf{Com}}
\newcommand{\open}{\mathsf{Open}}
\newcommand{\len}{\ell}
\newcommand{\domlen}{{\len_{\mathsf{dom}}}}
\newcommand{\comlen}{{\len_{\mathsf{com}}}}
\newcommand{\oplen}{{\len_{\mathsf{op}}}}

\newcommand{\ring}{\mathcal{R}}
\newcommand{\decomp}{\mathsf{dec}}
\newcommand{\binsca}{\mathsf{bin}}
\newcommand{\proj}{\mathsf{proj}}
\newcommand{\lbl}{\mathsf{label}}
\newcommand{\ball}[2]{\mathcal{B}_{#1, #2}}
\newcommand{\sis}{\mathsf{SIS}}
\newcommand{\tern}{\mathcal{T}}

\newcommand{\qkots}{q_\hots}
\newcommand{\qhvc}{q_\hvc}
\newcommand{\betaopen}{\beta_\mathsf{op}}

\newcommand{\bagg}{\beta_{\mathsf{agg}}}

\newcommand{\zip}{\mathsf{zip}}

\newcommand{\nnote}[1]{{\color{red}[\textbf{Nils} #1 ]}}
\newcommand{\gnote}[1]{{\color{red}[\textbf{Gotti:} #1]}}

% spaces

\newcommand{\moddomain}{A_{\mathsf{dom}}}
\newcommand{\modcommit}{A_{\mathsf{com}}}
\newcommand{\modopening}{A_{\mathsf{op}}}
\newcommand{\modlinear}[1]{A_{\mathsf{op,lin}}^{#1}}
\newcommand{\modencoded}[1]{A_{\mathsf{op}}^{#1}}
\newcommand{\modopk}{A_{\mathsf{opk}}}
\newcommand{\modsig}{A_{\mathsf{sig}}}

\newcommand*{\projring}{\proj_{\eta,\limbs}}
\newcommand*{\decompring}{\decomp_{\eta,\limbs}}
\newcommand*{\projmod}[1]{\proj_{#1}}
\newcommand*{\decompmod}[1]{\decomp_{#1}}

\newcommand{\hashajtai}{h_\mathrm{Ajtai}}
\newcommand{\limbs}{\kappa}
\newcommand*{\otspkkeylen}{\gamma}
% Used inside inkscape svg. This uses lots of LaTeX commands because the svg get messy otherwise.
\newcommand{\NODE}[1]{\ensuremath{v_{\scriptscriptstyle #1}}}
\newcommand{\MSG}[1]{\ensuremath{{\scriptstyle m}_{\scriptscriptstyle #1}}}
\newcommand{\MSGU}[1]{\ensuremath{{\scriptstyle u}_{\scriptscriptstyle #1}}} % decomposed message

% semantic markup
\newcommand*{\transpose}{\intercal}
\newcommand{\interval}[1]{[#1]}
\newcommand{\errorbound}{\varepsilon}
\newcommand{\emptylist}{\epsilon}
\newcommand*{\normalceil}[1]{\lceil #1 \rceil} % almost the same as the standard \ceil, but without \left and \right

\newcommand*{\sisadv}{\overline{\adv}}
\newcommand*{\sisadvkots}{\overline{\adv}}
\newcommand*{\hint}{\mathrm{hint}}

\newcommand*{\hvcencoded}{\hvc_{\textup{Encoded}}^{\textup{Chip}}}
\newcommand*{\hvcplain}{\hvc^{\textup{Chip}}_{0}}
\newcommand*{\otschip}{\hots^{\textup{Chip}}}
\newcommand*{\msigchip}{\msig^{\textup{Chip}}}

\newcommand*{\defring}{\ensuremath{\ZZ[X]/\langle X^n + 1\rangle}}
\newcommand*{\defringq}{\ensuremath{\ZZ_q[X]/\langle X^n + 1\rangle}}
\newcommand*{\defringqq}{\ensuremath{\ZZ_{q'}[X]/\langle X^n + 1\rangle}}

\newcommand*{\latring}{\Lambda_\ring}
\newcommand*{\latmodular}{\Lambda_{\ring,q}}

\newcommand*{\kernelbasis}{\mathcal{B}}
\newcommand*{\subrange}[1]{<#1}
% \newcommand*{\concat}{\Vert}

\newcommand*{\latencode}{\ensuremath{\mathsf{Encode}^{\textup{B}}_{\eta,q}}}
\newcommand*{\latdecode}{\ensuremath{\mathsf{Decode}^{\textup{B}}_{\eta,q}}}
\newcommand*{\openencode}{\ensuremath{\mathsf{Encode}_{\mathrm{op}}}}
\newcommand*{\opendecode}{\ensuremath{\mathsf{Decode}_{\mathrm{op}}}}
\DeclareMathOperator*{\Span}{Span}


\usetikzlibrary{shapes.geometric}
\usetikzlibrary{fit}
%\begin{document}
\begin{tikzpicture}[]
  \tikzset{
    short vector q/.style={outer sep=0pt,inner sep=0, minimum size=2mm,fill=blue,shape=diamond},
    short vector q'/.style={outer sep=0pt,inner sep=0, minimum size=2mm,fill=green,shape=circle},
    Rq elements/.style={outer sep=0pt,inner sep=0, minimum size=2mm,fill=red,shape=star}
  }
  \tikzset{
    level 1/.style={sibling distance=30mm,level distance=10mm},
    level 2/.style={sibling distance=15mm,level distance=10mm},
    level 3/.style={sibling distance=7.5mm,level distance=5mm},
    level 4/.style={sibling distance=6mm,level distance=5mm},
    level 5/.style={sibling distance=6mm,level distance=5mm},
    level 6/.style={level distance=8mm}
  };
  \node[short vector q] (ve) {} 
    child{node[short vector q] (v0) {}
      child{ node [short vector q] (v00) {}
        child[dashed]{
          coordinate (v000)
          child {
            edge from parent[draw=none]
            child[dashed] { node[short vector q] (vallzeroes) {}
              child[solid] { node[short vector q'] (u1) {}
                child { node[Rq elements] (m1) {}}
              }
            }
            child[missing]
          }
          child {
            edge from parent[draw=none]
            child[missing]
            child[dashed] {
              node[short vector q] (vzeroesone) {}
              child[solid] { node[short vector q'] (u2) {}
                child { node[Rq elements] (m2) {}}
              }
            }
          }
        }
        child[dashed]
      }
      child{ node [short vector q] (v01) {}
        child[dashed]
        child[dashed]
      }
    }
    child{node[short vector q] (v1) {}
      child{ node [short vector q] (v10) {}
        child[dashed]
        child[dashed]
      }
      child{ node [short vector q] (v11) {}
        child[dashed]
        child[dashed] {
          coordinate (v111)
          child {
            edge from parent[draw=none]
            child[dashed] {
                node[short vector q] (voneszero) {}
                child[solid] { node[short vector q'] (umaxminusone) {}
                  child { node[Rq elements] (mmaxminusone) {}}
                }
            }
            child[missing]
          }
          child {
            edge from parent[draw=none]
            child[missing]
            child[dashed] {
              node[short vector q] (vallones) {}
              child[solid] { node[short vector q'] (umax) {}
                child { node[Rq elements] (mmax) {}}
              }
            }
          }
        }
      }
    };
    \node[right=0mm of ve,inner sep=0mm] {\small$v_\varepsilon$};
    \foreach \nodename/\nodelabel in 
    {v0/v_0,
     v1/v_1,
     v00/v_{00},
     v01/v_{01},
     v10/v_{10},
     v11/v_{11},
     vallzeroes/v_{0\dots 0},
     vzeroesone/v_{0\dots 01},
     voneszero/v_{1\dots 10},
     vallones/v_{1\dots 1},
     u1/u_1,
     u2/u_2,
     umax/u_{2^\tau},
     umaxminusone/u_{2^\tau-1}
     }
     \node[right=0mm of \nodename,inner sep=0mm, yshift=-.5mm] {\small$\nodelabel$};
     \foreach \nodename/\nodelabel in 
    {m1/m_1,
     m2/m_2,
     mmax/m_{2^\tau},
     mmaxminusone/m_{2^\tau-1}
     }
     \node[below=1mm of \nodename,inner sep=0mm,anchor=north west, xshift=-.5em] {\small$\nodelabel$};
%     \node at ($(vzeroesone)!.5!(voneszero)$) {\Huge\ldots};
     \node at ($(u2)!.5!(umaxminusone)$) {\Huge\ldots};
%     \node at ($(m2)!.5!(mmaxminusone)$) {\Huge\ldots};
     \node[inner sep=0,minimum size=0] at ($(v000)+(270:2.5mm)$) {\myvdots};
     \node[inner sep=0,minimum size=0] at ($(v111)+(270:2.5mm)$) {\myvdots};
     \node[inner sep=0,minimum size=0] at ($(v01)+(270:7.5mm)$) {\myvdots};
     \node[inner sep=0,minimum size=0] at ($(v10)+(270:7.5mm)$) {\myvdots};
     
     \node[right=2.5cm of v1] (internalcheck) {$\hashajtai(v_{i\concat 0},v_{i\concat 1}) \equiv \proj(v_i) \pmod q$};
     \node at (vallones-| internalcheck) {$\hashajtai(u_i) \equiv \proj(v_{\binsca(i-1)}) \pmod q$};
     \node at (umax-| internalcheck) {$\proj(u_i) \equiv m_i \pmod q'$};
     
     \node[short vector q,left=6.5cm of ve,inner sep=0,yshift=-3mm] (svql) {};
     \node[anchor=west,at=(svql),yshift=.3ex] (svqlt) {: $\kappa$ many short $\ring$-elements};
     \node[short vector q',below=.9\baselineskip of svql,inner sep=0] (svqprimel) {};
     \node[anchor=west,at=(svqprimel),yshift=.4ex] (svqprimelt) {: $\kappa'$ many short $\ring$-elements};
     \node[Rq elements,below=.9\baselineskip of svqprimel,inner sep=0] (rql) {};
     \node[anchor=west,at=(rql),yshift=0ex] (rqlt) {: $\ring_{q'}$ element};
     \node[draw,rectangle,fit=(svql)(svqlt)(svqprimelt)(rqlt)] {};
\end{tikzpicture}
%\end{document}

\caption{Squirrel's homomorphic vector commitment. The bottom 2 rows serve the purpose to commit to $\ring_{q'}$ elements rather than to short vectors of $\ring_q$- or $\ring$-elements. The equations on the right are the constraints that link the layers together, ignoring shortness constraints. Note that all layers but the bottom one must contain short elements.}\label{fig:squirreltree}
\end{figure}

Since our homomorphic vector commitment is strongly based on Squirrel~\cite{CCS:FleSimZha22}, we recap their construction, albeit informally, in a bit more detail.
% Let us briefly and somewhat informally recap the homomorphic vector commitment from Squirrel~\cite{CCS:FleSimZha22}, which our HVC is based on.
Somewhat simplified, this commits to $2^\tau$ (small) entries from $\moddomain = \ring_q^\domlen$ by using a Merkle tree with a homomorphic hash function.
If we naively build a Merkle tree, this would mean that we construct a complete binary tree with $2^\tau$ leaves, where each leaf corresponds to an entry we want to commit to.
To each non-leaf node $v$, we associate the hash of its child nodes.
See \autoref{fig:squirreltree} for a visualization, ignoring the bottom two rows for now.
Concretely, the hash function utilized is Ajtai's hash function~\cite{ICALP:Ajtai99}, which hashes child nodes $\vec{c}_1, \vec{c}_2\in\ring_q^\domlen$ to
\[
\hashajtai(\vec{c}_1,\vec{c}_2) \coloneqq \vec{a}_1^\transpose \vec{c}_1 + \vec{a}_2^\transpose \vec{c}_2 \bmod q
\]
using a uniformly random $\ring_q$-linear map given by fixed public uninform $\vec{a}_1,\vec{a}_2\gets \ring_q^\domlen$.
Now, setting the relationship between parent node $\vec{p}$ and child nodes $\vec{c}_1, \vec{c}_2$ as $\vec{p}=\hashajtai(\vec{c}_1,\vec{c}_2)$ does not quite work: firstly, the range of the hash function is not $\moddomain$, which prevents iterating this construction.
Secondly, this hash function is only binding (based on some appropriate ring-SIS assumption) if we restrict its input to small elements.
To solve these issues, Squirrel chooses a second (public, fixed) linear function
\[
\proj\colon\,\ring_q^\domlen \to \ring_q
\]
and sets the equation that relates the parent node $\vec{p}\in\ring_q^\domlen$ with its children $\vec{c}_1,\vec{c}_2\in\ring_q^\domlen$ as
\begin{equation}
\hashajtai(\vec{c}_1,\vec{c}_2) = \proj(\vec{p}) \mod q\enspace. \label{eq:treerelation}
\end{equation}
%
One may view $\vec{p}$ as some kind of encoding of $\proj(\vec{p})$ here. Since $\vec{p}$ enters the hash function on the next layer of the tree as a child, it must be small (this is checked by the verification algorithms along with the linear relation above).
So to construct the tree, we need to be able to find small preimages of $\proj$.
In lattice terms, this means we need to solve some close(st) vector problem for the kernel of $\proj$.
An important observation is that this construction actually works for any $\proj$ for which we can find short preimages:
the homomorphic properties of the HVC are due to the fact that equation (\ref{eq:treerelation}) above
%\footnote{The point here is that what matters is that the verification equation is linear, not the map that finds the small preimage and is used to construct the Merkle tree. Indeed, the latter map is not linear.} 
is phrased in terms of $\ring_q$-linear maps, and the sum of small elements stays small.
We emphasize that what primarily matters here is the linearity properties of $\proj$ and the \emph{verification} equation. The map that finds the small preimage may be thought of as auxilliary and will not be linear.

Squirrel chooses $\proj$ as binary reconstruction $\proj(\vec{p}) = p_0 + 2p_1 + 4p_2 + \ldots$.
An algorithm to find a short inverse is then given by binary decomposition.

To commit to the correct domain $\ring_{q'}$, Squirrel adds some extra layers on the bottom of \autoref{fig:squirreltree}.

The main improvement from Chipmunk over Squirrel comes from choosing a different map for $\proj$ and its inverse: we propose to instead use $(2\eta+1)$-ary decomposition rather than binary decomposition.
This turns out to give significantly better parameters.

Some other differences in the actual construction are as follows:
\begin{itemize}
 \item We define the commitment (corresponding to the root of the Merkle tree) to be in non-decomposed form.
 \item We define $\proj$ and the $2\eta+1$-adic decomposition as maps over $\ring$ rather than $\ring_q$.
 \eprint{
 \item We constrain the size of $\proj(\vec{p})$ for any node of the tree.
 \item We use a more elaborate scheme to encode decommitments. This is explained in \autoref{sect:efficientencoding}.
 }
\end{itemize}

\subsection{A Homomorphic Vector Commitment based on Ring-SIS}\label{sec:our_hvc_construction}
To construct a homomorphic vector commitment with the desired properties, we will define $\proj$ and an inverse, called decomposition, as described above.
We use $(2\eta+1)$-ary decomposition for the latter, which allows us to map a ring element with possibly large norm to a vector of low norm ring elements.
To be able to use the greatest arity while minimizing the infinity norm of decomposed elements, we use a \emph{balanced} $(2\eta+1)$-ary decomposition, i.e.\ the decomposed elements have coefficients from $\{-\eta,\ldots,+\eta\}$ centered around 0.
We note that any even arity, such as the binary decomposition used by Squirrel~\cite{CCS:FleSimZha22}, is strictly worse than the next greater odd arity.
We then show that the projection function has nice homomorphic properties.

\begin{definition}[Projection onto $\ring$ elements]\label{def:proj}
  Let $\eta,\limbs\in\NN$.
  For any $\vec{b} \in \ZZ^{\limbs}$ we define the function 
  \[
    \projring\colon\, \ZZ^{\limbs} \to \ZZ,\quad \projring(\vec{b}) = \smashoperator{\sum_{j=1}^{\limbs}} b_j\cdot(2\eta+1)^{j-1}\enspace.
  \]
  We can extend this to a map 
  \[
    \projring\colon\, \ring^{\limbs} \to \ring,\quad \projring(\vec{b}) = \smashoperator{\sum_{j=1}^{\limbs}} b_j\cdot(2\eta+1)^{j-1}\enspace.
  \]
\end{definition}
% The parameter $q$ only affects the lenght of the domain and is solely included for notational consistency with Squirrel. \gnote{Actually, parameterizing by $\eta$ and $\limbs$ would make much more sense}
% For simplicity, all projection maps involved in a given instantiation of Chipmunk will use the same $\eta$. %, even if they use a different $q$.

\begin{definition}[Balanced $(2\eta+1)$-ary decomposition of $\ring$ elements]\label{def:decomp_R}
    Fix some odd arity $2\eta+1$ and let $\limbs\in\NN$ be the number of limbs.
%     Let $2\eta+1\in\NN$ be an odd integer and $\limbs\in\NN$ be the number of limbs.
    Then we can uniquely decompose any $a\in\ZZ$ into a balanced $(2\eta+1)$-ary decomposition with $\limbs$ limbs as
    \[
     a = \sum_{i=1}^{\limbs} a_i \cdot (2\eta+1)^{i-1}
    \]
    where $a_i \in \{-\eta,\ldots, \eta\}$ for all $1\leq i <\limbs$.
    An algorithm and proof of this statement is given below in \autoref{alg:decomp} and \autoref{prop:decomp}.
    Note that it is notationally convenient to allow arbitrarily sized $a$ in the definition and not bound $a_{\limbs}$, thereby putting all higher-order terms into $a_\limbs$.
    If we have the bound $\abs{a} < \tfrac{(2\eta+1)^\limbs}{2}$, then the most significant limb $a_{\limbs}$ will also be in $\{-\eta,\ldots,\eta\}$.
    
    We can extend this to a map on $\ring$ by essentially decomposing each coefficient, uniquely mapping a polynomial $a\in\ring$ to limbs $a_1\ldots,a_{\limbs}\in \ring$ such that
    \[
     a = \sum_{i=1}^{\limbs} a_i \cdot (2\eta+1)^{i-1}
    \]
    where $\norm{a_i}_{\infty} \leq \eta$ for all $1\leq i < \limbs$. If $\norm{a}_{\infty} < \tfrac{(2\eta+1)^\limbs}{2}$, we also have the bound $\norm{a_\limbs}_{\infty} \leq \eta$ for the most significant limb.
    
    Matching the notation from \autoref{def:proj}, we denote this decomposition map by $\decompring$, giving a map
    \[
     \decompring \colon\, \ring \to \ring^{\limbs},\quad a \mapsto (a_1,\ldots,a_{\limbs})\enspace.
    \]
%     Abusing notation, we can input elements $a\in\ring_q$ by decomposing their minimal representations in $a\in\ring$ with $\norm{a}_{\infty} < \tfrac{q}{2}$, giving a map
%     \[
%      \decomp_q \colon\, \ring_q \to \ring^{\lceil\log_{2\eta+1} q\rceil}\quad a \mapsto (a_0,\ldots,a_{\limbs-1})\enspace.
%     \]
% %     For any $a = \sum_{i=0}^{n-1} a_i\cdot x^{i-1}  \in \ring_q$,
%     denote by $(a_{i,1},\dots,a_{i,\lceil \log_{2\eta+1} q \rceil})^\transpose\in \{-\eta,\dots,\eta\}^{\lceil\log_{2\eta+1} q\rceil}$ the balanced $(2\eta+1)$-ary decomposition of $a_i$, i.e.,
%     \[
%     a_{i} \coloneqq \smashoperator{\sum_{j=1}^{\lceil\log_{2\eta+1} q\rceil}} a_{i,j}\cdot (2\eta+1)^{j-1}.
%     \]
%     We define the following decomposition of $a$ into polynomials with coefficients in $\{-\eta,\dots,\eta\}$:
%     \begin{equation*}
%         \decomp_q: \ring_q \to \ring^{\lceil\log_{2\eta+1} q\rceil},\quad
%         \decomp_q(a) = \left(\sum_{i=1}^{n} a_{i,1}\cdot x^{i-1}, \dots, \sum_{i=1}^{n} a_{i,\lceil\log_{2\eta+1} q\rceil}\cdot x^{i-1} \right).
%     \end{equation*}
\end{definition}
%
\begin{definition}[Projection and Decomposition for $\ring_q$]\label{def:proj_decomp_q}
Fix some odd arity $(2\eta+1)$ and let $q$ be prime.
Set $\limbs \coloneqq \ceil{\log_{2\eta+1}q}$
We denote by
\[
 \projmod{q}\colon\,\ring^\limbs \to \ring_q,\quad \projmod{q}(\vec{a}) \coloneqq \projring(\vec{a})\bmod q\in\ring_q
\]
and by
\[
 \decompmod{q}\colon\,\ring_q \to \ring^\limbs,\quad \decompmod{q}(a) \coloneqq \decompring(a'),\enspace,
\]
where $a'$ is the representative of $a$ in $\ring$ with coefficients in $\{-\frac{q-1}{2},\ldots,+\frac{q-1}{2}\}$.
\end{definition}
We remark that the only difference between $\projmod{q}$ and $\projring$ resp.\ between $\decompmod{q}$ and $\decompring$ is whether the non-decomposed element is in $\ring_q$ or $\ring$.
The decomposed elements are always from $\ring^\limbs$.
For $\projmod{q}$ and $\decompmod{q}$, the value of $\eta$ is not denoted explicitly.
This is done for notational consistency with Squirrel.
In our constructions, all uses of $\projmod{q}$ and $\decompmod{q}$ will use the same value for $\eta$, even if the values of $q$ differ.

% \gnote{There is really nothing to show here; this is obvious from the definition. To meaningfully prove something, we would need to actually give an algorithm for the decomposition. Note that we need $\ring$-linearity, not $\ring_q$-linearity.}

% \smallskip\noindent
The following proposition immediately follow from the definitions (for $\ring$-linearity, this follows from the examples given after \autoref{def:RModule}).
\begin{proposition}\label{prop:projanddecomp}
Let $q$ be an odd integer and fix some odd arity $2\eta+1$. The maps $\projmod{q}$ and $\projring$ defined above are $\ring$-linear.
The map $\decompring$ is a one-sided inverse to $\projring$, meaning that $\projring(\decompring(a)) = a$ for any $a\in\ring$.
Similarly, $\decompmod{q}$ is a one-sided inverse to $\projmod{q}$, meaning that $\projmod{q}(\decompmod{q}(a))= a$ for any $a\in\ring_q$.
For $a\in\ring_q$, we also have $\norm{\decompmod{q}(a)}_{\infty} \leq \eta$.
\end{proposition}

%
% The following two simple lemmas effectively state that the projection function is the inverse of the decomposition function and that the projection function is $\ring_q$-linear.
% \begin{lemma}\label{lem:projinvofbin}
%   For all primes $q$ and $a = \sum_{i=1}^{n} a_i\cdot x^{i-1} \in\ring_q$, it holds that $\proj_q(\decomp_q(a))=a$.
% \end{lemma}
% \begin{proof}
%     \begin{align*}
%       \proj_q(\decomp_q(a)) ={}& \proj_q\left(\sum_{i=1}^{n} a_{i,1}\cdot x^{i-1}, \dots, \sum_{i=1}^{n} a_{i,\lceil\log_{2\eta+1} q\rceil}\cdot x^{i-1}\right)\\
%       ={}& \smashoperator{\sum_{j=1}^{\lceil\log_{2\eta+1} q\rceil}} \Bigl((2\eta+1)^{j-1} \cdot \sum_{i=1}^{n} a_{i,j}\cdot x^{i-1}\Bigr)\\
%       ={}& \sum_{i=1}^{n} \Bigl(x^{i-1}\cdot \smashoperator{\sum_{j=1}^{\lceil\log_{2\eta+1} q\rceil}} (2\eta+1)^{j-1} \cdot a_{i,j}\Bigr)\\
%       ={}& \sum_{i=1}^{n} x^{i-1}\cdot a_i=a\tag*{\qed}
%     \end{align*}
% \end{proof}
%   
% 
% \begin{lemma}\label{lem:projislin}
% %   For all primes $q$ the projection function $\proj_q$ is $\ring_q$-linear, i.e., for any $\vec{b}^0,\vec{b}^1 \in \ring_q^{\lceil\log_{2\eta+1} q\rceil}$ and any $w^0,w^1 \in \ring_q$, $\proj_q(w^0\cdot\vec{b}^0+w^1\cdot\vec{b^1}) = w^0\cdot\proj_q(\vec{b^0}) + w^1\cdot\proj_q(\vec{b}^1)$.
% \end{lemma}
% \begin{proof}
%   \begin{align*}
%     \vspace{-1cm}
%     \proj_q(w^0\cdot\vec{b}^0+w^1\cdot\vec{b^1})
%     =&\smashoperator{\sum_{j=1}^{\lceil\log_{2\eta+1} q\rceil}} (2\eta+1)^{j-1}\cdot (w^0\cdot b^0_j+w^1\cdot b^1_j)\tag{\autoref{def:proj}}\\
%     =&w^0\cdot \Bigl(\smashoperator{\sum_{j=1}^{\lceil\log_{2\eta+1} q\rceil}} b^0_j\cdot (2\eta+1)^{j-1}\Bigr) + w^1\cdot\Bigl( \smashoperator{\sum_{j=1}^{\lceil\log_{2\eta+1} q\rceil}} b^1_j\cdot (2\eta+1)^{j-1}\Bigr)\\
%     =&w^0\cdot \proj_q(\vec{b}^0) + w^1\cdot\proj_q(\vec{b}^1)\tag{\autoref{def:proj}}
%   \end{align*}
%   \qed
% \end{proof}
%
For the sake of readability we will at times abuse notation slightly and apply $\decompmod{q}$ resp.\ $\decompring$ to \emph{vectors} of $\ring_q$ resp.\ $\ring$ elements, which is to be understood as the component-wise application of $\decompmod{q}$ resp.\ $\decompring$ with subsequent concatenation of the resulting vectors.
Similarly, $\projmod{q}$ resp.\ $\projring$ may be applied to vectors of a length that is a \emph{multiple} of $\limbs$ to result in a vector of $\ring_q$ resp.\ $\ring$ elements.
The above discussion generalizes to this extension.

\begin{figure}[hpt]
 \centering
 \begin{pcvstack}[center,boxed]
 \procedure{$\decompring(a)$}{
 r_{1}\coloneqq a\\
 \pcfor 1 \leq i \leq \limbs-1\\
 \quad \text{Choose $a_i\in\{-\eta,\ldots,+\eta\}$ with $a_i\equiv r_{i}\bmod (2\eta+1)$}\\
 \quad r_{i+1}\coloneqq \tfrac{r_{i}-a_i}{2\eta+1}\pccomment{Numerator is divisible by $2\eta+1$}\\
 % end for
 a_\limbs \coloneqq r_{\limbs}\\
 \pcreturn (a_1,\ldots,a_\limbs)
 }
 \end{pcvstack}
 \caption{%
 Algorithm for balanced $(2\eta+1)$-ary decomposition of integers $a\in\ZZ$. The corresponding algorithm for $a\in\ring$ works by applying this coefficient-wise.
 }
 \label{alg:decomp}
\end{figure}

\begin{proposition}[balanced $(2\eta+1)$-ary decomposition\label{prop:decomp}]
Let $\eta,\limbs\in\NN$. The algorithm in \autoref{alg:decomp} runs in polynomial time. For any $a\in\ZZ$, it outputs the unique $(a_1,\ldots,a_\limbs)$ with
     \begin{equation}\label{eq:decomp_proof}
     a = \sum_{i=1}^{\limbs} a_i \cdot (2\eta+1)^{i-1}
     \end{equation}
and $a_i\in\{-\eta,\ldots,+\eta\}$ for $1\leq i \leq \limbs-1$.
\end{proposition}
\begin{proof}
The algorithm is clearly polynomial time. $r-a_i$ is divisible by $2\eta+1$ by construction of $a_i$. By definition, $a_1,\ldots,a_{\limbs-1}\in\{-\eta,\ldots,+\eta\}$.
We show by induction that we have for all $1\leq i \leq \limbs -1$
% We claim that, after the $i$'th loop iteration (and with $i=0$ when entering the loop), we have the invariant
\[
a = r_{i}\cdot(2\eta+1)^{i-1} + \sum_{j=1}^{i-1} a_j(2\eta+1)^{j-1}\enspace.
\]
This is clear for $i=1$. Using induction, we compute
\begin{align*}
 {}&r_{i+1}\cdot(2\eta+1)^i + \sum_{j=1}^i a_j(2\eta+1)^{j-1}\\
 {}=&(r_i-a_i)\cdot(2\eta+1)^{i-1} + \sum_{j=1}^i a_j(2\eta+1)^{j-1} \tag{Def.\ of $r_{i+1}$}\\
 {}=&r_{i}\cdot(2\eta+1)^{i-1} + \sum_{j=1}^{i-1} a_j(2\eta+1)^{j-1} = a\tag{induction hypothesis}
\end{align*}
For $i=\limbs$, this yields $a = \sum_{i=1}^{\limbs} a_i \cdot (2\eta+1)^{i-1}$. For uniqueness, note that we just showed that the map $\{-\eta,\ldots,\eta\}^{\limbs-1}\times \ZZ \to \ZZ$, $(a_1,\ldots,a_\limbs)\mapsto \sum_{i=1}^{\limbs} a_i \cdot (2\eta+1)^{i-1}$ is surjective. Taking this modulo $(2\eta+1)^{\limbs-1}$ gives us that $\{-\eta,\ldots,+\eta\}^{\limbs-1}\to \ZZ_{(2\eta+1)^{\limbs-1}}$,
$(a_1,\ldots,a_{\limbs-1})\mapsto \sum_{i=1}^{\limbs-1} a_i \cdot (2\eta+1)^{i-1}\bmod (2\eta+1)^{\limbs-1}$ is surjective, hence injective (because domain and range have the same finite size). So $a_1,\ldots,a_{\limbs-1}$ are uniquely determined. Plugging this into \autoref{eq:decomp_proof} shows that $a_{\limbs}$ is uniquely determined as well.\qed
\end{proof}




This lets us define a labeling function for a full binary tree matching \autoref{fig:squirreltree}.

\begin{definition}[Labeled Full Binary Tree]\label{def:label}
  Let $n,q,q',\xi\in\NN$ with $n$ a power of two and $q,q'$ primes.
  Let $\vec{m}=(\vec{m}_{1},\dots,\vec{m}_{{2^{\tau}}})^\transpose\in(\ring_{q'}^\xi)^{2^{\tau}}$, $\vec{g} \in \ring_{q}^{\xi\ceil{\log_{2\eta+1} q'}}$ and $\vec{h}_0,\vec{h}_1 \in \ring_q^{\ceil{\log_{2\eta+1} q}}$ be fixed.
  We define the labeling function
  $\lbl_{\vec{g},\vec{h}_0,\vec{h}_1}\colon\,
  %\ring_{q}^{\xi\lceil\log_{2\eta+1} q'\rceil} \times (\ring_q^{\lceil\log_{2\eta+1} q\rceil})^2 \times 
  (\ring^\xi_{q'})^{2^{\tau}}\times \bin^{\leq \tau} \to \ring^{\lceil \log_{2\eta+1} q \rceil}$
  for a labeled full binary tree of depth $\tau$ as
  \[
    \lbl_{\vec{g},\vec{h}_0,\vec{h}_1}(\vec{m},v) \coloneqq 
      \begin{cases}
          \decompmod{q}(\vec{g}^\transpose\cdot \decompmod{q'}(\vec{m}_{v+1})) & \text{if } \abs{v}=\tau\\ \decompmod{q}\mleft(
              \begin{aligned}
                 \vec{h}_0^\transpose \cdot\lbl_{\vec{g},\vec{h}_0,\vec{h}_1}(\vec{m},v\concat 0)\\
               + \vec{h}_1^\transpose \cdot\lbl_{\vec{g},\vec{h}_0,\vec{h}_1}(\vec{m},v\concat 1)
               \end{aligned}
              \mright)& \text{if } \abs{v}< \tau
      \end{cases}\enspace.
  \]
For this, remember that multiplication of elements from $\ring_q$ and $\ring$ is always understood\footnote{as opposed to taking some canonical representative of $\ring_q$-elements in $\ring$ and then multiplying in $\ring$} to give an element in $\ring_q$. For $\vec{m}_{v+1}$, we interpret $v\in\bin^{\tau}$ as an integer in $\{0\ldots, 2^\tau-1\}$ in big-endian encoding and add 1 (i.e.\ the inverse of taking $\tilde t = \binsca(t-1)$).
\end{definition}

Using the labeling function, we can define Chipmunk's HVC as in \autoref{fig:hvcinst}.

\begin{definition}\label{def:hvc_chipmunk_unencoded}
Let $n, q, q', \alpha_w, \rho, \eta, \tau, \xi,\bagg$ be positive integers such that $n$ is a power of two and $q,q'$ are primes.
Let $\ring_q,\ring_{q'}$ be the polynomial rings $\defringq$ and $\defringqq$ respectively.
We define the homomorphic vector commitment \eprint{$\hvcplain$}\cameraready{$\hvccamera$} for domain $\moddomain = \ring_{q'}^\xi$ and vectors of length $2^\tau$ by the algorithms given in \autoref{fig:hvcinst}.
Its commitments and openings are from $\modcommit = \ring_q$ and $\modopening = (\ring^\limbs)^{2\tau}\times (\ring^{\limbs'})^{\xi}$, where $\limbs = \ceil{\log_{2\eta+1}q}, \limbs' = \ceil{\log_{2\eta+1}q'}$.
\end{definition}

\begin{figure}[pht]
\centering
\begin{pcvstack}[center,boxed]
\begin{pchstack}[center]
  \procedure{$\setup(\secparam)$}{
    \vec{g} \gets \ring_{q}^{\xi\kappa'}\\ % NOTE: \ring_q rather than \ring_{q'} is correct here.
    \vec{h}_0 \gets \ring_q^{\kappa}\\
    \vec{h}_1 \gets \ring_q^{\kappa}\\
    \pcreturn (\vec{g},\vec{h}_0,\vec{h}_1)
  }
  %
  \pchspace
  %
  \procedure{$\commit(\params,\vec{m})$}{
    \vec{p}_0 \coloneqq \lbl_{\vec{g},\vec{h}_0,\vec{h}_1}(\vec{m},\emptylist)\\
    \vec{c} \coloneqq \projmod{q}(\vec{p}_0)\\
    \pcreturn \vec{c}\in\ring_q
  }
\end{pchstack}
\begin{pchstack}[center]
    \procedure{$\open(\params,\vec{c},\vec{m},t)$}{
      \tilde t \coloneqq \binsca(t-1)\\
      \pcfor 1\leq j \leq \tau\\
      \quad \vec{p}_{j} \coloneqq \lbl_{\vec{g},\vec{h}_0,\vec{h}_1}(\vec{m},\tilde t_{< j}\concat \tilde t_{j})\\
      \quad \vec{s}_{j} \coloneqq \lbl_{\vec{g},\vec{h}_0,\vec{h}_1}(\vec{m},\tilde t_{< j}\concat (\tilde t_{j}\xor 1))\\
      \vec{u} \coloneqq \decompmod{q'}(\vec{m}_t)\\
      \pcreturn (\vec{p}_1,\dots,\vec{p}_\tau,\vec{s}_1,\dots,\vec{s}_\tau,\vec{u})
    }
  %
  \pchspace
  %
  \eprint{% cryptocode screws up LaTeX's \if -- WTF???
  \procedure{$\verify(\params,\vec{c},t, \vec{d},\beta)$}{
    \pcparse \vec{d} \pcas (\vec{p}_1,\dots,\vec{p}_\tau,\vec{s}_1,\dots,\vec{s}_\tau,\vec{u})\\
    \tilde t \coloneqq \binsca(t-1)\\
    \pcif \norm{\vec{u}} > \beta \pcor \vec{g}^\transpose \cdot \vec{u} \neq \projmod{q}(\vec{p}_\tau)\\
    \quad \pcreturn \bot\\
    %\vec{p}_0 \coloneqq \vec{c}\\
    \pcif \vec{c} \neq \vec{h}_{\tilde t_1}^\transpose\cdot \vec{p}_{1} + \vec{h}_{\tilde t_1 \xor 1}^\transpose \cdot \vec{s}_{1}\\
    \quad\pcreturn \bot\\
    \pcfor 2 \leq j \leq \tau\\
    \quad\pcif \projmod{q}(\vec{p}_{j-1}) \neq \vec{h}_{\tilde t_j}^\transpose\cdot \vec{p}_{j} + \vec{h}_{\tilde t_j \xor 1}^\transpose \cdot \vec{s}_{j}\\
    \quad\quad\pcreturn \bot\\
    \pcfor j \in \{1,\ldots,\tau\}\\
    \quad \pcif \norm{\vec{p}_{j}} > \beta \pcor \norm{\vec{s}_{j}} > \beta\\
    \quad \quad \pcreturn \bot \\
    \quad \pcif \norm{\projring(\vec{p}_j)} > \tfrac{q\beta}{2\eta} \pcor \norm{\projring(\vec{s}_j)} > \tfrac{q\beta}{2\eta}\\
    \quad \quad \pcreturn \bot\\
    \pcreturn \projmod{q'}(\vec{u}) \in\ring_{q'}^\xi
  }}
  \cameraready{% cryptocode screws up LaTeX's \if -- WTF???
  \procedure{$\verify(\params,\vec{c},t, \vec{d},\beta)$}{
    \pcparse \vec{d} \pcas (\vec{p}_1,\dots,\vec{p}_\tau,\vec{s}_1,\dots,\vec{s}_\tau,\vec{u})\\
    \tilde t \coloneqq \binsca(t-1)\\
    \pcif \norm{\vec{u}} > \beta \pcor \vec{g}^\transpose \cdot \vec{u} \neq \projmod{q}(\vec{p}_\tau)\\
    \quad \pcreturn \bot\\
    %\vec{p}_0 \coloneqq \vec{c}\\
    \pcif \vec{c} \neq \vec{h}_{\tilde t_1}^\transpose\cdot \vec{p}_{1} + \vec{h}_{\tilde t_1 \xor 1}^\transpose \cdot \vec{s}_{1}\\
    \quad\pcreturn \bot\\
    \pcfor 2 \leq j \leq \tau\\
    \quad\pcif \projmod{q}(\vec{p}_{j-1}) \neq \vec{h}_{\tilde t_j}^\transpose\cdot \vec{p}_{j} + \vec{h}_{\tilde t_j \xor 1}^\transpose \cdot \vec{s}_{j}\\
    \quad\quad\pcreturn \bot\\
    \pcfor j \in \{1,\ldots,\tau\}\\
    \quad \pcif \norm{\vec{p}_{j}} > \beta \pcor \norm{\vec{s}_{j}} > \beta\\
    \quad \quad \pcreturn \bot \\
%     \quad \pcif \norm{\projring(\vec{p}_j)} > \tfrac{q\beta}{2\eta} \pcor \norm{\projring(\vec{s}_j)} > \tfrac{q\beta}{2\eta}\\
%     \quad \quad \pcreturn \bot\\
    \pcreturn \projmod{q'}(\vec{u}) \in\ring_{q'}^\xi
  }}
\end{pchstack}
  \pcvspace
\begin{pchstack}
    \procedure{$\iverify(\params,\vec{c},t, \vec{d})$}{
      \pcreturn \verify(\params,\vec{c},t, \vec{d},\eta)
    }
    \pchspace
    \procedure{$\sverify(\params,\vec{c},t, \vec{d})$}{
      \pcreturn \verify(\params,\vec{c},t, \vec{d},\bagg)
    }
    \pchspace
    \procedure{$\wverify(\params,\vec{c},t, \vec{d})$}{
      \pcreturn \verify(\params,\vec{c},t, \vec{d},2\bagg)
    }
\end{pchstack}
\end{pcvstack}
\caption{%
% Note: The autoref{def:hvc_chipmunk_unencoded} is just to create a hyperlink in case the figure is placed badly.
The construction of the homomorphic vector commitment \eprint{$\hvcplain$}\cameraready{$\hvccamera$} for message space $\moddomain = \ring^\xi_{q'}$ based on a labeled binary tree, cf.~\autoref{def:hvc_chipmunk_unencoded}.
Commitments $\vec{c}$ are in $\modcommit = \ring_q$.
Openings are small elements in $\modopening = (\ring^\limbs)^{2\tau}\times (\ring^{\limbs'})^{\xi}$, where
$\limbs = \ceil{\log_{2\eta+1}q}$, $\limbs' = \ceil{\log_{2\eta+1}q'}$.
Let us clarify again that multiplication of $\ring_q$ with $\ring$ elements as done in the Ajtai hashes like $\vec{g}^\transpose\cdot\vec{u}$ is understood to give an element in $\ring_q$, i.e.\ we perform modular reduction here.
}
\label{fig:hvcinst}
\end{figure}


\begin{remark}\label{rmk:hvc}
Before proving security of \eprint{$\hvcplain$}\cameraready{$\hvccamera$}, let us give some remarks on the construction itself.
\begin{enumerate}
\eprint{%
\item Chipmunk's final homomorphic vector commitment actually employs a space-efficient non-trivial way to encode and decode (verifying) decommitments $\vec{d} = (\vec{p}_1,\ldots\,\vec{p}_\tau,\vec{s}_1,\ldots,\vec{s}_\tau,\vec{u})$.
% This is, strictly speaking, part of the opening and verification algorithms, but not included in \autoref{fig:hvcinst}.
To simplify the exposition, $\hvcplain$ in \autoref{fig:hvcinst} is described without these encoding and decoding schemes, which are formally part of the opening and verification algorithms.
We describe this encoding and decoding separately in \autoref{sect:efficientencoding}, giving an improved HVC denoted by $\hvcencoded$ there.%
}
\item The tree labels constructed by the labeling function that constitute the Merkle path $\vec{p}_j$ with its sibling nodes $\vec{s}_j$ are \emph{decomposed} elements, i.e.\ short elements in $\ring$.
For efficiency reasons, the commitment $\vec{c}$ itself is not $\vec{p}_0$, but rather in non-decomposed form. This is done to ensure the commitment is in $\ring_q$ rather than $\ring$, which is slightly more efficient when aggregating.
Regarding analysis, observe that if we set $\vec{p}_0$ as in the definition of $\commit$, the condition $\vec{c} = \vec{h}_{\tilde t_1}^\transpose\cdot \vec{p}_{1} + \vec{h}_{\tilde t_1 \xor 1}^\transpose \cdot \vec{s}_{1}$ is actually equivalent to
\[
\projmod{q}(\vec{p}_0) = \vec{h}_{\tilde t_1}^\transpose\cdot \vec{p}_{1} + \vec{h}_{\tilde t_1 \xor 1}^\transpose \cdot \vec{s}_{1}\enspace.
\]
Hence, we may treat this condition as the special case $j=1$ of the condition $\projmod{q}(\vec{p}_{j-1}) = \vec{h}_{\tilde t_j}^\transpose\cdot \vec{p}_{j} + \vec{h}_{\tilde t_j \xor 1}^\transpose \cdot \vec{s}_{j}$.
\eprint{%
\item Let $\limbs \coloneqq \ceil{\log_{2\eta+1} q}$.
The inequality checks in the definition of $\verify$ all compare elements from $\ring_q$ and are to be taken modulo $q$. By contrast, the norm-bounds are to be taken in $\ring$.
For the individual verification, the condition that $\norm{\projring(\vec{p}_j)} \leq \tfrac{q\beta}{2\eta}$ boils down to $\norm{\projring(\vec{p}_j)} \leq \tfrac{q}{2}$.
This is trivially satisfied by any decomposition of an element from $\ring_q$ and just means that $\vec{p}_j = \decompmod{q}(\projmod{q}(\vec{p}_j))$.
If we did not require this, a dishonestly generated signature could choose $\vec{p}_j$ as the decomposition of an element whose coefficients are not in $\{-\tfrac{q-1}{2},\ldots,\tfrac{q-1}{2}\}$, but still bounded by $\frac{(2\eta+1)^\limbs-1}{2}$.
% This may be possible because $q \neq (2\eta+1)^\limbs$.
In particular, if $q$ is significantly smaller than $(2\eta+1)^\limbs$, adding this condition actually gives a stronger shortness bound for the most significant limbs of the decomposition.
These tighter bounds are not present in Squirrel or in the extended abstract of this work\cite{TODO}, but they significantly help to make our encoding of openings both more efficient and easier to analyze later.%
}
\cameraready{%
\item The full version of this work\cite{TODO} contains a encoding scheme to transmit decommitments more efficiently. This requires some extra bounds checks on $\projring(\vec{p}_j)$ and $\projring(\vec{s}_j)$ inside $\verify$, which are not present here.
}
\end{enumerate}
\end{remark}

% Theorem about HVC
\begin{theorem}\label{theo:veccom}
  Let $n,q,q',\alpha_w,\rho,\eta,\tau,\xi,\bagg$ be positive integers and $0 < \errorbound \leq 1$ such that $n$ is a power of two, $q,q'$ are prime, and 
\[
\bagg \geq \eta\sqrt{2\alpha_w\rho
  \bigl(
  \ln\tfrac{2n}{\errorbound} +  \ln(2\tau \ceil{\log_{2\eta+1}q} + \xi\lceil\log_{2\eta+1}q'\rceil \eprint{+ 2\tau})
  \bigr)}\enspace.
\]
% \gnote{Simplify formula by using $\oplen$?}
  Let $\ring_q,\ring_{q'}$ be the polynomial rings $\defringq$ and $\defringqq$ respectively.
%  Let $\alpha$ be the smallest integer, such that $\binom{n}{\alpha}\cdot 2^\alpha \geq 2^\secpar$.
  If the $\sis_{\ring,q,2\ceil{\log_{2\eta+1} q},4\bagg}$ problem and the $\sis_{\ring,q,\xi\ceil{\log_{2\eta+1} q'} ,4\bagg}$ problem are hard,
  then \eprint{$\hvcplain$}\cameraready{$\hvccamera$} is an individually correct, $(\rho,\tern_{\alpha_w},\errorbound)$-probabilistically homomorphic, robustly homomorphic, and position binding HVC for domain $\ring^{\xi}_{q'}$ and vector length $2^\tau$.
\end{theorem}
\begin{proof}
  The theorem follows from \autoref{lem:veccomcorrectness}, \autoref{lem:hvcprobhom}, \autoref{lem:hvcrobhom}, and \autoref{lem:hvcposbind} proven below.\qed
\end{proof}

% Lemma about individual correctness
\begin{lemma}\label{lem:veccomcorrectness}
  Let $n,q,q',\alpha_w,\rho,\eta,\tau,\xi,\bagg$ be positive integers and $0 < \errorbound \leq 1$, such that $n$ is a power of two, $q,q'$ are prime.
  Let $\ring_q,\ring_{q'}$ be the polynomial rings $\defringq$ and $\defringqq$ respectively.
  Then \eprint{$\hvcplain$}\cameraready{$\hvccamera$} is an individually correct HVC for domain $\ring^\xi_{q'}$ and vector length $2^\tau$.
\end{lemma}
\begin{proof}
Let $\vec{m} \in (\ring_{q'}^{\xi})^{2^\tau}$, $\vec{c} = \commit(\params,\vec{m})$, $t\in[2^\tau]$, $(\vec{p}_1,\dots,\vec{p}_{\tau},\vec{s}_1, \dots, \vec{s}_{\tau},\vec{u})^\transpose = \open(\params,\vec{c},\vec{m},t)$. Let $\vec{p}_{0}, \tilde{t}$ be as in the definition of $\commit$.
We first observe that for all $j\in\{1,\ldots,\tau\}$ it holds in $\ring_q$ that
\begin{align*}
  \projmod{q}(\vec{p}_{j-1})
  ={}&\projmod{q}\left(\lbl_{\vec{g},\vec{h}_0,\vec{h}_1}(\vec{m},\tilde t_{< j})\tag{Def.\ of $\commit$ and $\open$}\right)\\
  ={}&\projmod{q}\left(\decompmod{q}\left(
      \begin{aligned}
        &\vec{h}_0^\transpose \cdot\lbl_{\vec{g},\vec{h}_0,\vec{h}_1}(\vec{m},\tilde t_{< j}\Vert 0)\\
        {}+ &\vec{h}_1^\transpose \cdot\lbl_{\vec{g},\vec{h}_0,\vec{h}_1}(\vec{m},\tilde t_{< j}\Vert 1)
      \end{aligned}
    \right)\right)\tag{\autoref{def:label}}\\
  ={}&\vec{h}_0^\transpose \cdot\lbl_{\vec{g},\vec{h}_0,\vec{h}_1}(\vec{m},\tilde t_{< j}\Vert 0) + \vec{h}_1^\transpose \cdot\lbl_{\vec{g},\vec{h}_0,\vec{h}_1}(\vec{m},\tilde t_{< j}\Vert 1)\tag{\autoref{prop:projanddecomp}}\\
  ={}&\vec{h}_{\tilde t_{j}}^\transpose \cdot\lbl_{\vec{g},\vec{h}_0,\vec{h}_1}(\vec{m},\tilde t_{< j}\Vert \tilde t_{j}) + \vec{h}_{\tilde t_{j}\xor 1}^\transpose \cdot\lbl_{\vec{g},\vec{h}_0,\vec{h}_1}(\vec{m},\tilde t_{< j}\Vert (\tilde t_{j}\xor 1))\\
  ={}&\vec{h}_{\tilde t_{j}}^\transpose \cdot\vec{p}_{j} + \vec{h}_{\tilde t_{j}\xor 1}^\transpose \cdot\vec{s}_{j}\tag{Def.\ of $\open$}.
\end{align*}
Observe that for $j=1$, this gives $\vec{c} = \vec{h}_{\tilde t_1}^\transpose\cdot \vec{p}_{1} + \vec{h}_{\tilde t_1 \xor 1}^\transpose \cdot \vec{s}_{1}$ in $\ring_q$.
Further it holds that
\begin{align*}
  \projmod{q}(\vec{p}_{\tau})
  ={}&\projmod{q}(\lbl_{\vec{g},\vec{h}_0,\vec{h}_1}(\vec{m},\tilde t)\tag{Def.\ of $\commit$ and $\open$})\\
  ={}&\projmod{q}(\decompmod{q}(\vec{g}^\transpose\cdot \decompmod{q'}(\vec{m}_t)))\tag{\autoref{def:label}}\\
  ={}&\vec{g}^\transpose\cdot \decompmod{q'}(\vec{m}_t)\tag{\autoref{prop:projanddecomp}}\\
  ={}&\vec{g}^\transpose\cdot\vec{u}\tag{Def.\ of $\open$}.
\end{align*}
%
Therefore it only remains to check that the norm bounds are not violated.
% For every $j\in[\tau]$, $\vec{p}_j$ and $\vec{s}_j$ are outputs of the $\lbl$ function and thus in the range of $\decomp_q$.
For every $j\in\interval\tau$, $\vec{p}_j$ and $\vec{s}_j$ are outputs of the $\lbl_{\vec{g},\vec{h}_0,\vec{h}_1}$ function and thus, by definition of $\lbl_{\vec{g},\vec{h}_0,\vec{h}_1}$, decompositions of elements from $\ring_q$.
Similarly, $\vec{u}$ is the output of $\decompmod{q'}$, applied to a vector of elements from $\ring_{q'}$.
By design, this implies that the resulting coefficients are in $\{-\eta,\dots,\eta\}$ and so the norm of each $\vec{p}_j$ and $\vec{s}_j$ as well as $\vec{u}$ is at most $\eta$.
\eprint{%
It also implies that applying $\projring$ to $\vec{p}_j$ or $\vec{s}_j$ gives back the representative with coefficients in $\{-\tfrac{q-1}{2},\ldots,\tfrac{q-1}{2}\}$ that was decomposed.
Consequently, we have $\norm{\projring(\vec{p}_j)}, \norm{\projring(\vec{s}_j)} \leq \tfrac{q-1}{2}$.%
}
\qed
\end{proof}


% By a union bound it is thus sufficient to show that each individual coefficient violates the bound with probability at most $2^{-\errorbound}/(n(2\tau\lceil\log_{2\eta+1} q\rceil + \xi\lceil\log_{2\eta+1} q'\rceil)+2\tau)$.
%   
%   For each individual $\vec{p}^i$, $\vec{s}^i$, $\vec{u}^i$ it holds by the definition of $\iverify$ that
%   \[
%     \norm{\vec{p}^i} \leq \eta, \quad \norm{\vec{s}^i} \leq \eta,\quad \text{and} \quad\norm{\vec{u}^i}\leq \eta.
%   \]
%   For each $\proj(\vec{p}_j^i)$ and $\proj(\vec{s}_j^i)$, by the definition of $\iverify$, we have
%   \[
%    \norm{\proj(\vec{p}_j^i)} < \tfrac{q}{2},\quad\text{and} \quad\norm{\proj(\vec{s}_j^i)} < \tfrac{q}{2}
%   \]
% 
%   Recall that each $w^i$ is a ternary polynomial with weight $\alpha_w$.
%   Therefore, each coefficient is a sum of the form
%   \(
%     \sum_{j=1}^{\alpha_w\ell}b_j c_j
%   \)
%   where $\abs{c_j}\leq \eta$ and $b_j$ is chosen uniformly from $\{-1,1\}$.
%   By linearity of expectation, the expected value of this sum is always zero and changing any summand can vary the sum by at most $2\eta$. We can thus apply McDiarmid's inequality~\cite{McDiarmid89} and the lower bound on $\bagg$ from the lemma statement to obtain the following bound on the probability that each individual coefficient exceeds the norm bound $\bagg$ for those cases where the bound is $\bagg$.


% Lemma about probabilistic homomorphism
\begin{lemma}\label{lem:hvcprobhom}
  Let $n,q,q',\alpha_w,\rho,\eta,\tau,\xi,\bagg$ be positive integers and $0 < \errorbound \leq 1$, such that $n$ is a power of two, $q,q'$ are prime,
%   $\limbs\coloneqq \ceil{\log_{2\eta+1}q}$, $\limbs'\coloneqq \ceil{\log_{2\eta+1}q'}$
and
  \[
  \bagg \geq \eta\sqrt{2\alpha_w\rho
  \bigl(
  \ln\tfrac{2n}{\errorbound} +  \ln(2\tau\kappa  + \xi\kappa' \eprint{+ 2\tau})
  \bigr)}\enspace,
  \]
  where $\kappa=\ceil{\log_{2\eta+1}q}$ and $\kappa'=\lceil\log_{2\eta+1}q'\rceil$.
  Let $\ring_q,\ring_{q'}$ be the polynomial rings $\defringq$ and $\defringqq$ respectively.
  Then \eprint{$\hvcplain$}\cameraready{$\hvccamera$} is a $(\rho,\tern_\alpha,\errorbound)$-probabilistically homomorphic HVC for domain $\ring^\xi_{q'}$ and vector length $2^\tau$.
\end{lemma}
\begin{proof}
Let $\params \gets \setup(\secparam)$, $\vec{c}^i \in \ring_q^\limbs$,
$1 \leq t \leq 2^\tau$, $\tilde{t} = \binsca(t-1)$,
% $t\in\interval{2^\tau}$, $\tilde{t} = \binsca(t-1)$,
$\vec{d}^i = (\vec{p}^i_1,\dots,\vec{p}^i_{\tau},\vec{s}^i_1, \dots, \vec{s}^i_{\tau},\vec{u})^\transpose \in \bigl(\ring^{\lceil\log_{2\eta+1} q\rceil}\bigr)^{2\tau} \times \ring^{\xi\lceil\log_{2\eta+1} q'\rceil}$
with $\iverify(\params,\vec{c}^i,t,\vec{d}^i) = \vec{m}_t^i \neq \bot$
as specified in \autoref{def:hvc}.
%
We first note that even for arbitrary $w^1,\dots,w^{\ell}\in\tern_{\alpha_w}$ it holds for all $2\leq j\leq \tau$ that
\begin{align*}
  \projmod{q}\Bigl(\sum_{i=1}^{\ell}w^i\cdot\vec{p}_{j-1}^i\Bigr) = &\sum_{i=1}^{\ell}w^i\cdot\projmod{q}(\vec{p}_{j-1}^i)\tag{\autoref{prop:projanddecomp}}\\
  ={}&\sum_{i=1}^{\ell}w^i\cdot(\vec{h}_{\tilde t_j}^\transpose\cdot \vec{p}^i_{j} + \vec{h}_{\tilde t_j\xor 1}^\transpose \cdot \vec{s}^i_{j})\tag{Def.\ of $\iverify$}\\
  ={}&\sum_{i=1}^{\ell}\vec{h}_{\tilde t_j}^\transpose\cdot w^i \vec{p}^i_{j} + \vec{h}_{\tilde t_j\xor 1}^\transpose\cdot w^i  \vec{s}^i_{j}\\  
  ={}& \vec{h}_{\tilde t_j}^\transpose\cdot \Bigl(\sum_{i=1}^{\ell}w^i\cdot\vec{p}^i_{j}\Bigr) + \vec{h}_{\tilde t_j \xor 1}^\transpose \cdot \Bigl(\sum_{i=1}^{\ell}w^i\cdot\vec{s}^i_{j}\Bigr).
\end{align*}
and similarly
\begin{align*}
  \projmod{q}\Bigl(\sum_{i=1}^{\ell}w^i\cdot\vec{p}_{\tau}^i\Bigr) ={}&\sum_{i=1}^{\ell}w^i\cdot\projmod{q}(\vec{p}_{\tau}^i)\tag{\autoref{prop:projanddecomp}}\\
  ={}&\sum_{i=1}^{\ell}w^i\cdot(\vec{g}^\transpose\cdot \vec{u}^i)\tag{Def.\ of $\iverify$}\\
  ={}&\vec{g}^\transpose\cdot\sum_{i=1}^{\ell}w^i\vec{u}^i 
\end{align*}
and similarly that 
\begin{align*}
  \sum_{i=1}^{\ell}w^i\cdot \vec{c}^i \equiv{}&\sum_{i=1}^{\ell}w^i\cdot \vec{h}_{\tilde t_1}^\transpose\cdot \vec{p}_{1}^i + \vec{h}_{\tilde t_1 \xor 1}^\transpose \cdot \vec{s}_{1}^i\\
  \equiv{}&\vec{h}_{\tilde t_1}^\transpose\cdot \sum_{i=1}^{\ell}w^i\cdot \vec{p}_{1}^i + \vec{h}_{\tilde t_1 \xor 1}^\transpose \cdot \sum_{i=1}^{\ell}w^i\cdot \vec{s}_{1}^i
\end{align*}
Therefore it only remains to verify that the norm-checks go through with sufficient probability.
Writing out the conditions, this means that we need to show that

\eprint{%
\begin{align*}
    P\coloneqq\Pr\Bigl[
      w^1,\dots,w^{\ell} \gets \tern_{\alpha_w}\colon\;
      \exists j\in\interval{\tau}\ldotp{}&
      \bigl\Vert\sum_{i=1}^{\ell}w^i\cdot \vec{p}_j^i\bigr\Vert > \bagg 
      \lor 
      \bigl\Vert\sum_{i=1}^{\ell}w^i\cdot \vec{s}_j^i\bigr\Vert > \bagg
      \lor
      {}\\{}&
      \bigl\Vert\sum_{i=1}^{\ell}w^i\cdot\vec{u}^i\bigr\Vert > \bagg
      \lor
      \bigl\Vert\sum_{i=1}^{\ell}w^i\cdot\projring(\vec{p}_j^i)\bigr\Vert > \tfrac{q\bagg}{2\eta}
      \lor
      {}\\{}&
      \bigl\Vert\sum_{i=1}^{\ell}w^i\cdot\projring(\vec{s}_j^i)\bigr\Vert > \tfrac{q\bagg}{2\eta}
    \Bigr] \leq \errorbound\enspace.
\end{align*}
}
\cameraready{%
\begin{align*}
    P\coloneqq\Pr\Bigl[
      w^1,\dots,w^{\ell} \gets \tern_{\alpha_w}\colon\;
      \exists j\in\interval{\tau}\ldotp{}&
      \bigl\Vert\sum_{i=1}^{\ell}w^i\cdot \vec{p}_j^i\bigr\Vert > \bagg
      \lor
      \bigl\Vert\sum_{i=1}^{\ell}w^i\cdot \vec{s}_j^i\bigr\Vert > \bagg
      \lor
      {}\\{}&
      \bigl\Vert\sum_{i=1}^{\ell}w^i\cdot\vec{u}^i\bigr\Vert > \bagg
    \Bigr] \leq \errorbound\enspace.
\end{align*}
}
Observe that this is an $\norm{.}_\infty$-bound for a total of
\[
  N_\textrm{bounds} \coloneqq \tau\ell \kappa + \tau\ell\kappa + \ell\xi\kappa' \eprint{+ \tau\ell + \tau\ell}
\]
many ring elements.
For each of the $N_\textrm{bounds}$ ring elements, we can individually apply \autoref{lem:normgrowth} with the same growth factor $\zeta =\tfrac{\bagg}{\eta}$.
Taking a $N_\textrm{bounds}$-fold union bound then gives
\[
 P \leq N_{\textrm{bounds}}\cdot 2n\exp\Bigl(-\frac{\bagg^2}{2\eta^2\alpha_w\rho}\Bigr)\enspace.
\]
Our condition on $\bagg$ is chosen exactly to guarantee that $\tfrac{\bagg^2}{2\eta^2\alpha_w\rho} \geq \ln\bigl(2nN_{\textrm{bounds}}\cdot \tfrac{1}{\errorbound} \bigr)$.
This gives $P\leq \errorbound$.
It follows that with probability at least $1-\errorbound$, the strong verification algorithm outputs
  \begin{align*}
    \projmod{q'}\bigl(\sum_{i=1}^{\ell}w^i \cdot \vec{u}^i\bigr)
    ={}&\sum_{i=1}^{\ell}w^i \cdot \projmod{q'}(\vec{u}^i) \tag{\autoref{prop:projanddecomp}}\\
    ={}&\sum_{i=1}^{\ell}w^i \cdot \iverify(\params,\vec{c}^i,t,\vec{d}^i)\tag{Def.\ of $\iverify$}\\
    ={}&\sum_{i=1}^{\ell}w^i \cdot \vec{m}_t^i\enspace,
  \end{align*}
  as required.%
  \qed
 
% To bound this probability, consider that the norm-bound is violated, iff the absolute value of at least one of the
% $n(2\tau\lceil\log_{2\eta+1} q\rceil  + \xi\lceil\log_{2\eta+1} q'\rceil + 2\tau)$
% coefficients in one of the sums 
% $\sum_{i=1}^{\ell} w^i\cdot\vec{p}^i$,
% $\sum_{i=1}^{\ell} w^i\cdot\vec{s}^i$,
% $\sum_{i=1}^{\ell} w^i\cdot\vec{u}^i$,
% $\sum_{i=1}^{\ell} w^i\cdot\proj(\vec{p}_j^i)$, and
% $\sum_{i=1}^{\ell} w^i\cdot\proj(\vec{s}_j^i)$
% is greater than its appropriate bound $\bagg$.
% By a union bound it is thus sufficient to show that each individual coefficient violates the bound with probability at most $2^{-\errorbound}/(n(2\tau\lceil\log_{2\eta+1} q\rceil + \xi\lceil\log_{2\eta+1} q'\rceil)+2\tau)$.
%   
%   For each individual $\vec{p}^i$, $\vec{s}^i$, $\vec{u}^i$ it holds by the definition of $\iverify$ that
%   \[
%     \norm{\vec{p}^i} \leq \eta, \quad \norm{\vec{s}^i} \leq \eta,\quad \text{and} \quad\norm{\vec{u}^i}\leq \eta.
%   \]
%   For each $\proj(\vec{p}_j^i)$ and $\proj(\vec{s}_j^i)$, by the definition of $\iverify$, we have
%   \[
%    \norm{\proj(\vec{p}_j^i)} < \tfrac{q}{2},\quad\text{and} \quad\norm{\proj(\vec{s}_j^i)} < \tfrac{q}{2}
%   \]
% 
%   Recall that each $w^i$ is a ternary polynomial with weight $\alpha_w$.
%   Therefore, each coefficient is a sum of the form
%   \(
%     \sum_{j=1}^{\alpha_w\ell}b_j c_j
%   \)
%   where $\abs{c_j}\leq \eta$ and $b_j$ is chosen uniformly from $\{-1,1\}$.
%   By linearity of expectation, the expected value of this sum is always zero and changing any summand can vary the sum by at most $2\eta$. We can thus apply McDiarmid's inequality~\cite{McDiarmid89} and the lower bound on $\bagg$ from the lemma statement to obtain the following bound on the probability that each individual coefficient exceeds the norm bound $\bagg$ for those cases where the bound is $\bagg$.
%   \begin{align*}
%     &\Pr\Bigl[\vec{b}\gets\{-1,1\}^{\alpha_w\ell} : \Bigl|\smashoperator{\sum_{j=1}^{\alpha_w\ell}}b_j c_j\Bigr| > \bagg\Bigr]\\
%     \leq{}& 2\cdot\exp\Bigl(-\frac{2\bagg^2}{\alpha_w\ell\cdot (2\eta)^2}\Bigr)\\
%     ={}& 2\cdot\exp\Bigl(-\frac{\bagg^2}{2\alpha_w\rho \eta^2}\Bigr)\\
%     \leq{}& 2\cdot\exp\Bigl(-\frac{\eta^2 2\alpha_w\rho(\errorbound + 1 +\log_2 n + \log_2(2\tau \lceil\log_{2\eta+1}q\rceil + \xi\lceil\log_{2\eta+1}q'\rceil))\cdot\ln2}{2\alpha_w\rho \eta^2}\Bigr)\\
%     ={}& 2\cdot 2^{-(\errorbound + 1 +\log_2 n + \log_2(2\tau \lceil\log_{2\eta+1}q\rceil + \xi\lceil\log_{2\eta+1}q'\rceil))\cdot}\\
%     ={}& 2^{-\errorbound}\cdot\frac{1}{n\cdot(2\tau \lceil\log_{2\eta+1}q\rceil + \xi\lceil\log_{2\eta+1}q'\rceil)}.
%   \end{align*}
%   %
%   Since only the ratio between $\bagg$ and $\eta$ enters here, the exact same argument applies to the bounds by $\tfrac{q\bagg}{2\eta}$.
%   
\end{proof}

\begin{lemma}\label{lem:hvcrobhom}
  Let $n,q,q',\alpha_w,\rho,\eta,\tau,\xi,\bagg$ be positive integers and $0<\errorbound\leq 1$, such that $n$ is a power of two, $q,q'$ are prime.
  Let $\ring_q,\ring_{q'}$ be the polynomial rings $\defringq$ and $\defringqq$ respectively.
  Then \eprint{$\hvcplain$}\cameraready{$\hvccamera$} is a robustly homomorphic HVC.
\end{lemma}
\begin{proof}
The proof of this lemma is taken almost verbatim from \cite{CCS:FleSimZha22}.
It deviates only insofar as the full construction and proof was split in two in \cite{CCS:FleSimZha22}, whereas it is combined in one here.
Since the proof is short, we include it here for the sake of completeness. %\gnote{Proof needs update.}
Let $\vec{c}^0,\vec{c}^1 \in \ring_q^\comlen$, and $\vec{d}^0, \vec{d}^1 \in \ring^\oplen$, and 
$1\leq t \leq 2^\tau$, $\tilde{t} = \binsca(t-1)$
be arbitrary, such that
\begin{equation}
    \sverify(\params,\vec{c}^0, t,\vec{d}^0)=\vec{m}^0 \quad \text{and} \quad \sverify(\params,\vec{c}^1, t,\vec{d}^1)=\vec{m}^1\label{eq:outputofsverify}
\end{equation}
with $\vec{m}^0,\vec{m}^1\neq \bot$.
Let $\vec{d}^i$ parse as $(\vec{p}^i_1,\dots,\vec{p}^i_{\tau},\vec{s}^i_1, \dots, \vec{s}^i_{\tau},\vec{u}^i)^\transpose$ for $i\in\bin$.
We first note that \emph{if} $\wverify(\params,\vec{c}^0-\vec{c}^1, t,\vec{d}^0-\vec{d}^1)\neq\bot$, then it holds in $\ring_{q'}^\xi$ that
\begin{align*}
  &\wverify(\params,\vec{c}^0-\vec{c}^1, t,\vec{d}^0-\vec{d}^1)\\
  ={}&\projmod{q'}(\vec{u}^0-\vec{u}^1)\tag{Def of $\sverify$}\\
  ={}&\projmod{q'}(\vec{u}^0)-\projmod{q'}(\vec{u}^1)\tag{\autoref{prop:projanddecomp}}\\
  ={}&\sverify(\params,\vec{c}^0, t,\vec{d}^0)-\sverify(\params,\vec{c}^1, t,\vec{d}^1)\tag{Def.\ of $\sverify$}\\
  ={}&\vec{m}^0- \vec{m}^1 \tag{\autoref{eq:outputofsverify}}.
\end{align*}
%
It thus remains to show that $\wverify(\params,\vec{c}^0-\vec{c}^1, t,\vec{d}^0-\vec{d}^1)\neq\bot$.
For this, let further $\vec{p}^i_0 = \vec{c}^i$.
By definition of the strong verification algorithm, and since $\vec{m}^0,\vec{m}^1\neq\bot$ it holds that for $i\in\bin$ 
and $j \in \interval{\tau}$ that the following two conditions hold
\begin{align}
  \norm{\vec{p}^i_{j}} \leq \bagg\quad\text{and}\quad
  \norm{\vec{s}^i_{j}} \leq \bagg\label{eq:robhomnormcheck}\\
  \projmod{q}(\vec{p}^i_{j-1}) = \vec{h}_{\tilde t_j}^\transpose\cdot \vec{p}^i_{j} + \vec{h}_{\tilde t_j\xor 1}^\transpose \cdot \vec{s}^i_{j}\enspace.\label{eq:robhompathcheck}
  \end{align}
  Similarly it holds that
  \begin{align}
  \norm{\vec{u}^i} \leq \bagg\quad \text{and}\quad \projmod{q}(\vec{p}^i_{\tau}) = \vec{g}^\transpose\cdot \vec{u}^i\label{eq:robhompaycheck}\enspace.
  \end{align}
  \eprint{%
  We also get the bounds on the projections
  \begin{align}
  \norm{\projring(\vec{p}_j^i)}\leq \frac{q\bagg}{2\eta}\quad\text{and}\quad \norm{\projring(\vec{s}_j^i)}\leq \frac{q\bagg}{2\eta}\enspace.\label{eq:robhomprojcheck}
  \end{align}
  }
From \autoref{eq:robhomnormcheck} and \autoref{eq:robhompaycheck} it follows that for all $j \in \interval{\tau}$
  \begin{align*}
    \norm{\vec{p}_j^0-\vec{p}_j^1} \leq& \norm{\vec{p}_j^0} + \norm{\vec{p}_j^1} \leq 2\bagg\\
    \norm{\vec{s}_j^0-\vec{s}_j^1} \leq& \norm{\vec{s}_j^0} + \norm{\vec{s}_j^1} \leq 2\bagg
  \end{align*}
  and
  \[
      \norm{\vec{u}^0-\vec{u}^1} \leq \norm{\vec{u}^0} + \norm{\vec{u}^1} \leq 2\bagg\enspace.
  \]
  \eprint{%
  From \autoref{eq:robhomprojcheck} and linearity of $\projring$, it follows that
\begin{align*}
 \norm{\projring(\vec{p}_j^0 - \vec{p}_j^1)} = \norm{\projring(\vec{p}_j^0)-\projring(\vec{p}_j^1)} \leq& \norm{\projring(\vec{p}_j^0)} + \norm{\projring(\vec{p}_j^1)} \leq \tfrac{q\bagg}{\eta}\\
 \norm{\projring(\vec{s}_j^0 - \vec{s}_j^1)} = \norm{\projring(\vec{s}_j^0)-\projring(\vec{s}_j^1)} \leq& \norm{\projring(\vec{s}_j^0)} + \norm{\projring(\vec{s}_j^1)} \leq \tfrac{q\bagg}{\eta}\enspace.
\end{align*}%
}
%
By Equations \ref{eq:robhompathcheck} and \ref{eq:robhompaycheck} and the linearity of $\projmod{q}$ it follows that for all $j \in \interval{\tau}$, it holds in $\ring_q$ that
  \begin{align*}
    \projmod{q}(\vec{p}^0_{j-1}-\vec{p}^1_{j-1})
    ={} &\projmod{q}(\vec{p}^0_{j-1})-\projmod{q}(\vec{p}^1_{j-1})\tag{\autoref{prop:projanddecomp}}\\
    ={} &(\vec{h}_{\tilde t_j}^\transpose\cdot \vec{p}^0_{j} + \vec{h}_{\tilde t_j\xor 1}^\transpose \cdot \vec{s}^0_{j})- (\vec{h}_{\tilde t_j}^\transpose\cdot \vec{p}^1_{j} + \vec{h}_{\tilde t_j\xor 1}^\transpose \cdot \vec{s}^1_{j})\tag{\autoref{eq:robhompathcheck}}\\
    ={} &\vec{h}_{\tilde t_j}^\transpose\cdot (\vec{p}^0_{j} - \vec{p}^1_{j}) + \vec{h}_{\tilde t_j\xor 1}^\transpose \cdot (\vec{s}^0_{j} - \vec{s}^1_{j})\enspace.
  \end{align*}
  and
  \begin{align*}
    \projmod{q}(\vec{p}^0_{\tau}-\vec{p}^1_{\tau})
    ={} &\projmod{q}(\vec{p}^0_{\tau})-\projmod{q}(\vec{p}^1_{\tau})\tag{\autoref{prop:projanddecomp}}\\
    ={} &(\vec{g}^\transpose\cdot \vec{u}^0 - \vec{g}^\transpose \cdot \vec{u}^1)\tag{\autoref{eq:robhompaycheck}}\\
    ={} &\vec{g}^\transpose\cdot (\vec{u}^0 - \vec{u}^1)\enspace.
  \end{align*}
  Thus, all checks in the weak verification algorithm go through and $\wverify(\params,\vec{c}^0-\vec{c}^1, t,\vec{d}^0-\vec{d}^1)\neq\bot$.\qed
\end{proof}

\begin{lemma}\label{lem:hvcposbind}
  Let $n,q,q',\alpha_w,\rho,\eta,\tau,\xi,\bagg$ be positive integers and $0 < \errorbound \leq 1$, such that $n$ is a power of two, $q,q'$ are prime.
  Let $\ring_q,\ring_{q'}$ be the polynomial rings $\defringq$ and $\defringqq$ respectively.
  If the $\sis_{\ring,q,2\normalceil{\log_{2\eta+1} q},4\bagg}$ problem and the $\sis_{\ring,q,\xi\normalceil{\log_{2\eta+1} q'},4\bagg}$ problem are hard, then \eprint{$\hvcplain$}\cameraready{$\hvccamera$} is position binding.
\end{lemma}
\begin{proof}
This proof once again follows very closely the proof shown in \cite{CCS:FleSimZha22}.
% \gnote{Needs update to get equalities vs. equalities mod $q$ correct}
We will prove this lemma by leveraging that any pair of valid decommitments for different messages will lead to a collision somewhere in the generalized hash tree, which can be turned into a solution for one of the SIS instances. 

  Let $\adv$ be an arbitrary PPT adversary against the position binding property of the construction.
%  
  By the law of total probability it holds that
  \begin{align*}
    &\Pr[\vec{m}_0\neq \vec{m}_1 \land \bot\not\in\{\vec{m}_0,\vec{m}_1\}]\\
    ={}& \Pr[\vec{m}_0\neq \vec{m}_1 \land \bot\not\in\{\vec{m}_0,\vec{m}_1\} \land \projmod{q}(\vec{p}_\tau^0) = \projmod{q}(\vec{p}_\tau^1)]\\& 
    +
    \Pr[\vec{m}_0\neq \vec{m}_1 \land \bot\not\in\{\vec{m}_0,\vec{m}_1\} \land \projmod{q}(\vec{p}_\tau^0) \neq \projmod{q}(\vec{p}_\tau^1)]\enspace.
  \end{align*}
%
  We now bound the two probabilities separately.
%  
  \begin{align*}
    &\Pr[\vec{m}_0\neq \vec{m}_1 \land \bot\not\in\{\vec{m}_0,\vec{m}_1\} \land \projmod{q}(\vec{p}_\tau^0) = \projmod{q}(\vec{p}_\tau^1)]\\
    \leq{}&\Pr[\projmod{q'}(\vec{u}^0) \neq \projmod{q'}(\vec{u}^1) \land \vec{g}^\transpose\cdot \vec{u}^0 = \vec{g}^\transpose\cdot \vec{u}^1 \land \norm{\vec{u}^0}\leq 2\bagg \land \norm{\vec{u}^1}\leq 2\bagg]\tag{Def.\ of $\wverify$}\\
    \leq{}& \Pr[\vec{u}^0 \neq \vec{u}^1 \land \vec{g}^\transpose\cdot (\vec{u}^0-\vec{u}^1) = 0 \land \norm{\vec{u}^0-\vec{u}^1}\leq 4\bagg]\\
    = {}& \Pr[(\vec{u}^0-\vec{u}^1) \in \ball{4\bagg}{q}^{\xi\normalceil{\log_{2\eta+1}q'}}\setminus \{\vec{0}\} \land \vec{g}^\transpose\cdot (\vec{u}^0-\vec{u}^1) = 0]\\
    \leq{}& \negl\enspace,
  \end{align*}
  where the last inequality follows from the assumed hardness of the $\sis_{\ring,q,\xi\normalceil{\log_{2\eta+1} q'},4\bagg}$ problem and the fact that all involved algorithms are PPT.
  
  We now analyze 
  \[
    \Pr[\vec{m}_0\neq \vec{m}_1 \land \bot\not\in\{\vec{m}_0,\vec{m}_1\} \land \vec{p}_\tau^0 \bmod q \neq \vec{p}_\tau^1 \bmod q]\enspace.
  \]
  We construct a PPT algorithm $\sisadv$ that solves the $\sis_{\ring,q,2 \normalceil{\log_{2\eta+1} q},4\bagg}$ problem as follows.
  Upon input $\vec{a}=(a_1,\dots,a_{2 \normalceil{\log_{2\eta+1} q}})^\transpose$,
  $\sisadv$ sets 
  $\vec{h}_0\coloneqq(a_1,\dots,a_{\normalceil{\log_{2\eta+1} q}})^\transpose$ and 
  $\vec{h}_1 \coloneqq (a_{\normalceil{\log_{2\eta+1} q}+1},\dots,\allowbreak a_{2\normalceil{\log_{2\eta+1} q}})^\transpose$,
  samples $\vec{g} \gets \ring_q^{\xi\normalceil{\log_{2\eta+1} q'}}$, sets $\params \coloneqq (\vec{g}, \vec{h}_0, \vec{h}_1)$ and runs 
  $(\vec{c}, t,\vec{d}^0,\vec{d}^1) \gets \adv(\params)$.
%   \gnote{Fixed typo, changing $t$ to $\tilde t$ here. Also renamed $\bdv$ to $\sisadv$ to avoid notational clash with ball. @Nils: Please double check I did not mess up.}
  For $i\in\bin$ let $m^i \coloneqq \wverify(\params,\vec{c},t,\vec{d}^i)$.
  If $\vec{m}^0 = \vec{m}^1$, $\bot\in\{\vec{m}^0,\vec{m}^1\}$, or $\projmod{q}(\vec{p}_\tau^0) = \projmod{q}(\vec{p}_\tau^1)$, $\sisadv$ aborts.
  Otherwise, parse $\vec{d}^i$ as $(\vec{p}^i_1,\dots,\vec{p}^i_{\tau},\vec{s}^i_1,\dots,\vec{s}^i_\tau,\vec{u}^i)$, set $\vec{p}^i_0 \coloneqq \decompmod{q}(\vec{c})$, $\tilde{t}\coloneqq \binsca(t-1)$.
  
  Let $j^*\in \interval{\tau+1}$ be the \emph{largest} index, such that $\projmod{q}(\vec{p}^0_{j^*-1})\neq\projmod{q}(\vec{p}^1_{j^*-1})$.
  Note that such an index always exists, since $\vec{p}^0_0 = \decompmod{q}(\vec{c}) = \vec{p}^1_0$, and that $j^* < \tau$, since $\projmod{q}(\vec{p}_\tau^0) \neq \projmod{q}(\vec{p}_\tau^1)$.
  If $\tilde t_{j^*-1} = 0$, $\sisadv$ outputs $
  \vec{z}\coloneqq(\vec{p}_{j^*}^0, \vec{s}_{j^*}^0)^\transpose - (\vec{p}_{j^*}^1, \vec{s}_{j^*}^1)^\transpose$, if $\tilde t_{j^*-1} = 1$, $\sisadv$ outputs $\vec{z}\coloneqq(\vec{s}_{j^*}^0, \vec{p}_{j^*}^0)^\transpose - (\vec{s}_{j^*}^1, \vec{p}_{j^*}^1)^\transpose$.
  
  We now analyze the success probability of $\sisadv$.
  It holds that $\projmod{q}(\vec{p}_{j^*-1}^0) = \projmod{q}(\vec{p}_{j^*-1}^1)$ and by the definition of the weak verification algorithm that
  \begin{align*}
    &\vec{h}_{\tilde t_{j^*}}^\transpose\cdot\vec{p}_{j^*}^0 + \vec{h}_{\tilde t_{j^*}\xor 1}^\transpose\cdot\vec{s}_{j^*}^0 = \vec{h}_{\tilde t_{j^*}}^\transpose\cdot\vec{p}_{j^*}^1 + \vec{h}_{\tilde t_{j^*}\xor 1}^\transpose\cdot\vec{s}_{j^*}^1\\
    \iff&\vec{h}_{\tilde t_{j^*}}^\transpose\cdot(\vec{p}_{j^*}^0-\vec{p}_{j^*}^1) + \vec{h}_{\tilde t_{j^*}\xor 1}^\transpose\cdot(\vec{s}_{j^*}^0-\vec{s}_{j^*}^1) = 0\\
    \iff&\vec{a}^\transpose\cdot\vec{z} = \vec{0} \enspace.
  \end{align*}
  It further holds by the definition of the weak verification algorithm that 
  \[
  \norm{\vec{p}_{j^*}^0} \leq 2\bagg,\quad \norm{\vec{s}_{j^*}^0} \leq 2\bagg, \quad\norm{\vec{p}_{j^*}^1} \leq 2\bagg, \quad \norm{\vec{s}_{j^*}^1} \leq 2\bagg\enspace.
  \]
  Therefore, the norm of $\vec{z}$ can be bounded as
  \[
  \norm{\vec{z}} \leq \max\{\norm{\vec{p}_{j^*}^0},\norm{\vec{s}_{j^*}^0}\}+\max\{\norm{\vec{p}_{j^*}^1},\norm{\vec{s}_{j^*}^1}\} \leq 4\bagg\enspace.
  \]
  It remains to show that $\vec{z}\neq 0$.
  Since $j^*$ is the \emph{largest} index such that 
  \[
    \projmod{q}(\vec{p}_{j^*-1}^0) =  \projmod{q}(\vec{p}_{j^*-1}^1)\enspace,
  \]
  it holds that
  \[
    \projmod{q}(\vec{p}_{j^*}^0) \neq \projmod{q}(\vec{p}_{j^*}^1)
  \]
  and thereby that
  \[
    \vec{p}_{j^*}^0 \neq \vec{p}_{j^*}^1\enspace.
  \]
  Therefore $\vec{z}\neq\vec{0}$.
  Thus, whenever $\adv$ is successful, $\sisadv$ is successful with probability $1$ and we can conclude that
  \begin{align*}
  \negl \geq{}& \Pr[
      \vec{a} \gets \ring_q^{2\normalceil{\log_{2\eta+1} q}}; \vec{z}\gets\sisadv(\vec{a}) \colon \vec{z}\in\ball{4\bagg}{q}^{2\normalceil{\log_{2\eta+1} q}}\setminus\{\vec{0}\} \land \vec{a}^\transpose\vec{z}\equiv 0
    ]\\
    ={}&
    \Pr[\vec{m}_0\neq \vec{m}_1 \land \bot\not\in\{\vec{m}_0,\vec{m}_1\} \land \vec{p}_\tau^0 \bmod q\neq \vec{p}_\tau^1\bmod q]\enspace.
  \end{align*}
  Combining the above, it follows that
  \[
    \Pr[\vec{m}_0\neq \vec{m}_1 \land \bot\not\in\{\vec{m}_0,\vec{m}_1\}] \leq \negl\enspace,
  \]
  as required.\qed
\end{proof}

\cameraready{%
\begin{remark}
Let $n, q, q',\alpha_w,\rho, \eta, \tau, \xi, \bagg$ be positive integers such that $n$ is a power of two and $q,q'$ prime.
Let us collect in \autoref{table:hvcsizescamera} the individual components of $\hvccamera$ and look at the bit-sizes of commitments and (individually or strongly verifying) openings as functions of the parameters.
Note that the strongly verifying case will correspond to the constribution for the size of aggregated signatures later in \autoref{sec:nidv}, and this size is the most important metric we want to minimize.

A commitment is a (non-short) single element from $\ring_q$. This means we can use $n\ceil{\log q}$ bits to store it.\footnote{In principle, we could do $\ceil{n\log q}$ by using some clever arithmetic encoding; hoever, for simplicity, we assume here that every coefficient is stored individually.}
An opening consists of $\oplen = 2\tau\ceil{\log_{2\eta+1}q} + \xi\ceil{\log_{2\eta+1}q'}$ many elements from $\ring$.
Each element is $\norm{.}_\infty$-bounded: for individually verifying openings, the bound is $\eta$, giving $n\oplen\ceil{\log (2\eta+1)}$ bits if done naively. For strongly verifying opening, the bound is $\bagg$, giving $n\oplen\ceil{\log (2\bagg+1)}$ bits. For honestly generated%
\footnote{%
We need here that $\projmod{q}(\vec{p}_j)$ uniquely determines $\vec{p}_j$. This is the case for \emph{honestly generated}, unaggregated commitments. Note that we require them to be honestly generated rather than individually verifying: a dishonest committer could choose some $\vec{p}_j$ with a coordinate $x$ that is not fully reduced modulo $q$ but still satisfies $\abs{\projring(x)} < \eta$. In that case, $\projmod{q}(\vec{p}_j)$ only determines $\vec{p}_j \bmod q$, which is not enough.%
}
unaggregated commitments, we can actually do better: as in a usual Merkle tree opening, we do not need the $\vec{p}_j$'s as part of the decommitments but only the $\vec{s}_j$'s, as the verifier can actually compute all $\projmod{q}(\vec{p}_j)$ by themselves and the $\projmod{q}(\vec{p}_j)$ uniquely determine the $\vec{p}_j$. This means that only $\bigl(\tau\limbs + \xi\limbs'\bigr)n\cdot \ceil{\log(2\eta+1)}$ bits need to be transmitted. In the full version\cite{TODO}, we present a generalization of this trick that can also be applied to aggregated decommitments.

\begin{table}\centering
 \begin{tabular}{cc@{\hskip 3.5ex}l}
  & & size in bits \\\toprule
  commitments & & $n\ceil{\log q}$\\
  \hline
  \multirow{2}{*}{opening} &individually verifying & $\bigl(\tau\limbs + \xi\limbs'\bigr)n\cdot \ceil{\log(2\eta+1)}$\\\cline{2-3}
                           &strongly verifying & $\bigl(2\tau\limbs + \xi\limbs'\bigr)n\cdot \ceil{\log(2\bagg+1)}$\\
                           \hline
 \end{tabular}
 \medskip % Alternatively, put caption on top (some style guides suggest this)
 \caption{\cameraready{Bitlength of our HVC construction $\hvccamera$. We denote by $\limbs = \ceil{\log_{2\eta+1}q}$ and $\limbs' = \ceil{\log_{2\eta+1}q'}$ the number of limbs for the decompositions of $\ring_q$ resp.\ $\ring_{q'}$ elements. Note that in the individually verifying case, we do not include the $\vec{p}_j$ here, but only the sibling nodes $\vec{s}_j$.}}
 \label{table:hvcsizescamera}
\end{table}
\end{remark}
}

% !TEX root = ../main.tex

\section{Key-Homomorphic One-Time Signatures}\label{sec:otms}

In this section, we define and instantiate the notion of a key-homomorphic one-time signature scheme that we will need in our final construction.
Intuitively, a one-time signature is unforgeable as long as at most one signature for some message is published under a given public key.
We call such a scheme key-homomorphic, if the a linear combination of separate signatures for the same message verifies under the linear combination of the corresponding public keys, while still being unforgeable.
We present a construction of this primitive, which is similar to previous one-time signature schemes by Boneh and Kim~\cite{BonKim2020} and Lyubashevsky and Micciancio~\cite{TCC:LyuMic08}.

\begin{definition}[One-Time Signature]\label{def:hots}
  Let $\ring$ be a ring.
  A key-homomorphic one-time signature scheme (KOTS) over $\ring$ with public key length $\opklen$ and signature length $\siglen$ is defined by four PPT algorithms $\hots=(\setup,\kgen,\sign,\verify)$.
  \begin{description}
    \item[$\params\gets\setup(\secparam)$] The setup algorithm takes as input the security parameter and outputs public parameters.
    \item[$(\osk,\opk) \gets \kgen(\params)$] The key generation algorithm takes as input the public parameters and outputs a key pair with $\opk\in\ring_q^\opklen$.
    \item[$\vec{\sigma} \gets \sign(\params,\osk,m)$] The signing algorithm takes as input the public parameters, a one-time signing key, and a message and outputs a signature $\sigma\in\ring_q^\siglen$.
    \item[$b\gets \iverify(\params,\opk,m,\sigma)$] The individual verification algorithm takes as input the public parameters, a verification key, a message, and a candidate signature and outputs a bit indicating acceptance/rejection.
    \item[$b\gets \sverify(\params,\opk,m,\sigma)$] The strong verification algorithm has the same input and output domains as the individual verification algorithm.
    \item[$b\gets \wverify(\params,\opk,m,\sigma)$] The weak verification algorithm has the same input and output domains as the individual verification algorithm..
  \end{description}
\end{definition}
\begin{definition}[Individual Correctness]
Let $\hots$ be a one-time signature scheme.
$\hots$ is individually correct, if for all security parameters $\secpar\in\NN$, parameters $\params \gets \setup(\secparam)$, key pairs $(\osk,\opk) \gets \kgen(\params)$, messages $m\in\bin^*$, and signatures $\vec{\sigma} \gets \sign(\params,\osk,m)$ it holds that
\[
  \iverify(\params,\opk,m,\sigma)=1
\]
\end{definition}

We require that individually verifying signatures can be homomorphically aggregated by computing a random linear combination of them.
Such aggregated signatures should still \emph{strongly} verify with high probability over the choice of the random linear combination.

\begin{definition}[Probabilistic Homomorphism]
  Let $\hots$ be a one-time signature scheme with public key length $\opklen$ and signature length $\siglen$.
  Let $\rho,\varepsilon \in \NN$ and $W\subseteq \ring$.
  $\hots$ is $(\rho,W,\varepsilon)$-probabilistically homomorphic, if 
  for all security parameters $\secpar\in\NN$, number of aggregated signatures $\ell\in[\rho]$, parameters $\params \gets \setup(\secparam)$, public keys $\opk^i \in \ring_q^\opklen$, messages $m\in\bin^*$ and signatures $\vec{\sigma}^i\in\ring_q^{\siglen}$ with $\iverify(\params,\opk^i,m,\vec{\sigma}^i)$ it holds that
  \[
    \Pr\mleft[
      w^1,\dots,w^\ell\gets W
      :
      \sverify(\params,\sum_{i=1}^{\ell}w^i\cdot\opk^i,m,\sum_{i=1}^{\ell}w^i\cdot\vec{\sigma}^i) = 1
    \mright]\geq 1-2^{-\varepsilon}
  \]
\end{definition}

As with the vector commitments from the previous section, we additionally require that a further limited homomorphism still holds, even for maliciously \emph{aggregated} signatures.
For any two, even maliciously generated, signatures that \emph{strongly} verify under potentially maliciously generated public keys, their difference will still \emph{weakly} verify.

\begin{definition}[Robust Homomorphism]
  \label{def:malhomhots}
  Let $\hots$ be a one-time signature scheme with public key length $\opklen$ and signature length $\siglen$.
  $\hots$ is robustly homomorphic if for all security parameters $\secpar\in\NN$, public parameters $\params\gets\setup(\secparam)$, messages $m\in\bin^*$, (possibly malformed) public keys $\opk^0,\opk^1 \in \ring_q^\opklen$, and (possibly malformed) signatures $\vec{\sigma}^0,\vec{\sigma}^1\in\ring_q^\siglen$ with
  \[
    \sverify(\params,\opk^0, m,\vec{\sigma}^0)=1 \quad \text{and} \quad \sverify(\params,\opk^1, m,\vec{\sigma}^1)=1
  \]
  it holds that
  \[
    \wverify(\params,\opk^0-\opk^1, m,(\vec{\sigma}^0-\vec{\sigma}^1))=1.
  \]
\end{definition}

We define a multi-user version of (one-time) existential unforgeability, this will allow for a tighter proof of the synchronized multi-signature scheme.
The definition is further strengthened by allowing the adversary to produce forgeries not just under one of the given public keys, but also under mildly rerandomized public key.

\begin{definition}[Multi-User Existential Unforgeability under Rerandomized Keys]
  A $(\rho,W,\epsilon)$-homomorphically correct KOTS is $W'$-existentially unforgeable under rerandomized keys (EUF-RK), if for all security parameters $\secpar$, any $T=\poly\in\NN$ and all stateful PPT algorithms $\adv$ it holds that
  \[
    \Pr\mleft[
      \begin{aligned}
      \params \gets \setup(\secparam);\\
      \forall i \in [T-1]\ldotp (\osk_i,\opk_i) \gets \kgen(\params);\\
      (i^*,m^*,\sigma^*,w^*) \gets \adv^{\widetilde\sign(\cdot,\cdot)}(\params,\opk_0,\dots,\opk_{T-1});\\
%      \sigma \gets \sign(\params,\osk,m);\\
%      (m^*,\sigma^*,w_0^*,w_1^*) \gets \adv(\sigma)\\
      \end{aligned}:
      \begin{aligned}
      \wverify(\params,w^*\cdot\opk_{i^*},m^*,\sigma^*) = 1\\
      \land m^*\not\in Q_{i^*} \land \abs{Q_{i^*}}\leq 1 \land w^* \in W'
      \end{aligned}
    \mright]\leq \negl,
  \]
  where the oracle $\widetilde\sign(\cdot,\cdot)$ is defined as $\widetilde\sign(i,m) := \sign(\osk_i,m)$ and $Q_i$ denotes the set of messages for which a signing query with index $i$ has been made.
\end{definition}

\begin{figure}
\centering
\begin{pcvstack}[center,boxed]
\begin{pchstack}[center]
  \procedure{$\setup(\secparam)$}{
    \vec{a} \gets \ring_q^\gamma\\
    \pcreturn \vec{a}
  }
  \pchspace
  \procedure{$\kgen(\params)$}{
    \vec{s}_0 \gets \ball_{\varphi}^\gamma\\
    \vec{s}_1 \gets \ball_{\varphi\cdot\alpha_H}^{\gamma}\\
    v_0 := \vec{a}^\intercal\cdot \vec{s}_0\\
    v_1 := \vec{a}^\intercal\cdot \vec{s}_1\\
    \pcreturn ((\vec{s}_0,\vec{s}_1)(v_0,v_1))
  }
  \pchspace
  \procedure{$\sign(\params,\osk,m)$}{
    \pcparse \osk \pcas (\vec{s}_0,\vec{s}_1)\\
    \vec{\sigma} := \vec{s}_0\cdot H(m)+\vec{s}_1\\
    \pcreturn \vec{\sigma}
  }
  \end{pchstack}
  \pchspace
  \begin{pchstack}
  \begin{pcvstack}
  \procedure{$\iverify(\params,\opk,m,\vec{\sigma})$}{
    \pcreturn \verify(\params,\opk,m,\vec{\sigma},2\varphi\alpha_H)
  }
  \pcvspace
  \procedure{$\sverify(\params,\opk,m,\vec{\sigma})$}{
    \pcreturn \verify(\params,\opk,m,\vec{\sigma},\beta_\sigma)
  }
  \pcvspace
  \procedure{$\wverify(\params,\opk,m,\vec{\sigma})$}{
    \pcreturn \verify(\params,\opk,m,\vec{\sigma},2\beta_\sigma)
  }
  \end{pcvstack}
  \pchspace
  \procedure{$\verify(\params,\opk,m,\vec{\sigma},\beta')$}{
    \pcparse \opk \pcas (v_0,v_1)\\
    \pcif \norm{\vec{\sigma}} > \beta'\\
    \quad \pcreturn 0\\
    \pcif \vec{a}^\intercal\cdot \vec{\sigma} \neq v_0\cdot H(m) + v_1\\
    \quad \pcreturn 0\\
    \pcreturn 1
  }
\end{pchstack}
\end{pcvstack}
\caption{Description of the key-homomorphic one-time signature scheme. $H$ is a collision-resistant hash function mapping bit-strings to $\tern_{\alpha_H}$.}
\label{fig:otsconstruction}
\end{figure}

\nnote{Describe how scheme is derived from the work of Boneh and Kim~\cite{BonKim2020}.}
\begin{theorem}\label{theo:kots}
Let $\secpar, \varepsilon, \alpha_w$, $\alpha_H$, $\delta$, $\varphi$, $\gamma$, $\rho$, $\beta_\sigma$, $n$, $q$ be integers such that $q$ is prime and $q > 16\alpha_w\alpha_H\varphi$, $n$ is a power of two, $2^{2\secpar}\leq \tern_{\alpha_H} \leq 2^{2\secpar+\delta}$, $\beta_\sigma \geq 4\varphi\alpha_H\sqrt{\tfrac{1}{2}\alpha_w\rho(\varepsilon+1+\log_2n\gamma)\cdot\ln2}$, and $\gamma\geq((3\secpar+\delta)/n+\log_2q)\log^{-1}_2(\varphi+\tfrac{1}{2})$.
Let $H : \bin^* \to \tern_{\alpha_H}$ be a hash function.
Let $W' = \{w_0-w_1\mid w_0,w_1\in\tern_{\alpha_w} \land w_0\neq w_1\}$.
If the $\sis_{\ring,q,\gamma,2\beta_\sigma+4\alpha_w\alpha_H\varphi}$ problem is hard and $H$ is collision resistant, then the construction from~\autoref{fig:otsconstruction} is an individually correct, $(\rho,\tern_{\alpha_w},\varepsilon)$-probabilistically homomorphic, robustly homomorphic KOTS that is $W'$-multi-user existentially unforgeable under rerandomized keys.
\end{theorem}
\begin{proof}
The proof closely follows the corresponding proof in \cite{CCS:FleSimZha22} with adapted parameters.
The theorem follows from \autoref{lem:kots_ind_correct}, \autoref{lem:kots_correct}, \autoref{lem:kots_homomorphic}, and \autoref{lem:kots_sis}.\qed
\end{proof}

The following four lemmas state that our construction satisfies the desired homomorphic properties and that it is unforgeable.
\begin{lemma}\label{lem:kots_ind_correct}
Let $\secpar, \varepsilon, \alpha_w$, $\alpha_H$, $\varphi$, $\gamma$, $\rho$, $\beta_\sigma,n,q$ be positive integers, such that $n$ is a power of two, $q$ is prime.
  Let $\ring_q$ be the polynomial ring $\ZZ_q[x]/\langle x^n+1\rangle$.
Let $H : \bin^* \to \tern_{\alpha_H}$ be a hash function.
  Then the construction from \autoref{fig:otsconstruction} is a individually correct one time signature scheme.
\end{lemma}
\begin{proof}
  Let $\params \gets \setup(\secparam)$, $(\osk,\opk) \gets \kgen(\params)$, $m\in\bin^*$ and $\vec{\sigma} \gets \sign(\params,\osk,m)$ be arbitrary.
  We first observe that the check on the \emph{value} of the signature goes through, as
  \begin{align*}
    \vec{a}^\intercal\vec{\sigma}
    ={}&\vec{a}^\intercal(\vec{s}_0\cdot H(m) + \vec{s}_1)\tag{Def. of $\sign$}\\
    ={}&\vec{a}^\intercal\vec{s}_0\cdot H(m) + \vec{a}^\intercal\vec{s}_1 \tag{Distributivity}\\
    ={}&v_0\cdot H(m) + v_1. \tag{Def. of $\kgen$}
  \end{align*}
  The signature also does not violate the norm bound, as
  \begin{align*}
    \norm{\vec{\sigma}}
    ={}&\norm{\vec{s}_0\cdot H(m) + \vec{s}_1} \tag{Def. of $\sign$}\\
    \leq{}&\norm{\vec{s}_0\cdot H(m)} + \norm{\vec{s}_1}\\
    \leq{}&\norm{\vec{s}_0}\cdot\norm{H(m)}_1 + \norm{\vec{s}_1} \tag{\autoref{lem:ternbound}}\\
    ={}&2\varphi\alpha_H \tag{Def. of $\kgen$}.
  \end{align*}
  The lemma thus follows.\qed
\end{proof}


\begin{lemma}\label{lem:kots_correct}
Let $\secpar, \varepsilon, \alpha_w$, $\alpha_H$, $\varphi$, $\gamma$, $\rho$, $\beta_\sigma$, $n$, $q$ be positive integers, such that \[\beta_\sigma \geq 4\varphi\alpha_H\sqrt{\tfrac{1}{2}\alpha_w\rho(\varepsilon+1+\log_2n\gamma)\cdot\ln2}.\]
  Let $\ring_q$ be the polynomial ring $\ZZ_q[x]/\langle x^n+1\rangle$.
  Let $H : \bin^* \to \tern_{\alpha_H}$ be a hash function.
  Then the construction from \autoref{fig:otsconstruction} is a $(\rho,\tern_{\alpha_w},\varepsilon)$-probabilistically homomorphic one time signature scheme.
\end{lemma}

\begin{proof}
  Let $\ell\in[\rho]$, $m\in\bin^*$, and $\params\gets\setup(\secparam)$ and for $i\in[\ell]$ let $\opk^i=(v_0,v_1) \in \ring_q^2$ and $\vec{\sigma}^i \in \ring_q^\gamma$ be arbitrary such that for all $i\in[\ell]$, $\iverify(\params,\opk^i,m,\vec{\sigma}^i)=1$.
  
  We first note that even for arbitrary $w_1,\dots,w_\ell \in \tern_\alpha$ it holds that
  \begin{align*}
    \vec{a}^\intercal\cdot \sum_{i=1}^{\ell-1}w^i\cdot\vec{\sigma}^i
    ={}&\sum_{i=1}^{\ell}w^i\cdot\vec{a}^\intercal\vec{\sigma}^i \tag{Distributivity}\\
    ={}&\sum_{i=1}^{\ell}w^i\cdot(v^i_0\cdot H(m)+v^i_1) \tag{Def. of $\iverify$}\\
%    ={}&\sum_{i=1}^{\ell}w^iv^i_0\cdot H(m)+ \sum_{i=1}^{\ell}w^iv^i_1 \tag{Distributivity}\\
    ={}&\Bigl(\sum_{i=1}^{\ell}w^iv^i_0\Bigr)\cdot H(m)+ \Bigl(\sum_{i=1}^{\ell}w^iv^i_1\Bigr) \tag{Distributivity}.
  \end{align*}
  
  Therefore, it only remains to verify that the norm-check goes through with sufficient probability.
  I.e., that
  \[
    \Pr\mleft[
      w^1,\dots,w^\ell\gets \tern_{\alpha_w}
      :
      \Bigl\Vert\sum_{i=1}^{\ell} w^i\cdot\vec{\sigma}^i\Bigr\Vert \leq \beta_\sigma\mright] \leq 2^{-\varepsilon}.
  \]
  To bound this probability, consider that the norm-bound is violated iff the absolute value of at least one of the $n\gamma$ coefficients in $\sum_{i=1}^{\ell} w^i\cdot\vec{\sigma}^i$ is greater than $\beta_\sigma$.
  By a union bound it is thus sufficient to show that each individual coefficient violates the bound with probability at most $2^{-\varepsilon}/(n\gamma)$
  
  For each individual $\vec{\sigma}^i$ it holds by the definition of $\iverify$ that
    $\norm{\vec{\sigma}^i} \leq 2\varphi\alpha_H$
  Therefore, each coefficient is a sum of the form
  \(
    \sum_{j=1}^{\alpha_w\ell}b_j c_j
  \)
  where $\abs{c_j}\leq 2\varphi\alpha_H$ and $b_j$ is chosen uniformly from $\{-1,1\}$.
  By linearity of expectation, the expected value of this sum is always zero and changing any summand can vary the sum by at most $4\varphi\alpha_H$. We can thus apply McDiarmid's inequality~\cite{McDiarmid89} and the lower bound on $\beta_\sigma$ from the lemma statement to obtain the following bound on the probability that each individual coefficient exceeds the norm bound $\beta_\sigma$:
  \begin{align*}
    \Pr\Bigl[\vec{b}\gets\{-1,1\}^{\alpha_w\ell} : \Bigl|\smashoperator{\sum_{j=1}^{\alpha_w\ell}}b_j c_j\Bigr| > \beta_\sigma\Bigr]
    \leq{}& 2\cdot\exp\Bigl(-\frac{2\beta_\sigma^2}{\alpha_w\ell\cdot (4\varphi\alpha_H)^2}\Bigr)\\
    \leq{}& 2\cdot\exp\Bigl(-\frac{2\beta_\sigma^2}{\alpha_w\rho\cdot (4\varphi\alpha_H)^2}\Bigr)\\
    \leq{}& 2\cdot\exp\Bigl(-\frac{2(4\varphi\alpha_H)^2\cdot\tfrac{1}{2}\alpha_w\rho(\varepsilon+1+\log_2n\gamma)\cdot\ln2}{\alpha_w\rho\cdot (4\varphi\alpha_H)^2}\Bigr)\\
    ={}& 2\cdot\exp(-(\varepsilon+1+\log_2n\gamma)\cdot\ln2)\\
    ={}& 2^{-\varepsilon-\log_2n\gamma} = 2^{-\varepsilon}\cdot \frac{1}{n\gamma}
  \end{align*}
  It thus follows that with probability at least $1-2^\varepsilon$ the strong verification algorithm outputs $1$ as required.
\end{proof}


\begin{lemma}\label{lem:kots_homomorphic}
  Let $\secpar, \alpha_H$, $\varphi$, $\gamma$, $\beta_\sigma,q,n,\secpar$ be positive integers.
  Let $\ring_q$ be the polynomial ring $\ZZ_q[x]/\langle x^n+1\rangle$.
  Let $H : \bin^* \to \tern_{\alpha_H}$ be a hash function.
  Then the construction from \autoref{fig:otsconstruction} is a robustly homomorphic.
\end{lemma}

\begin{proof}
  Let $\params\gets\setup(\secparam)$, $m\in\bin^*$, $\opk^0=(v^0_0,v^0_1),\opk^1=(v^1_0,v^1_1)\in \ring^2_q$, and $\vec{\sigma}^0,\vec{\sigma}^1 \in \ring_q^\gamma$ be arbitrary such that $\sverify(\params,\opk^0,m,\vec{\sigma}^0)=1$ and $\sverify(\params,\opk^1,m,\vec{\sigma}^1)=1$.
  
  By the definition of the strong verification algorithm, it holds that
  \begin{equation*}
     \norm{(\vec{\sigma}^0-\vec{\sigma}^1)}
    \leq \norm{\vec{\sigma}^0}+\norm{\vec{\sigma}^1}
    \leq 2\beta_\sigma,
  \end{equation*}
  thus the norm check goes through.
  It remains to verify that the second check also goes through.
  \begin{align*}
     \vec{a}^\intercal\cdot (\vec{\sigma}^0-\vec{\sigma}^1)
    ={}& \vec{a}^\intercal\cdot \vec{\sigma}^0- \vec{a}^\intercal\cdot\vec{\sigma}^1\\
    ={}& (v^0_0\cdot H(m) + v^0_1) - (v^1_0\cdot H(m) + v^1_1)\tag{Def of $\sverify$}\\
    ={}& (v^0_0-v^1_0)\cdot H(m) + (v^0_1-v^1_1).
  \end{align*}
  Therefore, the lemma statement follows.\qed
\end{proof}


\begin{lemma}\label{lem:kots_sis}
  Let $n,\gamma,q,\alpha_H,\alpha_w,\delta,\secpar$ be positive integers with $q$ prime and $n$ a power of two, such that $q > 16 \alpha_w \alpha_H\varphi$, $\gamma\geq((3\secpar+\delta)/n+\log_2q)\log^{-1}_2(\varphi+\tfrac{1}{2})$, and $2^{2\secpar} \leq \abs{\tern_{\alpha_H}} \leq 2^{2\secpar + \delta}$.
  Let $H : \bin^* \to \tern_{\alpha_H}$ be a hash function.
  If the $\sis_{\ring,q,\gamma,2\beta_\sigma + 4\alpha_w\alpha_H\varphi}$ problem is hard and $H$ is collision resistant, then the construction from \autoref{fig:otsconstruction} is existentially unforgeable under rerandomized keys.
\end{lemma}

\begin{proof}
  Let $\adv$ be an arbitrary adversary against the multi-user $W'$-existentially unforgeability under rerandomized keys with success probability $\nu(\secpar)$.
  We construct an algorithm $\bdv$ that solves $\sis_{\ring,q,\gamma,2\beta_\sigma + 4\alpha_w\alpha_H\varphi}$ as follows.
  Given $\vec{a}\in\ring_q^\gamma$, $\bdv$ honestly chooses secret keys $(\vec{s}^i_0,\vec{s}^i_1) \in \ball_{\varphi}^\gamma\times \ball_{\varphi\alpha_H}^\gamma$ uniformly at random for $i\in[T-1]$ and invokes $\adv$ on public keys $(v^i_0,v^i_1)$, with $v^i_b := \vec{a}^\intercal\cdot s^i_b$.
  Whenever $\adv$ sends a signing query $(i,m)$, $\bdv$ will respond with the honestly computed signature $\vec{\sigma}:=\vec{s}^i_0\cdot H(m)+ \vec{s}^i_1$.
  Eventually $\adv$ outputs a candidate forgery $(i^*,m^*,\vec{\sigma}^*,w^*)$ and $\bdv$ will compute a signature on the same message as $\vec{\sigma}' := w^*\cdot\vec{s}^{i^*}_0\cdot H(m^*)+ w^*\cdot\vec{s}^{i^*}_1$.
  It then outputs $\vec{\sigma}^*-\vec{\sigma}'$.
  
  To analyze the success probability of $\bdv$ suppose that $\adv$ outputs a \emph{valid} forgery.
  I.e., at most a single query was asked for index $i^*$, said query was \emph{not} $m^*$, $w^*\in W'$ and $\wverify(\vec{a},(w^*v^{i^*}_0,w^*v^{i^*}_1),m^*,\allowbreak\vec{\sigma}^*)=1$.
  From this and the definition of $\vec{\sigma}'$ above it follows that
  \begin{align*}
       \vec{a}^\intercal\cdot(\vec{\sigma}^*-\vec{\sigma}') ={}& \vec{a}^\intercal\vec{\sigma}^*-\vec{a}^\intercal\vec{\sigma}'\\ 
    ={}& (w^*\cdot v^{i^*}_0 H(m) + w^*\cdot v^{i^*}_1) - \vec{a}^\intercal(w^*\cdot \vec{s}^{i^*}_0\cdot H(m^*)+ w^*\cdot\vec{s}^{i^*}_1)\\
    ={}& (w^*\cdot v^{i^*}_0 H(m) + w^*\cdot v^{i^*}_1) - (w^*\cdot\vec{a}^\intercal\vec{s}^{i^*}_0\cdot H(m^*)+ w^*\cdot\vec{a}^\intercal\vec{s}^{i^*}_1)\\
    ={}& (w^*\cdot v^{i^*}_0\cdot H(m) + w^*\cdot v^{i^*}_1) - (w^*\cdot v^{i^*}_0\cdot H(m) + w^*\cdot v^{i^*}_1) = 0.
  \end{align*}
  as required for a solution to the SIS problem.
  
  Next, to argue that $\norm{\vec{\sigma}^*-\vec{\sigma}'}\leq 2\beta_\sigma + 4\alpha_w\alpha_H\varphi$, note that the weak verification algorithm guarantees that $\norm{\vec{\sigma}^*} \leq 2\beta_\sigma$.
  Further, since $w^*\in W'$ there exist $w_0,w_1 \in \tern_{\alpha_w}$ such that $w^* = w_0-w_1$ and $\norm{w^*}_1 \leq \norm{w_0}_1 + \norm{w_1}_1 = 2\alpha_w$.
  We can thus bound the norm of $\vec{\sigma}'$ as
  \begin{align*}
    \norm{\vec{\sigma}'} ={}& \norm{w^*\cdot\vec{s}^{i^*}_0\cdot H(m^*)+ w^*\cdot\vec{s}^{i^*}_1}\tag{Def. of $\sign$}\\
    ={}&\norm{w^*\cdot\vec{s}^{i^*}_0\cdot H(m^*)}+ \norm{w^*\cdot\vec{s}^{i^*}_1}\tag{Triangle Inequality}\\
    ={}&\norm{w^*}_1\cdot\norm{H(m^*)}_1\cdot\norm{\vec{s}^{i^*}_0} + \norm{w^*}_1\cdot\norm{\vec{s}^{i^*}_1}\tag{\autoref{lem:ternbound}}\\
    ={}&4\alpha_w\alpha_H\varphi.\tag{$w^*\in W'$ and $H(m^*)\in\tern_{\alpha_H}$}
  \end{align*}
  It follows that $\norm{\vec{\sigma}^*-\vec{\sigma}'} \leq \norm{\vec{\sigma}^*}+\norm{\vec{\sigma}'} \leq 2\beta_\sigma + 4\alpha_w\alpha_H\varphi$ as required.

  Finally, we need to argue that $\vec{\sigma}^*-\vec{\sigma}'\neq 0$.
  This is the case iff $\vec{\sigma}^* \neq \vec{\sigma}'$.
  It thus suffices to bound the probability, that $\vec{\sigma}^*=\vec{\sigma}'$.

  To this end, we observe by \autoref{lem:keyhidden} that, since $\adv$ has learned at most a single signature under $(v_0^{i^*},v_1^{i^*})$, the corresponding $(\vec{s}_0^{i^*},\vec{s}_1^{i^*})$ remains information-theoretically hidden from $\adv$ among at least 2 possible secret keys.
  Once $\adv$ outputs a valid forgery $(i^*,m^*,\vec{\sigma}^*,w^*)$ the signing key used for the forgery becomes uniquely determined by \autoref{lem:nilssupportivechildsupport} as long as $H(m^*)\neq H(m)$ which is guaranteed with overwhelming probability by the collision resistance of $H$.
  It follows that $\sigma^* \neq \sigma'$ with probability at least $1/2 - \negl$.
  Therefore, the success probability of our reduction $\bdv$ is $(1/2 - \negl) \nu(\secpar)$ and since the SIS problem is assumed to be hard, $\nu(\secpar)$ must therefore be negligible in $\secpar$.
  \qed
\end{proof}

\begin{lemma}\label{lem:keyhidden}
  Let $n,\gamma,q,\alpha_H, \delta, \varphi,\secpar$ be positive integers such that $\gamma\geq((3\secpar+\delta)/n+\log_2q)\log^{-1}_2(\varphi+\tfrac{1}{2})$ and $\abs{\tern_{\alpha_H}} \leq 2^{2\secpar + \delta}$, let $\ring=\ZZ[x]/(x^n+1)$. Then for any $\vec{a}\in\ring_q^\gamma$ and uniformly chosen $(\vec{s}_0,\vec{s}_1)\in \ball_{\varphi}^\gamma \times \ball_{\varphi\alpha_H}^\gamma$ it holds with probability at least $1-2^{-\lambda}$ that for every $c\in \tern_{\alpha_H}$ there exists $(\vec{s}'_0,\vec{s}'_1)\in \ball_{\varphi}^\gamma \times \ball_{\varphi\alpha_H}^\gamma$such that $(\vec{s}'_0,\vec{s}'_1)\neq(\vec{s}_0,\vec{s}_1)$, $(\vec{a}^\intercal\cdot\vec{s}'_0,\vec{a}^\intercal\cdot\vec{s}'_1) = (\vec{a}^\intercal\cdot\vec{s}_0,\vec{a}^\intercal\cdot\vec{s}_1)$ and $\vec{s}'_0\cdot c + \vec{s}'_1 = \vec{s}_0\cdot c + \vec{s}_1$.
\end{lemma}
\begin{proof}
  We define a function $f_{\vec{a}, c}$ that maps any secret key $(\vec{s}_0, \vec{s}_1)$ to a pair of public key and signature defined as $((\vec{a}^\intercal\cdot\vec{s}_0,\vec{a}^\intercal\cdot\vec{s}_1), \vec{s}_0\cdot c + \vec{s}_1)$.
  We will show that the domain of this function is at least $2^{3\secpar + \delta}$ times larger than the range.
  The number of possible secret keys is $(2\varphi+1)^{n\gamma} \cdot (2\varphi\alpha_H+1)^{n\gamma}$.
  The number of possible signatures is at most $(4\varphi\alpha_H + 1)^{n\gamma}$.
  For fixed values $\vec{a}, c, \vec{s}_0\cdot c + \vec{s}_1$, we observe that once $\vec{a}^\intercal\cdot\vec{s}_0$ is fixed, the second component $\vec{a}^\intercal\cdot\vec{s}_1 = \vec{a}^\intercal \cdot ((\vec{s}_0\cdot c + \vec{s}_1) - \vec{s}_0 \cdot c)$ is uniquely determined.
  Thus for a fixed signature, there are at most $q^n$ many possible public keys and therefore the size of the range of $f_{\vec{a}, c}$ is at most $(4 \varphi\alpha_H + 1)^{n\gamma} \cdot q^n$.
  We observe that
  \begin{align*}\frac{(2\varphi+1)^{n\gamma} \cdot (2\varphi\alpha_H+1)^{n\gamma}}{(4 \varphi\alpha_H + 1)^{n\gamma} \cdot q^n}
   \geq{}& \frac{(2\varphi+1)^{n\gamma} \cdot (2\varphi\alpha_H+1)^{n\gamma}}{(4 \varphi\alpha_H + 2)^{n\gamma} \cdot q^n}\\
   ={}& \frac{(2\varphi+1)^{n\gamma}}{2^{n\gamma} \cdot q^n}\\
   ={}& (\varphi+\frac{1}{2})^{n\gamma} \cdot q^{-n}\\
   ={}& 2^{\log_2(\varphi+\frac{1}{2})\cdot{n\gamma} - n \log_2 q}
  \end{align*}
  Using the condition on $\gamma$ from the lemma statement, one can see that
  \[
    \log_2(\varphi+\frac{1}{2})\cdot{n\gamma} - n \log_2 q
    \geq n\Bigl(\frac{3\secpar+\delta}{n}+\log_2q\Bigr) - n \log_2 q = 3\secpar+\delta
  \]
  and thus, as claimed the domain of $f_{\vec{a},c}$ is at least $2^{3\secpar + \delta}$ times larger than its range.
  
  Using Lemma 4.1 from~\cite{TCC:LyuMic08}, the probability, over a uniformly chosen secret key, that there exists $(\vec{s}'_0,\vec{s}'_1)\in \ball_{1}^\gamma \times \ball_{\beta_s}^\gamma$ such that $(\vec{s}'_0,\vec{s}'_1)\neq(\vec{s}_0,\vec{s}_1)$, $(\vec{a}^\intercal\cdot\vec{s}'_0,\vec{a}^\intercal\cdot\vec{s}'_1) = (\vec{a}^\intercal\cdot\vec{s}_0,\vec{a}^\intercal\cdot\vec{s}_1)$ and $\vec{s}'_0\cdot c + \vec{s}'_1 = \vec{s}_0\cdot c + \vec{s}_1$ is at least $1-2^{-3\secpar-\delta}$.
  By union bounding over all possible hash values $c \in \tern_{\alpha_H}$ and observing that $\tern_{\alpha_H} \leq 2^{2\secpar + \delta}$ the lemma statement follows. \qed
\end{proof}

\begin{lemma}\label{lem:nilssupportivechildsupport}
Let $n,\gamma,q,\alpha_H, \alpha_w$ be positive integers with $q$ prime and $n$ a power of two such that $q > 16 \alpha_w \alpha_H\varphi$ and let $\ring=\ZZ[x]/(x^n+1)$. Let $\vec{a}\in\ring_q^\gamma$, $c_0,c_1 \in \tern_{\alpha_{H}}$, $w_0, w_1 \in \tern_{\alpha_w}$, and $\sigma_0,\sigma_1 \in \ring$ be arbitrary ring elements such that $c_0\neq c_1$ and $w_0 \neq w_1$. Then there exists at most a single pair of vectors $(\vec{s}_0,\vec{s}_1)\in\ball^\gamma_{\varphi}\times \ball^\gamma_{\varphi\alpha_H}$, such that
    \[\vec{s}_0\cdot c_0 + \vec{s}_1 = \sigma_0 \quad\text{and}\quad (w_0 - w_1) \cdot (\vec{s}_0\cdot c_1 + \vec{s}_1) = \sigma_1.\]
\end{lemma}
 \begin{proof}
    Let $(\vec{s}_0, \vec{s}_1)\in\ball^\gamma_{\varphi}\times \ball^\gamma_{\varphi\alpha_H}$ and $(\vec{s}'_0, \vec{s}'_1)\in\ball^\gamma_{\varphi}\times \ball^\gamma_{\varphi\alpha_H}$ be two secret keys, such that 
    \begin{equation}
    \vec{s}_0\cdot c_0 + \vec{s}_1 = \vec{s}'_0\cdot c_0 + \vec{s}'_1 \implies (\vec{s}_0 - \vec{s}'_0)\cdot c_0 + (\vec{s}_1 - \vec{s}'_1) = 0 \label{hello}
    \end{equation}
    and 
    \begin{equation}
    \begin{aligned}
    &(w_0 - w_1) \cdot (\vec{s}_0\cdot c_1 + \vec{s}_1) = (w_0 - w_1) \cdot (\vec{s}'_0\cdot c_1 + \vec{s}'_1)\\ \implies& (w_0 - w_1)((\vec{s}_0 - \vec{s}'_0)\cdot c_1 + (\vec{s}_1 - \vec{s}'_1)) = 0 \label{kitty}
    \end{aligned}
    \end{equation}
    Equation~\ref{hello} implies that 
    \[
    (w_0 - w_1)((\vec{s}_0 - \vec{s}'_0)\cdot c_0 + (\vec{s}_1 - \vec{s}'_1)) = 0.
    \]
    Combined with Equation~\ref{kitty}, we get that in $\ring_q$
    \begin{equation}
    (w_0 - w_1)(\vec{s}_0 - \vec{s}'_0) (c_0 - c_1)  = 0 \label{herekittykitty}
    \end{equation}
    Since $w_0,w_1\in\tern_{\alpha_w}$, $\vec{s}_0,\vec{s}'_0 \in \ball^\gamma_{\varphi}$, and $c_0,c_1\in\tern_{\alpha_{H}}$, it holds by \autoref{lem:ternbound} that
    \[
      \norm{(w_0 - w_1)(\vec{s}_0 - \vec{s}'_0) (c_0 - c_1)} \leq \norm{w_0 - w_1}_1\cdot\norm{c_0 - c_1}_1\cdot \norm{(\vec{s}_0 - \vec{s}'_0)} \leq 8\alpha_w\alpha_H\varphi \leq \frac{q-1}{2}.
    \]
    Therefore \autoref{herekittykitty} also holds in $\ring$.
    Since $w_0 \neq w_1$, $c_0 \neq c_1$, and $\ring$ is an integral domain, it follows that $\vec{s}_0 = \vec{s}'_0$.
    By Equation~\ref{hello}, it must therefore hold that $(\vec{s}_0, \vec{s}_1) = (\vec{s}'_0, \vec{s}'_1)$.
    \qed
\end{proof}

% !TEX root = ../main.tex

\section{Synchronized Multi-Signatures}\label{sec:nidv}

In this section, we show how the tools developed in the previous sections can be combined to yield a synchronized multi-signature with the desired properties.
We do this in a manner that is almost identical to the way Squirrel~\cite{CCS:FleSimZha22} does it, except that our aggregation is modified to use a rejection sampling technique that allows us to reduce the signature size.
Roughly speaking, a public key in the multi-signature scheme is a vector commitment to a vector of independent one-time signature public keys.
To sign a message at time $t$, the signer publishes an opening to the key in vector position $t$ and signs the message with that key.

To aggregate these signatures, the construction computes a random linear combination of them, using weights derived using a random oracle.
The uniform distribution of weights allows us to leverage the probabilistic homomorphism of the KOTS and HVC schemes, such that this aggregation procedure will be successful with probability $1-2^{-(\epsilon-1)}$.
By rejecting unsuccessful attempts and retrying a number of times, the overall probability of an aggregation failure can be made negligible.

We will now formally define the requirements for a synchronized multi-signature scheme.
Once again, our definitions follow the definitions of Fleischhacker, Simkin, and Zhang \cite{CCS:FleSimZha22}.
In contrast to their work, however, we define a significantly stronger notion of correctness for aggregated signatures.
More concretely, \cite{CCS:FleSimZha22} only required that aggregation is successful for honestly generated keys and signatures.
We, on the other hand, require that any sequence of \emph{individually valid} signatures can be successfully aggregated.\footnote{It is worth noting, that the  \emph{construction} of Squirrel~\cite{CCS:FleSimZha22} actually satisfies this stronger notion. It was just never defined or proven.}

\begin{definition}[Synchronized Multi-Signatures]\label{def:multisig} A synchronized $\rho$-wise multi-signature scheme for $2^\tau$ time periods is defined by six PPT algorithms $\msig=(\setup, \kgen, \sign,\allowbreak \aggregate,\allowbreak \iverify, \averify)$.
\begin{description}
  \item[$\params\gets\setup(\secparam)$] The setup algorithm takes as input the security parameter and the maximum number of time periods and outputs public parameters $\params$. \gnote{shouldn't $\tau$ or $2^\tau$ be an input?}
  \item[$(\sk,\pk)\gets\kgen(\params)$] The key generation algorithm takes as input the public parameters and outputs a key-pair.
  \item[$\sigma\gets\sign(\params,\sk,t,m)$] The signing algorithm takes as input the public parameters, a secret key, a time period $0 \leq t < 2^\tau $, and a message and outputs a signature.
  \item[$\aggsig\gets\aggregate(\params,\pubkeys,t,m,\signatures)$] The aggregation algorithm takes as input the public parameters, a list of public keys, a time period $0 \leq t < 2^\tau$, a message, and a list of signatures, where $\abs{\pubkeys}=\abs{\signatures}\leq\rho$ and outputs an aggregated signature or an error $\bot$.
  \item[$b \gets \iverify(\params,\pk,t,m,\sigma)$] The deterministic individual verification algorithm takes as input the public parameters, a public key, a time period $0 \leq t < 2^\tau $, a message, and a signature and outputs a bit indicating acceptance/rejection.
  \item[$b \gets \averify(\params,\pubkeys,t,m,\aggsig)$] The deterministic aggregated verification algorithm takes as input the public parameters, a list of public keys, a time period $0 \leq t < 2^\tau$, a message, and an aggregated signature and outputs a bit indicating acceptance/rejection.
\end{description}
\end{definition}

\begin{definition}[Individual Correctness]
  Let $\msig$ be a synchronized $\rho$-wise multi-signature scheme for $2^\tau$ time periods.
  $\msig$ is individually correct if for all security parameters $\secpar\in\NN$, public parameters $\params\gets\setup(\secparam)$, key pairs $(\sk,\pk)\gets\kgen(\params)$, time periods $0 \leq t < 2^\tau$, message $m\in\bin^*$, and signatures $\sigma\gets\sign(\params,\sk,t,m)$ it holds that
  \[
    \iverify(\params,\pk,t,m,\sigma) = 1\enspace.
  \]
\end{definition}

\begin{definition}[Aggregation Correctness with Rogue Keys and Signatures]
  Let $\msig$ be a synchronized $\rho$-wise multi-signature scheme for $2^\tau$ time periods.
  $\msig$ has correct aggregations in the presence of rogue keys and signatures if for all security parameters $\secpar\in\NN$, public parameters $\params\gets\setup(\secparam)$, number of aggregated signatures $\ell\in[\rho]$, time periods $0\leq t< 2^\tau$, messages $m\in\bin^*$, public keys $\pubkeys = (\pk^1,\dots,\pk^\ell)$ and signatures $\signatures = (\sigma^1,\dots,\sigma^\ell)$, such that for all $i\in[\ell]$, $\iverify(\params,\pk^i,t,m,\sigma^i) = 1$ it holds that
  \[
    \Pr[\aggsig\gets\aggregate(\params,\pubkeys,t,m,\signatures)\colon \averify(\params,\pubkeys,t,m,\aggsig)=1]=1-\negl\enspace.
  \]
\end{definition}

\begin{definition}[Unforgeability]\label{def:multisigunforge}
  Let $\msig$ be a synchronized $\rho$-wise multi-signature scheme for $2^\tau$ time periods.
  $\msig$ is unforgeable if for all security parameters $\secpar\in\NN$, and all PPT algorithms $\adv$ it holds that
\[
  \Pr\left[\begin{aligned}\params\gets{}&\setup(\secparam),\\ (\sk^*,\pk^*)\gets&\kgen(\params)\\(\pubkeys,t,m,\aggsig)\gets{}&\adv^{\sign(\params,\sk^*,\cdot,\cdot)}(\params,\pk^*)\end{aligned}\colon
  \begin{aligned}
  &\averify(\params,\pubkeys,t,m,\aggsig)=1\\
  {}\land{}&\pk^*\in\pubkeys\\
  {}\land{}&\forall\; (t',m',\sigma') \in\queries\ldotp\; (t',m') \neq (t,m)\\
  {}\land{}&\forall\; t' \ldotp\;\abs{\queries_{t'}}\leq 1
  \end{aligned}\right] \leq \negl\enspace,
\]
where $\queries$ denotes the set of signing queries made by $\adv$ and $\queries_{t'}$ denotes the set of signing queries made for timeslot $t'$.
\end{definition}

\begin{figure}[p]
\centering
\begin{pchstack}[center,boxed]
\begin{pcvstack}
  \procedure{$\setup(\secparam)$}{
%    \delta \coloneqq \lceil\log\tau\rceil\\
    \hotsparams\gets\hots.\setup(\secparam)\\
    \hvcparams\gets\hvc.\setup(\secparam)\\
    \pcreturn \params\coloneqq (\hotsparams,\hvcparams)
  }
  \pcvspace
  \procedure{$\kgen(\params)$}{
%    \pcparse \params \pcas (\hotsparams,\hvcparams)\\
    \pcforeach 0 \leq i < 2^\tau\\
    \quad(\osk^i,\opk^i) \gets \hots.\kgen(\hotsparams)\\
    \osks = (\osk^0,\dots,\osk^{2^\tau-1})\\
    \opks = (\opk^0,\dots,\opk^{2^\tau-1})\\
    \vec{c} \gets \hvc.\commit(\hvcparams,\opks)\\
    \pcreturn (\sk,\pk) \coloneqq ((\osks,\opks),\vec{c})
  }
  \pcvspace
  \procedure{$\aggregate(\params,\pubkeys,t,m,\signatures)$}{
    \pcif \abs{\signatures} \neq \abs{\pubkeys}\\
    \quad \pcreturn \bot\\
    \pcfor (\pk,\sigma) \in \zip(\pubkeys,\signatures)\\
    \quad \pcif \iverify(\params,\pk,t,m,\sigma) = 0\\
    \quad \quad \pcreturn \bot\\
    j \coloneqq 0\\
    \pcdo[]\\
    \quad j\coloneqq j+1\\
    \quad (w^0,\dots,w^{\abs{\pubkeys}}) \coloneqq H(t,m,\pubkeys,j)\\
    \quad \otsig \coloneqq \sum_{i=1}^{\abs{\pubkeys}} w^i\cdot \otsig^i\\
    \quad \vec{d} \coloneqq \sum_{i=1}^{\abs{\pubkeys}} w^i\cdot \vec{d}^i\\
    \pcwhile j < \chi \pcand \averify(\params,\pubkeys,t,m,(\otsig,\vec{d},j))=0\\
    \pcreturn \aggsig \coloneqq (\otsig,\vec{d},j)
  }
\end{pcvstack}
\pchspace
\begin{pcvstack}
  \procedure{$\sign(\params,\sk,t,m)$}{
    \otsig \gets \hots.\sign(\hotsparams,\osk_t,m)\\
    \vec{d} \gets \hvc.\open(\hvcparams,\vec{c},\opks,t)\\
    \pcreturn \sigma \coloneqq (\otsig,\vec{d})
  }
  \pcvspace
  \procedure{$\iverify(\params,\pk,t,m,\sigma)$}{
    \opk \gets \hvc.\sverify(\hvcparams,\vec{c},t,\vec{d})\\
    \pcif t \geq 2^\tau \pcor \opk = \bot\\
    \quad \pcreturn 0\\
    \pcelse\\
    \quad \pcreturn \hots.\sverify(\hotsparams,\opk,m,\otsig)
  }
  \pcvspace
  \procedure{$\averify(\params,\pubkeys,t,m,\aggsig)$}{
    (w^1,\dots,w^{\abs{\pubkeys}}) \coloneqq H(t,m,\pubkeys,j)\\
    \vec{c} \coloneqq \sum_{i=1}^{\abs{\pubkeys}} w^i\cdot \vec{c}_i\\
    \opk \gets \hvc.\sverify(\hvcparams,\vec{c},t,\vec{d})\\
    \pcif \abs{\pubkeys} > \rho \pcor \opk = \bot\\
    \quad \pcreturn 0\\
    \pcelse\\
    \quad \pcreturn \hots.\sverify(\hotsparams,\opk,m,\otsig)
  }
\end{pcvstack}
\end{pchstack}
%\tikz[remember picture,overlay]{
%    \draw ($(kgen-for-head)+(1ex,-.3\baselineskip)$) -- ($(kgen-for-foot)+(1ex,-.5\baselineskip)$);
%}
\caption{The Chipmunk synchronized multi-signature scheme based on homomorphic vector commitments and key-homomorphic one-time signatures.}
\label{fig:nidvconst}
\end{figure}
%
\iffalse
The following lemma will be useful for proving the security of our construction in~\autoref{thm:mainconstruction}, specifically it will be useful during the security reduction to the underlying one-time signature scheme.
Intuitively, the lemma shows that two valid aggregate signatures that are created using vectors of random weights that differ in one position, allow for extracting a valid one-time signature and key.

\begin{lemma}\label{lem:forkingisuseful}
  Let $\secparam,\tau,\rho$ be positive integers.
  Let $\params\gets\setup(\secparam)$ and $(\sk^*,\pk^*=\vec{c}^*) \gets\kgen(\params)$ be fixed.
  Let $\ell\in[\rho]$, $t\in[2^\tau]$, $m\in\bin^*$, $\pubkeys = (\pk^1,\dots,\pk^{\ell})$ with $\pk^j=\pk^*$, $\aggsig^0=(\otsig^0,\vec{d}^0)$, $\aggsig^1=(\otsig^1,\vec{d}^1)$, and let $H_0,H_1$ be two random oracles, such that 
  \begin{align*}
    (w^1,\dots,w^{\ell}) &\coloneqq H_0(t,m,\pubkeys)\\
    (w^1,\dots,w^{j-1},\hat{w}^j,w^{j+1},\dots,w^{\ell}) &\coloneqq H_1(t,m,\pubkeys)
  \end{align*}
  with $w^j\neq \hat{w}^j$ and
  \[
    \averify^{H_0}(\params,\pubkeys,t,m,\aggsig^0)=1 \quad \text{and} \quad \averify^{H_1}(\params,\pubkeys,t,m,\aggsig^1)=1.
  \]
  Then, for $\opk^*\gets\hvc.\wverify(\hvcparams,(w^j-\hat{w}^j)\cdot \vec{c}^*,t,\vec{d}^0-\vec{d}^1)$ it holds that
  \[
    \opk^* \neq \bot \quad \text{and} \quad \hots.\wverify(\hotsparams,\opk^*,m,\otsig^0-\otsig^1) = 1.
  \]
\end{lemma}

\begin{proof}
    Since 
    \[
    \averify^{H_0}(\params,\pubkeys,t,m,\aggsig^0)=1 \quad \text{and} \quad \averify^{H_1}(\params,\pubkeys,t,m,\aggsig^1)=1,
    \]
    it must hold by definition of the aggregated verification algorithm that 
    \[
      \hvc.\sverify(\hvcparams,\smashoperator{\sum_{i\in[\ell]}} w^i\cdot \vec{c}^i,t,\vec{d}^0) = \opk^0 
    \]
    and
    \[   
      \hvc.\sverify(\hvcparams,\hat{w}^j\cdot \vec{c}^j+\smashoperator{\sum_{i\in[\ell]\setminus\{j\}}} w^i\cdot \vec{c}^i,t,\vec{d}^1)= \opk_1
    \]
    for $\opk^0,\opk^1\neq\bot$.
    Thus by \autoref{def:malhomhvc} it holds that
    \begin{align*}
      \opk^* =& \hvc.\wverify(\hvcparams,(w^j-\hat{w}^j)\cdot \vec{c}^*,t,\vec{d}^0-\vec{d}^1)\\
      =&\hvc.\wverify(\hvcparams,\Bigl(\smashoperator{\sum_{i\in[\ell]}} w^i\cdot \vec{c}^i\Bigr) - \Bigl(\hat{w}^j\cdot \vec{c}^j+\smashoperator{\sum_{i\in[\ell]\setminus\{j\}}} w^i\cdot \vec{c}^i\Bigr),t,\vec{d}^0-\vec{d}^1)\\
       =& (\opk^0-\opk^1).
    \end{align*}
    Further, by definition of the aggregated verification algorithm it must also hold that
    \[
      \hots.\sverify(\hotsparams,\opk^0,m,\otsig^0) =1 \quad \text{and} \quad \hots.\sverify(\hotsparams,\opk^1,m,\otsig^1)=1
    \]
    Thus, by \autoref{def:malhomhots} it holds that
    \begin{align*}
      &\hots.\wverify(\hotsparams,\opk^*,m,\otsig^0-\otsig^1)\\
      =&\hots.\wverify(\hotsparams,\opk^0-\opk^1,m,\otsig^0-\otsig^1)=1 \tag*{\qed}
    \end{align*}
\end{proof}
\fi
The following theorem now states the security of the construction of Chipmunk presented in~\autoref{fig:nidvconst}.
\begin{theorem}\label{thm:mainconstruction}
Let $\secpar,n,q',\varepsilon,\xi,\chi,\tau$ be positive integers with $n$ being a power of two, $q'$ being prime, and $\chi \geq \secpar/(\varepsilon-1)$.
Let $\ring_{q'}$ be the polynomial ring $\ZZ_{q'}[x]/\langle x^n+1\rangle$.
Let $W \subseteq \ring$ be a set such that $\abs{W} > 2^\secpar$ and let $W' \coloneqq \{w^0-w^1| w^0,w^1 \in W\}$.
Let $H \colon\; \bin^* \to W^\rho$ be a random oracle.
Let $\hots$ be a key homomorphic one-time signature scheme with public keys in $\ring_{q'}^\xi$ and let $\hvc$ be a homomorphic vector commitment for domain $\ring_{q'}^\xi$.

If $\hots$ is individually correct, $(\rho,W,\varepsilon)$-probabilistically homomorphic, robustly homomorphic, and $W'$-multi-user existentially unforgeable under rerandomized keys and $\hvc$ is individually correct, $(\rho,W,\varepsilon)$-probabilistically homomorphic, robustly homomorphic, and position-binding, then Chipmunk, as described in \autoref{fig:nidvconst}, is an unforgeable synchronized $\rho$-wise multi-signature scheme that is individually correct and has correct aggregations in the presence of rogue keys and signatures.
\end{theorem}

\begin{proof}
  The theorem follows immediately from \autoref{lem:msigindcorrect}, \autoref{lem:msigaggcorrect}, \autoref{lem:msigunf}.\qed
\end{proof}

\begin{lemma}\label{lem:msigindcorrect}
Let $\secpar,n,q',\varepsilon,\xi,\chi,\rho,\tau$ be positive integers with $n$ being a power of two, $q'$ being prime.
Let $\ring_{q'}$ be the polynomial ring $\ZZ_{q'}[x]/\langle x^n+1\rangle$.
Let $\hots$ be a key homomorphic one-time signature scheme with public keys in $\ring_{q'}^\xi$ and let $\hvc$ be a homomorphic vector commitment for domain $\ring_{q'}^\xi$.

If both $\hots$ and $\hvc$ are individually correct, then Chipmunk, as described in \autoref{fig:nidvconst}, is individually correct.
\end{lemma}
\begin{proof}
  Let $\params=(\hotsparams,\hvcparams)\gets\setup(\secparam)$, $(\sk,\pk) = ((\osks,\opks),\vec{c})\gets\kgen(\params)$, $0\leq t< 2^\tau$,$m\in\bin^*$, and $\sigma = (\otsig,\vec{d}) \gets \sign(\params,\sk,t,m)$.
  By definition of the signing algorithm it holds that
  \[
    \otsig \gets \hots.\sign(\hotsparams,\osk^t,m) \quad \text{and} \quad \vec{d} \gets \hvc.\open(\hvcparams,\vec{c},\opks,t).
  \]
  By definition of the key generation algorithm it further holds that
  \[
    (\osk^t,\opk^t) \gets \hots.\kgen(\hotsparams)
  \]
  From the individual correctness of $\hvc$ and the definition of the individual verification algorithm it follows that
  \[
    \opk^t = \opk \gets \hvc.\sverify(\params_\hvc,\vec{c},t,\vec{d})
  \]
  which finally implies by the individual correctness of $\hots$ that
  \[
    \hots.\sverify(\params_\hots,\opk,m,\otsig) = 1\enspace.
  \]
  Individual correctness thus follows.\qed
\end{proof}

\begin{lemma}\label{lem:msigaggcorrect}
  Let $\secpar,n,q',\varepsilon,\xi,\chi,\rho,\tau$ be positive integers with $n$ being a power of two, $q'$ being prime and $\chi \geq \secpar/(\epsilon-1)$.
Let $\ring_{q'}$ be the polynomial ring $\ZZ_{q'}[x]/\langle x^n+1\rangle$.
Let $W \subseteq \ring$ be a set and let $W' \coloneqq \{w^0-w^1| w^0,w^1 \in W\}$.
Let $H\colon\; \bin^* \to W^\rho$ be a random oracle.
Let $\hots$ be a key homomorphic one-time signature scheme with public keys in $\ring_{q'}^\xi$ and let $\hvc$ be a homomorphic vector commitment for domain $\ring_{q'}^\xi$.

If both $\hots$ and $\hvc$ are $(\rho,W,\epsilon)$-probabilistically homomorphic, then Chipmunk, as described in \autoref{fig:nidvconst}, has correct aggregations in the presence of rogue keys and signatures.
\end{lemma}
\begin{proof}
  Let $\params = (\params_\hvc,\params_\hots)\gets\setup(\secparam)$, $\ell \in [\rho]$, $0\leq t< 2^\tau$, $m\in\bin^*$, $\pubkeys = (\vec{c}^1,\dots,\vec{c}^\ell)$, and $\signatures = (\sigma^1,\dots,\sigma^\ell)$ with 
  $\sigma^i=(\otsig,\vec{d})$, be arbitrary, such that for all $i\in[\ell]$, $\iverify(\params,\vec{c}^i,t,m,\sigma^i)$.
  
  The aggregation algorithm makes up to $\chi$ attempts to aggregate the signature and will only output an \emph{invalid} signature, if all $\chi$ attempts fail.
  It thus suffices to analyse the probability with which all attempts fail.
  
  Attempt $j$ is performed by computing weights $(w^1,\dots,w^\ell) \coloneqq H(t,m,\pubkeys,j)$ and computing 
  \[
    \otsig \coloneqq \sum_{i=1}^{\abs{\pubkeys}} w^i\cdot \otsig_i \quad \text{and}\quad \vec{d} \coloneqq \sum_{i=1}^{\abs{\pubkeys}} w^i\cdot \vec{d}_i\enspace.
  \]
  Let $\vec{c} = \sum_{i\in[\ell]}w^i\cdot\vec{c}^i$.
  Since the signatures individually verify, there exists a well-defined $\opk^i \coloneqq \hvc.\iverify(\params,\vec{c}^i,t,\vec{d}^i)$ for all $i\in[\ell]$.
  Since further $H$ is a random oracle we can apply the $(\rho,W,\epsilon)$-probabilistic homomorphism of both $\hvc$ to conclude that
  \[
    \Pr\Bigl[\hvc.\sverify(\params_\hvc,\vec{c},t,\vec{d}) \neq \sum_{i\in[\ell]} w^i\cdot\opk^i\Bigr] < 2^{\-\epsilon}\enspace.
  \]
  and
  \[
    \Pr\Bigl[\hots.\sverify\Bigl(\params_\hots,\sum_{i\in[\ell]} w^i\cdot\opk^i, m, \otsig\Bigr) = 0\Bigr] < 2^{\-\epsilon}\enspace.
  \]
  The aggregation attempt fails iff either of these conditions is violated.
  Therefore, by a union bound, each individual attempt fails with probability at most $2^{-(\epsilon-1)}$.
  Since each attempt is an independent Bernoulli trial the probability of overall failure of all $\chi\geq \secpar/(\epsilon-1)$ attempts can be bounded by $2^{-(\epsilon-1)\chi} \leq 2^{-\secpar}$.
  Hence, aggregation will succeed with overwhelming probability.\qed
\end{proof}

\begin{lemma}\label{lem:msigunf}
Let $\secpar,n,q',\varepsilon,\xi,\chi,\rho,\tau$ be positive integers with $n$ being a power of two, $q'$ being prime.
Let $\ring_{q'}$ be the polynomial ring $\ZZ_{q'}[x]/\langle x^n+1\rangle$.
Let $W \subseteq \ring$ be a set such that $|W| > 2^\secpar$ and let $W' \coloneqq \{w^0-w^1| w^0,w^1 \in W\}$.
Let $H\colon\; \bin^* \to W^\rho$ be a random oracle.
Let $\hots$ be a key homomorphic one-time signature scheme with public keys in $\ring_{q'}^\xi$ and let $\hvc$ be a homomorphic vector commitment for domain $\ring_{q'}^\xi$.

If $\hots$ is $W'$-multi-user existentially unforgeable under rerandomized keys and $\hvc$ is position-binding, then Chipmunk, as described in \autoref{fig:nidvconst}, is unforgeable.
\end{lemma}

\begin{proof}
  The proof for this lemma remains essentially identical to the proof of unforgeability for Squirrel~\cite{CCS:FleSimZha22}.
  The entire argument is only concerned with the \emph{aggregated verification} algorithm, the unforgeability of $\hots$ and the position binding of $\hvc$.
  None of the differences between Chipmunk and Squirrel affect these parts, with the tiny exception that the random oracle during verification now takes the additional input $j$.
  Literally, the only necessary change in the proof is, therefore, that during the technically tedious forking lemma setup, the simulated random oracle needs to also take $j\in[\chi]$ as input.
  As such, we omit the proof here and refer the interested reader to the full version of the original Squirrel paper~\cite{EPRINT:FleSimZha22}.
  We stress that the proof of unforgeability in \cite{EPRINT:FleSimZha22} relies on a variant \cite{CCS:BelNev06} of the forking lemma \cite{EC:PoiSte96}, which uses a rewinding strategy that does not apply to quantum algorithms.\qed
  \iffalse
  Let $\adv$ be an arbitrary PPT algorithm that makes at most $p=\poly$ queries to the random oracle and which breaks unforgeability with probability $\nu(\secpar)$
  If $\adv$ outputs a forgery $(\pubkeys^*,t^*,m^*,\aggsig^*=(\otsig,\vec{d}^*,j^*))$, we assume without loss of generality that $\adv$ always queries $(\pubkeys^*,t^*,m^*,j^*)$ to the random oracle.
  
%  Let $\IG$ be the algorithm that
\begin{figure}\centering
\begin{pchstack}[boxed,center]
  \begin{pcvstack}
  \procedure{$\IG(\secparam)$}{
    \params\gets \setup(\secparam,\rho,\tau)\\
    (\sk^*,c^*)\gets\kgen(\params)\\
    \pcreturn ((\params,\pk^*),(\params,\sk^*))
  }
  \pcvspace
  \procedure{$H(t,m,\pubkeys,j)$}{
    \pcif ((t,m,\pubkeys,j),\vec{w}) \in \Hlist\\
    \quad \pcreturn \vec{w}\\
    \pcparse \pubkeys \pcas (c_0,\dots,c_{\ell-1})\\
    \vec{w} \coloneqq (h_{i\rho+1},\dots,h_{i\rho+\ell})\\
    \pcfor j \in [\ell]\\
    \quad \pcif \vec{c}_j = \pk^*\\
    \quad\quad \vec{w} \coloneqq \biggl(\begin{aligned}w^{0},\dots,w^{j-1},w^{\ell-1},\pclb w^{j+1},\dots,w^{\ell-2},w^{j}\end{aligned}\biggr)\\
    i \coloneqq i+1\\
    \Hlist \coloneqq \Hlist \cup ((t,m,\pubkeys,j),\vec{w})\\
    \pcreturn \vec{w}
  }
  \end{pcvstack}
  \pchspace
  \procedure{$\bdv^{\sign(\params,\sk^*,\cdot,\cdot)}(\params,c^*,h_1,\dots,h_{p\rho})$}{
    i \coloneqq 0\\
    \Hlist \coloneqq \emptyset\\
    (\pubkeys^*,t^*,m^*,\aggsig^*) \gets \adv^{\sign(\params,\sk^*,\cdot,\cdot),H(\cdot,\cdot,\cdot)}(\params,\pk^*)\\
    \pcif \averify^{H(\cdot,\cdot,\cdot)}(\params,\pubkeys^*,t^*,m^*,\aggsig^*) = 0\\
    \quad \pcreturn (0,0,\bot)\\
    \pcif \pk^*\not\in\pubkeys^*\pcor \exists\; \sigma\ldotp\ (t^*,m^*,\sigma)\in\queries\\
    \quad \pcreturn (0,0,\bot)\\
    \pcfor t' \in [2^\tau]\\
    \quad\pcif \abs{\queries_{t'}}> 1\\
    \quad\quad \pcreturn (0,0,\bot)\\
    \pcfor k \in [i]\\
    \quad ((t_k,m_k,\pubkeys_k,j_k),\vec{w}_k) = \Hlist_k\\
    \quad \pcif (t_k,m_k,\pubkeys_k,j_k) = (t^*,m^*,\pubkeys^*,j^*)\\
    \quad\quad \pcreturn \biggl(1,j \rho+\abs{\pubkeys^*},\biggl(\begin{aligned}\pubkeys^*,t^*,m^*,\pclb\aggsig^*,\vec{w}_j\end{aligned}\biggr)\biggr)\\
    \pcreturn (0,0,\bot)
  }
\end{pchstack}
\caption{The setup for the forking lemma based on attacker $\adv$.}
\label{fig:fork}
\end{figure}
\begin{claim}\label{claim:bisasgoodasa}
  Let $\IG$ and $\bdv$ be as defined in \autoref{fig:fork}. Then it holds that
  \[\Pr\left[\begin{aligned}((\params,\pk^*),(\params,\sk^*)) \gets \IG(\secparam);\\ h_1,\dots,h_{p}\gets W;\\ (b,i,%(\pubkeys,t,m,\aggsig,\vec{w})
  \omega) \gets \bdv^{\sign(\params,)}(x,h_1,\dots,h_{p})\end{aligned} : b = 1\right] = \epsilon(\secpar).\]
\end{claim}
\begin{proof}
  The input generation algorithm $\IG$ performs exactly the same setup expected by $\adv$ and then $\bdv$ simply executes $\adv$, perfectly simulating the random oracle by lazy sampling. The only interesting part of the simulation is the fact that $\bdv$ reorders the used randomness if a query includes the challenge public key $\pk^*$.
  This is necessary to later make use of \autoref{lem:moregeneralforking} without modification, but does not impact the simulation at all: the random values $h_0$ through $h_{p\cdot\rho}$ are all distributed independently and the swapping takes place independently of their values. Therefore the distribution of the random oracle answers is identical with or without swapping.
  After executing $\adv$, the algorithm $\bdv$ checks whether $\adv$ would have been successful according to \autoref{def:multisigunforge} and outputs $b=1$ iff $\adv$ was successful. Therefore, the claim follows. \qed
\end{proof}
  
  By combining \autoref{claim:bisasgoodasa} with \autoref{lem:moregeneralforking} we can conclude that
  \begin{equation}
    \epsilon \leq \frac{p}{\abs{\tern_\alpha}} + \sqrt{p\cdot\Pr\left[\begin{aligned}((\params,\pk^*),(\params,\sk^*)) \gets \IG(\secparam);\\ (b,\omega^0,\omega^1)\gets \fork^{\sign(\params,\sk^*,\cdot,\cdot)}_\bdv(\params,\pk^*)\end{aligned} : b=1 \right]}.\label{eq:forkingresult}
  \end{equation}
  It remains to bound the probability
  \[
    \Pr\left[\begin{aligned}((\params,\pk^*),(\params,\sk^*)) \gets \IG(\secparam);\\ (b,\omega^0,\omega^1)\gets \fork^{\sign(\params,\sk^*,\cdot,\cdot)}_\bdv(\params,\pk^*)\end{aligned} : b=1 \right]
  \]
  
  For any output $(1,(\pubkeys^0,t^0,m^0,(\sigma^0,d^0),\vec{w}^0),(\pubkeys^1,t^1,m^1,(\sigma^1,d^1),\vec{w}^1))$ of the forking algorithm it holds by definition of $\bdv$ and $\fork_\bdv$ that $(\pubkeys^0,t^0,m^0)=(\pubkeys^1,t^1,m^1)$ because these are inputs to the random oracle \emph{before} the fork occurs.
  To improve readability we thus introduce a modified forking algorithm $\widetilde\fork_\bdv$ defined in \autoref{fig:modfork}.
  \begin{figure}[tb]
  \centering
  \fbox{
  \procedure{$\widetilde\fork_\bdv^{\sign(\params,\sk^*)}(\params,\pk^*)$}{
    (b,(\pubkeys^0,t^0,m^0,\aggsig^0,\vec{w}^0),(\pubkeys^1,t^1,m^1,\aggsig^1,\vec{w}^1)) \gets \fork^{\sign(\params,\sk^*,\cdot,\cdot)}_\bdv(\params,\pk^*)\\
    \pcreturn (b,\pubkeys^0,t^0,m^0,\aggsig^0,\vec{w}^0,\aggsig^1,\vec{w}^1)
  }
  }
  \caption{The modified forking algorithm.}\label{fig:modfork}
  \end{figure}
  Obviously, it holds that
  \begin{align}
%    \begin{aligned}
    &\Pr\left[\begin{aligned}((\params,\pk^*),(\params,\sk^*)) \gets \IG(\secparam);\\ (b,\omega^0,\omega^1)\gets \fork^{\sign(\params,\sk^*,\cdot,\cdot)}_\bdv(\params,\pk^*)\end{aligned} : b=1 \right]\nonumber\\
    =&\Pr\left[\begin{aligned}((\params,\pk^*),(\params,\sk^*)) \gets& \IG(\secparam);\\ (b,\pubkeys,t,m,\aggsig^0,\vec{w}^0,\aggsig^1,\vec{w}^1)\gets& \widetilde\fork^{\sign(\params,\sk^*,\cdot,\cdot)}_\bdv(\params,\pk^*)\end{aligned} : b=1 \right]\label{eq:forkprob}
%    \end{aligned}
  \end{align}
  Let $(b,\pubkeys,t,m,(\sigma^0,d^0),\vec{w}^0,(\sigma^1,d^1),\vec{w}^1) \gets \widetilde\fork_\bdv^{\sign(\params,\sk^*,\cdot,\cdot)}(\params,\pk^*)$ be an execution of the modified forking algorithm with \[\pk^* = c^*\quad\text{and}\quad\sk^* = ((\osk^*_0,\dots,\osk^*_{2^\tau-1}),(\opk^*_0,\dots,\opk^*_{2^\tau-1})).\]
  %  We define $(\pubkeys,t,m) = (\pubkeys^0,t^0,m^0)=(\pubkeys^1,t^1,m^1)$.
  Let $j$ denote the index, such that $\pubkeys_j=\pk^*$.
  We then define \[\widetilde w \coloneqq (w^0_j - w^1_j)\quad\text{and}\quad
%  \[
%    \widetilde c \coloneqq %\sum_{i=0}^{\ell-1} w^0_i\cdot c_i - \sum_{i=0}^{\ell-1} w^1_i\cdot c_i = 
%    c_j \cdot (w^0_j - w^1_j) 
%  \]
%  
    \widetilde\opk \coloneqq  \hvc.\wverify(\params_\hvc,\widetilde w\cdot c^* ,t,(d^0-d^1)).
  \]
  The probability from \autoref{eq:forkprob} can then be split as
  \begin{equation}
  \begin{aligned}
    &\Pr\left[\begin{aligned}((\params,\pk^*),(\params,\sk^*)) \gets& \IG(\secparam);\\ (b,\pubkeys,t,m,\aggsig^0,\vec{w}^0,\aggsig^1,\vec{w}^1)\gets& \widetilde\fork^{\sign(\params,\sk^*,\cdot,\cdot)}_\bdv(\params,\pk^*)\end{aligned} : b=1 \right]\\
    =&\Pr\left[\begin{aligned}((\params,\pk^*),(\params,\sk^*)) \gets& \IG(\secparam);\\ (b,\pubkeys,t,m,\aggsig^0,\vec{w}^0,\aggsig^1,\vec{w}^1)\gets& \widetilde\fork^{\sign(\params,\sk^*,\cdot,\cdot)}_\bdv(\params,\pk^*)\end{aligned} : \begin{aligned} &\widetilde\opk = \widetilde w \cdot \opk^*_{t} \\ \land& b=1\end{aligned}\right]\\%\cdot\Pr[\widetilde\opk = \opk^*_{t^0}(w^0_j-w^1_j)]\\
    &+\Pr\left[\begin{aligned}((\params,\pk^*),(\params,\sk^*)) \gets& \IG(\secparam);\\ (b,\pubkeys,t,m,\aggsig^0,\vec{w}^0,\aggsig^1,\vec{w}^1)\gets& \widetilde\fork^{\sign(\params,\sk^*,\cdot,\cdot)}_\bdv(\params,\pk^*)\end{aligned} :\begin{aligned} &\widetilde\opk \neq \widetilde w \cdot \opk^*_{t} \\ \land& b=1\end{aligned}\right]%\cdot\Pr[\widetilde\opk \neq \opk^*_{t^0}(w^0_j-w^1_j)].
  \end{aligned}
  \end{equation}
  and we can bound the two parts separately.
  
  Consider $\rdv_0$ described in \autoref{fig:redhvc} as an adversary against the position binding of $\hvc$.
  \begin{figure}\centering\fbox{
  \procedure{$\rdv_0(\hvcparams,\tau)$}{
    \hotsparams\gets\hots.\setup(\secparam)\\
    \params \coloneqq (\hotsparams,\hvcparams,\tau)\\
    ((\osks^*,\opks^*),c^*) \gets \kgen(\params)\\
    (b,\pubkeys,t,m,{\aggsig}_0,\vec{w}_0,{\aggsig}_1,\vec{w}_1)\gets\widetilde\fork_\bdv^{\sign(\params,(\osks^*,\opks^*),\cdot,\cdot)}(\params,c^*)\\
    \pcif b=1\\
    \quad\pcparse \pubkeys \pcas (\vec{c}^0,\dots,\vec{c}^\ell)\\
    \quad\pcparse {\aggsig}_0 \pcas (\vec{\sigma}_0, \vec{d}_0)\\
    \quad\pcparse {\aggsig}_1 \pcas (\vec{\sigma}_1, \vec{d}_1)\\
    \quad\pcfor j\in[\ell]\\
    \quad\quad \pcif c^j = c^*\\
    \quad\quad\quad \pcreturn (\widetilde w\cdot c^*,t,\widetilde w\cdot \open(\hvcparams,c^*,\opks^*,t),(d^0-d^1))\\
    \pcreturn \bot
  }}
  \caption{A reduction that uses the forking algorithm $\widetilde\fork_\bdv$ to attack the position binding of $\hvc$.}\label{fig:redhvc}
  \end{figure}
  To analyse the success probability of $\rdv_0$, consider an execution $(c,t,d_0,d_1)\gets\rdv_0(\hvcparams,\tau)$.
  According to the definitions above, it holds that
  \begin{align*}
    & \hvc.\wverify(\hvcparams,c,t,d_0)\\
%    =& \hvc.\wverify(\hvcparams,\widetilde w\cdot c^*,t^0,\widetilde w\cdot\open(\hvcparams,c,\opks^*,t))\\
    =& \hvc.\wverify(\hvcparams,w^j_0\cdot c^* - w^j_1\cdot c^*,t,w^j_0\cdot\open(\hvcparams,c,\opks^*,t) - w^j_0\cdot\open(\hvcparams,c,\opks^*,t))\\
    =& \hvc.\sverify(\hvcparams,w^j_0\cdot c^*,t^0,w^j_0\cdot\open(\hvcparams,c,\opks^*,t))\\ &- \hvc.\sverify(\hvcparams,w^j_1\cdot c^*,t^0,w^j_1\cdot\open(\hvcparams,c,\opks^*,t))
    \tag{Robust Homomorphism}\\
    =& w^j_0\cdot \opk^*_t - w^j_1\cdot \opk^*_t = \widetilde w\cdot \opk^*_t \tag{Homomorphic correctness}
  \end{align*}
  and
  \[
    \hvc.\wverify(\hvcparams,c,t,d_1) = \hvc.\wverify(\hvcparams,\widetilde w\cdot c^*,t,(d^0-d^1)) = \widetilde\opk
  \]
  We then have that
    \begin{align}
    &\Pr\left[
      \begin{aligned}
      \hvcparams\gets&\mspace{\medmuskip}\hvc.\setup(\secparam,\tau);\\
      (c,t,d_0,d_1) \gets&\mspace{\medmuskip} \rdv_0(\params);\\
      m_0 \gets&\mspace{\medmuskip} \hvc.\wverify(\params,c,t,d_0);\\
      m_1 \gets&\mspace{\medmuskip} \hvc.\wverify(\params,c,t,d_1)
      \end{aligned}:
      m_0\neq m_1 \land\bot\not\in\{m_0,m_1\}
    \right]\notag\\
    =&\Pr\left[
      \begin{aligned}
      ((\params,\pk^*),(\params,\sk^*)) \gets \IG(\secparam);\\
      (b,\pubkeys,t,m,\aggsig^0,\vec{w}^0,\aggsig^1,\vec{w}^1)\gets \widetilde\fork^{\sign(\params,\sk^*,\cdot,\cdot)}_\bdv(\params,\pk^*);\\
      m_0 \coloneqq \widetilde w\cdot \opk^*_t;
      m_1 \coloneqq \widetilde\opk
      \end{aligned}:
      \begin{aligned}
      &m_0\neq m_1\\ \land&\bot\not\in\{m_0,m_1\}
      \end{aligned}
    \right]\label{eq:redhvcsimcorrect}\\
    =&\Pr\left[
      \begin{aligned}
      ((\params,\pk^*),(\params,\sk^*)) \gets \IG(\secparam);\\
      (b,\pubkeys,t,m,\aggsig^0,\vec{w}^0,\aggsig^1,\vec{w}^1)\gets \widetilde\fork^{\sign(\params,\sk^*,\cdot,\cdot)}_\bdv(\params,\pk^*)
      \end{aligned}:
      \widetilde\opk \neq \widetilde w\cdot \opk^*_t \land \widetilde\opk \neq \bot
    \right]\label{eq:redhvcm0notbot}\\
    \geq&\Pr\left[
      \begin{aligned}
      ((\params,\pk^*),(\params,\sk^*)) \gets \IG(\secparam);\\
      (b,\pubkeys,t,m,\aggsig^0,\vec{w}^0,\aggsig^1,\vec{w}^1)\gets \widetilde\fork^{\sign(\params,\sk^*,\cdot,\cdot)}_\bdv(\params,\pk^*)
      \end{aligned}:
      \widetilde\opk \neq \widetilde w\cdot \opk^*_t \land b=1
    \right]\label{eq:redhvc}
  \end{align}
  where \autoref{eq:redhvcsimcorrect} follows because the inputs $\rdv_0$ provides to the forking algorithm are distributed identically to those sampled by $\IG$.
  \autoref{eq:redhvcm0notbot} follows because by the above observation $(w^j_0-w^j_1)\cdot\opk^*_t$ cannot be $\bot$.
  Finally \autoref{eq:redhvc} follows from \autoref{lem:forkingisuseful} and the fact that the forking algorithm only outputs $b=1$ if both multi-signatures verify.
  
  Since the homomorphic vector commitment is assumed to be position binding, it thus follows that
  \begin{equation}
    \negl \geq \Pr\left[\begin{aligned}((\params,\pk^*),(\params,\sk^*)) \gets& \IG(\secparam);\\ (b,\pubkeys,t,m,\aggsig^0,\vec{w}^0,\aggsig^1,\vec{w}^1)\gets& \widetilde\fork^{\sign(\params,\sk^*,\cdot,\cdot)}_\bdv(\params,\pk^*)\end{aligned} :\begin{aligned} &\widetilde\opk \neq \widetilde w \cdot \opk^*_{t} \\ \land& b=1\end{aligned}\right]\label{opkstardifferentisnegl}
  \end{equation}
  
  Consider now $\rdv_1$ described in \autoref{fig:redhots} as an adversary against the existential unforgeability under rerandomized keys of $\hots$.
  \begin{figure}\centering
%  \begin{pchstack}[boxed]
  \fbox{\procedure{$\rdv_1(\hotsparams,\opk_0,\dots\opk_{2^\tau-1})$}{
    \hvcparams\gets\hvc.\setup(\secparam)\\
    \params \coloneqq (\hotsparams,\hvcparams,\tau)\\
    \opks^* \coloneqq (\opk_0,\dots,\opk_{2^\tau-1})\\
    c^* \gets \hvc.\commit(\hvcparams,\opks^*)\\
%    e \coloneqq 0\\
    (b,\pubkeys,t,m,\aggsig^0,\vec{w}^0,\aggsig^1,\vec{w}^1)\gets\widetilde\fork_\bdv^{\widetilde{\sign}(\cdot,\cdot)}(\params,c^*)\\
    \pcif b=1\\
    \quad\pcparse \pubkeys \pcas (c_0,\dots,c_\ell)\\
    \quad\pcparse \aggsig^0 \pcas (\sigma^0, d^0)\\
    \quad\pcparse \aggsig^1 \pcas (\sigma^1, d^1)\\
    \quad\pcfor j\in[\ell-1]\\
    \quad\quad \pcif c_j = c^*\\
    \quad\quad\quad \pcreturn (t,m,(\sigma^0-\sigma^1),w^0_j-w^1_j)\\
    \pcreturn \bot
  }}
%  \pchspace
%  \procedure{$\widetilde{\sign}(t,m)$}{
%    \pcif t = t^*\\
%%    \quad \pcif e=0\\
%    \quad \pcoutput m \pcand\pcreceive \otsig\\
%%    \quad\quad e \coloneqq 1\\
%%    \quad \pcelse\\
%%    \quad \quad \pcabort\pcand\pcreturn\bot\\
%    \pcelse\\
%    \quad\otsig \gets \hots.\sign(\hotsparams,\osk_t,m)\\
%%    \otsig \gets \hots.\sign(\hotsparams,\osk_t,m)\\
%    d \gets \hvc.\open(\hvcparams,c^*,\opks^*,t)\\
%    \pcreturn (\otsig,d)
%  }
%  \end{pchstack}
  \caption{A reduction that uses the forking algorithm $\widetilde\fork_\bdv$ to attack the multi-user existential unforgeability under rerandomized keys of $\hots$.}\label{fig:redhots}
  \end{figure}
  We analyze the success probability of $\rdv_1$ by observing that
  \begin{align}
    &\Pr\left[
      \begin{aligned}
      \hotsparams \gets \hots.\setup(\secparam);\\
      \forall i \in [T-1]\ldotp (\osk_i,\opk_i) \gets \hots.\kgen(\params);\\
      (i^*,m^*,\sigma^*,\widetilde w) \gets \rdv_1^{\widetilde\sign(\cdot,\cdot)}(\hotsparams,\params,\opks);\\
      \end{aligned}:
      \begin{aligned}
      \hots.\wverify(\params,\widetilde\cdot\opk_{i^*},m^*,\sigma^*) = 1\\
      \land m^* \not\in Q_{i^*} \land \widetilde w\in W'
      \end{aligned}
    \right]
    \\
%    =&\Pr\left[
%      \begin{aligned}
%      ((\params,\pk^*),(\params,\sk^*)) \gets \IG(\secparam);\\
%      (\osk,\opk) \gets\mspace{\medmuskip} \kgen(\params);\\
%      m \gets\mspace{\medmuskip} \rdv_1(\params,\opk);\\
%      \sigma \gets\mspace{\medmuskip} \sign(\params,\osk,m);\\
%      (m^*,\sigma^*,w^*) \gets\mspace{\medmuskip} \rdv_1(\sigma)\\
%      \end{aligned}:\begin{aligned}
%      \hots.\wverify(\params,w^*\cdot\opk,m^*,\sigma^*) = 1\\
%      \land m^* \neq m \land \norm{w^*}\in \tern_\alpha
%      \end{aligned}
%    \right]\\
    \geq&\Pr\left[
      \begin{aligned}
      ((\params,\pk^*),(\params,\sk^*)) \gets \IG(\secparam);\\
%      t^* \gets [2^\tau-1]\\
      \left(\begin{aligned}b,\pubkeys,t,m,\\\aggsig^0,\vec{w}^0,\\\aggsig^1,\vec{w}^1\end{aligned}\right)\gets \widetilde\fork^{\sign(\params,\sk^*,\cdot,\cdot)}_\bdv(\params,\pk^*)\\
      \end{aligned}:
      \begin{aligned}
        \hots.\wverify(\params,\widetilde w\cdot\opk_{t},m,\sigma^0-\sigma^1) = 1\label{eq:succifdiffverandguesscorr}\\
        \land (\nexists \sigma'\ldotp (m,t,\sigma') \in Q)
      \end{aligned}
    \right]\\
    \geq&\Pr\left[
      \begin{aligned}
      ((\params,\pk^*),(\params,\sk^*)) \gets \IG(\secparam);\\
      \left(\begin{aligned}b,\pubkeys,t,m,\\\aggsig^0,\vec{w}^0,\\\aggsig^1,\vec{w}^1\end{aligned}\right)\gets \widetilde\fork^{\sign(\params,\sk^*,\cdot,\cdot)}_\bdv(\params,\pk^*)\\
      \end{aligned}:
      \begin{aligned}
        \widetilde \opk = \widetilde w\cdot\opk_{t}\\
        \land \hots.\wverify(\params,\widetilde \opk,m,\sigma^0-\sigma^1) = 1\\
        \land (\nexists \sigma'\ldotp (m,t,\sigma') \in Q)
      \end{aligned}
    \right]\label{eq:guessindependent}\\
    \geq&\Pr\left[
      \begin{aligned}
      ((\params,\pk^*),(\params,\sk^*)) \gets \IG(\secparam);\\
      \left(\begin{aligned}b,\pubkeys,t,m,\\\aggsig^0,\vec{w}^0,\\\aggsig^1,\vec{w}^1\end{aligned}\right)\gets \widetilde\fork^{\sign(\params,\sk^*,\cdot,\cdot)}_\bdv(\params,\pk^*)\\
      \end{aligned}:
      \begin{aligned}
        \widetilde \opk = \widetilde w\cdot\opk_{t^*}\\
        \land b=1
      \end{aligned}
    \right].\label{eq:usethatforkingworks}
  \end{align}
  Here \autoref{eq:succifdiffverandguesscorr} follows by observing that the inputs $\rdv_1$ provides to the forking algorithm are distributed identically to those sampled by $\IG$ and $\rdv_1$ is successful whenever the difference of the two signatures verifies.
 Finally, \autoref{eq:usethatforkingworks} follows from \autoref{lem:forkingisuseful} and the fact that the forking algorithm only outputs $b=1$ if both multi-signatures verify.
  
Since $\hots$ is assumed to be multi-user existentially unforgeable under rerandomized keys, it thus follows that
  \begin{equation}
    \negl \geq \Pr\left[
      \begin{aligned}
      ((\params,\pk^*),(\params,\sk^*)) \gets \IG(\secparam);\\
      \left(\begin{aligned}b,\pubkeys,t,m,\\\aggsig^0,\vec{w}^0,\\\aggsig^1,\vec{w}^1\end{aligned}\right)\gets \widetilde\fork^{\sign(\params,\sk^*,\cdot,\cdot)}_\bdv(\params,\pk^*)\\
      \end{aligned}:
      \begin{aligned}
        \widetilde \opk = \widetilde w\cdot\opk_{t^*}\\
        \land b=1
      \end{aligned}
    \right]\label{opkstarsameisnegl}
  \end{equation}
  Combining Equations~\ref{opkstardifferentisnegl} and \ref{opkstarsameisnegl} with \autoref{eq:forkingresult} we can conclude that
  \[
    \epsilon \leq \frac{p}{\abs{\tern_\alpha}} + \sqrt{p\cdot\negl}
  \]
  which implies that $\epsilon$ is negligible and concludes the proof.\qed
  \fi
\end{proof}


%\begin{table}
%  \input{scripts/chipmunk/params_table}
%\end{table}


\begin{acks}
   Nils Fleischhacker was supported by the \grantsponsor{DFG}{Deutsche Forschungsgemeinschaft}{https://www.dfg.de/} (DFG, German Research Foundation) under Germany's Excellence Strategy - \grantnum{DFG}{EXC 2092 CASA - 390781972}.
\end{acks}


%% Bibliography
\bibliographystyle{ACM-Reference-Format}
\bibliography{../cryptobib/abbrev0,../cryptobib/crypto,../extraref}
%\newpage
%\section*{Supplementary Material}
%\input{sections/appendix}
\begin{appendix}
  % !TEX root = ../main.tex
%%% NOTE: I changed notations. Be sure that gamma is the correct thing.

% \begin{table}
% \centering
% \begin{tabular}{@{\makebox[3em][r]{\rownumber\space}} l@{\hspace{3em}}rl}
% \toprule
%  \multicolumn{1}{@{\makebox[3em][r]{\#\space}} c}{Source}&\multicolumn{2}{c}{Constraint}\\
% \midrule
%  \autoref{lem:kots_correct}& $\beta_\sigma \geq$&$ 4\varphi\alpha_H\sqrt{\tfrac{1}{2}\alpha_w\rho(\varepsilon+1+\log_2n\gamma)\cdot\ln2}$\\
%  \autoref{lem:keyhidden}& $\gamma\geq$&$((3\secpar+\delta)/n+\log_2q)\log^{-1}_2(\varphi+\tfrac{1}{2})$\\
%  \autoref{lem:keyhidden}& $\abs{\tern_{\alpha_H}} \leq$&$ 2^{2\secpar + \delta}$\\
%  \autoref{lem:nilssupportivechildsupport}&$q>$&$ 16 \alpha_w \alpha_H\varphi$\\
%  \autoref{lem:kots_sis}&$\abs{\tern_{\alpha_H}} \geq$&$ 2^{2\secpar}$\\
%  \autoref{lem:hvcprobhom}&$\bagg \geq$&$ \eta\sqrt{2\alpha_w\rho(\epsilon + 1 + \log_2 n + \log_2(2\tau \lceil\log_{2\eta+1}q\rceil + \xi\lceil\log_{2\eta+1}q'\rceil))\cdot\ln2}$
% %\bottomrule
% \end{tabular}
% \caption{The constraints a set of Chipmunk parameters needs to satisfy.}\label{tab:constraints}
% \end{table}
\section{Concrete Parameters}\label{sec:concreteparams}

We used a script\footnote{
\url{https://github.com/GottfriedHerold/Chipmunk}%label removed -- Gotti, was causes errors due to multiple-defined label
}
to find concrete parameters that allow for instantiating Chipmunk based on a hard ring-SIS problem.
We have used a fixed ring dimension $n = 512$.
A selection of possible parameter choices is given in Table~\ref{tab:param}.

% ┌──────────┬───────┬────────┬───────────┬───────────┬───────┬───────────┬───────┬─────────┬──────────────┬──────────┬───────┬────────────┬─────────┬────────┐
% │   secpar │   tau │    rho │ epsilon   │   alpha_w │   chi │   alpha_H │   phi │   gamma │   beta_sigma │       q' │   eta │   beta_agg │       q │ size   │
% ├──────────┼───────┼────────┼───────────┼───────────┼───────┼───────────┼───────┼─────────┼──────────────┼──────────┼───────┼────────────┼─────────┼────────┤
% │      112 │    21 │   1024 │ 1/65536   │        16 │    11 │        37 │    13 │       6 │       775143 │  3168257 │    29 │      25133 │  202753 │ 120 KB │
% │      112 │    21 │   8192 │ 1/65536   │        16 │    11 │        37 │    16 │       6 │      2698381 │ 10872833 │    49 │     120108 │  962561 │ 135 KB │
% │      112 │    21 │ 131072 │ 1/65536   │        16 │    11 │        37 │    13 │       7 │      8803786 │ 35301377 │    99 │     970665 │ 7790593 │ 158 KB │
% │      112 │    24 │   1024 │ 1/65536   │        16 │    11 │        37 │    13 │       6 │       775143 │  3168257 │    29 │      25202 │  202753 │ 135 KB │
% │      112 │    24 │   8192 │ 1/65536   │        16 │    11 │        37 │    16 │       6 │      2698381 │ 10872833 │    49 │     120438 │  964609 │ 152 KB │
% │      112 │    24 │ 131072 │ 1/65536   │        16 │    11 │        37 │    13 │       7 │      8803786 │ 35301377 │    99 │     973331 │ 7790593 │ 178 KB │
% │      112 │    26 │   1024 │ 1/65536   │        16 │    11 │        37 │    13 │       6 │       775143 │  3168257 │    29 │      25243 │  202753 │ 145 KB │
% │      112 │    26 │   8192 │ 1/65536   │        16 │    11 │        37 │    16 │       6 │      2698381 │ 10872833 │    50 │     123098 │  995329 │ 163 KB │
% │      112 │    26 │ 131072 │ 1/65536   │        16 │    11 │        37 │    13 │       7 │      8803786 │ 35301377 │    99 │     974934 │ 7806977 │ 191 KB │
% │      128 │    21 │   1024 │ 1/65536   │        19 │    13 │        44 │     9 │       7 │       698123 │  2856961 │    31 │      29276 │  249857 │ 121 KB │
% │      128 │    21 │   8192 │ 1/65536   │        19 │    13 │        44 │     8 │       8 │      1761048 │  7114753 │    12 │      32252 │  270337 │ 167 KB │
% │      128 │    21 │ 131072 │ 1/65536   │        19 │    13 │        44 │     9 │       8 │      7924716 │ 31776769 │    17 │     182757 │ 1492993 │ 197 KB │
% │      128 │    24 │   1024 │ 1/65536   │        19 │    13 │        44 │     9 │       7 │       698123 │  2856961 │    31 │      29357 │  249857 │ 136 KB │
% │      128 │    24 │   8192 │ 1/65536   │        19 │    13 │        44 │     8 │       8 │      1761048 │  7114753 │    12 │      32339 │  270337 │ 188 KB │
% │      128 │    24 │ 131072 │ 1/65536   │        19 │    13 │        44 │     9 │       8 │      7924716 │ 31776769 │    17 │     183255 │ 1492993 │ 222 KB │
% │      128 │    26 │   1024 │ 1/65536   │        19 │    13 │        44 │     9 │       7 │       698123 │  2856961 │    31 │      29405 │  249857 │ 146 KB │
% │      128 │    26 │   8192 │ 1/65536   │        19 │    13 │        44 │     8 │       8 │      1761048 │  7114753 │    12 │      32392 │  270337 │ 202 KB │
% │      128 │    26 │ 131072 │ 1/65536   │        19 │    13 │        44 │     9 │       8 │      7924716 │ 31776769 │    17 │     183554 │ 1492993 │ 239 KB │
% └──────────┴───────┴────────┴───────────┴───────────┴───────┴───────────┴───────┴─────────┴──────────────┴──────────┴───────┴────────────┴─────────┴────────┘

\bgroup
\setlength{\tabcolsep}{0.5em}
\renewcommand{\arraystretch}{1.05}
\begin{table}\centering
    \begin{tabular}{ccr|cc|cccrr|crr|c}%\hline
  
      \multicolumn{3}{c|}{Parameter Sets}
      & \multicolumn{2}{c|}{}
      & \multicolumn{5}{c|}{KOTS Parameters}  
      & \multicolumn{3}{c|}{HVC Parameters}  
      &  {\bf Agg. Sig. Size}\\%\cline{1-3}\cline{5-9}\cline{10-12}
      
      $\lambda$      & $\tau$
      & \multicolumn{1}{c|}{$\rho$}& $\alpha_w$ & $\chi$  &$\alpha_H$& $\varphi$ 
      & $\gamma$                      & \multicolumn{1}{c}{$\beta_\sigma$}     
      & \multicolumn{1}{c|}{$q'$}                     & $\eta$       
      & $\bagg\quad$                  & \multicolumn{1}{c|}{$q$} 
      & (Kilobytes) \\\toprule
  
  
      % &       &        &            &  &  \multicolumn{4}{c||}{HVC}  &  & \multicolumn{2}{c||}{KOTS}  & (Kilobytes) \\\hline\hline
      &       &   1024 &         16 &        11 &        37 &         13 &       6 &        775,143 &  3,168,257 &    29 &       25,133 &   202,753 & 120 \\%\cline{3-13}
      &    21 &   8192 &         16 &        11 &        37 &         16 &       6 &      2,698,381 & 10,872,833 &    49 &      120,108 &   962,561 & 135 \\%\cline{3-13}
      &       & 131072 &         16 &        11 &        37 &         13 &       7 &      8,803,786 & 35,301,377 &    99 &      970,665 & 7,790,593 & 158 \\\cline{2-14}
  
      &       &   1024 &         16 &        11 &        37 &         13 &       6 &        775,143 &  3,168,257 &    29 &       25,202 &   202,753 & 135 \\%\cline{3-13}
  112 &    24 &   8192 &         16 &        11 &        37 &         16 &       6 &      2,698,381 & 10,872,833 &    49 &      120,438 &   964,609 & 152 \\%\cline{3-13}
      &       & 131072 &         16 &        11 &        37 &         13 &       7 &      8,803,786 & 35,301,377 &    99 &      973,331 & 7,790,593 & 178 \\\cline{2-14}
  
      &       &   1024 &         16 &        11 &        37 &         13 &       6 &        775,143 &  3,168,257 &    29 &       25,243 &   202,753 & 145 \\%\cline{3-13}
      &    26 &   8192 &         16 &        11 &        37 &         16 &       6 &      2,698,381 & 10,872,833 &    50 &      123,098 &   995,329 & 163 \\%\cline{3-13}
      &       & 131072 &         16 &        11 &        37 &         13 &       7 &      8,803,786 & 35,301,377 &    99 &      974,934 & 7,806,977 & 191 \\\midrule
  
      &       &   1024 &         19 &        13 &        44 &          9 &       7 &        698,123 &  2,856,961 &    31 &       29,276 &   249,857 & 121 \\%\cline{3-13}
      &    21 &   8192 &         19 &        13 &       44 &          8 &       8 &      1,761,048 &  7,114,753 &    12 &       32,252 &   270,337 & 167 \\%\cline{3-13}
      &       & 131072 &         19 &        13 &       44 &          9 &       8 &      7,924,716 & 31,776,769 &    17 &      182,757 & 1,492,993 & 197 \\\cline{2-14}
  
      &       &   1024 &         19 &        13 &       44 &          9 &       7 &        698,123 &  2,856,961 &    31 &       29,357 &   249,857 & 136 \\%\cline{3-13}
  128 &    24 &   8192 &         19 &        13 &       44 &          8 &       8 &      1,761,048 &  7,114,753 &    12 &       32,339 &   270,337 & 188 \\%\cline{3-13}
      &       & 131072 &         19 &        13 &       44 &          9 &       8 &      7,924,716 & 31,776,769 &    17 &      183,255 & 1,492,993 & 222 \\\cline{2-14}
  
      &       &   1024 &         19 &        13 &       44 &          9 &       7 &        698,123 &  2,856,961 &    31 &       29,405 &   249,857 & 146 \\%\cline{3-13}
      &    26 &   8192 &         19 &        13 &       44 &          8 &       8 &      1,761,048 &  7,114,753 &    12 &       32,392 &   270,337 & 202 \\%\cline{3-13}
      &       & 131072 &         19 &        13 &       44 &          9 &       8 &      7,924,716 & 31,776,769 &    17 &      183,554 & 1,492,993 & 239 \\\bottomrule
    \end{tabular}
    \caption{Parameter sets for Chipmunk for a fixed ring dimension $n=512$.}\label{tab:param}
  \end{table}
\egroup


  \ifeprint
  % !TEX root = ../main.tex
\section{Application to Ethereum}\label{sec:ethereum}

% \cameraready{This section (including this text itself) is not \textbackslash input'ed for the cameraready version -- color-coding this as with other text when compiling both versions does not work and I'm too lazy to fix the LaTeX for this.}

One important class of applications is proof-of-stake blockchains such as Ethereum. %other example?
In this section, we briefly discuss how one could apply Chipmunk for Ethereum and explore the tradeoff space involved.
Note, however, that our focus is primarily on Chipmunk and not the blockchain's consensus protocol, which is out of scope of this paper;
an actual application will have to consider interactions between these two aspects in much more detail.

For simplicity\footnote{In particular, we ignore that the validator set may change and that we may form committees of the validator set to only have those vouch}, we assume that we have a public, fixed set of validators of size $\rho$, who vouch for the validity of each block, and are economically incentivized to do so.
Vouching for a block is done by essentially signing it (called an attestation). The consensus protocol gives a natural notion of time slots (currently 12 seconds for Ethereum) and each honest validator must only sign at most one block per time slot. This means that our synchronized multi-signatures fit this setting very well:

assume that we have some synchronized $\rho$-wise multi-signature $\msig = (\setup, \kgen, \sign, \allowbreak \aggregate,\allowbreak \iverify, \averify)$ for $2^{\tau}$ time periods for some $\tau$ with $\params\gets\setup(1^{\secpar})$. Assume for now each validator $v_i$ has some keypair $(\sk_i,\pk_i)\gets \kgen(\params)$ bound to its identity and we have somehow disseminated those public keys.
Simplifying a bit, for time slot $0\leq t <2^\tau$, each validator $v_i$ would attest to some block $m_i$ by signing it as $\sigma_i\gets\sign(\params,\sk_i,t,m_i)$ and send $m_i, \sigma_i$ to some aggregator. Usually, most honest validators will sign the same message. The aggregator checks each such individual signature, picks a message with the most valid signatures and aggregates the valid signatures for this message. The aggregate signature, together with the set of public keys that were used, is then broadcast and ends up on the blockchain.

For Ethereum, the size of the validator set is about $800.000$ at the time of this writing, with most validators actually attesting most of the time, so the parameters for $\msig$ need to support aggregating $\rho\approx 2^{20}$ signatures\footnote{Remember that, for simplicity, we ignore the option of forming committees out of the validator set and having only those be eligible to attest}. Note that no matter the used multi-signature scheme, transmitting which public keys were included in the aggregate will already take up about $800.000$ bits, i.e.\ 100KB if encoded as a bitfield. The information who signed is relevant for the consensus mechanism independently of the actual signature scheme and needs to work in the worst-case, so this cost seems unavoidable. %In particular, this limits what optimizing the signatures for size can achieve.

Interestingly, it turns out that we may be better off choosing a relatively small value for $\tau$ and re-key often. Notably, if we use Chipmunk, there are four main bottlenecks.
\begin{itemize}
\item communication between validators and the aggregator
\item aggregation time
\item size of the aggregate signature.
\item dissemination of public keys
\end{itemize}
For the communication between validators and the aggregator, note that signatures for consecutive slots actually may share a significant portion of their Merkle paths in the vector commitment part.
Consider the case that a validator signs for (almost) all timeslots and communicates with the same aggregator each slot.
Then over the course of $2^\tau$ timeslots, we need to communicate all $2^{\tau+1} - 1$ labels of the labelled full binary tree for the HVC openings, which amounts to approx.\ 2 labels on average per slot.
This can be smoothened by communicating labels in advance, so we assume that per slot we communicate 2 labels, 1 KOTS public key and 1 KOTS signature.
Observe that this does not depend on $\tau$.

Note this this does not fit what Ethereum currently does with BLS\cite{AC:BonLynSha01} signatures, where somewhat simplified, aggregates get aggregated in multiple steps along an aggregation tree. Chipmunk as presented in this paper does not support this (at least not with the given security analysis and noise bounds derived from that). We note that even with such an extension of Chipmunk to tree-based aggregation, the verifier would, to derive the random weights via hashing, need to know the structure of the aggregation tree rather than just, as presented here, the set of public keys whose signatures end up in the final result. Unfortunately, needing to transmtting this extra information may be prohibitive.

% While this is not the case at the moment, Ethereum plans (for unrelated reasons) to create a special role called builder who would perform the aggregation.\gnote{TODO:Citation? Anything better than \url{https://ethereum.org/en/roadmap/pbs/}?} Builders are assumed to be very computationally powerful and have large bandwidth.

Each aggregate signatures consist of (encodings of) $2\tau$ nodes in the Merkle path, 1 aggregated decomposed KOTS keys and 1 aggregated KOTS signature.
For our parameter choices, this gives a size of \gnote{TODO: Concrete values (as a function of $\tau$)}.

\bigskip
Above, we ignored dissemination of public keys, but of course this needs to be taken into account.
When choosing a small value of $\tau$, we actually need to change keys often.
To change a key, we need to broadcast a public key, i.e.\ a single $\ring_q$ element.
During $2^\tau$ timeslots, we need to change $\rho$ keys, so on average this gives a cost of $\frac{2^\tau}{\rho}$ times the cost of a single $\ring_q$ element.

This gives a tradeoff, parameterized by $\tau$, of aggregate signature size vs.\ frequency of re-keying.

A naive way of disseminating keys is inside the actual blocks of the blockchain (i.e.\ together with the aggregated signatures).
In this case, we can choose $\tau$ to minimize the total size needed (on average) per block.
\gnote{State where the optimum is; needs the above TODO resolved.}
However, for reasons related to how actual blockbuilding works, it is preferable to not rekey each validator exactly once during $2^\tau$ slots, but rather to store up to $T$ Chipmunk public keys for each validator for the next up to $T\times 2^\tau$ slots and top this number up to $T$ when rekeying.
In particular, any re-keying disseminates a key that is used far in the future.
For that reason, the latency requirement for re-keying messages is several orders of magnitudes weaker than for the aggregate signatures;
note that latency is actually a significant reason why the blocksize is a bottleneck, so it might be worthwhile to lower $\tau$ even further, only commit to re-keying on-chain and disseminate the actual keys off-chain.
However, a detailed analysis of such a strategy is out of scope of this paper.

  \fi
\end{appendix}
\end{document}



